\pdfbookmark[1]{Typographical Conventions}{Typographical Conventions}
\chapter*{Typographical Conventions}
I use a combination of colors and fonts to make distinctions between terms in
different languages clear.
Often, programs from more than one language will appear in the same equation.
While which language each program belongs to will be clear from the structure of
the equations, I use colors and fonts to make the distinction more apparent.
These colors and fonts are chosen to be distinguishable in color, in grayscale,
and with various forms of color blindness.

Each chapter will involve at least two languages: a source and a target
language.
Source languages expressions and metavariables are typeset in a \sfonttext{blue,
  non-bold, sans-serif font}.
Target language expressions and metavariables are typeset in \tfonttext{bold,
  red, serif font}.
These different fonts do not indicate \emph{particular} languages---they only
indicate source and target.
I will remind the reader which language is which each time a new language is
introduced.

For languages other than source and target languages, such as examples of
concepts or from related work, I use a {black, non-bold, serif font}.

I use emphasis to introduce a word with a \deftech{technical definition}---\ie,
a specific technical use that may differ from the normal English definition.

\begin{typographical}
I will remind the reader of these typographical conventions in the section I
first use them, or when languages change, in a typographical note like this.
\end{typographical}
