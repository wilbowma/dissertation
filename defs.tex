% Misc

\let\lemma\undefined
\newtheorem{lemma}[theorem]{Lemma}
\Crefname{lemma}{Lemma}{Lemmas}

\newtheorem{conjecture}[theorem]{Conjecture}

\newcommand{\code}[1]{\mintinline{agda}{#1}}
% Don't want omitthis because it is conditionally displayed
\newcommand{\discard}[1]{}

\newenvironment{bnfgrammar}
{\begin{displaymath}\begin{array}{lrcl}}
{\end{array}\end{displaymath}}

\newcommand{\bnfnewline}{
  \\[6pt]
}

\newenvironment{reductionrules}
{\begin{displaymath}\begin{array}{rlll}}
{\end{array}\end{displaymath}}

\newcommand{\stepnewline}{
  \\[4pt]
}

\newcommand{\False}{\ensuremath{\bot}\xspace}
\newcommand{\sFalse}{\ensuremath{\sfontsym{\False}}\xspace}
\newcommand{\tFalse}{\ensuremath{\tfontsym{\False}}\xspace}

%% Stuff that belongs in mttex
\newenvironment{proofcases}
{\begin{enumerate}[label={\bfseries Case:},itemindent=2em]\ignorespacesafterend}
{\end{enumerate}}

\newenvironment{lemmacases}
{~\begin{enumerate}\ignorespacesafterend}
{\end{enumerate}\ignorespacesafterend}

\newcommand{\with}[1]{\mathrm{with}\:#1}

%% Coercion
\newexpr{coe}{2}{\kwopen{\langfont{subst}}_{#1}\mathpunct{}#2}

\makeatletter
%% NB: Removes parens around argument from \@fune
\renewcommand{\@fune}[3]{%
  \kwopen{\langsymfont{\uplambda}}%
  {#1} \mathbin{:} {#2}%
  \mathclose{.}%
  \mathpunct{}#3%
}

%% NB: use ; as separator instead of |
\renewcommand{\@casee}[5]{%
  \kwopen{\langfont{case}}%
  #1%
  \kwbin{\langfont{of}}%
  #2 \mathpunct{.} #3%
  \mathpunct{\langsymfont{;}}%
  #4 \mathpunct{.} #5%
}

\newexpr{nlet}{2}{%
  \kwopen{\langfont{let}}%
  #1%
  \kwbin{\langfont{in}}%
  #2}
\makeatother

\newcommand{\len}[1]{|#1|}
\newtheorem*{digression}{Digression}
\newtheorem*{typographical}{Typographical Note}

%% The Generic Language
%% For when I don't want to distinguish between languages, but still want to write terms in prefix.

\newcommand{\gfont}{\mathrm}
\newcommand{\gfonttext}{\nop}
\newcommand{\gfontsym}{\nop}
\newcommand{\gcolor}{\nop}
\newlanguage{\gcolor}{\gfont}{\gfontsym}{g}
{}
{lenv/\Gamma}
{pi,pair,fun,bool,sigma,sum,unit}
{fun,app,pair,if,fst,snd,true,false,case,sum,unit,nfun}

\newmetavars{\nop}{\nop}{g}{x,e,U,A,B,K,C,k}

\newcommand{\gFalse}{\False}
\newcommand{\ginje}{\gsume}

\newcommand{\gjudg}{\judg}

\newcommand{\model}[1]{\sembrace{#1}^\circ}

%% The Source Language Macros
%% These macros are used for both CoC and ECC

% The slang macro should be redefine in each part.
\renewcommand{\slang}{ECC\xspace}
\renewcommand{\scolor}[1]{\mathcolor{RoyalBlue}{#1}}

\newlanguage{\scolor}{\sfont}{\sfontsym}{s}
{x,y,e,t,U,i,f,A,B,C,v,a,n,k,z,N,M,K,V,p,g}
{lenv/\Gamma,alpha/\alpha,beta/\beta,subst/\gamma}
{pi,fun,sigma,type,dpair,sum,prop,set,bool,npi,star,box,unit,exist,code}
{fun,app,alet,let,fst,snd,dpair,pair,sum,case,true,false,if,nfun,unit,pack,unpack,coe,npair,clo}
% TODO Do I need nfune and npity?

\newcommand{\sinje}{\ssume}

\renewcommand{\step}{\vartriangleright}
\renewcommand{\stepstar}{\step^*}

\newcommand{\subtypesym}{\preceq}
\newcommand{\sstepjudg}[4][\step]{\wf{#2}{#3\mathbin{#1}#4}}
\newcommand{\sequivjudg}[3]{\wf{#1}{#2\mathbin{\equiv}#3}}
\newcommand{\ssubtyjudg}[3]{\wf{#1}{#2\mathbin{\subtypesym}#3}}
\newcommand{\styjudg}[3]{\judg{#1}{#2}{#3}}
\newcommand{\swf}[1]{\wf{}{#1}}

\newcommand{\ecceqv}[4]{\judg{#1}{#2 \approx #3}{#4}}

\newcommand{\seval}[1]{\sfont{eval}(#1)}

\newcommand{\ssubstok}{\wf}

%% Parametric CPS Macros:
\newcommand{\cpsslang}{CoC$^{D}$\xspace}
\newcommand{\eccidk}{\sfont{id}}
\newcommand{\cps}{CoC$^{k}$}
\newcommand{\cpstlang}{\cps\xspace}
\newcommand{\cpsprime}{\tprime}

\newlanguage{\tcolor}{\tfont}{\tfontsym}{cps}
{k,x,y,e,t,U,A,B,z,f,v,l}
{alpha/\alpha,lenv/\Gamma,subst/\gamma,K/\kappa}
{pi,pair,sigma,fun,sum,prop,set,type,box,star,bool}
{fun,app,alet,let,fst,snd,pair,dpair,nfun,sum,case,true,false,if}

%\newcommand{\cpsstarty}{\cpspropty}
%\newcommand{\cpsboxty}{\cpstypety{}}

\newcommand{\cpscappe}[3]{\cpsappe{#1}{\tfontsym{@}~#2~#3}}
\newcommand{\cpsncappe}[3]{\cpsappe{#1}{#2~#3}}
\newcommand{\cpsidk}{\tfont{id}}

\newcommand{\cpswf}{\swf}
\newcommand{\cpstyjudg}{\styjudg}
\newcommand{\cpsstepjudg}{\sstepjudg}
\newcommand{\cpsequivjudg}{\sequivjudg}
\newcommand{\cpssubtyjudg}{\ssubtyjudg}

\newcommand{\cpsterm}[1]{\textsc{cps}\sembrace{#1}}
\newcommand{\cpstype}[1]{\textsc{cpsT}\sembrace{#1}}

\newcommand{\cbnsym}[1]{\leadsto^{n}_{#1}}
\newcommand{\cbnsymK}{\cbnsym{\kappa}}
\newcommand{\cbnsymU}{\cbnsym{U}}
\newcommand{\cbnsymA}{\cbnsym{A}}
\newcommand{\cbnsymAdiv}{\cbnsym{A^{\div}}}
\newcommand{\cbnsyme}{\cbnsym{e}}
\newcommand{\cbnsymlenv}{\cbnsym{}}
\newcommand{\cbnU}[3]{\wf{#1}{#2\mathbin{\cbnsymU}#3}}
\newcommand{\cbnK}[4]{\judg{#1}{#2}{#3\mathbin{\cbnsymK}#4}}
\newcommand{\cbnA}[4]{\judg{#1}{#2}{#3\mathbin{\cbnsymA}#4}}
\newcommand{\cbnAdiv}[4]{\judg{#1}{#2}{#3\mathbin{\cbnsymAdiv}#4}}
\newcommand{\cbne}[4]{\judg{#1}{#2}{#3\mathbin{\cbnsyme}#4}}
\newcommand{\cbnlenv}[2]{\wf{}{#1\cbnsymlenv}#2}

\newcommand{\cbnnametext}{CPS$^{n}$\xspace}
\newcommand{\cbnname}[1][{}]{CPS$^{n}#1$}
\newcommand{\cbnAname}{\cbnname[_A]}
\newcommand{\cbnArule}[1]{\refrule*{cbnA:#1}{\cbnAname-#1}}
\newcommand{\defcbnArule}[1]{\defrule*{cbnA:#1}{\cbnAname-#1}}
\newcommand{\cbnAdivname}{\cbnname[_{A^\div}]}
\newcommand{\cbnAdivrule}[1]{\refrule*{cbnAdiv:#1}{\cbnAdivname-#1}}
\newcommand{\defcbnAdivrule}[1]{\defrule*{cbnAdiv:#1}{\cbnAdivname-#1}}
\newcommand{\cbnename}{\cbnname[_e]}
\newcommand{\cbnerule}[2][\techprefix]{\refrule*[#1]{cbne:#2}{\cbnename-#2}}
\newcommand{\defcbnerule}[1]{\defrule*{cbne:#1}{\cbnename-#1}}
\newcommand{\cbnlenvname}{\cbnname[_\Gamma]}
\newcommand{\cbnlenvrule}[2][\techprefix]{\refrule*[#1]{cbnlenv:#2}{\cbnlenvname-#2}}
\newcommand{\defcbnlenvrule}[1]{\defrule*{cbnlenv:#1}{\cbnlenvname-#1}}
\newcommand{\cbnKname}{\cbnname[_\kappa]}
\newcommand{\cbnKrule}[1]{\refrule*{cbnK:#1}{\cbnKname-#1}}
\newcommand{\defcbnKrule}[1]{\defrule*{cbnK:#1}{\cbnKname-#1}}
\newcommand{\cbnUname}{\cbnname[_U]}
\newcommand{\cbnUrule}[1]{\refrule{\cbnUname-#1}}
\newcommand{\defcbnUrule}[1]{\defrule*{cbnU:#1}{\cbnUname-#1}}

\newcommand{\cbvsym}[1]{\leadsto^{v}_{#1}}
\newcommand{\cbvsymK}{\cbvsym{\kappa}}
\newcommand{\cbvsymU}{\cbvsym{U}}
\newcommand{\cbvsymA}{\cbvsym{A}}
\newcommand{\cbvsymAdiv}{\cbvsym{A^{\div}}}
\newcommand{\cbvsyme}{\cbvsym{e}}
\newcommand{\cbvsymlenv}{\cbvsym{}}
\newcommand{\cbvU}[3]{\wf{#1}{#2\mathbin{\cbvsymU}#3}}
\newcommand{\cbvK}[4]{\judg{#1}{#2}{#3\mathbin{\cbvsymK}#4}}
\newcommand{\cbvA}[4]{\judg{#1}{#2}{#3\mathbin{\cbvsymA}#4}}
\newcommand{\cbvAdiv}[4]{\judg{#1}{#2}{#3\mathbin{\cbvsymAdiv}#4}}
\newcommand{\cbve}[4]{\judg{#1}{#2}{#3\mathbin{\cbvsyme}#4}}
\newcommand{\cbvlenv}[2]{\wf{}{#1\cbvsymlenv}#2}

\newcommand{\cbvnametext}{CPS$^{v}$\xspace}
\newcommand{\cbvname}[1][{}]{CPS$^{v}#1$}
\newcommand{\cbvAname}{\cbvname[_A]}
\newcommand{\cbvArule}[1]{\rulename{\cbvAname-#1}}
\newcommand{\cbvAdivname}{\cbvname[_{A^div}]}
\newcommand{\cbvAdivrule}[1]{\rulename{\cbvAdivname-#1}}
\newcommand{\cbvename}{\cbvname[_e]}
\newcommand{\cbverule}[1]{\rulename{\cbvename-#1}}
\newcommand{\cbvlenvname}{\cbvname[_\Gamma]}
\newcommand{\cbvlenvrule}[1]{\rulename{\cbvlenvname-#1}}
\newcommand{\cbvKname}{\cbvname[_\kappa]}
\newcommand{\cbvKrule}[1]{\refrule*{cbvK:#1}{\cbvKname-#1}}
\newcommand{\defcbvKrule}[1]{\defrule*{cbvK:#1}{\cbvKname-#1}}
\newcommand{\cbvUname}{\cbvname[_U]}
\newcommand{\cbvUrule}[1]{\refrule*{cbvU:#1}{\cbvUname-#1}}
\newcommand{\defcbvUrule}[1]{\defrule*{cbvU:#1}{\cbvUname-#1}}

% CPS Model translation

\newcommand{\anfsym}{\leadsto_{\circ}}
\newcommand{\anfjudg}[4]{\judg{#1}{#2}{#3 \mathrel{\anfsym} #4}}
\newcommand{\cpsmodel}[1]{{#1}^\circ}

\newcommand{\cccolor}[1]{#1}
\newcommand{\ccfont}[1]{#1}
\newcommand{\ccfonttext}[1]{\textit{\cccolor{#1}}}
\newcommand{\ccfontsym}[1]{#1}
\newcommand{\ccFalse}{\ccfont{\False}}

\newlanguage{\cccolor}{\ccfont}{\ccfontsym}{cc}
{k,x,y,e,t,U,A,B,z,f,p}
{alpha/\alpha,lenv/\Gamma,subst/\gamma,K/\kappa}
{pi,star,box,pair,sigma,fun}
{nfun,fun,app,alet,let,fst,snd,pair,dpair}

\newcommand{\ccstepjudg}[4][\step]{\wf{#2}{#3\mathbin{#1}#4}}
\newcommand{\ccequivjudg}[3]{\wf{#1}{#2\mathbin{\equiv}#3}}
\newcommand{\cctyjudg}[3]{\judg{#1}{#2}{#3}}
\newcommand{\ccidk}{\ccfont{id}}

\newcommand{\paramtr}[1]{\sembrace{#1}^{\mathcal{P}}}

%% Target language for ANF and CC


\renewcommand{\tcolor}[1]{\mathcolor{OrangeRed}{#1}}

%% Heterogeneous plugging
%% (heterogeneous hole-with)
\newcommand{\hhw}[1]{\boldsymbol{\langle}\!\boldsymbol{\langle}#1\boldsymbol{\rangle}\!\boldsymbol{\rangle}}

%% Heterogeneous subst
\newcommand{\hsubst}[3]{#1[#2/\!/#3]}

\newlanguage{\tcolor}{\tfont}{\tfontsym}{t}
{x,y,N,V,K,M,A,B,U,i,f,j,e,C,k,z,v,p,E}
{lenv/\Gamma,alpha/\alpha,beta/\beta}
{pi,fun,sigma,type,dpair,sum,prop,bool}
{fun,app,let,fst,snd,dpair,pair,sum,case,coe,true,false,if,nlet}

\newcommand{\tinje}{\tsume}

\newcommand{\tstepjudg}[4][\mapsto]{#3 #1 #4}
\newcommand{\tequivjudg}{\sequivjudg}
\newcommand{\tsubtyjudg}{\ssubtyjudg}
\newcommand{\ttyjudg}{\styjudg}
\newcommand{\twf}{\swf}

% ANF Translation
\newcommand{\anfslang}{ECC$^{D}$\xspace}
\newcommand{\anftlang}{ECC$^{A}$\xspace}
\newcommand{\anf}[2]{\sembrace{#1}{#2}}
\newcommand{\anfh}[1]{\sembrace{#1}}

\newcommand{\teval}[1]{\tfont{eval}(#1)}

% blue on the outside but red on the inside
%\newcommand{\sbound}[1]{\sfont{\boldsymbol{\big((}}#1\sfont{\boldsymbol{)\big)}}}
\newcommand{\sbound}[1]{\mathcal{ST}(#1)}

% Exported definitions of the translation of #1
\newcommand{\tedefs}[1]{\mathtt{defs}{({#1})}}
\newcommand{\edefs}[1]{\tedefs{\anfh{#1}}}

% Exported hole of the translation of #1
\newcommand{\tehole}[1]{\mathtt{hole}{({#1})}}
\newcommand{\ehole}[1]{\tehole{\anfh{#1}}}

%% Closure Conversion:
\newcommand{\absccslang}{ECC$^{D}$\xspace}
\newcommand{\abscctlang}{ECC$^{CC}$\xspace}
\newcommand{\absccmodel}[1]{\sembrace{#1}^{\circ}}
\newcommand{\modelsym}{\leadsto_{\circ}}
\newcommand{\mjudg}[4]{\judg{#1}{#2}{#3 \mathbin{\modelsym} #4}}

\newcommand{\sNat}{\sfont{Nat}}
\newcommand{\sS}{\sfont{S}}

\makeatletter
\renewcommand{\@existty}[3]{
  \kwopen{\langsymfont{\exists}}%
  {#1} \mathbin{:} {#2}%
  \mathclose{.}%
  \mathpunct{}#3%
}
\makeatother

\newcommand{\codety}[4]{
  \kwopen{#2{Code}}(#3)\mathpunct{.}#4%
}

\newcommand{\cloe}[4]{
  #2{\langle\!\langle}#3\mathpunct{#2{,}}#4#2{\rangle\!\rangle}%
}

\newcommand{\mcloe}[2]{\langle\!\langle#1,#2\rangle\!\rangle}

\newlanguage{\tcolor}{\tfont}{\tfontsym}{t}
{n,t}
{alpha/\alpha,subst/\gamma}
{unit,pair,nsigma,exist,npi,code}
{unit,npair,dnpair,nfun,clo,pack,npack,npacko,unpack,alet}

\newcommand{\tstarty}{\tpropty}
\newcommand{\tboxty}{\ttypety{}}

\newcommand{\DFV}[2]{\metafont{FV}{(#2, #1)}}

\newcommand{\cctrans}[1]{\sembrace{#1}}
\newcommand{\ccsym}{\leadsto}
\newcommand{\ccenvjudg}[2]{\wf{}{#1\mathbin{\ccsym}#2}}
\newcommand{\ccjudg}[4]{\judg{#1}{#2}{#3 \mathbin{\ccsym} #4}}

%% Parametric Closure Conversion:

% Translucent functions

\makeatletter

\newcommand{\trlufunarrow}{\Rightarrow}

% trlufunty formats a translucent function type.
%
% #1 : The pre-formatted argument to the function
% #2 : The pre-formatted result of the function
\newtype{trlufun}{2}{#1 \mathbin{\langsymfont{\trlufunarrow}} #2}

\renewcommand{\@nexistty}[2]{\kwopen{\langsymfont{\exists}}#1\mathclose{.}\mathpunct{}#2}

\makeatother

\newcommand{\tr}[1]{{#1}^+}
\newcommand{\pccslang}{\cpsslang}
\newcommand{\pcctlang}{CoC$^{CC}$\xspace}
\newcommand{\pccwf}{\swf}
\newcommand{\pcctyjudg}{\styjudg}
\newcommand{\pccstepjudg}{\sstepjudg}
\newcommand{\pccequivjudg}{\sequivjudg}
\newcommand{\pccsubtyjudg}{\ssubtyjudg}

\newcommand{\pcccolor}{\tcolor}
\newcommand{\pccfont}{\tfont}
\newcommand{\pccfontsym}{\tfontsym}
\newcommand{\pccfonttext}{\tfonttext}
\newlanguage{\pcccolor}{\pccfont}{\pccfontsym}{pcc}
{k,x,y,e,t,U,f,A,B,b,v,a,n,c}
{alpha/\alpha,lenv/\Gamma,subst/\gamma}
{pi,npi,nsigma,exist,unit,trlufun,pair,nexist,sigma,bool,box,star,fun}
{nfun,app,alet,if,dnpair,fst,snd,prj,pack,unpack,unit,packo,dpair,npair,let,nunpack,npack,npacko,pair,true,false}

% Allow currying pcccode
\newcommand{\pcccodety}[2]{
  \kwopen{\pccfont{Code}}#1\mathpunct{.}#2%
}

\newcommand{\pccsym}{\leadsto}
\newcommand{\pccenvjudg}[2]{\wf{}{#1\mathbin{\pccsym}#2}}
\newcommand{\pccjudg}[4]{\judg{#1}{#2}{#3 \mathbin{\pccsym} #4}}

\newcommand{\pccsubstok}{\ssubstok}
