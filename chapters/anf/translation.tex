\newcommand{\FigTrans}[1][t]{
  \begin{figure}[#1]
    \judgshape{\anf{\se}{\tK} = \tM}
    \begin{displaymath}
      \begin{array}{rcl}
        \anfh{\se} & \defeq & \anf{\se}{\hole}
        \\

        \anf{\sx}{\tK} &\defeq& \tK\hw{\tx}
        \\

        \anf{\spropty}{\tK} &\defeq& \tK\hw{\tpropty}
        \\

        \anf{\stypety{i}}{\tK} &\defeq& \tK\hw{\ttypety{i}}
        \\

        \anf{\spity{\sx}{\sA}{\sB}}{\tK} & \defeq & \tK\hw{\tpity{\tx}{\anfh{\sA}}{\anfh{\sB}}}
        \\

        \anf{\sfune{\sx}{\sA}{\se}}{\tK} & \defeq & \tK\hw{\tfune{\tx}{\anfh{\sA}}{\anfh{\se}}}
        \\

        \anf{\sappe{\seone}{\setwo}}{\tK} & \defeq & \anf{\seone}{\tlete{\txone}{\hole}{(\anf{\setwo}{\tlete{\txtwo}{\hole}{\tK\hw{\tappe{\txone}{\txtwo}}}})}}
        \\

        \anf{\ssigmaty{\sx}{\sA}{\sB}}{\tK} & \defeq & \tK\hw{\tsigmaty{\tx}{\anfh{\sA}}{\anfh{\sB}}}
        \\

        \anf{\sdpaire{\seone}{\setwo}{\sA}}{\tK} & \defeq & \anf{\seone}{\tlete{\txone}{\hole}{\anf{\setwo}{(\tlete{\txtwo}{\hole}{\tK\hw{(\tdpaire{\txone}{\txtwo}{\anfh{\sA}})}})}}}
        \\

        \anf{\sfste{\se}}{\tK} & \defeq & \anf{\se}{\tlete{\tx}{\hole}{\tK\hw{\tfste{\tx}}}}
        \\

        \anf{\ssnde{\se}}{\tK} & \defeq & \anf{\se}{\tlete{\tx}{\hole}{\tK\hw{\tsnde{\tx}}}}
        \\

        \anf{\slete{\sx}{\se}{\sepr}}{\tK} & \defeq & \anf{\se}{\tlete{\tx}{\hole}{\anf{\sepr}{\tK}}}
      \end{array}
    \end{displaymath}
    \caption{ANF Translation from \slang to \tlang}
    \label{fig:anf:trans}
  \end{figure}
}

\section{{{ANF}} Translation}
\label{sec:anf:trans}

\FigTrans

The \tech{ANF} translation is presented in \fullref[]{fig:anf:trans}.
The translation is defined inductively on the syntax of the source term, and is
indexed by a current \tech{continuation}.
The translation is essentially standard.
When translating a \tech{value} such as \im{\sx}, \im{\sfune{\sx}{\sA}{\se}},
and \im{\stypety{i}}, we essentially plug the \tech{value} into the current
\tech{continuation}, recursively translating the sub-expressions of the
\tech{value} if applicable.
For non-values such as application, we make sequencing explicit by recursively
translating each sub-expression with a \tech{continuation} that binds the result
of the sub-expression and will perform the rest of the \tech{computation}.

Note that if the translation must produce type annotations for input to a
continuation, then defining the translation and typing preservation proof are
somewhat more complicated.
For instance, if we required the \tfonttext{let}-bindings in the target
language to have type annotations for bound expressions, then we would need to
modify the translation to produce those annotations.
This requires defining the translation over typing derivations, so the compiler
has access to the type of the expression and not only its syntax.
I discuss the implications of this in \fullref[]{sec:anf:discuss}.

Next, I show \tech{type preservation} following essentially the standard
architecture presented \fullref[]{chp:type-pres}.
A few additional lemmas are required, and some lemma statements are non-standard,
as I discuss next.

After proving type preservation, I prove correctness of separate compilation for
the ANF machine semantics.
I use the same notion of linking and observations as defined in \fullref[]{chp:source}.
This proof is straightforward from the meta-theory about the machine semantics
proved in \fullref[]{sec:anf:target}, and from equivalence preservation.

\subsection{Type Preservation}
To prove \tech{type preservation}, I follow the same recipe as presented in
\fullref[]{chp:type-pres}.
First, I show \tech{compositionality}, which states that the translation
commutes with substitution, \ie, that substituting first and then translating is
equivalent to translating first and then substituting.
This proof is somewhat non-standard for ANF since the notion of composition in
ANF is not the usual substitution.
Next, I show that \tech{reduction and} \tech{conversion} are preserved up to
\tech{equivalence}, as is standard.
Note that for this theorem, we are interested in the \tech{conversion} semantics used
for definitional \tech{equivalence}, not in the machine semantics.
Then, I show \tech{equivalence preservation}: if two terms are definitionally
\tech{equivalent} in the source, then their translations are definitionally
\tech{equivalent}.
Finally, I can show \tech{type preservation} of the \tech{ANF} translation,
using \tech{continuation} typing to express the inductive invariant required for
\tech{ANF}.
The lemma for \tech{type preservation} is non-standard compared to the generic
statement given in \fullref[]{chp:type-pres}.
The \tech{continuation} typing allows us to formally state \tech{type
  preservation} in terms of the intuitive reason it should be true: because the
definitions expressed by the \tech{continuation} typing suffice to prove
equivalence between a \tech{computation} variable and the original
\tech*{depended upon}{depended-upon} expression.

\begin{typographical}
Recall from \fullref[]{sec:anf:target}, I shift from the \tfonttext{target
  language font} to the \sfonttext{source language font} whenever the term is no
longer in \tech{ANF}, such as when performing standard substitution or
conversion.
\end{typographical}

Before I proceed, I state a property about the syntax produced by the \tech{ANF}
translation, in particular, that the \tech{ANF} translation does produce syntax
in \tech{ANF}.
The proof is straightforward so I elide it.

\begin{theorem}[Normal Form]
  \label{cor:anf:form}
  \im{\anf{\se}{\tKpr} = \tlete{\txone}{\tNone}{\dots\tlete{\txin{n}}{\tNin{n}}{\tKpr\hw{\tNin{n+1}}}}}
\end{theorem}

{
\allowdisplaybreaks % Let latex break proofs on a page boundary
As discussed in \fullref[]{sec:anf:target}, composition in ANF is somewhat non-standard.
Normally, we compose via substitution and compositionality
is \im{\anfh{\subst{\se}{\sepr}{\sx}} \equiv
  \subst{\anfh{\se}}{\anfh{\sepr}}{\tx}}, which says we can either compose
then translate or translate then compose.
However most composition in \tech{ANF} goes through \tech{continuations}, not
through substitution, since only \tech*{anf:value}{values} can be substituted in
\tech{ANF}.
The primary \tech{compositionality} lemma (\fullref[]{lem:anf:comp-stack}) tells
us that we can either first translate a program \im{\se} under
\tech{continuation} \im{\tK} and then compose it with a \tech{continuation}
\im{\tKpr}, or we can first compose the \techs{continuation} \im{\tK} and
\im{\tKpr} and then translate \im{\se} under the composed \tech{continuation}.
Note that this proof is entirely within \tlang; there are no language boundaries.
\begin{lemma}[Compositionality]
  \label{lem:anf:comp-stack}
  \im{\tKpr\hhw{\anf{\se}{\tK}} = \anf{\se}{\tKpr\hhw{\tK}}}
\end{lemma}
\begin{proof}
  By induction on the structure of \im{\se}.
  All \tech{value} cases are trivial.
  The cases for non-values are all essentially similar, by definition of
  composition for \tech{continuations} or \tech*{anf:config}{configurations}.
  I give some representative cases.
  \begin{proofcases}
    \item \im{\se = \sx}

      Must show \im{\tKpr\hhw{\tK\hw{\tx}} = \tKpr\hhw{\tK\hw{\tx}}}, which is trivial.
    \item \im{\se = \spity{\sx}{\sA}{\sB}}

      Must show that \im{\tKpr\hhw{\tK\hw{\tpity{\tx}{\anfh{\sA}}{\anfh{\sB}}}} =
        \tKpr\hhw{\tK{\hw{\tpity{\tx}{\anfh{\sA}}{\anfh{\sB}}}}}}, which is trivial.
      Note that we need not appeal to induction, since the recursive
      translation does not use the current continuation for values.
    \item \im{\se = \sappe{\seone}{\setwo}}
      Must show that
      \begin{align}
        & \tKpr\hhw{(\anf{\seone}{(\tlete{\txone}{\hole}{(\anf{\setwo}{\tlete{\txtwo}{\hole}{\tK\hw{\tappe{\txone}{\txtwo}}}})})})} \nonumber \\
        =~& (\anf{\seone}{(\tlete{\txone}{\hole}{(\anf{\setwo}{\tlete{\txtwo}{\hole}{\tKpr\hhw\tK\hw{\tappe{\txone}{\txtwo}}}})})}) \nonumber
      \end{align}

      The proof follows essentially from the definition of continuation composition.
      \begin{align}
        & \tKpr\hhw{(\anf{\seone}{(\tlete{\txone}{\hole}{(\anf{\setwo}{\tlete{\txtwo}{\hole}{\tK\hw{\tappe{\txone}{\txtwo}}}})})})} \nonumber \\
        =~& (\anf{\seone}{\tKpr\hhw{(\tlete{\txone}{\hole}{(\anf{\setwo}{\tlete{\txtwo}{\hole}{\tK\hw{\tappe{\txone}{\txtwo}}}})})}}) \\
        & \text{by the induction hypothesis applied to \im{\seone}} \nonumber \\
        =~& (\anf{\seone}{(\tlete{\txone}{\hole}{\tKpr\hhw{(\anf{\setwo}{\tlete{\txtwo}{\hole}{\tK\hw{\tappe{\txone}{\txtwo}}}})}})}) \\
        & \text{by definition of continuation composition} \nonumber \\
        =~ & (\anf{\seone}{(\tlete{\txone}{\hole}{(\anf{\setwo}{\tKpr\hhw{\tlete{\txtwo}{\hole}{\tK\hw{\tappe{\txone}{\txtwo}}}}})})}) \\
        & \text{by the induction hypothesis applied to \im{\setwo}} \nonumber \\
        =~ & (\anf{\seone}{(\tlete{\txone}{\hole}{(\anf{\setwo}{\tlete{\txtwo}{\hole}{\tKpr\hhw\tK\hw{\tappe{\txone}{\txtwo}}}})})}) \\
        & \text{by definition of continuation composition} \nonumber \qedhere
      \end{align}
  \end{proofcases}
\end{proof}

Next I show \tech{compositionality} of the translation with respect to
substitution.
While the proof relies on the previous lemma, this lemma is different in that
substitution is the primary means of composition within the type system.
We must essentially show that substitution is equivalent to composing via
\tech{continuations}.
Since standard substitution does not preserve \tech{ANF}, this lemma does not
equate \tlang terms, but \slang terms that have been transformed via \tech{ANF}
translation.
We will again use language boundaries to indicate a shift from \tech{ANF} to
non-\tech{ANF} terms.
Note that this lemma relies on uniqueness of names.
\begin{lemma}[Substitution]
  \label{lem:anf:subst}
  \im{\anf{\subst{\se}{\sepr}{\sx}}{\tK} \equiv \sbound{\subst{(\anf{\se}{\tK})}{\anfh{\sepr}}{\tx}}}
\end{lemma}
\begin{proof}
  By induction on the structure of \im{\se}
  I give the key cases.
  \begin{proofcases}
    \item \im{\se = \sx}

      Must show that \im{\anf{\sepr}{\tK} \equiv \sbound{\subst{(\anf{\sx}{\tK})}{\anfh{\sepr}}{\tx}}}
      \begin{align}
        & \sbound{\subst{\anf{\sx}{\tK}}{\anfh{\sepr}}{\tx}} \nonumber \\
        =~& \sbound{\subst{\tK\hw{\tx}}{\anfh{\sepr}}{\tx}} \\
        =~& \sbound{\tK\hw{\anfh{\sepr}}} \\
        \equiv~& \tK\hhw{\anfh{\sepr}} & \text{by \fullref[]{lem:anf:tgt:het-sound}} \\
        \equiv~& \anf{\sepr}{\tK} & \text{by \fullref[]{lem:anf:comp-stack}}
      \end{align}

    \item \im{\se = \spropty}
      Trivial.

    \item \im{\se = \spity{\sxpr}{\sA}{\sB}}

      Must show that
      \im{\anf{\subst{\spity{\sxpr}{\sA}{\sB}}{\sepr}{\sx}}{\tK}
        \equiv
        \sbound{\subst{(\anf{\spity{\sxpr}{\sA}{\sB}}{\tK})}{\anfh{\sepr}}{\sx}}}
      \begin{align}
        & \anf{\subst{\spity{\sxpr}{\sA}{\sB}}{\sepr}{\sx}}{\tK} \nonumber \\
        =~& \anf{\spity{\sxpr}{\subst{\sA}{\sepr}{\sx}}{\subst{\sB}{\sepr}{\sx}}}{\tK} \\
        =~& \tK\hw{\tpity{\txpr}{\anfh{\subst{\sA}{\sepr}{\sx}}}{\anfh{\subst{\sB}{\sepr}{\sx}}}} \\
        \equiv~& \tK\hw{\tpity{\txpr}{\sbound{\subst{\anfh{\sA}}{\anfh{\sepr}}{\tx}}}{\sbound{\subst{\anfh{\sB}}{\anfh{\sepr}}{\tx}}}} \\
        & \text{by the induction hypothesis} \nonumber \\
        =~& \sbound{\subst{\tK\hw{\tpity{\txpr}{\anfh{\sA}}{\anfh{\sB}}}}{\anfh{\sepr}}{\tx}} \\
        & \text{by definition of substitution} \nonumber \\
        =~& \sbound{\subst{(\anf{\spity{\sxpr}{\sA}{\sB}}{\tK})}{\anfh{\sepr}}{\tx}} \\
        & \text{by definition} \nonumber
      \end{align}

    \item \im{\se = \sappe{\seone}{\setwo}}

      Must show that
      \im{\anf{\subst{(\sappe{\seone}{\setwo})}{\sepr}{\tx}}{\tK} \equiv \sbound{\subst{(\anf{\sappe{\seone}{\setwo}}{\tK})}{\anfh{\sepr}}{\tx}}}
      \begin{align}
        & \anf{\subst{(\sappe{\seone}{\setwo})}{\sepr}{\sx}}{\tK} \nonumber \\
        =~& \anf{\sappe{\subst{\seone}{\sepr}{\sx}}{\subst{\setwo}{\sepr}{\sx}}}{\tK} \\
        & \text{by substitution} \nonumber \\
        =~& \anf{\subst{\seone}{\sepr}{\sx}}{\tlete{\txone}{\hole}{\anf{\subst{\setwo}{\sepr}{\sx}}{\tlete{\txtwo}{\hole}{\tK\hw{\tappe{\txone}{\txtwo}}}}}} \\
        & \text{by translation} \nonumber \\
        \equiv~& \anf{\subst{\seone}{\sepr}{\sx}}{\tlete{\txone}{\hole}{\sbound{\subst{(\anf{\setwo}{\tlete{\txtwo}{\hole}{\tK\hw{\tappe{\txone}{\txtwo}}}})}{\anfh{\sepr}}{\tx}}}} \\
        & \text{by IH applied to \im{\seone}} \nonumber \\
        \equiv~& \subst{\anf{\seone}{\tlete{\txone}{\hole}{\subst{\anf{\setwo}{\tlete{\txtwo}{\hole}{\tK\hw{\tappe{\txone}{\txtwo}}}}}{\anfh{\sepr}}{\sx}}}}{\anfh{\sepr}}{\tx} \\
        & \text{by IH applied to \im{\setwo}} \nonumber \\
        =~& \sbound{\subst{(\anf{\seone}{\tlete{\txone}{\hole}{\anf{\setwo}{\tlete{\txtwo}{\hole}{\tK\hw{\tappe{\txone}{\txtwo}}}}}})}{\anfh{\sepr}}{\tx}} \\
        & \text{by substitution} \nonumber \\
        =~& \sbound{\subst{(\anf{\sappe{\seone}{\setwo}}{\tK})}{\anfh{\sepr}}{\tx}} \\
        & \text{by substitution} \nonumber
      \end{align}
  \end{proofcases}
\end{proof}

Since \tech{equivalence} is part of the type system, to show \tech{type preservation},
we must show that \tech{equivalence} is preserved.
I first show that reduction is preserved up to \tech{equivalence},
then \tech{conversion}, and finally that \tech{equivalence} is preserved.
The proofs are straightforward; intuitively, \tech{ANF} is just adding a bunch
of \(\zeta\)-reductions.
\begin{lemma}[Preservation of Reduction]
  \label{lem:anf:step}
  If \im{\sstepjudg[\step]{\slenv}{\se}{\sepr}} then \im{\sequivjudg{\anfh{\slenv}}{\anfh{\se}}{\anfh{\sepr}}}.
\end{lemma}
\begin{proof}
  By cases on \im{\sstepjudg[\step]{\slenv}{\se}{\sepr}}.
  I give the key cases.
  \begin{proofcases}
    \item \im{\sstepjudg[\step_{\delta}]{\slenv}{\sx}{\sepr}}

      We must show that \im{\sequivjudg{\anfh{\slenv}}{\anfh{\sx}}{\anfh{\sepr}}}

      We know that \im{\sx = \sepr \in \slenv}, and by definition \im{\tx =
        \anfh{\sepr} \in \anfh{\slenv}}, so the goal follows by definition.

    \item \im{\sstepjudg[\step_{\beta}]{\slenv}{\sappe{\sfune{\sx}{\sA}{\seone}}{\setwo}}{\subst{\seone}{\setwo}{\sx}}}

      We must show \im{\sequivjudg{\anfh{\slenv}}{\anfh{\sappe{\sfune{\sx}{\sA}{\seone}}{\setwo}}}{\anfh{\subst{\seone}{\setwo}{\sx}}}}
      \begin{align}
        & \anfh{\sappe{\sfune{\sx}{\sA}{\seone}}{\setwo}} \nonumber \\
        =~& \anf{\sfune{\sx}{\sA}{\seone}}{\tlete{\txone}{\hole}{\anf{\setwo}{\tlete{\txtwo}{\hole}{\tappe{\txone}{\txtwo}}}}} \\
        =~&
        \tlete{\txone}{(\tfune{\tx}{\anfh{\sA}}{\anfh{\seone}})}{\anf{\setwo}{\tlete{\txtwo}{\hole}{\tappe{\txone}{\txtwo}}}}
        \\
        \stepstar~& \anf{\setwo}{\tlete{\txtwo}{\hole}{\tappe{\tfune{\tx}{\anfh{\sA}}{\anfh{\seone}}}{\txtwo}}} \\
        =~& \tlete{\txtwo}{\hole}{\tappe{(\tfune{\tx}{\anfh{\sA}}{\anfh{\seone}})}{\txtwo}}\hhw{\anfh{\setwo}}
        & \text{(\fullref[]{lem:anf:comp-stack})} \\
        \equiv~& \slete{\sxtwo}{\anfh{\setwo}}{\sappe{(\tfune{\tx}{\anfh{\sA}}{\anfh{\seone}})}{\sxtwo}}
        & \text{(\fullref[]{lem:anf:tgt:het-sound})} \\
        \step_{\zeta}~& \tappe{(\tfune{\tx}{\anfh{\sA}}{\anfh{\seone}})}{\anfh{\setwo}} \\
        \step_{\beta}~& \sbound{\subst{\anfh{\seone}}{\anfh{\setwo}}{\tx}} \\
        \equiv~& \anfh{\subst{\seone}{\setwo}{\sx}} & \text{(\fullref[]{lem:anf:subst})}
      \end{align}
  \end{proofcases}
\end{proof}

Next I show that \tech{conversion} is preserved up to \tech{equivalence}.
\begin{lemma}[Preservation of Conversion]
  \label{lem:anf:red}
  If \im{\sstepjudg[\stepstar]{\slenv}{\se}{\sepr}} then \im{\sequivjudg{\anfh{\slenv}}{\anfh{\se}}{\anfh{\sepr}}}
\end{lemma}
\begin{proof}
  By induction on the structure of \im{\sstepjudg[\stepstar]{\slenv}{\se}{\sepr}}.
  \begin{proofcases}
  \item \refrule[src]{Red-Refl}, trivial.
  \item \refrule[src]{Red-Trans}, by \fullref[]{lem:anf:step} and the induction hypothesis.
  \item \refrule[src]{Red-Cong-Let}

    We have that \im{\sstepjudg[\stepstar]{\slenv}{\slete{\sx}{\seone}{\se}}{\slete{\sx}{\seone}{\sepr}}}
    and \im{\sstepjudg[\stepstar]{\slenv}{\se}{\sepr}}.

    We must show that
    \im{\sequivjudg{\anfh{\slenv}}{\anfh{\slete{\sx}{\seone}{\se}}}{\anfh{\slete{\sx}{\seone}{\sepr}}}}
    \begin{align}
      & \anfh{\slete{\sx}{\seone}{\se}} \nonumber \\
      =~& \anfh{\subst{\slete{\sx}{\seone}{\sy}}{\se}{\sy}} \\
      \equiv~& \sbound{\subst{\anfh{\slete{\sx}{\seone}{\sy}}}{\anfh{\se}}{\ty}} \nonumber \\
      & \text{by \fullref{lem:anf:subst}} \\
      \equiv~& \sbound{\subst{\anfh{\slete{\sx}{\seone}{\sy}}}{\anfh{\sepr}}{\ty}} \nonumber \\
      & \text{by the induction hypothesis applied to \im{\se \stepstar \sepr}} \\
      \equiv~& \anfh{\subst{\slete{\sx}{\seone}{\sy}}{\sepr}{\sy}} \nonumber \\
      & \text{by \fullref[]{lem:anf:subst}} \\
      =~& \anfh{\slete{\sx}{\seone}{\sepr}}
    \end{align}

  \end{proofcases}
\end{proof}

The previous two lemmas imply equivalence preservation.
Including \(\eta\)-equivalence makes this non-trivial, but not hard.
\begin{lemma}[Preservation of Equivalence]
  \label{lem:anf:eqv-pres}
  If \im{\sequivjudg{\slenv}{\se}{\sepr}} then \im{\sequivjudg{\anfh{\slenv}}{\anfh{\se}}{\anfh{\sepr}}}
\end{lemma}
\begin{proof}
  By induction on the derivation of \im{\sequivjudg{\slenv}{\se}{\sepr}}.
  \begin{proofcases}
    \item \refrule*[src]{eqv}{\im{\equiv}}
      Follows by \fullref[]{lem:anf:red}.
    \item \refrule*[src]{eqv-eta1}{\im{\equiv}-\im{\eta_1}}

      By \fullref[]{lem:anf:red}, we know \im{\anfh{\se} \equiv \anfh{\sfune{\sx}{\sA}{\seone}}}.
      By transitivity, it suffices to show
      \im{\anfh{\sfune{\sx}{\sA}{\seone}} \equiv \anfh{\sepr}}.

      By \refrule*[anfapp]{eqv-eta1}{\im{\equiv}-\im{\eta_1}}, since
      \im{\anfh{\sfune{\sx}{\sA}{\seone}} =
        \tfune{\tx}{\anfh{\sA}}{\anfh{\seone}}}, it suffices to show that
      \im{\anfh{\seone} \equiv
        \sappe{\anfh{\sepr}}{\sxtwo}}
      \begin{align}
        & \anfh{\seone} \nonumber \\
        \equiv~& \anfh{\sappe{\sepr}{\sxtwo}} & \text{by the induction hypothesis} \\
        =~& \anf{\sepr}{\tlete{\txone}{\hole}{\tappe{\txone}{\txtwo}}} \\
        =~& (\tlete{\txone}{\hole}{\tappe{\txone}{\txtwo}})\hhw{\anfh{\sepr}} & \text{by \fullref[]{lem:anf:comp-stack}} \\
        \equiv~& \slete{\sxone}{\anfh{\sepr}}{\tappe{\sxone}{\sxtwo}} & \text{by \fullref[]{lem:anf:tgt:het-sound}} \\
        \step_{\zeta}~& \sappe{{\anfh{\sepr}}}{\sxtwo}
      \end{align}

    \item \refrule*[src]{eqv-eta2}{\im{\equiv}-\im{\eta_2}} Essentially similar to the previous case. \qedhere
  \end{proofcases}
\end{proof}

Since I implement \tech{cumulative} \tech{universes} through subtyping, we must
also show subtyping is preserved.
The proof is completely uninteresting, except insofar as it is simple, while it
seems to be impossible for CPS translation \cite{bowman2018:cps-sigma}.
I discuss this further in \fullref[]{sec:anf:discuss}.
\begin{lemma}[Preservation of Subtyping]
  \label{lem:anf:subtype-pres}
  If \im{\ssubtyjudg{\slenv}{\se}{\sepr}} then \im{\ssubtyjudg{\anfh{\slenv}}{\anfh{\se}}{\anfh{\sepr}}}
\end{lemma}
\begin{proof}
  By induction on the structure of \im{\ssubtyjudg{\slenv}{\se}{\sepr}}.
  \begin{proofcases}
    \item \refrule*[src]{sub-eqv}{\im{\subtypesym}-\im{\equiv}}. Follows by \fullref[]{lem:anf:eqv-pres}.
    \item \refrule*[src]{sub-trans}{\im{\subtypesym}-Trans}. Follows the induction hypothesis.
    \item \refrule*[src]{sub-prop}{\im{\subtypesym}-Prop}. Trivial, since \im{\anfh{\spropty} = \tpropty} and \im{\anfh{\stypety{0}} = \ttypety{0}}.
    \item \refrule*[src]{sub-cum}{\im{\subtypesym}-Cum}. Trivial, since \im{\anfh{\stypety{i}} = \ttypety{i}} and \im{\anfh{\stypety{i+1}} = \ttypety{i+1}}.
    \item \refrule*[src]{sub-pi}{\im{\subtypesym}-Pi}.

      We must show that
      \im{\ssubtyjudg{\anfh{\slenv}}{\anfh{\spity{\sxone}{\sAone}{\sBone}}}{\anfh{\spity{\sxtwo}{\sAtwo}{\sBtwo}}}}

      By definition of the translation, we must show \im{\ssubtyjudg{\anfh{\slenv}}{\tpity{\txone}{\anfh{\sAone}}{\anfh{\sBone}}}{\tpity{\txtwo}{\anfh{\sAtwo}}{\anfh{\sBtwo}}}}.

      Note that if we lifted the continuations in type annotations \im{\sAone} and
      \im{\sAtwo} outside the \tfonttext{\im{\Pi}}, as CBPV suggests we should, we
      would need a new subtyping rule that allows subtyping \tfonttext{let}
      expressions.
      As it is, we proceed by \refrule*[anfapp]{sub-pi}{\im{\subtypesym}-Pi}.

      It suffices to show that
      \begin{enumerate}
        \item \im{\sequivjudg{\anfh{\slenv}}{\anfh{\sAone}}{\anfh{\sAtwo}}}, which follows by the induction hypothesis.
        \item \im{\ssubtyjudg{\anfh{\slenv},\txone:\anfh{\sAtwo}}{\anfh{\sBone}}{\subst{\anfh{\sBtwo}}{\txone}{\txtwo}}}, which follows by the induction hypothesis.
      \end{enumerate}
    \item \refrule*[src]{sub-sig}{\im{\subtypesym}-Sig}. Similar to previous case.
  \end{proofcases}
\end{proof}

I now prove \tech{type preservation}, with a suitably strengthened induction
hypothesis.
I prove that, given a well-typed source term \im{\se} of type \im{\sA}, and a
continuation \im{\tK} that expects the definitions \im{\edefs{\se}}, expects the
term \im{\ehole{\se}}, and has result type \im{\tB}, the
translation \im{\anf{\se}{\tK}} is well typed.

The structure of the lemma and its proof are a little surprising.
Intuitively, we would expect to show something like ``if \im{\se : \sA} then
\im{\anfh{\se} : \anfh{\sA}}''.
I will ultimately prove this, \fullref{thm:anf:type-pres}, but we need a
stronger lemma first.
Since the translation is pushing \tech{computation} inside-out (since
\tech{continuations} compose inside-out), the \tech{type-preservation} lemma and proof
are essentially inside-out.
Instead of the expected statement, we must show that if we have a
\tech{continuation} \im{\tK} that expects \im{\anfh{\se} : \anfh{\sA}}, then we get a
term \im{\anf{\se}{\tK}} of some arbitrary type \im{\tB}.
Of course, in order to show that, we will have to show that
\im{\anfh{\se} : \anfh{\sA}} and then appeal to \fullref{lem:anf:ecca:cut}.
Furthermore, each appeal to the inductive hypothesis will have to establish that
we can in fact create well-typed \tech{continuations} from the assumption that \im{\anfh{\se}
  : \anfh{\sA}}.

Wielding our propositions-as-types hat, we can view this theorem as in
accumulator-passing style, where the well-typed \tech{continuation} is an accumulator
expressing the inductive invariant for \tech{type preservation}.

I begin with a minor technical lemma that will come in useful in the proof of
type preservation.
This lemma allows us to establish that a \tech{continuation} is well typed when it
expects an inductively smaller translated term in its hole.
It also tells us, formally, that the inductive hypothesis implies the \tech{type
  preservation} theorem we expect.
\begin{lemma}
  \label{lem:anf:ih-empty}
  If for all \im{\styjudg{{\slenv}}{{\se}}{{\sA}}} and
  \im{\styjudg{\anfh{\slenv},\edefs{\se}}{\tK}{(\ehole{\se} :
      \anfh{\sA}) \Rightarrow \tB}} we know that
  \im{\styjudg{\anfh{\slenv}}{\anf{\se}{\tK}}{\tB}}, then
  \im{\styjudg{\anfh{\slenv},\edefs{\se}}{\ehole{\se}}{\anfh{\sA}}}
  (and, incidentally, \im{\styjudg{\anfh{\slenv}}{\anfh{\se}}{\anfh{\sA}}})
\end{lemma}
\begin{proof}
  Note that by \fullref{cor:anf:form} and the definitions of \im{\edefs{\se}} and \im{\ehole{\se}},
  \im{\styjudg{\anfh{\slenv},\edefs{\se}}{\ehole{\se}}{\anfh{\sA}}} is a sub-derivation of
  \im{\styjudg{\anfh{\slenv}}{\anfh{\se}}{\anfh{\sA}}}, so
  it suffices to show that \im{\styjudg{\anfh{\slenv}}{\anfh{\se}}{\anfh{\sA}}}.
  By the premise \im{\styjudg{\anfh{\slenv}}{\anf{\se}{\tK}}{\tB}}, it suffices
  to show that \im{\hole : (\_ : \anfh{\sA}) \Rightarrow \anfh{\sA}}, which is
  true by \refrule{K-Empty}.
\end{proof}

\begin{lemma}[Type and Well-formedness Preservation]
  \label{lem:anf:type-pres}
  \begin{lemmacases}
    \item If \im{\swf{\slenv}} then \im{\swf{\anfh{\slenv}}}
    \item If \im{\styjudg{{\slenv}}{{\se}}{{\sA}}}, and
      \im{\styjudg{\anfh{\slenv},\edefs{\se}}{\tK}{(\ehole{\se} : \anfh{\sA}) \Rightarrow \tB}} then
      \im{\styjudg{\anfh{\slenv}}{\anf{\se}{\tK}}{\tB}}
  \end{lemmacases}
\end{lemma}
\begin{proof}
  The proof is by induction on the mutually defined judgments \im{\swf{\slenv}} and \im{\styjudg{\slenv}{\se}{\sA}}.
  The key cases are the typing rules that use dependency, that is,
  \refrule[srcapp]{Snd}, \refrule[srcapp]{App}, and \refrule[src]{Let}.
  I give these cases, although they are essentially similar, and a couple of other
  representative cases, which are uninteresting.
  I strongly recommend that the reader read the proof case for
  \refrule[srcapp]{Snd}, in which I take care to spell out the interesting aspects
  of the proof.
  \begin{proofcases}
  \item \refrule[src]{Prop}

    We must show that
    \im{\styjudg{\anfh{\slenv}}{\anf{\spropty}{\tK}}{\tB}}.

    By definition of the translation, it suffices to show that
    \im{\styjudg{\anfh{\slenv}}{\tK\hw{\tpropty}}{\tB}}.

    Note that \im{\edefs{\spropty} = \cdot}; this property holds for all values.

    By \fullref{lem:anf:ecca:cut}, it suffices to show that
    \begin{enumerate}[leftmargin=*]
    \item \im{\ehole{\spropty} = \tpropty}, which is true by definition of the translation, and
    \item \im{\tpropty : \anfh{\stypety{1}}}, which is true by \refrule[anfapp]{Prop}, since \im{\anfh{\stypety{1}} = \ttypety{1}}.
    \end{enumerate}

  \item \refrule[srcapp]{Lam}

    We must show that
    \im{\styjudg{\anfh{\slenv}}{\anf{\sfune{\sx}{\sApr}{\sepr}}{\tK}}{\tB}}.

    That is, by definition of the translation,
    \im{\styjudg{\anfh{\slenv}}{\tK\hw{\tfune{\tx}{\anfh{\sApr}}{\anfh{\sepr}}}}{\tB}}.

    Recall that \im{\edefs{\sfune{\sx}{\sApr}{\sepr}} = \cdot}, since values export no definitions.

    By \fullref[]{lem:anf:ecca:cut}, it suffices to show that
    \im{\styjudg{\anfh{\slenv}}{\tfune{\tx}{\anfh{\sApr}}{\anfh{\sepr}}}{\anfh{\spity{\sx}{\sApr}{\sBpr}}}}.

    By definition, \im{\anfh{\spity{\sx}{\sApr}{\sBpr}} =
      \tpity{\tx}{\anfh{\sApr}}{\anfh{\sBpr}}}, we must show
    \im{\styjudg{\anfh{\slenv}}{\tfune{\tx}{\anfh{\sApr}}{\anfh{\sepr}}}{\tpity{\tx}{\anfh{\sApr}}{\anfh{\sBpr}}}}.

    By \refrule[anfapp]{Lam}, it suffices to show

    \im{\styjudg{\anfh{\slenv},\tx:\anfh{\sApr}}{\anfh{\sepr}}{\anfh{\sBpr}}}.

    Note that \im{\styjudg{\anfh{\slenv}}{\hole}{(\_ : \anfh{\sBpr}) \Rightarrow \anfh{\sBpr}}}.

    So, the goal follows by the induction hypothesis applied to \im{\styjudg{\slenv,\sx:\sApr}{\sepr}{\sBpr}} with \im{\tK = \hole}

    \item \refrule[srcapp]{Snd}

      We must show \im{\styjudg{\anfh{\slenv}}{\anf{\ssnde{\se}}{\tK}}{\tB}},
      where
      \im{\styjudg{\anfh{\slenv},\edefs{\ssnde{\se}}}{\tK}{(\ehole{\ssnde{\se}} : \anfh{\subst{\sBpr}{\sfste{\se}}{\sx}}) \Rightarrow\tB}}
      and \im{\styjudg{\slenv}{\se}{\ssigmaty{\sx}{\sApr}{\sBpr}}}.

      That is, by definition of the translation, we must show,

      \im{\styjudg{\anfh{\slenv}}{\anf{\se}{\tlete{\txpr}{\hole}{\tK\hw{\tsnde{\txpr}}}}}{\tB}}.

      Let \im{\tKpr = {\tlete{\txpr}{\hole}{\tK\hw{\tsnde{\txpr}}}}}.

      Note that we know nothing further about the structure of the term
      we're trying to type check, \im{\anfh{\se}{\tKpr}}.
      Therefore, we cannot appeal to any typing rules directly.
      This happens because \im{\se} is a \tech{computation}, and the translation of
      \techs{computation} composes \tech{continuations}, which occurs ``inside-out''.
      Instead, my proof proceeds ``inside out'': we build up typing invariants
      in a well-typed \tech{continuation} \im{\tKpr} (that is, I build up
      definitions in an accumulator) and then appeal to the induction hypothesis
      for \im{\se} with \im{\tKpr}.
      Intuitively, some later case of the proof that knows more about the
      structure of \im{\se} will be able to use this well-typed
      \tech{continuation} to proceed.

      So, by the induction hypothesis, applied to
      \im{\styjudg{\slenv}{\se}{\ssigmaty{\sx}{\sApr}{\sBpr}}} with \im{\tKpr},
      to complete this proof case, it suffices to show that:
      \im{\styjudg{\anfh{\slenv},\edefs{\se}}{\tlete{\txpr}{\hole}{\tK\hw{\tsnde{\txpr}}}}{(\ehole{\se} : \anfh{\ssigmaty{\sx}{\sApr}{\sBpr}}) \Rightarrow \tB}}

      By \refrule{K-Bind}, it suffices to show that
      \begin{enumerate}[leftmargin=*]
      \item \label{prf:anf:type-pres:snd-0}
        \im{\styjudg{\anfh{\slenv},\edefs{\se}}{\ehole{\se}}{\anfh{\ssigmaty{\sx}{\sApr}{\sBpr}}}},
           which follows by \fullref[]{lem:anf:ih-empty} applied to the
           induction hypothesis for
           \im{\styjudg{\slenv}{\se}{\ssigmaty{\sx}{\sApr}{\sBpr}}}
      \item \label{prf:anf:type-pres:snd-1} \im{\styjudg{\anfh{\slenv},\edefs{\se},\txpr = \ehole{\se}}{\tK\hw{\tsnde{\txpr}}}{\tB}}.
      \end{enumerate}

      To complete this case of the proof, it suffices to show
      \hyperref[prf:anf:type-pres:snd-1]{Item (b)}.

      Note \im{\edefs{\ssnde{\se}} = (\edefs{\se},\txpr=\ehole{\se})} and \im{\ehole{\ssnde{\se}} = \tsnde{\txpr}}.

      So, by \fullref{lem:anf:ecca:cut}, given the type of \im{\tK}, it suffices to
      show that

      \im{\styjudg{\anfh{\slenv},\edefs{\se},\txpr=\ehole{\se}}{\tsnde{\txpr}}{\anfh{\subst{\sBpr}{\sfste{\se}}{\sx}}}}.

      By \fullref[]{lem:anf:subst}, \im{\anfh{\subst{\sBpr}{\sfste{\se}}{\tx}} \equiv
      \subst{\anfh{\sBpr}}{\anfh{\sfste{\se}}}{\tx}}.

      By \refrule[anfapp]{Conv}, it suffices to show that
      \im{\styjudg{\anfh{\slenv},\edefs{\se},\txpr=\ehole{\se}}{\tsnde{\txpr}}{\subst{\anfh{\sBpr}}{\anfh{\sfste{\se}}}{\tx}}}.

      Note that we cannot show this by the typing rule
      \refrule[anfapp]{Snd}, since the substitution
      \im{\subst{\anfh{\sBpr}}{\anfh{\sfste{\se}}}{\tx}} copies an apparently
      arbitrary expression \im{\anfh{\sfste{\se}}} into the type, instead of the
      expected sub-expression \im{\tfste{\txpr}}.
      That is, by the typing rules, all we can show is that
      \im{\tsnde{\txpr} : \subst{\anfh{\sBpr}}{\tfste{\txpr}}{\tx}}, but we must show
      \im{\tsnde{\txpr} : \subst{\anfh{\sBpr}}{\anfh{\sfste{\se}}}{\tx}}.
      The translation has disrupted the \tech{dependency} on \im{\se}, changing the
      type that depended on the specific value \im{\se} into a type that
      depends on an apparently arbitrary value \im{\txpr}.
      This is the problem discussed in \fullref[]{sec:anf:ideas}.
      It is also where the definitions we have accumulated in our
      \tech{continuation} typing save us.
      We can show that \im{\anfh{\sfste{\se}} \equiv \tfste{\txpr}}, under
      the definitions we have accumulated from \tech{continuation} typing.
      This follows by \fullref[]{lem:anf:export-equal}.

      Therefore, by \refrule[anfapp]{Conv}, it suffices to show that
      \im{\styjudg{\anfh{\slenv},\edefs{\se},\txpr=\ehole{\se}}{\tsnde{\txpr}}{\subst{\anfh{\sBpr}}{\tfste{\txpr}}{\tx}}}.

      By \refrule[anfapp]{Snd}, it suffices to show
      \im{\styjudg{\anfh{\slenv},\edefs{\se},\txpr=\ehole{\se}}{\txpr}{\tsigmaty{\tx}{\anfh{\sApr}}{\anfh{\sBpr}}}},
      which follows since, as we showed in
      \hyperref[prf:anf:type-pres:snd-0]{Item (a)}, \im{\anfh{\se} :
        \tsigmaty{\tx}{\anfh{\sApr}}{\anfh{\sBpr}}}.

    \item \refrule[srcapp]{App}

      We must show that \im{\styjudg{\anfh{\slenv}}{\anf{\sappe{\seone}{\setwo}}{\tK}}{\tB}}.

      That is, by definition, \im{\styjudg{\anfh{\slenv}}{\anf{\seone}{\tlete{\txone}{\hole}{\anf{\setwo}{\tlete{\txtwo}{\hole}{\tK\hw{\tappe{\txone}{\txtwo}}}}}}}{\tB}}.

      Again, we know nothing about the structure of \im{\anf{\seone}{\tKpr}}, so we must proceed inside-out.

      By the inductive hypothesis applied to \im{\styjudg{\slenv}{\seone}{\subst{\sBpr}{\seone}{\sx}}}, it suffices to show that

      \im{\styjudg{\anfh{\slenv},\edefs{\seone}}{\begin{stackTL}\tlete{\txone}{\hole}{\\\,\,\anf{\setwo}{\tlete{\txtwo}{\hole}{\tK\hw{\tappe{\txone}{\txtwo}}}}}}{(\ehole{\seone} : \anfh{\spity{\sx}{\sApr}{\sBpr}}) \Rightarrow \tB}\end{stackTL}}

      To show this, by \refrule{K-Bind}, it suffices to show

      \begin{enumerate}[leftmargin=*]
      \item
        \im{\styjudg{\anfh{\slenv},\edefs{\seone}}{\ehole{\seone}}{\anfh{\spity{\sx}{\sApr}{\sBpr}}}},
        which follows by \fullref[]{lem:anf:ih-empty}
        applied to the induction hypothesis for
        \im{\styjudg{\slenv}{\seone}{\spity{\sx}{\sApr}{\sBpr}}}
      \item \im{\styjudg{\anfh{\slenv},\edefs{\seone},\txone=\ehole{\seone}}{\anf{\setwo}{\tlete{\txtwo}{\hole}{\tK\hw{\tappe{\txone}{\txtwo}}}}}{\tB}}
      \end{enumerate}

      By the inductive hypothesis applied to \im{\styjudg{\slenv}{\setwo}{\sApr}}, it suffices to show that

      \im{\styjudg{\anfh{\slenv},\edefs{\seone},\txone=\ehole{\seone},\edefs{\setwo}}{\tlete{\txtwo}{\hole}{\tK\hw{\tappe{\txone}{\txtwo}}}}{\tB}}.

      By \refrule{K-Bind}, it suffices to show

      \im{\styjudg{\anfh{\slenv},\edefs{\seone},\txone=\ehole{\seone},\edefs{\setwo},\txtwo=\ehole{\setwo}}{\tK\hw{\tappe{\txone}{\txtwo}}}{\tB}}.

      Let \im{\tlenvpr = {\anfh{\slenv},\edefs{\seone},\txone=\ehole{\seone},\edefs{\setwo},\txtwo=\ehole{\setwo}}}, for brevity.

      By \fullref{lem:anf:ecca:cut}, we must show
      \im{\styjudg{\tlenvpr}{\tappe{\txone}{\txtwo}}{\anfh{\subst{\sBpr}{\setwo}{\sx}}}}.

      By \fullref[]{lem:anf:subst} and \refrule[anfapp]{Conv}, it suffices to show
      \im{\styjudg{\tlenvpr}{\tappe{\txone}{\txtwo}}{\subst{\anfh{\sBpr}}{\anfh{\setwo}}{\sx}}}.

      As in the proof case for \refrule[srcapp]{Snd}, we cannot proceed directly by
      \refrule[anfapp]{App}, since we see a disrupted \tech{dependency}.
      This dependent application whose type depends on the argument being the
      specific value \im{\anfh{\setwo}} now finds the argument \im{\txtwo}.
      (This issue comes up in type-preservation for call-by-value \tech{CPS}
      (\fullref[]{chp:cps}).)
      But, again, we know by \fullref[]{lem:anf:export-equal} that under these
      exported definitions, \im{\anfh{\setwo} \equiv \txtwo}.
      So by \refrule[anfapp]{Conv}, it suffices to show \im{\styjudg{\tlenvpr}{\tappe{\txone}{\txtwo}}{\subst{\anfh{\sBpr}}{\txtwo}{\sx}}}.
      By \refrule[anfapp]{App} it suffices to show
      \begin{enumerate}[leftmargin=*]
      \item
        \im{\styjudg{\tlenvpr}{\txone}{\tpity{\tx}{\anfh{\sApr}}{\anfh{\sBpr}}}},
        which follows by \fullref[]{lem:anf:ih-empty} applied to the induction
        hypothesis for \im{\styjudg{\slenv}{\seone}{\spity{\sx}{\sApr}{\sBpr}}}
      \item \im{\styjudg{\tlenvpr}{\txtwo}{\anfh{\sApr}}},
        which follows by \fullref[]{lem:anf:ih-empty} applied to the induction
        hypothesis for \im{\styjudg{\slenv}{\setwo}{\sApr}}.
      \end{enumerate}

    \item \refrule[src]{Let}

      We must show that \im{\styjudg{\anfh{\slenv}}{\anf{\slete{\sx}{\seone}{\setwo}}{\tK}}{\tB}}.

      That is, by definition, \im{\styjudg{\anfh{\slenv}}{\anf{\seone}{\tlete{\txone}{\hole}{\anf{\setwo}{\tK}}}}{\tB}}.

      By the induction hypothesis applied to \im{\styjudg{\slenv}{\seone}{\sA}}, it suffices to show that

      \im{\styjudg{\anfh{\slenv},\edefs{\seone}}{\tlete{\txone}{\hole}{\anf{\setwo}{\tK}}}{(\ehole{\seone} : \anfh{\sA}) \Rightarrow \tB}}.

      By \refrule{K-Bind}, it suffices to show
      \begin{enumerate}[leftmargin=*]
      \item
        \im{\styjudg{\anfh{\slenv},\edefs{\seone}}{\ehole{\seone}}{\anfh{\anfh{\sA}}}},
        which follows by \fullref[]{lem:anf:ih-empty} applied to the induction
        hypothesis for \im{\styjudg{\slenv}{\seone}{\sA}},
      \item \label{prf:anf:type-pres:let-1} \im{\styjudg{\anfh{\slenv},\edefs{\seone},\txone=\ehole{\seone}}{\anf{\setwo}{\tK}}{\tB}}.
      \end{enumerate}

      \hyperref[prf:anf:type-pres:let-1]{Item (b)} follows from the
      induction hypothesis applied to
      \im{\styjudg{\slenv,\sx=\seone}{\setwo}{\sBpr}} with \im{\tK} (the \emph{same}
      well-typed \im{\tK} that we have from our current premise), if we can show
      that \im{\tK} is well typed as follows:

      \im{\styjudg{\anfh{\slenv},\edefs{\seone},\txone=\ehole{\seone},\edefs{\setwo}}{\tK}{(\ehole{\setwo} : \anfh{\subst{\sBpr}{\seone}{\sxone}}) \Rightarrow \tB}}

      Currently, we know by our premises that

      \im{\styjudg{\anfh{\slenv},\edefs{\slete{\sxone}{\seone}{\setwo}}}{\tK}{(\ehole{\slete{\sxone}{\seone}{\setwo}} : \anfh{\subst{\sBpr}{\seone}{\sxone}}) \Rightarrow \tB}}

      So it suffices to show that
      \begin{enumerate}[leftmargin=*]
      \item \im{\edefs{\slete{\sxone}{\seone}{\setwo}} =
        (\edefs{\seone},\txone=\ehole{\seone},\edefs{\setwo})}
      \item \im{\ehole{\slete{\sxone}{\seone}{\setwo}} = \ehole{\setwo}}
      \end{enumerate}

      both of which are straightforward by definition.
  \end{proofcases}
\end{proof}

\begin{theorem}[Type Preservation]
  \label{thm:anf:type-pres}
  If \im{\styjudg{\slenv}{\se}{\sA}} then \im{\styjudg{\anfh{\slenv}}{\anfh{\se}}{\anfh{\sA}}}
\end{theorem}
\begin{proof}
  By \fullref[]{lem:anf:type-pres}, it suffices to show that
  \im{\styjudg{\anfh{\slenv}}{\hole}{(\_ : \anfh{\sA}) \Rightarrow \anfh{\sA}}}, which is trivial.
\end{proof}

\subsection{Compiler Correctness}
I prove correctness of separate compilation with respect to the \tech{ANF}
machine semantics.
I use the same definitions of linking and \tech{observations} as in
\fullref[]{chp:source} for \slang.
It is simple to lift the \tech{ANF} translation to closing substitutions.

Correctness of separate compilation is non-standard compared to
\fullref[]{chp:type-pres}, in that I define \tech{ANF} semantics separate from
\tech{conversion}.
This theorem states that we can either link then run a program in the source language
semantics, \ie, using the \tech{conversion} relation, or separately compile the
term and its closing substitution then run in the \tech{ANF} machine semantics.
Either way, we get related \tech{observations}.
\begin{theorem}[Separate Compilation Correctness]
  \label{thm:anf:sep-comp}
  If \im{\wf{\slenv}{\se}} and
  \im{\wf{\slenv}{\ssubst}}, then
  \im{\seval{\ssubst(\se)} \approx \teval{\anfh{\ssubst}(\anfh{\se})}}.
\end{theorem}
\begin{proof}
  Because \im{\equiv} corresponds to \im{\approx} on \tech{observations}, it suffices
  to show that the following diagram commutes, which it does because
  \im{\seval{\ssubst(\se)} \equiv \ssubst(\se)} (by definiton),
  \im{\ssubst(\se) \equiv \anfh{\ssubst(\se)}} (by \fullref[]{lem:anf:eqv-pres}),
  \im{\anfh{\ssubst}(\anfh{\se}) \equiv \anfh{\ssubst(\se)}} (by \fullref[]{lem:anf:subst}),
  and \im{\teval{\anfh{\ssubst}(\anfh{\se})} \equiv {\anfh{\ssubst}(\anfh{\se})}} (by \fullref{thm:anf:tgt:eval-sound}).

  \begin{nop}
    \begin{tikzcd}
      \seval{\ssubst(\se)} \arrow[r, "\equiv"] \arrow[d, "\equiv"] & {\anfh{\ssubst(\se)}} \arrow[d, "\equiv"] \\
      \teval{\anfh{\ssubst(\anfh{\se})}} \arrow[r, "\equiv"] & \anfh{\ssubst(\anfh{\se})}
    \end{tikzcd}
  \end{nop}
\end{proof}
}

\begin{corollary}[Whole-Program Correctness]
  If \im{\wf{\cdot}{\se}} then \im{\seval{\se} \approx \anfh{\teval{\se}}}.
\end{corollary}
