\newcommand{\FigECCASyntax}[1][t]{
  \begin{figure}[#1]
    \begin{bnfgrammar}
      \bnflabel{Universes} &
      \tU & \!\!\bnfdef & \tpropty \bnfalt \ttypety{i}
      \bnfnewline

      \bnflabel{Values} & \tV & \bnfdef & \tx \bnfalt \tU \bnfalt \tpity{\tx}{\tM}{\tM} \bnfalt \tfune{\tx}{\tM}{\tM} \bnfalt \tsigmaty{\tx}{\tM}{\tM}
      \\&& \bnfalt & \tdpaire{\tV}{\tV}{\tM}
      \bnfnewline

      \bnflabel{Computations} & \tN & \bnfdef & \tV \bnfalt \tappe{\tV}{\tV} \bnfalt \tfste{\tV} \bnfalt \tsnde{\tV}
      \bnfnewline

      \bnflabel{Configurations} & \tM,\tA,\tB & \bnfdef & \tN \bnfalt \tlete{\tx}{\tN}{\tM}
      \bnfnewline

      \bnflabel{Continuations} & \tK  & \bnfdef & \hole \bnfalt \tlete{\tx}{\hole}{\tM}
    \end{bnfgrammar}
    \caption{\tlang Syntax}
    \label{fig:anf:ecca:syntax}
  \end{figure}
}

\newcommand{\FigECCARed}[1][t]{
  \begin{figure}[#1]
    \judgshape{\tM \mapsto \tMpr}
    \begin{reductionrules}
      \tK\hw{\tappe{(\tfune{\tx}{\tA}{\tM})}{\tV}} & \mapsto_{\beta} & \tK\hhw{\subst{\tM}{\tV}{\tx}}
      \stepnewline

      \tK\hw{\tfste{\tpaire{\tVone}{\tVtwo}}} & \mapsto_{\pi_{1}} & \tK\hw{\tVone}
      \stepnewline

      \tK\hw{\tsnde{\tpaire{\tVone}{\tVtwo}}} & \mapsto_{\pi_{2}} & \tK\hw{\tVtwo}
      \stepnewline

      \tlete{\tx}{\tV}{\tM} & \mapsto_{\zeta} & \subst{\tM}{\tV}{\tx}
    \end{reductionrules}
    \judgshape{\tstepjudg[\mapsto^*]{}{\tM}{\tMpr}}
    \begin{mathpar}
      \inferrule*[right=\defrule{RedA-Refl}]
      {~}
      {\tstepjudg[\mapsto^*]{}{\tM}{\tM}}

      \inferrule*[right=\defrule{RedA-Trans}]
      {\tM \mapsto \tMone \\
       \tstepjudg[\mapsto^*]{}{\tMone}{\tMpr}}
      {\tstepjudg[\mapsto^*]{}{\tM}{\tMpr}}
    \end{mathpar}
    \judgshape{\teval{\tM} = \tV}
    \begin{displaymath}
      \begin{array}{rcll}
        \teval{\tM} & = & \tV & \text{if \im{\wf{}{\tM}} and \im{\tM \mapsto^* \tV} and \im{\tV \not\mapsto \tVpr}}
      \end{array}
    \end{displaymath}
    \caption{\tlang Evaluation}
    \label{fig:anf:ecca:red}
  \end{figure}
}

\newcommand{\FigECCAHetero}[1][t]{
  \begin{figure}[#1]
    \judgshape{\tK\hhw{\tM} = \tM}
    \begin{displaymath}
      \begin{array}{rcl}
        \tK\hhw{\tN} &\defeq& \tK\hw{\tN} \\
        \tK\hhw{\tlete{\tx}{\tNpr}{\tM}} &\defeq& \tlete{\tx}{\tNpr}{\tK\hhw{\tM}} \\
      \end{array}
    \end{displaymath}
    \judgshape{\tK\hhw{\tK} = \tK}
    \begin{displaymath}
      \begin{array}{rcl}
        \tK\hhw{\hole} &\defeq& \tK \\
        \tK\hhw{\tlete{\tx}{\hole}{\tM}} &\defeq& \tlete{\tx}{\hole}{\tK\hhw{\tM}} \\
      \end{array}
    \end{displaymath}
    \judgshape{\hsubst{\tM}{\tMpr}{\tx} = \tM}
    \begin{displaymath}
      \begin{array}{rcl}
        \hsubst{\tM}{\tMpr}{\tx} &\defeq& (\tlete{\tx}{\hole}{\tM})\hhw{\tMpr}
      \end{array}
    \end{displaymath}
    \caption{\tlang Composition of Configurations}
    \label{fig:anf:ecca:hsubst}
  \end{figure}
}

\newcommand{\FigECCAStackTypes}[1][t]{
  \begin{figure}[#1]
    \judgshape{\styjudg{\tlenv}{\tK}{(\tN : \tA)} \Rightarrow \tB}
    \begin{mathpar}
      \inferrule*[right=\defrule{K-Empty}]
      {~}
      {\styjudg{\tlenv}{\hole}{(\tN : \tA) \Rightarrow \tA}}

      \inferrule*[right=\defrule{K-Bind}]
      {\styjudg{\tlenv}{\tN}{\tA} \\
       \styjudg{\tlenv,\ty=\tN}{\tM}{\tB}}
      {\styjudg{\tlenv}{\tlete{\ty}{\hole}{\tM}}{(\tN : \tA) \Rightarrow \tB}}
    \end{mathpar}
    \caption{\tlang Continuation Typing}
    \label{fig:anf:ecca:stack-types}
  \end{figure}
}

\newcommand{\FigStackExports}[1][t]{
  \begin{figure}[#1]
    \judgshape{\tedefs{\tM} = \tlenv}
    \begin{displaymath}
      \tedefs{\tM} = \txone=\tNone,\dots,\txin{n}=\tNin{n} \hspace{-1em}\qquad \where{\tM = \tlete{\txone}{\tNone}{\dots\tlete{\txin{n}}{\tNin{n}}{\tNin{n+1}}}}
    \end{displaymath}
    \judgshape{\tehole{\tM} = \tN}
    \begin{displaymath}
      \tehole{\tM} = \tNin{n+1} \qquad \where{\tM = \tlete{\txone}{\tNone}{\dots\tlete{\txin{n}}{\tNin{n}}{\tNin{n+1}}}}
    \end{displaymath}
    \caption{\tlang Continuation Exports}
    \label{fig:anf:ecca:exports}
  \end{figure}
}

\section{{{ANF}} Intermediate Language}
\label{sec:anf:target}

\FigECCASyntax
The target language, \tlang, is an ANF-restricted subset of \slang.
For now, I exclude \techs{dependent conditional} from \slang and from \tlang; I
return to them in \fullref[]{sec:anf:discuss}.
I continue to use the same typing and conversion rules as \slang, which are
permitted to break \tech{ANF} when computing term equivalence during type
checking.
However, I define an \tech{ANF}-preserving machine-like semantics for
evaluation of program configurations.
Note that this means the definitional equivalence is not suitable for equational
reasoning about run-time terms (\eg, reasoning about optimizations), without
\tech{ANF} translation afterwards.\footnote{This ability to break \tech{ANF}
  locally to support reasoning is similar to the language \(F_J\) of
  \citet{maurer2017}, which does not enforce \tech{ANF} syntactically, but is
  meant to support \tech{ANF} transformation and optimization with join points.}

\begin{typographical}
Although \tlang is a restriction of \slang, I typeset it as separate language for
clarity, and use the shift in fonts to indicate an explicit shift in how I am
treating terms, \ie, as either \tech{ANF}-restricted terms still suitable for
evaluation, or as unrestricted terms that we can type check but cannot run in
the \tech{ANF} semantics any longer.
\end{typographical}

I give the syntax for \tlang in \fullref[]{fig:anf:ecca:syntax}.
As discussed in \fullref[]{chp:type-pres}, the goal is to make an explicit
distinction between values and computations.
In \tlang, I do this by imposing a syntactic distinction between
\deftech*{value,values}{values} \im{\tV} which do not reduce,
\deftech*{comp,computation,computations}{computations} \im{\tN} which
eliminate values and can be composed
using \deftech*{continuation,continuation}{continuations} \im{\tK}, and
\deftech*{config,configuration,configurations}{configurations} \im{\tM} which
intuitively represent whole programs ready to be executed.
A continuation \im{\tK} is a program with a hole, and is composed
\im{\tK\hw{\tN}} with a computation \im{\tN} to form a configuration \im{\tM}.
For example, \im{(\tlete{\tx}{\hole}{\tsnde{\tx}})\hw{\tN} =
  \tlete{\tx}{\tN}{\tsnde{\tx}}}.
Since continuations are not first-class objects in the language, we cannot
express control effects---continuations are syntactically guaranteed to be used
linearly.
%
\footnote{The reader familiar with proof theory
  may wish to consider configurations and continuations as the same term under a
  \emph{stoup}~\cite{girard1991}, an environment expressing either zero or one
  linear computation variables \im{\hole}; when the stoup is empty, we have a
  configuration and otherwise we have a continuation.}
%
Note that despite the syntactic distinction, I still do not enforce a
\tech{phase distinction}---\tech*{anf:config}{configurations} (programs) can
appear in \techs{type}.

\FigECCAHetero
In \tech{ANF}, all \tech{continuations} are left associated, so substitution can
break \tech{ANF}.
Note that \(\beta\)-reduction takes an \tech{ANF} \tech{configuration}
\im{\tK\hw{\tappe{(\tfune{\tx}{\tA}{\tM})}{\tV}}} but would na\"ively produce
\im{\tK\hw{\subst{\tM}{\tV}{\tx}}}.
While the substitution \im{\subst{\tM}{\tV}{\tx}} is well-defined, substituting
the resulting term, itself a \emph{\tech{configuration}}, into the
\tech{continuation} \im{\tK} could result in the non-\tech{ANF} term
\im{\tlete{\tx}{\tM}{\tMpr}}.
In \tech{ANF}, \tech{configurations} cannot be nested.


To ensure \tech{reduction} preserves \tech{ANF}, I define composition of a
\tech{continuation} \im{\tK} and a \tech{configuration} \im{\tM},
\fullref[]{fig:anf:ecca:hsubst}, typically called \emph{renormalization} in the
literature~\cite{sabry1997:on-cbv,kennedy2007}.
When composing a \tech{continuation} with a \tech{configuration},
\im{\tK\hhw{\tM}}, we essentially unnest all continuations so they remain
left-associated.%
\footnote{
  Some work uses an append notation, \eg, \im{\tM ::
    \tK}~\cite{sabry1997:on-cbv}, suggesting we are appending \im{\tK} onto the
  continuation for \im{\tM}; I prefer notation that evokes composition.
}
Note that these definitions are simplified because I am ignoring
capture-avoiding substitution.

\paragraph{Digression on composition in ANF}
In the literature, the composition operation \im{\tK\hhw{\tM}} is usually
introduced as \emph{renormalization}, as if the only intuition for why it exists
is ``well, it happens that \tech{ANF} is not preserved under \(\beta\)-reduction''.
It is not mere coincidence; the intuition for this operation is composition, and
having a syntax for composing terms is not only useful for stating
\(\beta\)-reduction, but useful for all reasoning about \tech{ANF}!
This should not come as a surprise---compositional reasoning is useful.
The only surprise is that the composition operation is not the usual
one used in programming language semantics, \ie, substitution.
In \tech{ANF}, as in monadic normal form, substitution can be used to compose
any \tech{expression} with a \emph{\tech{value}}, since names are
\tech{values} and \tech{values} can always be replaced by \tech{values}.
But substitution cannot just replace a name, which is a \emph{\tech{value}},
with a \emph{\tech{computation}} or \emph{\tech{configuration}}.
That wouldn't be well-typed.
So how do we compose \tech{computations} with \tech{configurations}?
We can use \tfonttext{let}, as in \im{\tlete{\ty}{\tN}{\tM}}, which we can
imagine as an explicit substitution.
In monadic form, there is no distinction between \tech{computations} and
\tech{configurations}, so the same term works to compose \tech{configurations}.
But in \tech{ANF}, we have no object-level term to compose \emph{configurations}
or \emph{continuations}.
We cannot substitute a \tech{configuration} \im{\tM} into a \tech{continuation}
\im{\tlete{\ty}{\hole}{\tMpr}}, since this would result in the non-ANF (but
valid monadic) expression \im{\tlete{\ty}{\tM}{\tMpr}}.
Instead, ANF requires a new operation to compose configurations:
\im{\tK\hhw{\tM}}.
This operation is more generally known as \emph{hereditary
  substitution}~\cite{watkins2003:clf}, a form of substitution that maintains
canonical forms.
So we can think of it as a form of substitution, or, simply, as composition.

\FigECCARed
I present the call-by-value (\deftech{CBV}) evaluation semantics for \tlang in
\fullref[]{fig:anf:ecca:red}.
It is essentially standard, but recall that \(\beta\)-reduction produces a
\tech{configuration} \im{\tM} which must be composed with the
current \tech{continuation} \im{\tK}.
This semantics is only for the evaluation of \tech*{anf:config}{configurations};
during type checking, we continue to use the type system and conversion relation
defined in \fullref[]{chp:source}.

\subsection{The Essence of Dependent Continuation Typing}
\FigECCAStackTypes
I define \tech{continuation} typing in \fullref[]{fig:anf:ecca:stack-types}.
The type \im{(\tN : \tA) \Rightarrow \tB} of a \tech{continuation} expresses that this
\tech{continuation} expects to be composed with a term equal (syntactically) to the
\tech*{anf:comp}{computation} \im{\tN} of type \im{\tA} and returns a
result of type \im{\tB} when completed.
This is the formal statement that \im{\tN} is \tech{depended upon} (in the sense
introduced in \fullref[]{chp:source}) in the rest of the \tech{computation}, and
is key to recovering the dependency disrupted during \tech{ANF} translation.
For the empty \tech{continuation} \im{\hole}, \im{\tN} is arbitrary since an empty
\tech{continuation} has no ``rest of the program'' that could depend on anything.

Intuitively, what we want from \tech{continuation} typing is a compositionality
property---that we can reason about the types of
\tech*{anf:config}{configurations} \im{\tK\hw{\tN}} by composing the typing
derivations for \im{\tK} and \im{\tN}.
To get this property, a \tech{continuation} type must express not merely the
\emph{\tech{type}} of its hole \im{\tA}, but exactly \emph{which \tech{term}}
\im{\tN} will be bound in the hole.
We see this formally from the typing rule \refrule[anfapp]{Let} (the same for \tlang as
for \slang in \fullref[]{chp:source}), since showing that
\im{\tlete{\ty}{\tN}{\tM}} is well-typed requires showing that \im{\ty = \tN
  \vdash \tM}, that is, requires knowing the definition \im{\ty = \tN}.
If we omit the expression \im{\tN} from the type of \tech{continuations}, we know there
are some \tech{configurations} \im{\tK\hw{\tN}} that we cannot type check
\emph{compositionally}.
Intuitively, if all we knew about \im{\ty} was its \tech{type}, we would be in
exactly the situation of trying to type check a continuation that has
abstracted some dependent type that depends on the \emph{specific} \im{\tN} into
one that depends on an \emph{arbitrary} \im{\ty}.
I prove that \tech{continuation} typing is compositional in this way,
\fullref{lem:anf:ecca:cut}.

Note that the type of the result in a \tech{continuation} type cannot depend on
the term that will be plugged in for the hole, \ie, for a \tech{continuation}
\im{\tK : (\tN : \tA) \Rightarrow \tB}, \im{\tB} does not depend on \im{\tN}.
This is not important for \tech{ANF} translation, but is interesting as it
provides insight into related work as I discuss in \fullref[]{sec:anf:discuss}.
The restriction is not necessary, and is not true in all systems, but turns out
to be true in ANF.
To see this, first note that the initial \tech{continuation} must be empty and
thus \emph{cannot} have a result type that depends on its hole.
The ANF translation will take this initial empty continuation and compose it
with intermediate continuations \im{\tKpr}.
Since composing any continuation \im{\tK : (\tN : \tA) \Rightarrow \tB} with any
continuation \im{\tKpr} results in a new continuation with the final result type
\im{\tB}, then the composition of any two continuations cannot depend on the
type of the hole.
This is similar to how, in CPS, the answer type doesn't matter and might as well
be \im{\bot}.

\subsection{Meta-theory}
{
\allowdisplaybreaks % Let latex break proofs on a page boundary
Since \tlang is a syntactic discipline in \slang, we inherit most of the
meta-theory from \slang, notably: logical consistency, type safety, and
decidability~\cite{luo1989,severi1994:dpts}.
There are some new meta-theoretic questions to answer, though, such as: Is
\tech{ANF} evaluation sound? Does \tech{continuation} typing make sense?

First, I prove that \tech{ANF} evaluation semantics is sound with respect to
definitional equivalence.
That is, running in the \tech{ANF} evaluation semantics produces an equivalent
value to normalization in the equivalence relation.
The heart of this proof is actually \emph{\tech{naturality}}, a property found
in the literature on continuations that essentially expressed freedom from
control effects~\cite{thielecke2003}.

When computing definitional equivalence, we end up with terms that are not in
ANF, and can no longer be used in the ANF evaluation semantics.
This is not a problem; we could always ANF translate the resulting term if needed.
To make it clear which terms are in ANF, and which are not, I leave terms and subterms that are
in ANF in the \tfonttext{target language font}, and write terms or subterms that are not in ANF
in the \sfonttext{source language font}.
Meta-operations like substitution may be applied to ANF (\tfonttext{red}) terms,
but result in non-ANF (\sfonttext{blue}) terms.
Since substitution leaves no visual trace of its blueness, I wrap such terms in
a distinctive language boundary such as \im{\sbound{\subst{\tM}{\tMpr}{\tx}}}
and \im{\sbound{\tK\hw{\tM}}}.
The boundary indicates the term is a target (\(\mathcal{T}\)) term on the inside
but a source (\(\mathcal{S}\)) term on the outside.
The boundary is only meant to communicate with the reader that a term is no
longer in \tech{ANF}; it has no meaning operationally.

First, I prove that composing \tech{continuation} in \tech{ANF} is sound with
respect to substitution.
This is an expression of \tech{naturality} in \tech{ANF}: composing a term
\im{\tM} with its \tech{continuation} \im{\tK} in \tech{ANF} is equivalent to running
\im{\tM} to a value and substituting the result into the \tech{continuation} \im{\tK}.
\begin{lemma}[Naturality]
  \label{lem:anf:tgt:het-sound}
  \im{\tK\hhw{\tM} \equiv \sbound{\tK\hw{\tM}}}
\end{lemma}
\begin{proof}
  By induction on the structure of \im{\tM}
  \begin{proofcases}
    \item \im{\tM = \tN} trivial
    \item \im{\tM = \tlete{\tx}{\tNpr}{\tMpr}}.

      Must show that \im{\tlete{\tx}{\tNpr}\tK\hhw{\tMpr} \equiv \sbound{\tK\hw{\tlete{\tx}{\tNpr}{\tM}}}}.
      Note that this requires breaking \tech{ANF} while computing equivalence.
      \begin{align}
        &\tlete{\tx}{\tNpr}\tK\hhw{\tMpr} \nonumber \\
        \step_{\zeta}~& \sbound{\subst{\tK\hhw\tMpr}{\tNpr}{\tx}} \\
        &\text{note that this substitution is undefined in \tech{ANF}} \nonumber \\
        =~& \tK\hhw{\sbound{\subst{\tMpr}{\tNpr}{\tx}}} \\
        &\text{by ignoring capture-avoiding substitution} \nonumber \\
        \vartriangleleft^*~& \sbound{\tK\hw{\tlete{\tx}{\tNpr}{\tM}}} \\
        & \text{by \(\zeta\)-reduction and congruence} \nonumber \qedhere
      \end{align}
  \end{proofcases}
\end{proof}

Next I show that the \tech{ANF} evaluation semantics is sound with respect to
definitional equivalence.
This is also central to my later proof of compiler correctness.
To do that, I first show that the small-step semantics is sound.
Then I show soundness of the evaluation function.
\begin{lemma}[Small-step soundness]
  \label{lem:anf:tgt:step-sound}
  If \im{\tstepjudg[\mapsto]{}{\tM}{\tMpr}} then \im{\sequivjudg{}{\tM}{\tMpr}}
\end{lemma}
\begin{proof}
  By cases on \im{\tstepjudg[\mapsto]{}{\tM}{\tMpr}}.
  Most cases follow easily from the \slang reduction relation and congruence.
  I give representative cases.
  \begin{proofcases}
    \item \im{\tK\hw{\tappe{(\tfune{\tx}{\tA}{\tMone})}{\tV}} \mapsto_{\beta} \tK\hhw{\subst{\tMone}{\tV}{\tx}}}

      Must show that \im{\tK\hw{\tappe{(\tfune{\tx}{\tA}{\tMone})}{\tV}} \equiv \tK\hhw{\subst{\tMone}{\tV}{\tx}}}
      \begin{align}
        & \tK\hw{\tappe{(\tfune{\tx}{\tA}{\tMone})}{\tV}} \nonumber \\
        \stepstar~& \sbound{\tK\hw{\subst{\tMone}{\tV}{\tx}}} & \text{by \(\beta\) and congruence} \\
        \equiv~& \tK\hhw{\subst{\tMone}{\tV}{\tx}}  & \text{by \fullref[]{lem:anf:tgt:het-sound}}
      \end{align}

    \item \im{\tK\hw{\tfste{\tpaire{\tVone}{\tVtwo}}} \mapsto_{\pi_1} \tK\hw{\tVone}}

      Must show that \im{\tK\hw{\tfste{\tpaire{\tVone}{\tVtwo}}} \equiv
        \tK\hw{\tVone}}, which follows by \im{\step_{\pi_1}} and congruence. \qedhere
  \end{proofcases}
\end{proof}

\begin{theorem}[Evaluation soundness]
  \label{thm:anf:tgt:eval-sound}
  \im{\sequivjudg{}{\teval{\tM}}{\tM}}
\end{theorem}
\begin{proof}
  By induction on the length \im{n} of the reduction sequence given by \im{\teval{\tM}}.
  Note that, unlike \tech{conversion}, the \tech{ANF} evaluation semantics
  have no congruence rules, so this \emph{can} be proved on the length of reduction sequences.
  \begin{proofcases}
    \item \im{n = 0} By \refrule[anfapp]{Red-Refl} and \refrule*[anfapp]{eqv}{\im{\equiv}}.
    \item \im{n = i+1} Follows by \fullref[]{lem:anf:tgt:step-sound} and the induction hypothesis. \qedhere
  \end{proofcases}
\end{proof}

I prove that plugging a well-typed term into a well-typed \tech{continuation} results
in a well-typed term of the expected type.
This theorem corresponds to the Rule \rulename{Cut} rule from sequent calculus, and
tells us that \tech{continuation} typing allows for compositional reasoning about
\tech{configurations} \im{\tK\hw{\tN}} whose result types do not
depend on \im{\tN}.
\begin{lemma}[Cut]
  \label{lem:anf:ecca:cut}
  If \im{\styjudg{\tlenv}{\tK}{(\tN : \tA) \Rightarrow \tB}} and
  \im{\styjudg{\tlenv}{\tN}{\tA}} then \im{\styjudg{\tlenv}{\tK\hw{\tN}}{\tB}}.
\end{lemma}
\begin{proof}
  By cases on \im{\styjudg{\tlenv}{\tK}{(\tN : \tA) \Rightarrow \tB}}
  \begin{proofcases}
    \item \im{\styjudg{\tlenv}{\hole}{(\tN : \tA) \Rightarrow \tA}}, trivial
    \item \im{\styjudg{\tlenv}{\tlete{\ty}{\hole}{\tM}}{(\tN : \tA) \Rightarrow \tB}}

      We must show that \im{\styjudg{\tlenv}{\tlete{\ty}{\tN}{\tM}}{\tB}}, which
      follows directly from \refrule[anfapp]{Let} since, by the continuation typing
      derivation, we have that \im{\styjudg{\tlenv,\ty=\tN}{\tM}{\tB}} and \im{\ty \not\in \FV(\tB)}. \qedhere
  \end{proofcases}
\end{proof}

\FigStackExports
To reason inductively about ANF terms, we need to separate a configuration
\im{\tM} into its exported definitions \im{\edefs{\tM}} and its underlying
computation \im{\ehole{\tM}}.
In \fullref[]{fig:anf:ecca:exports}, I define \im{\edefs{\tM}} to be the
definitions exported by the ANF term \im{\tM}.
These are the definitions that will be in scope for a continuation \im{\tK} when
composed with \im{\tM}, \ie, in scope for \im{\tK} in \im{\tK\hhw{\tM}}.
I define this as \im{\ehole{\tM}}, also in \fullref[]{fig:anf:ecca:exports}.
Note that \im{\ehole{\tM}} will only be well typed in the environment for
\im{\tM} extended with the definitions \im{\edefs{\tM}}.

I show that a configuration is nothing more than its exported definitions and
underlying computation, \ie, that in a context with the exports of
\im{\edefs{\tM}}, \im{\ehole{\tM} \equiv \tM}.
In essence, this lemma shows how ANF converts a dependency on a
\emph{configuration} \im{\tM} into a series of dependencies on \emph{values},
\ie, the names
\im{\txin{0},\dots,\txin{n+1}} in \im{\edefs{\tM}}.
Note that the ANF guarantees that all dependent typing rules, like
\im{\tappe{\tV}{\tVpr} : \subst{\tB}{\tVpr}{\tx}}, only depend on
\tech*{anf:value}{values}.
This lemma allows us to recover the dependency on a
\tech*{anf:config}{configuration}.
\begin{lemma}
  \label{lem:anf:export-equal}
  \im{\sequivjudg{\tedefs{\tM}}{\tehole{\tM}}{\tM}}
\end{lemma}
\begin{proof}
  Note that the exports \im{\tedefs{\tM}} are exactly the definitions from
  the syntax of \im{\tM}.
  Inlining those definitions via \(\delta\)-reduction is the same as reducing
  \im{\tM} via \(\zeta\)-reduction.
  \begin{align}
    \tM =~&(\tlete{\txone}{\tNone}{\dots\tlete{\txin{n}}{\tNin{n}}{\tNin{n+1}}}) \\
    \step_{\zeta}^n~& \subst{\tNin{n+1}}{\tNone\dots\tNin{n}}{\txone\dots\txin{n}}
    \end{align}

  And \im{\tehole{\tM} = \tNin{n+1} \step_{\delta}^n \subst{\tNin{n+1}}{\tNone\dots\tNin{n}}{\txone\dots\txin{n}}}
\end{proof}
}
