\section{Related and Future Work}
\label{sec:anf:discuss}
\label{sec:anf:related}

\subsection{Comparison to CPS}
\label{sec:discuss:cps}
ANF is usually seen in opposition to CPS, so I briefly discuss similarities and
differences between our type-preserving ANF and prior work on-type preserving
CPS.
ANF is favored as a compiler intermediate representation, although not
universally.
\citet{maurer2017} argue for ANF, over alternatives such as CPS, because ANF
makes control flow explicit but keeps evaluation order implicit, automatically
avoids administrative redexes, simplifies many optimizations, and keeps code in
direct style.
\citet{kennedy2007} argues the opposite---that CPS is preferred to ANF---and
summarizes the arguments for and against.

Most recent work on CPS translation of dependently typed language has focused on
expressing control effects.
\citet{pedrot2017:lola} uses a non-standard CPS translation to internalize
classical reasoning in the Calculus of Inductive Constructions (CIC).
\citet{cong2018:lam-pi-s-r,cong2018:tfp} develop CPS translations to
express delimited control effects, via \texttt{shift} and \texttt{reset}, in a
dependently typed language.
\citet{miquey2017} uses a CPS translation to model a dependent sequent calculus.
When expressing control effects in dependently typed languages, it is
\emph{necessary} to prevent certain dependencies from being expressed to
maintain consistency~\cite{barthe2002,herbelin2005}, therefore these
translations do not try to recover dependencies in the way we discuss in
\fullref[]{sec:anf:ideas}.

My own CPS translation, \fullref[]{chp:cps} (published before writing this
chapter as \citet{bowman2018:cps-sigma}) does avoid control effects and seeks to
develop a type preserving translation.
The new typing rule I add is similar to my \refrule{K-Bind} in \tlang, and is
used for the same purpose: to recover disrupted dependencies.
Unfortunately, that encoding does not scale to higher universes, and relies on
interpreting all functions as parametric (discussed further in
\fullref[]{chp:conclusions}).
By contrast, this ANF translation works with higher universes and, since \tlang
is a subset of \slang, the ANF translation is orthogonal to parametricity.

\subsection{Branching and Join Points}
The \tech{ANF} translation presented so far does not support all of \slang; in
particular, it omits the \tech{dependent conditional}.
This is primarily for simplicity, as many of the problems with \tech{ANF} are
orthogonal to \tech{dependent conditional}.
However, \tech{dependent conditionals} do introduce non-trivial challenges for
\tech{ANF} translation.
In this section, I give the translation and argue that it is \tech{type preserving}.

It is well-known that \tech{ANF} in the presence of branching constructs, such
as \texttt{if}, can cause considerable code duplication for branches.
For instance, supposing we have an (for the moment, non-dependent) \texttt{if},
the na\"ive \tech{ANF} translation is the following.
%
\begin{displaymath}
  \anf{\sife{\se}{\seone}{\setwo}}{\tK} = \anf{\se}{\tlete{\tx}{\hole}{\tife{\tx}{(\anf{\seone}{\tK})}{(\anf{\setwo}{\tK})}}}
\end{displaymath}
%
Notice that the current \tech{continuation} \im{\tK} is duplicated in the branches.

The well-known solution is to add a \emph{join point}---essentially, a
continuation that is used only for avoiding code duplication in branch
constructs.
Using join points, we would translate \sfonttext{if} as follows.
%
\begin{displaymath}
  \anf{\sife{\se}{\seone}{\setwo}}{\tK} =
    \anf{\se}{\tlete{\tx}{\hole}{\begin{stackTL}\tlete{\tj}{\tfune{\txpr}{\tA}{\tK\hw{\txpr}}}{
            \\\,\,\quad\begin{stackTL}
              \tife{\tx}{\anf{\seone}{\tlete{\txone}{\hole}{~\tappe{\tj}{\txone}}}\\\quad\,\,\,\,}{\,\,\anf{\setwo}{\tlete{\txtwo}{\hole}{~\tappe{\tj}{\txtwo}}}}
          \end{stackTL}}}
    \end{stackTL}}
\end{displaymath}
%
Instead of duplicating \im{\tK}, we create a join point \im{\tj} which is called in the branches.

Extending the translation to support join points requires (for decidable
type checking) that the translation generate a type annotation \im{\tA} for the
join point, where \im{\tA} is the translation of the type of the \tfonttext{if}
statement.
It is easy to extend the translation to be defined on typing derivations; in
fact, the rest of the translation defined in this dissertation are defined on
typing derivations, and the proof architecture described in
\fullref[]{chp:type-pres} is designed to support this.
The only disadvantage is that structuring the translation this way disallows
typed equivalence, unless the problem discussed in \fullref[]{chp:type-pres} can
be solved.

The real problem arises when we have \emph{\techs{dependent conditional}}, in
which the result type of the branches can depend on the scrutinee of the
conditional.
Recall that typing rule for dependent \texttt{if}, reproduced below.
%
\begin{displaymath}
  \inferrule
  {\styjudg{\slenv,\sy:\sboolty}{\sB}{\sU} \\
   \styjudg{\slenv}{\se}{\sboolty}\\
   \styjudg{\slenv}{\seone}{\subst{\sB}{\struee}{\sy}} \\
   \styjudg{\slenv}{\setwo}{\subst{\sB}{\sfalsee}{\sy}}}
  {\styjudg{\slenv}{\sife{\se}{\seone}{\setwo}}{\subst{\sB}{\se}{\sy}}}
\end{displaymath}

We can describe the problem clearly using \tech{continuation} typing.
The \tech{ANF} translation of this term is with respect to \tech{continuation} \im{\tK
  : (\ehole{\sife{\se}{\seone}{\setwo}} : \anfh{\subst{\sB}{\se}{\sy}})
  \Rightarrow \tBpr}.
However, the \tech{ANF} translation will use \im{\tK} in an ill-typed way,
producing in one branch \im{\anfh{\seone}{\tK}}, and in the other branch
\im{\anfh{\setwo}{\tK}}.
For the first branch, for instance, we must show that now \im{\tK : (\ehole{\seone} :
\anfh{\subst{\sB}{\struee}{\sy}}) \Rightarrow \tBpr}.
The definitions, \im{\edefs{\se}}, introduced by the translation of \texttt{if}
are not sufficient to show that this type is equivalent to the type of \im{\tK}
expected for the translation of \texttt{if}.
I discuss an essentially similar problem for \tech{CPS} with
\techs{dependent conditional} in \fullref[]{chp:cps}.

We could resolve this, it seems, if we could assume that while type checking the
first branch that \im{\se = \struee}, and similarly for the second branch that
\im{\se = \sfalsee}.
But one of these is not true; we cannot have that both \im{\se = \struee} and
\im{\se = \sfalsee}.
Since in \tlang we require reduction under both branches during type checking,
reduction under this inconsistent assumption could cause divergence.

To enable us to make inconsistent assumptions like the above but ensure
strong normalization, my idea is to use explicit equality proofs and explicit
coercions.
The coercions would block reduction until they have a \tech{proof} of
equality, and since we will never end up with a \tech{proof} that
\im{\struee = \sfalsee}, the inconsistent assumption can never be used in
reduction.
But the coercion would allow the ANF translation to type check.

With this idea, the type rule for \sfonttext{if} would be the following.
%
\begin{displaymath}
  \inferrule
  {\styjudg{\slenv,\sy:\sboolty}{\sB}{\sU} \\
   \styjudg{\slenv}{\se}{\sboolty} \\
   \styjudg{\slenv,\sp:\se = \struee}{\seone}{\subst{\sB}{\struee}{\sy}} \\
   \styjudg{\slenv,\sp:\se = \sfalsee}{\setwo}{\subst{\sB}{\sfalsee}{\sy}}}
  {\styjudg{\slenv}{\sife{\se}{\seone}{\setwo}}{\subst{\sB}{\se}{\sy}}}
\end{displaymath}
%
Then, the \tech{ANF} na\"ive translation for \tech{dependent conditionals} would
be the following; I give the join-point translation shortly.
%
\begin{displaymath}
  \anf{\sife{\se}{\seone}{\setwo}}{\tK} = \anf{\se}{
    \tlete{\begin{stackTL}\tx}{\hole}{\\
      \begin{stackTL}\tife{\tx}{\anf{\seone}{\tlete{\txone}{\hole}{\tK\hw{\tcoee{\tp}{\txone}}}}\\\quad\,\,\,\,}{\,\,\,{\anf{\setwo}{\tlete{\txtwo}{\hole}{\tK\hw{\tcoee{\tp}{\txtwo}}}}}}}
      \end{stackTL}
      \end{stackTL}}
\end{displaymath}
%
The term \im{\tcoee{\tp}{\te}} is an elimination for the identity type
(derivable from only axiom \im{J}, a standard axiom admitted in many dependent
type theories), with the following standard semantics.
Recall that the result type of \im{\tK} cannot depend on the term in its hole,
so we do not need a similar conversion for the result of \im{\tK}.
%
\begin{mathpar}
  \inferrule
   {\styjudg{\tlenv,\tx:\tA}{\tB}{\tU} \\
    \styjudg{\tlenv}{\tp}{\teone = \tetwo} \\
    \styjudg{\tlenv}{\te}{\subst{\tB}{\teone}{\tx}}}
   {\styjudg{\tlenv}{\tcoee{\tp}{\te}}{\subst{\tB}{\tetwo}{\tx}}}

   \inferrule
   {\styjudg{\tlenv}{\te}{\tA}}
   {\styjudg{\tlenv}{\tappe{\tfonttext{refl}}{\te}}{\te = \te}}

   \tcoee{{\tappe{\tfonttext{refl}}{\te}}}{\te} \step \te
\end{mathpar}

The translation with join points is below, and requires no further additions.
\begin{displaymath}
  \anf{\sife{\se}{\seone}{\setwo}}{\tK} =
  \anf{\se}{\tlete{\tx}{\hole}{\begin{stackTL}\tlete{\tj}{\tfune{\txpr}{\tA}{\tK\hw{\txpr}}}{
          \\\quad
          \begin{stackTL}\tife{\tx}{\anf{\seone}{\tlete{\txone}{\hole}{\tappe{\tj}{\tcoee{\tp}{\txone}}}}\\\quad\,\,\,\,}{\,\,\anf{\setwo}{\tlete{\txtwo}{\hole}{\tappe{\tj}{\tcoee{\tp}{\txtwo}}}}}
    \end{stackTL}}}
  \end{stackTL}}
\end{displaymath}
The type annotation \im{\tA} would be the translation of the type source
\sfonttext{if} expression.
This is why, formally, the translation must be defined over typing derivations
instead of syntax.

I conjecture that this translation would be type preserving, but a full
investigation is left as future work.

\citet{cong2018:lam-pi-s-r} use a similar approach to extend the \tech{CPS}
translation presented in \fullref[]{chp:cps} to inductive types with dependent
case analysis.

\subsection{Dependent Pattern Matching and Commutative Cuts}
The above problem with \techs{dependent conditional} is exactly the problem of
commutative cuts for case analysis~\cite{boutillier2012,herbelin2009:pps-talk}.
Formally, the problem of commutative cuts can be phrased: Is the following
transformation \tech{type preserving}?
%
\begin{displaymath}
  \sK\hw{\sife{\se}{\seone}{\setwo}} \leadsto \sife{\se}{\sK\hw{\seone}}{\sK\hw{\setwo}}
\end{displaymath}
%
\tech{ANF} necessarily performs this transformation, as shown in the previous section.

Ignoring \tech{ANF} for a moment, in general this \emph{is not} \tech{type
  preserving}, and \tech{stack} typing shows why.
Suppose that we have a non-\tech{ANF} \tech{stack} \im{\sK} with the following
typing derivation.
%
\begin{mathpar}
  \inferrule
  {\styjudg{\slenv,\sypr:\sboolty}{\sBpr}{\sUpr} \\
    \styjudg{\slenv,\sy:\subst{\sBpr}{\se}{\sypr}}{\sB}{\sU} \\
   \slenv \vdash {\sife{\se}{\seone}{\setwo}} : \subst{\sBpr}{\se}{\sypr}}
  {\slenv \vdash \sK : ({\sife{\se}{\seone}{\setwo}} : \subst{\sBpr}{\se}{\sypr}) \Rightarrow \subst{\sB}{({\sife{\se}{\seone}{\setwo}})}{\sy}}
\end{mathpar}
%
Note that the result type of this \tech{stack}, \im{\sBpr}, \emph{may} depend on
the term in the hole, since this is not in \tech{ANF}.
The problem is that after the commutative cut, we must show that
\im{\sK\hw{\seone}} and \im{\sK\hw{\setwo}} are well-typed in the branches,
which is not true in general since \im{\sK} expects
\im{\sife{\se}{\seone}{\setwo}}.

If we add the equalities \im{\se = \struee} and \im{\se = \sfalsee} while type
checking the branches, as proposed in the previous section, then it appears that
we can make the terms \im{\sK\hw{\seone}} and \im{\sK\hw{\setwo}} well-typed, although
it requires a generalization of \fullref[]{lem:anf:ecca:cut}.
However, we still cannot show the commutative cut is \tech{type preserving},
since the result type \im{\sB} also depends on the term in the hole.
And now, the two branches of the \sfonttext{if} have different types:
\im{\sK\hw{\seone} : \subst{\sB}{\seone}{\sy}}, while \im{\sK\hw{\setwo} :
  \subst{\sB}{\setwo}{\sy}}.
Since the branches of an \sfonttext{if} must have the same type (up to
equivalence), it appears that we must show \im{\seone \equiv \setwo}.

In fact, what we need to show is essentially another, smaller, commutative cut.
Viewing \im{\sB} as a type-level \tech{stack}, we must show
\im{\sB\hw{\sife{\se}{\seone}{\setwo}} \equiv
  \sife{\se}{\sB\hw{\seone}}{\sB\hw{\setwo}}}.
This is smaller in the sense that the type of this type cannot contain a
commutative cut.
For booleans, we could pursue this by adding yet another appeal to
\im{J}, but this approach does not scale to indexed inductive types.

But, there is a solution: we can make the commutative cut type preserving, even
for general inductive types.
\citet{boutillier2012} give an extension to CIC that allows for typing commutative
cuts, in particular, by relaxing the termination checker.
Explaining the solution is out of scope for this work.
I only want to point out two things.
First, the solution adds an environment of \emph{equalities} to the termination
checker, just as I propose for \tech{stack} typing and for typing \sfonttext{if}
above.
Second, the solution requires \emph{axiom K}, a very strong requirement which is
inconsistent with certain extensions to type theory, such as univalence.
This raises the question: does \tech{ANF} in general require axiom \emph{K}?

I conjecture the answer is no, and the reason is the interesting property we
observed in \fullref[]{sec:anf:target}: in \tech{ANF}, the result type cannot
depend on the term in the hole.
This additional structure seems to avoid some problems of commutative
cuts, and I hope that it will be enough to scale \tech{ANF} to indexed inductive
types without additional requirements on the type theory.
I have one additional reason to be hopeful: even if we need a stronger axiom than
\im{J}, work on dependent pattern matching suggests that univalence may replace
axiom \emph{K} for our purposes.

Recent work on dependent pattern matching creates typing rules similar to what we
suggest above to yield additional equalities during a pattern
match~\cite{barras2008:new-cic-elim,cockx2016}.
There is another unfortunate similarity: some work on dependent pattern matching
requires axiom \emph{K}.
In particular, \citet{barras2008:new-cic-elim} give a new eliminator for CIC
which adds additional equalities while checking branches of an elimination, and show
that this new typing rule is equivalent to axiom \emph{K}.
\citet{cockx2016} discuss a proof-relevant view of unification, in the context
of Agda's dependent pattern matching.
They note that the heterogeneous equalities usually required by dependent
pattern matching require axiom \emph{K} to be useful.
They also take a different approach, and build on an idea from homotopy type
theory that one equality can ``layer over'' another, to get a proof relevant
unification algorithm that does not rely on \emph{K}, and yet yields the
additional equalities for dependent pattern matching.

\subsection{Dependent Call-by-Push-Value and Monadic Form}
Call-by-push-value (\deftech{CBPV}) is similar to the \tech{ANF} target
language, and to \tech{CPS} target languages.
In essence, \tech{CBPV} is a \(\lambda\) calculus in monadic normal form
suitable for reasoning about \tech{CBV} or \tech{CBN}, due to explicit
sequencing of computations.
It has \techs{value}, \techs{computation}, and \techs{stack}, similar to
\tech{ANF}, and has compositional typing rules (which inspired much of my own
presentation).
The particular structure of \tech{CBPV} is beneficial when modeling effects; all
\techs{computation} should be considered to carry an arbitrary effect, while
\techs{value} do not.

Work on designing a dependent call-by-push-value (\deftech{dCBPV}) runs into
some of the same design issues that we see in \tech{ANF}
\cite{ahman2017:dissertation,vakar2017:dissertation}, but critically, avoids the
central difficulties introduced in \fullref[]{sec:anf:ideas}.
The reason is essentially that monadic normal form is more compositional than
\tech{CPS} or \tech{ANF}, so \tech{dependency} is not disrupted in the same way.

Recall from \fullref[]{sec:anf:target} that our definition of composition was
entirely motivated by the need to compose \tech*{anf:config}{configurations} and
\techs{stack}.
In \tech{CBPV}, and monadic form generally, there is no distinction between
\tech{computation} and \tech*{anf:config}{configurations}, and \texttt{let} is
free to compose \tech*{anf:config}{configurations}.
This means that \tech*{anf:config}{configurations} can return intermediate
\techs{computation}, instead of composing the entire rest of the stack inside
the body of a \texttt{let}.
The monadic translation of \im{\ssnde{\se}} from \fullref[]{sec:anf:ideas}, which is
problematic in \tech{CPS} and \tech{ANF}, is given below and is easily
\tech{type preserving}.
%
\begin{displaymath}
  \anfh{\ssnde{\se} : \subst{\sB}{\se}{\sy}} = \tlete{\tx}{\anfh{\se}}{\tsnde{\tx}} : \subst{\anfh{\sB}}{\anfh{\se}}{\ty}
\end{displaymath}

Note that since \texttt{let} can bind the ``configuration'' \im{\anfh{\se}}, the
typing rule \rulename{Let} and the compositionality lemma suffice to show
\tech{type preservation}, without any reasoning about definitions.
In fact, we don't even need \emph{definitions} for monadic form; we only need a
dependent result type for \tfonttext{let}.
The dependent typing rule for \tfonttext{let} without definition is essentially
the rule given by \citet{vakar2017:dissertation}, called the dependent Kleisli
extension, to support the \tech{CBV} monadic translation of type theory into
\tech{dCBPV}, and the \tech{CBN} translation with strong dependent pairs.
\citet{vakar2017:dissertation} observes that without the dependent Kleisli
extension, \tech{CBV} translation is ill-defined (not \tech{type preserving}),
and \tech{CBN} only works for dependent elimination of positive types.
A similar observation was made independently in my own work with Nick Rioux,
Youyou Cong, and Amal Ahmed~(\citeyear{bowman2018:cps-sigma}, presented in
\fullref[]{chp:cps}): type-preserving \tech{CBV} \tech{CPS} fails for \(\Pi\)
types, in addition to the well-known
result that the \tech{CBN} translation failed for \(\Sigma\)
types~\cite{barthe2002}.

If we restrict the language so that types can only depend on
\emph{\techs{value}}, then the extension with dependent \texttt{let} is not
necessary.
This restriction seems sensible in the context of modeling effects.
\citet{ahman2017:dissertation} in \deftech{eMLTT}, a co-discovered variant of
dependent \tech{CBPV}, avoids dependent let altogether, but comes up with many
useful models of dependent types with effects.
\citet{ahman2017:dissertation}, however, does not give a translation from type
theory into \tech{eMLTT}, and it seems likely that an extension with dependent
let would be required to do so.
(This would be necessary to build on \tech{eMLTT} as a compiler IL.)
However, as \citet{vakar2017:dissertation} points out, it is not clear what it
\emph{means} to have a type depend on an effectful computation, and trying to do
so makes it impossible to model effects in \tech{dCBPV} the way one would hope.

In \tech{eMLTT}, stacks cannot have a result type that depends on the value it
is composed with, just as in our \rulename{K-Bind} rule.
However, the \tech{dCBPV} of \citet{vakar2017:dissertation} \emph{does} allow
the result type of stacks to depend on values, but only on values.
It is unclear what trade-offs each approach presents.
