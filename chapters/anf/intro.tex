In this chapter, I develop the first of the two front-end translations for a
\tech{type-preserving} compiler for \slang: \tech{ANF} translation.
The target language for this translation is \tlang, a variant of \slang
syntactically restricted to \tech{ANF}.
Recall from \fullref[]{chp:type-pres} that the first translation in the model
compiler imposes a distinction between \tech{computation} and \tech{values}.
A-normal form (\tech{ANF}) translation accomplishes this goal and can take the
place of \tech{CPS} translation in the model compiler.
(In fact, \tech{ANF} was introduced as the essence of
\tech{CPS}~\cite{flanagan1993}.)

\begin{digression}
  The \emph{A} in A-normal form has no further meaning.
  The \emph{A} comes from the set of axioms and reductions introduced by
  \citet{sabry1992} to reason about CPS.
  \citet{flanagan1993} use the \emph{A} as a label in a commutative diagram,
  observing that there should exist a direct translation that produces a normal
  form equivalent to the CPS/optimization/un-CPS process found in practice.
  The programs produced by this translation are in normal form with respect to
  the \emph{A}-reductions of \citeauthor{sabry1992}, \ie,
  in \emph{A}-normal form.
\end{digression}

I start by explaining the key problems that arise when preserving dependent
types through \tech{ANF} translation and the main idea to my solution.
I then move on to the formal development of the target language, the
translation, and proofs of \tech{type preservation} and \tech{compiler
correctness}.
I conclude with a discussion of related problems, and a brief comparison
of \tech{CPS} and \tech{ANF} in the context of dependent-type preservation, but
leave a full discussion of \tech{CPS} until \fullref[]{chp:cps}.

\begin{typographical}
 In this chapter, I define fixed source and target languages.
 I typeset the source language, \slang, in a \emph{\sfonttext{blue, non-bold,
 sans-serif font}}, and the target language, \tlang, in a
 \emph{\tfonttext{bold, red, serif font}}.
\end{typographical}
