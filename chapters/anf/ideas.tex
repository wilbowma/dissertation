\section{Main Ideas}
\label{sec:anf:ideas}
The idea behind ANF is to make control flow explicit in the syntax of a program
so transfer of control can be expressed as a jump.
ANF encodes \emph{computation} (\eg, reducing an expression to a value) as
sequencing simple intermediate computations with \im{let} expressions.
To reduce \im{\se} to a value in a high-level language, we need to describe the
evaluation order and control flow of each language primitive.
For example, if \im{\se} is an application \im{\sappe{\seone}{\setwo}} and we
want a call-by-value semantics, we say the language first evaluates
\im{\seone} to a value, then evaluates \im{\setwo} to a value, then performs
the function application.
ANF makes this explicit in the syntax by decomposing \im{\se} into a
series of primitive computations sequenced by \tfonttext{let}, as in
\im{\tnlete{\txin{0}=\tNin{0},...,\txin{n}=\tNin{n}}{\tN}}, where each
\im{\tNin{i}} is either a value or a primitive computation applied
to a value.
The ANF translation of \im{\sappe{\seone}{\setwo}} is the following.
\begin{mathpar}
  \tnlete{\begin{stackTL}
      \txin{1_0}=\tNin{1_0},...,\txin{1_n}=\tNin{1_n},
      \txin{1}=\tNin{1}\\
      \txin{2_0}=\tNin{2_0},...,\txin{2_n}=\tNin{2_n},
      \txin{2}=\tNin{2}\\}
  {(\tappe{\txone}{\txtwo})}
  \end{stackTL}
\end{mathpar}
\begin{mathpar}
  \text{where~}\begin{stackTL}
    \sembrace{\seone} = (\tnlete{\txin{1_0}=\tNin{1_0},...,\txin{1_n}=\tNin{1_n}}{\tNin{1}})\\
    \sembrace{\setwo} = (\tnlete{\txin{2_0}=\tNin{2_0},...,\txin{2_n}=\tNin{2_n}}{\tNin{2}})\\
  \end{stackTL}
\end{mathpar}

Roughly, we can think of this translation as
\im{\tnlete{\txone=\sembrace{\seone},\txtwo=\sembrace{\setwo}}{\tappe{\txone}{\txtwo}}}.
This rough translation is only ANF if \im{\seone} and \im{\setwo} are values or
primitive computations.
In general, the ANF translation reassociates all the intermediate computations
from \im{\sembrace{\seone}} and \im{\sembrace{\setwo}} so there are no nested
\tfonttext{let} expressions.
Once in ANF, it is simple to formalize a machine semantics to implement
evaluation by always reducing the left-most computation, which will be a
primitive operation.

The problem in developing a type-preserving translation for a dependently typed
language is that changing the structure of a program disrupts
the \tech{dependencies} described in \fullref[]{chp:source}, \ie,
an expression \im{\sepr} whose type and evaluation depends on a sub-expression
\im{\se}.
Recall that I call to a sub-expression such as \im{\se} \emph{depended upon}.
This pattern happens in dependent elimination forms, such as application and
projection.
For example, for a dependent function \im{\seone : \spity{\sx}{\sA}{\sB}},
the application is typed as \im{\sappe{\seone}{\setwo} : {\subst{\sB}{\setwo}{\sx}}}.
Notice that the \emph{depended upon} sub-expression \im{\setwo} is copied into the type.
If we transform the expression \im{\sappe{\seone}{\setwo}}, we can easily change
the type.
For example, recall the ANF translation for this expression given earlier.
\begin{displaymath}
\tnlete{\begin{stackTL}
      \txin{1_0}=\tNin{1_0},...,\txin{1_n}=\tNin{1_n},
      \txin{1}=\tNin{1}\\
      \txin{2_0}=\tNin{2_0},...,\txin{2_n}=\tNin{2_n},
      \txin{2}=\tNin{2}\\}
  {(\tappe{\txone}{\txtwo})}
  \end{stackTL}
\end{displaymath}
Using the standard typing rule for dependent \tfonttext{let} (without
definitions), the type of the ANF translation above is
\im{\subst{\sembrace{\sB}}{\txtwo}{\tx}} (the type of the body), with all
\tfonttext{let}-bindings substituted into this type, \ie, roughly
\im{\subst{\subst{\sembrace{\sB}}{\txtwo}{\tx}}{\tNin{2_0}}{\txin{2_0}}[...][\tNin{2_n}/\txin{2_n}][\tNtwo/\txtwo]}.
To show type preservation, we must show that the target type system can prove
that this type is equivalent to the translation of the original type, \ie,
\im{\sembrace{\subst{\sB}{\setwo}{\sx}}}.
Intuitively, it \emph{ought} to be true that the intermediate computations
\im{(\tNin{2_0},...,\tNin{2_n},\tNin{2})} are
equivalent to \im{\sembrace{\setwo}}, since those are the computations that
must happen to compute the value of \im{\setwo}.
Therefore, the two types \emph{ought} to be equivalent.

But the intuitive argument that ANF ought to be type preserving is wrong.
Instead of the \im{(\tappe{\txone}{\txtwo})} appearing in the body directly as
above, consider the expression below in which \im{(\tappe{\txone}{\txtwo})} is
bound and then used.
\begin{mathpar}
  \tnlete{...}{\tlete{\ty}{(\tappe{\txone}{\txtwo})}{\tappe{\tf}{\ty}}}

  \where{\tf : (\subst{\sembrace{\sB}}{\sembrace{\setwo}}{\tx}) \to \tC}
\end{mathpar}
To show type preservation, we now need to reestablish a dependent type that is
\tfonttext{let}-bound, instead of in the body of a \tfonttext{let}.
The type derivation fails, as follows.
\begin{mathpar}
  \inferrule
      {\inferrule
        {~}
        {... \vdash {(\tappe{\txone}{\txtwo})} : \subst{\sembrace{\sB}}{\txtwo}{\tx}}\\
    \inferrule
        {\text{fails}}
    {...,\ty:\subst{\sembrace{\sB}}{\txtwo}{\tx} \vdash \tappe{\tf}{\ty} : \tC}}
  {... \vdash \tlete{\ty}{(\tappe{\txone}{\txtwo})}{\tappe{\tf}{\ty}} : \subst{\tC}{(\tappe{\txone}{\txtwo})}{\ty}}
\end{mathpar}
This fails since \im{\tf} expects
\im{\ty : (\subst{\sembrace{\sB}}{\sembrace{\setwo}}{\tx})}, but is applied to
\im{\ty : \subst{\sembrace{\sB}}{\txtwo}{\tx}}.
We cannot show the two types are equal without substituting all the (elided)
bindings as described earlier, but that substitution happens after it is needed.
The problem is that the dependent typing rule for \tfonttext{let} only binds a
depended-upon expression in the type of the \emph{body} of the \tfonttext{let},
not in the types of \emph{bound expressions}.

The typing rule for \tfonttext{let} only allows dependency in the type of the
overall expression---that is, in the ``output'' of the typing judgment---but we
need some way to thread dependencies into the sub-derivations while checking
bound expressions.
For example, if we could express dependencies in the \emph{input}, something
like \im{... \txtwo = \sembrace{\setwo} \vdash {(\tappe{\txone}{\txtwo})} :
  \subst{\sembrace{\sB}}{\sembrace{\setwo}}{\tx}}, then we could complete the
derivation and prove type-preservation.
And this is exactly the intuition we formalize.

We formalize this intuition using \emph{definitions}~\cite{severi1994:dpts}
introduced in \fullref[]{chp:source}.
Recall that the typing rule for \tfonttext{let} with definitions is the following.
\begin{mathpar}
  \inferrule
  {\ttyjudg{\tlenv}{\te}{\tA}\\
   \ttyjudg{\tlenv,\tx=\te}{\tepr}{\tApr}}
  {\ttyjudg{\tlenv}{\tlete{\tx}{\te}{\tepr}}{\subst{\tApr}{\te}{\tx}}}
\end{mathpar}

The definition \im{\tx = \te} is introduced when type checking the body of the
\tfonttext{let}, and can be used to solve type equivalence in sub-derivations,
instead of only in the substitution \im{\subst{\tApr}{\te}{\tx}} in the "output"
of the typing rule.
While this is an extension to the type theory (so one may worry that the
\tech{ANF} translation applies to fewer dependently typed languages), it is a
standard extension that is admissible in any Pure Type System
(PTS)~\cite{severi1994:dpts}, and is a feature already found in dependently
typed languages such as Coq.

Using definitions, we can prove that, under the definitions \im{\txin{2_0} =
  \tNin{2_0}, ..., \txin{2_n} = \tNin{2_n},\txtwo = \tNtwo} produced from the ANF translation
\im{\sembrace{\setwo}} (written \im{\edefs{\setwo}}), the desired equivalence
\im{\txtwo \equiv \sembrace{\setwo}} holds.
We use \im{\ehole{\setwo}} to refer to the innermost computation produced by
the ANF translation, in this case \im{\ehole{\setwo} = \tNtwo}.
Then we can prove \im{\edefs{\setwo} \vdash \ehole{\setwo} \equiv \anfh{\setwo}}
(formally captured by \fullref[]{lem:anf:export-equal}).
Since we also have the definition, \im{\txtwo = \tNtwo}, we conclude that
\im{\edefs{\setwo} \vdash \txtwo \equiv \anfh{\setwo}}.

To formalize this type preservation argument, we need to step back and define
the ANF translation precisely.
In the source, looking at an expression such as \im{\sappe{\seone}{\setwo}}, we
do not know whether the expression is embedded in a larger context.
This matters in ANF, since we can no longer compose expressions by nesting, but
instead must compose expressions with \tfonttext{let}.
This is why the above example translation disassembled the translation of
\im{\seone} and \im{\setwo} into a new, larger, \tfonttext{let} expression.
To formalize the ANF translation, it helps to have a more compositional syntax
for translating an expression and composing it with an unknown context.

I develop a compositional translation by indexing the ANF translation by a target language
(non-first-class) \emph{continuation} \im{\tK} representing the rest of the computation in which a translated expression will be used.
A continuation \im{\tK} is a program with a hole (single linear variable)
\im{\hole}, and can be composed with a computation \im{\tK\hw{\tN}} to form a
program \im{\tM}.
In ANF, there are only two continuations: either \im{\hole} or
\im{\tlete{\tx}{\hole}{\tM}}.
Using continuations, we define ANF translation for functions and application as
follows.
\begin{displaymath}
  \begin{aligned}[t]
    \anf{\sfune{\sx}{\sA}{\se}}{\tK} &= \tK\hw{\tfune{\tx}{(\anf{\sA}{\hole})}{(\anf{\se}{\hole})}} \\
    \anf{\sappe{\seone}{\setwo}}{\tK} &= \anf{\seone}{\tlete{\txone}{\hole}{\anf{\setwo}{\tlete{\txtwo}{\hole}{\tK\hw{\tappe{\txone}{\txtwo}}}}}}
  \end{aligned}
\end{displaymath}
This allows us to focus on composing the primitive operations instead of
reassociating \tfonttext{let} bindings.

The key typing rule for continuations is the following.
\begin{mathpar}
  \inferrule*[right=\rulename{K-Bind}]
  {\styjudg{\tlenv}{\tN}{\tA} \\
   \styjudg{\tlenv,\ty=\tN}{\tM}{\tB}}
  {\styjudg{\tlenv}{\tlete{\ty}{\hole}{\tM}}{(\tN : \tA) \Rightarrow \tB}}
\end{mathpar}
The type \im{(\tN : \tA) \Rightarrow \tB} of continuations describes that the
continuation must be composed with the term \im{\tN} of type \im{\tA}, and the
result will be of type \im{\tB}.
This expresses syntactically the intuition that continuations must be used
linearly to avoid control effects, which are known to cause inconsistency with dependent
types~\cite{barthe2002,herbelin2005}.
Note that this type allows us to introduce the definition \im{\ty = \tN} via the
type, before we know how the continuation is used.\footnote{This is
  essentially a singleton type, but as mentioned in \fullref[]{chp:source}, I
  avoid explicit encoding with singleton types to focus on the intuition:
  tracking dependencies.}
I discuss this rule further in \fullref[]{sec:anf:target}.
The \fullref{lem:anf:ecca:cut} expresses that continuation typing is not an
extension to the target type theory, which is important to ensure ANF can be
applied in practice.

The key lemma to prove type preservation is the following.
\begin{lemma}
  If \im{\styjudg{\slenv}{\se}{\sA}} and \im{\ttyjudg{\tlenv,\edefs{\se}}{\tK}{(\ehole{\se}:\sembrace{\sA}) \Rightarrow \tB}}, then \im{\ttyjudg{\tlenv}{\anf{\se}{\tK}}{\tB}}.
\end{lemma}
\noindent This lemma captures the fact that each time we build a new \im{\tK} in the ANF translation, we must show it is
well-typed, and that is where we apply the reasoning about definitions.
Proving that \im{\tK} has the above type requires proving \im{\tlenv,\edefs{\se} \vdash
  \ehole{\se} : \anfh{\sA}}.
For our running example, this means proving
\im{\tlenv,\edefs{\sappe{\seone}{\setwo}} \vdash \tappe{\txone}{\txtwo} :
  \anfh{\subst{\sB}{\setwo}{\sx}}}.
Recall that \im{\tappe{\txone}{\txtwo} : \subst{\anfh{\sB}}{\txtwo}{\tx}}.
As we saw earlier, definitions allow us to prove that \im{\tlenv,\edefs{\sappe{\seone}{\setwo}}
  \vdash \txtwo \equiv \sembrace{\setwo}}; therefore, combined with compositionality
(\im{\anfh{\subst{\sB}{\setwo}{\sx}} \equiv
  \subst{\anfh{\sB}}{\anfh{\setwo}}{\tx}}, \fullref[]{lem:anf:subst}), the proof is complete!
The earlier failing derivation now succeeds:

\begin{mathpar}
  \inferrule
      {\inferrule
        {\fullref[]{lem:anf:subst} \\
          \fullref[]{lem:anf:export-equal}}
        {\edefs{\sappe{\seone}{\setwo}} \vdash {(\tappe{\txone}{\txtwo})} : \subst{\sembrace{\sB}}{\anfh{\setwo}}{\tx}}\quad
    \inferrule
        {\text{succeeds!}}
    {\ty:\subst{\sembrace{\sB}}{\anfh{\setwo}}{\tx} \vdash \tappe{\tf}{\ty} : \tC}}
  {\edefs{\sappe{\seone}{\setwo}} \vdash \tlete{\ty}{(\tappe{\txone}{\txtwo})}{\tappe{\tf}{\ty}} : \subst{\tC}{(\tappe{\txone}{\txtwo})}{\ty}}
\end{mathpar}
