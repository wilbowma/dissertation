\newcommand{\FigCCCCSyntax}[1][t]{
  \begin{figure}[#1]
    \begin{bnfgrammar}
        \bnflabel{Universes} & \tU & \bnfdef & \tpropty \bnfalt \ttypety{i}
        \bnfnewline
        \bnflabel{Expressions} & \te,\tA,\tB & \bnfdef &
                                                         \tx
                                                         \bnfalt \tU
                                                         \bnfalt \tcodety{\txpr:\tApr,\tx:\tA}{\tB}
                                                         \bnfalt \tnfune{(\txpr:\tApr,\tx:\tA)}{\te}
                                                         \\ && \bnfalt &
                                                         \tpity{\tx}{\tA}{\tB}
                                                         \bnfalt \tcloe{\te}{\tepr}
                                                         \bnfalt \tappe{\te}{\tepr}
                                                         \bnfalt \tsigmaty{\tx}{\tA}{\tB}
                                                         \\ && \bnfalt &
                                                         \tdpaire{\teone}{\tetwo}{\tsigmaty{\tx}{\tA}{\tB}}
                                                         \bnfalt \tfste{\te}
                                                         \bnfalt \tsnde{\te}
                                                         \bnfalt \tboolty
                                                         \bnfalt \ttruee
                                                         \\ && \bnfalt & \tfalsee
                                                         \bnfalt \tife{\te}{\teone}{\tetwo}

                                                         \bnfalt \tlete{\tx}{\te}{\te}
                                                         \bnfalt \tunitty
                                                         \bnfalt \tunite
    \end{bnfgrammar}
    \caption{\tlang Syntax}
    \label{fig:abs-cc:cc-cc:syntax}
  \end{figure}
}

\newcommand{\FigCCCCSugar}[1][t]{
  \begin{figure}[#1]
    \begin{displaymath}
      \begin{array}{rcl}
        \tpaire{\teone}{\tetwo} & = & \tdpaire{\teone}{\tetwo}{\tsigmaty{\tx}{\tA}{\tB}}
        \\ && \text{where \im{\teone:\tA} and \im{\tetwo : \tB} are inferred from context} \\
        \tnpaire{\tei...} & = & \tpaire{\tein{0}}{\tpaire{\tein{1}}{...\tpaire{\tein{n}}{\tunite}}} \\
        \tnsigmaty{(\txi:\tAi...)} & = & \tsigmaty{\txin{0}}{\tAin{0}}{\tsigmaty{\txin{1}}{\tAin{1}}{...\tsigmaty{\txin{n}}{\tAin{n}}{\tunitty}}} \\
        \tlete{\tnpaire{\txi...}}{\te}{\tepr} & = & \begin{stackTL}\tlete{\txin{0}}{\tfste{\te}}{\\\quad\tlete{\txin{1}}{\tfste{\tsnde{\te}}}{...\\\qquad\tlete{\txin{n}}{\tfste{\tsnde{...\tsnde{\te}}}}{\tepr}}}\end{stackTL} \\
      \end{array}
    \end{displaymath}
    \caption{\tlang Syntactic Sugar}
    \label{fig:abs-cc:cc-cc:sugar}
  \end{figure}
}

\newcommand{\FigCCCCRedFull}[1][t]{
  \begin{figure}[#1]
    \judgshape{\tstepjudg[\step]{\tlenv}{\te}{\tepr}}
    \begin{reductionrules}
      \tlenv \vdash\tappe{\tcloe{\tnfune{\txpr:\tApr,\tx:\tA}{\teone}}{\tepr}}{\te}
      & \step_{\beta} & \subst{\subst{\teone}{\tepr}{\txpr}}{\te}{\tx}
      \stepnewline

      \tlenv \vdash \tfste{\tpaire{\teone}{\tetwo}} & \step_{\pi_{1}} & \teone
      \stepnewline

      \tlenv \vdash \tsnde{\tpaire{\teone}{\tetwo}} & \step_{\pi_{2}} & \tetwo
      \stepnewline

      \tlenv \vdash \tife{\ttruee}{\teone}{\tetwo} & \step_{\iota_{1}} & \teone
      \stepnewline

      \tlenv \vdash \tife{\tfalsee}{\teone}{\tetwo} & \step_{\iota_{2}} & \tetwo
      \stepnewline

      \tlenv \vdash \tx & \step_{\delta} & \te & \where{\tx = \te \in \tlenv}
      \stepnewline

      \tlenv \vdash \tlete{\tx}{\te}{\teone} & \step_{\zeta} & \subst{\teone}{\te}{\tx}
    \end{reductionrules}
    \caption{\tlang Reduction}
    \label{fig:abs-cc:cc-cc:red-full}
  \end{figure}
}

\newcommand{\FigCCCCEval}[1][t]{
  \begin{figure}[#1]
    \judgshape{\teval{\te} = \tv}
    \begin{displaymath}
      \begin{array}{rcll}
        \teval{\te} & = & \tv & \text{if \im{\twf{}{\te}} and \im{\te \stepstar \tv}}
      \end{array}
    \end{displaymath}
    \caption{\tlang Evaluation}
    \label{fig:abs-cc:cc-cc:eval}
  \end{figure}
}

\newcommand{\FigCCCCRed}[1][t]{
  \begin{figure}[#1]
    \judgshape{\tstepjudg[\step]{\tlenv}{\te}{\tepr}}
    \begin{reductionrules}
      & \vdots &
      \stepnewline

      \tlenv \vdash\tappe{\tcloe{\tnfune{\txpr:\tApr,\tx:\tA}{\teone}}{\tepr}}{\te}
      & \step_{\beta} & \subst{\subst{\teone}{\tepr}{\txpr}}{\te}{\tx}
      \stepnewline
    \end{reductionrules}
    \caption{\tlang Reduction (excerpts)}
    \label{fig:abs-cc:cc-cc:red}
  \end{figure}
}

\newcommand{\FigCCCCConv}[1][t]{
  \begin{figure}[#1]
    \begin{mathpar}
      \inferrule*[right=\defrule{Red-Refl}]
      {~}
      {\tstepjudg[\stepstar]{\tlenv}{\te}{\te}}

      \inferrule*[right=\defrule{Red-Trans}]
      {\tstepjudg{\tlenv}{\te}{\teone} \\
       \tstepjudg[\stepstar]{\tlenv}{\teone}{\tepr}}
      {\tstepjudg[\stepstar]{\tlenv}{\te}{\tepr}}

      \inferrule*[right=\defrule{Red-Cong-T-Code}]
      {\tstepjudg[\stepstar]{\tlenv}{\tAone}{\tAonepr} \\
       \tstepjudg[\stepstar]{\tlenv,\tn:\tAonepr}{\tAtwo}{\tAtwopr} \\
       \tstepjudg[\stepstar]{\tlenv,\tn:\tAonepr,\tx:\tAtwopr}{\tB}{\tBpr}}
      {\tstepjudg[\stepstar]{\tlenv}{\tcodety{\tn:\tAone,\tx:\tAtwo}{\tB}}{\tcodety{\tn:\tAonepr,\tx:\tAtwopr}{\tBpr}}}

      \inferrule*[right=\defrule{Red-Cong-Code}]
      {\tstepjudg[\stepstar]{\tlenv}{\tAone}{\tAonepr} \\
       \tstepjudg[\stepstar]{\tlenv,\tn:\tAonepr}{\tAtwo}{\tAtwopr} \\
       \tstepjudg[\stepstar]{\tlenv,\tn:\tAonepr,\tx:\tAtwopr}{\te}{\tepr}}
      {\tstepjudg[\stepstar]{\tlenv}{\tnfune{(\tn:\tAone,\tx:\tAtwo)}{\te}}{\tnfune{(\tn:\tAonepr,\tx:\tAtwopr)}{\tepr}}}

      \inferrule*[right=\defrule{Red-Cong-Pi}]
      {\tstepjudg[\stepstar]{\tlenv}{\tA}{\tApr} \\
       \tstepjudg[\stepstar]{\tlenv,\tx:\tApr}{\te}{\tepr}}
      {\tstepjudg[\stepstar]{\tlenv}{\tpity{\tx}{\tA}{\te}}{\tpity{\tx}{\tApr}{\tepr}}}

      \inferrule*[right=\defrule{Red-Cong-Clo}]
      {\tstepjudg[\stepstar]{\tlenv}{\teone}{\teonepr} \\
       \tstepjudg[\stepstar]{\tlenv}{\tetwo}{\tetwopr}}
      {\tstepjudg[\stepstar]{\tlenv}{\tcloe{\teone}{\tetwo}}{\tcloe{\teonepr}{\tetwopr}}}

      \inferrule*[right=\defrule{Red-Cong-App}]
      {\tstepjudg[\stepstar]{\tlenv}{\teone}{\teonepr} \\
       \tstepjudg[\stepstar]{\tlenv}{\tetwo}{\tetwopr}}
      {\tstepjudg[\stepstar]{\tlenv}{\tappe{\teone}{\tetwo}}{\tappe{\teonepr}{\tetwopr}}}

      \inferrule*[right=\defrule{Red-Cong-Sig}]
      {\tstepjudg[\stepstar]{\tlenv}{\tA}{\tApr} \\
       \tstepjudg[\stepstar]{\tlenv,\tx:\tApr}{\te}{\tepr}}
      {\tstepjudg[\stepstar]{\tlenv}{\tsigmaty{\tx}{\tA}{\te}}{\tsigmaty{\tx}{\tApr}{\tepr}}}

      \inferrule*[right=\defrule{Red-Cong-Pair}]
      {\tstepjudg[\stepstar]{\tlenv}{\teone}{\teonepr} \\
       \tstepjudg[\stepstar]{\tlenv}{\tetwo}{\tetwopr} \\
       \tstepjudg[\stepstar]{\tlenv}{\tA}{\tApr}}
      {\tstepjudg[\stepstar]{\tlenv}{\tdpaire{\teone}{\tetwo}{\tA}}{\tdpaire{\teonepr}{\tetwopr}{\tApr}}}

      \inferrule*[right=\defrule{Red-Cong-Fst}]
      {\tstepjudg[\stepstar]{\tlenv}{\te}{\tepr}}
      {\tstepjudg[\stepstar]{\tlenv}{\tfste{\te}}{\tfste{\tepr}}}

      \inferrule*[right=\defrule{Red-Cong-Snd}]
      {\tstepjudg[\stepstar]{\tlenv}{\te}{\tepr}}
      {\tstepjudg[\stepstar]{\tlenv}{\tsnde{\te}}{\tsnde{\tepr}}}

      \inferrule*[right=\defrule{Red-Cong-If}]
      {\tstepjudg[\stepstar]{\tlenv}{\te}{\tepr} \\
       \tstepjudg[\stepstar]{\tlenv}{\teone}{\teonepr} \\
       \tstepjudg[\stepstar]{\tlenv}{\tetwo}{\tetwopr}}
      {\sstepjudg[\stepstar]{\tlenv}{\tife{\te}{\teone}{\tetwo}}{\tife{\tepr}{\teonepr}{\tetwopr}}}


      \inferrule*[right=\defrule{Red-Cong-Let}]
      {\tstepjudg[\stepstar]{\tlenv}{\teone}{\teonepr} \\
       \tstepjudg[\stepstar]{\tlenv,\tx = \tepr}{\tetwo}{\tetwopr}}
      {\tstepjudg[\stepstar]{\tlenv}{\tlete{\tx}{\teone}{\tetwo}}{\tlete{\tx}{\teonepr}{\tetwopr}}}
    \end{mathpar}
    \caption{\tlang Conversion}
    \label{fig:abs-cc:cc-cc:red-cong}
  \end{figure}
}

\newcommand{\FigCCCCEqv}[1][t]{
  \begin{figure}[#1]
    \begin{mathpar}
      \inferrule*[right=\defrule*{eqv}{\im{\equiv}}]
        {\tstepjudg[\stepstar]{\tlenv}{\teone}{\te} \\
         \tstepjudg[\stepstar]{\tlenv}{\tetwo}{\te}}
        {\tequivjudg{\tlenv}{\teone}{\tetwo}}

      \inferrule*[right=\defrule*{eqv-eta1}{\im{\equiv}-Clo\im{_1}}]
      {\tstepjudg[\stepstar]{\tlenv}{\teone}{\tcloe{\tnfune{(\txpr:\tApr,\tx:\tA)}{\teonepr}}{\tepr}}
        \\
        \tstepjudg[\stepstar]{\tlenv}{\tetwo}{\tetwopr} \\
        \tequivjudg{\tlenv,\tx:\tA}{\subst{\teone}{\tepr}{\txpr}}{\tappe{\tetwopr}{\tx}}}
        {\tequivjudg{\tlenv}{\teone}{\tetwo}}

      \inferrule*[right=\defrule*{eqv-eta2}{\im{\equiv}-Clo\im{_2}}]
      {\tstepjudg[\stepstar]{\tlenv}{\tetwo}{\tcloe{\tnfune{(\txpr:\tApr,\tx:\tA)}{\tetwopr}}{\tepr}}
        \\
        \tstepjudg[\stepstar]{\tlenv}{\teone}{\teonepr} \\
        \tequivjudg{\tlenv,\tx:\tA}{\tappe{\teonepr}{\tx}}{\subst{\tetwopr}{\tepr}{\txpr}}}
      {\tequivjudg{\tlenv}{\teone}{\tetwo}}
    \end{mathpar}
    \caption{\tlang Equivalence}
    \label{fig:abs-cc:cc-cc:eqv}
  \end{figure}
}

\newcommand{\FigCCCCSub}[1][t]{
  \begin{figure}[#1]
    \judgshape{\tsubtyjudg{\tlenv}{\tA}{\tB}}
    \begin{mathpar}
      \inferrule*[right=\defrule*{sub-equiv}{\im{\subtypesym}-\im{\equiv}}]
      {\tequivjudg{\tlenv}{\tA}{\tB}}
      {\tsubtyjudg{\tlenv}{\tA}{\tB}}

      \inferrule*[right=\defrule*{sub-trans}{\im{\subtypesym}-Trans}]
      {\tsubtyjudg{\tlenv}{\tA}{\tApr} \\
       \tsubtyjudg{\tlenv}{\tApr}{\tB}}
      {\tsubtyjudg{\tlenv}{\tA}{\tB}}

      \inferrule*[right=\defrule*{sub-prop}{\im{\subtypesym}-Prop}]
      {~}
      {\tsubtyjudg{\tlenv}{\tpropty}{\ttypety{0}}}

      \inferrule*[right=\defrule*{sub-cum}{\im{\subtypesym}-Cum}]
      {~}
      {\tsubtyjudg{\tlenv}{\ttypety{i}}{\ttypety{i+1}}}

      \inferrule*[right=\defrule*{sub-code}{\im{\subtypesym}-Code}]
      {\tequivjudg{\tlenv}{\tAone}{\tAtwo} \\
       \tequivjudg{\tlenv,\tnone:\tAone}{\tAonepr}{\tAtwopr} \\
       \tsubtyjudg{\tlenv,\tnone:\tAone,\txone:\tAonepr}{\tBone}{\subst{\subst{\tBtwo}{\tnone}{\tntwo}}{\txone}{\txtwo}}}
      {\tsubtyjudg{\tlenv}{\tcodety{\tnone:\tAone,\txone:\tAonepr}{\tBone}}{\tcodety{\tntwo:\tAtwo,\txtwo:\tAtwopr}{\tBtwo}}}

      \inferrule*[right=\defrule*{sub-pi}{\im{\subtypesym}-Pi}]
      {\tequivjudg{\tlenv}{\tAone}{\tAtwo} \\
       \tsubtyjudg{\tlenv,\txone:\tAone}{\tBone}{\subst{\tBtwo}{\txone}{\txtwo}}}
      {\tsubtyjudg{\tlenv}{\tpity{\txone}{\tAone}{\tBone}}{\tpity{\txtwo}{\tAtwo}{\tBtwo}}}

      \inferrule*[right=\defrule*{sub-sig}{\im{\subtypesym}-Sig}]
      {\tsubtyjudg{\tlenv}{\tAone}{\tAtwo} \\
       \tsubtyjudg{\tlenv,\txone:\tAtwo}{\tBone}{\subst{\tBtwo}{\txone}{\txtwo}}}
      {\tsubtyjudg{\tlenv}{\tsigmaty{\txone}{\tAone}{\tBone}}{\tsigmaty{\txtwo}{\tAtwo}{\tBtwo}}}
    \end{mathpar}
    \caption{\tlang Subtyping}
    \label{fig:abs-cc:cc-cc:sub}
  \end{figure}
}

\newcommand{\UnitTypingRules}[1][\defrule]{
  \inferrule*[right=#1{T-Unit}]
  {\twf{\tlenv}}
  {\ttyjudg{\tlenv}{\tunitty}{\tpropty}}

  \inferrule*[right=#1{Unit}]
  {\twf{\tlenv}}
  {\ttyjudg{\tlenv}{\tunite}{\tunitty}}
  \makeatother
}

\newcommand{\ClosureTypingRules}[1][\defrule]{
  \inferrule*[right=#1{T-Code-Prop}]
  {\ttyjudg{\tlenv,\txpr:\tApr,\tx:\tA}{\tB}{\tpropty}}
  {\ttyjudg{\tlenv}{\tcodety{\txpr:\tApr,\tx:\tA}{\tB}}{\tpropty}}

  \inferrule*[right=#1{T-Code-Type}]
  {\ttyjudg{\tlenv,\txpr:\tApr,\tx:\tA}{\tB}{\ttypety{i}}}
  {\ttyjudg{\tlenv}{\tcodety{\tx:\tA,\txpr:\tApr}{\tB}}{\ttypety{i}}}

  \inferrule*[right=#1{Code}]
  {\ttyjudg{\cdot,\txpr:\tApr,\tx:\tA}{\te}{\tB}}
  {\ttyjudg{\tlenv}{\tnfune{(\txpr:\tApr,\tx:\tA)}{\te}}{\tcodety{\txpr:\tApr,\tx:\tA}{\tB}}}

  \inferrule*[right=#1{Clo}]
  {\ttyjudg{\tlenv}{\te}{\tcodety{\txpr:\tApr,\tx:\tA}{\tB}} \\
   \ttyjudg{\tlenv}{\tepr}{\tApr}}
  {\ttyjudg{\tlenv}{\tcloe{\te}{\tepr}}{\tpity{\tx}{\subst{\tA}{\tepr}{\txpr}}{\subst{\tB}{\tepr}{\txpr}}}}
}

\newcommand{\FigCCCCTypingNew}[1][t]{
  \begin{figure}[#1]
    \begin{mathpar}
      \cdots

      \ClosureTypingRules

      \UnitTypingRules
    \end{mathpar}
    \caption{\tlang Typing (excerpts)}
    \label{fig:abs-cc:cc-cc:type}
  \end{figure}
}

\newcommand{\FigCCCCTypingFullTwo}[1][t]{
  \begin{figure}[#1]
    \begin{mathpar}
      \ClosureTypingRules[\rulename]

      \UnitTypingRules[\rulename]
    \end{mathpar}
    \caption{\tlang Typing (2/2)}
    \label{fig:abs-cc:cc-cc:type2}
  \end{figure}
}

\newcommand{\FigCCCCTypingFullOne}[1][t]{
  \begin{figure}[#1]
    \judgshape{\ttyjudg{\tlenv}{\te}{\tt}}
    \begin{mathpar}
      \inferrule*[right=\defrule{Var}]
      {\tx : \tA \in \tlenv \\
       \twf{\tlenv}}
      {\ttyjudg{\tlenv}{\tx}{\tA}}

      \inferrule*[right=\defrule{Prop}]
      {\twf{\tlenv}}
      {\ttyjudg{\tlenv}{\tpropty}{\ttypety{1}}}

      \inferrule*[right=\defrule{Type}]
      {\twf{\tlenv}}
      {\ttyjudg{\tlenv}{\ttypety{i}}{\ttypety{i+1}}}

      \inferrule*[right=\defrule{Pi-Prop}]
      {\ttyjudg{\tlenv}{\tA}{\tU} \\
       \ttyjudg{\tlenv,\tx:\tA}{\tB}{\tpropty}}
      {\ttyjudg{\tlenv}{\tpity{\tx}{\tA}{\tB}}{\tpropty}}

      \inferrule*[right=\defrule{Pi-Type}]
      {\ttyjudg{\tlenv}{\tA}{\ttypety{i}} \\
       \ttyjudg{\tlenv,\tx:\tA}{\tB}{\ttypety{i}}}
      {\ttyjudg{\tlenv}{\tpity{\tx}{\tA}{\tB}}{\ttypety{i}}}

      \inferrule*[right=\defrule{App}]
      {\ttyjudg{\tlenv}{\te}{\tpity{\tx}{\tApr}{\tB}} \\
        \ttyjudg{\tlenv}{\tepr}{\tApr}}
      {\ttyjudg{\tlenv}{\tappe{\te}{\tepr}}{\subst{\tB}{\tepr}{\tx}}}

      \inferrule*[right=\defrule{Sig}]
      {\ttyjudg{\tlenv}{\tA}{\ttypety{i}} \\
        \ttyjudg{\tlenv,\tx:\tA}{\tB}{\ttypety{i}}}
      {\ttyjudg{\tlenv}{\tsigmaty{\tx}{\tA}{\tB}}{\ttypety{i}}}

      \inferrule*[right=\defrule{Pair}]
      {\ttyjudg{\tlenv}{\teone}{\tA} \\
       \ttyjudg{\tlenv}{\tetwo}{\subst{\tB}{\teone}{\tx}}}
      {\ttyjudg{\tlenv}{\tdpaire{\teone}{\tetwo}{\tsigmaty{\tx}{\tA}{\tB}}}{\tsigmaty{\tx}{\tA}{\tB}}}

      \inferrule*[right=\defrule{Fst}]
      {\ttyjudg{\tlenv}{\te}{\tsigmaty{\tx}{\tA}{\tB}}}
      {\ttyjudg{\tlenv}{\tfste{\te}}{\tA}}

      \inferrule*[right=\defrule{Snd}]
      {\ttyjudg{\tlenv}{\te}{\tsigmaty{\tx}{\tA}{\tB}}}
      {\ttyjudg{\tlenv}{\tsnde{\te}}{\subst{\tB}{\tfste{\te}}{\tx}}}

      \inferrule*[right=\defrule{Bool}]
      {\twf{\tlenv}}
      {\ttyjudg{\tlenv}{\tboolty}{\tpropty}}

      \inferrule*[right=\defrule{True}]
      {\twf{\tlenv}}
      {\ttyjudg{\tlenv}{\ttruee}{\tboolty}}

      \inferrule*[right=\defrule{False}]
      {\twf{\tlenv}}
      {\ttyjudg{\tlenv}{\tfalsee}{\tboolty}}

      \inferrule*[right=\defrule{If}]
      {\ttyjudg{\tlenv,\ty:\tboolty}{\tB}{\tU} \\
       \ttyjudg{\tlenv}{\teone}{\subst{\tB}{\ttruee}{\ty}} \\
       \ttyjudg{\tlenv}{\tetwo}{\subst{\tB}{\tfalsee}{\ty}}}
      {\ttyjudg{\tlenv}{\tife{\te}{\teone}{\tetwo}}{\subst{\tB}{\te}{\ty}}}

      \inferrule*[right=\defrule{Let}]
      {\ttyjudg{\tlenv}{\te}{\tA} \\
        \ttyjudg{\tlenv,\tx:\tA,\tx=\te}{\tepr}{\tB}}
      {\ttyjudg{\tlenv}{\tlete{\tx}{\te}{\tepr}}{\subst{\tB}{\te}{\tx}}}

      \inferrule*[right=\defrule{Conv}]
      {\ttyjudg{\tlenv}{\te}{\tA} \\
       \ttyjudg{\tlenv}{\tB}{\tU} \\
       \tsubtyjudg{\tlenv}{\tA}{\tB}}
      {\ttyjudg{\tlenv}{\te}{\tB}}
    \end{mathpar}
    \caption{\tlang Typing (1/2)}
    \label{fig:abs-cc:cc-cc:type1}
  \end{figure}
}

\newcommand{\FigCCCCWF}[1][t]{
  \begin{figure}[#1]
    \judgshape{\twf{\tlenv}}
    \begin{mathpar}
      \inferrule*[right=\defrule{W-Empty}]
      {~}
      {\twf{\cdot}}

      \inferrule*[right=\defrule{W-Assum}]
      {\twf{\tlenv} \\
       \ttyjudg{\tlenv}{\tA}{\tU}}
      {\twf{\tlenv,\tx:\tA}}

      \inferrule*[right=\defrule{W-Def}]
      {\twf{\tlenv} \\
       \ttyjudg{\tlenv}{\te}{\tA}}
      {\twf{\tlenv,\tx = \te}}
    \end{mathpar}
    \caption{\tlang Well-formed Environments}
    \label{fig:abs-cc:cc-cc:wf}
  \end{figure}
}

\newcommand{\FigCCCCModelPartOne}[1][t]{
  \begin{figure}[#1]
    \judgshape{\mjudg{\tlenv}{\te}{\tA}{\se}}
    \begin{mathpar}
      \inferrule*[right=\defrule{M-T-Unit}]
      {~}
      {\mjudg{\tlenv}{\tunitty}{\tpropty}{\spity{\salpha}{\spropty}{\spity{\sx}{\salpha}{\salpha}}}}

      \inferrule*[right=\defrule{M-Unit}]
      {~}
      {\mjudg{\tlenv}{\tunite}{\tunitty}{\sfune{\salpha}{\spropty}{\sfune{\sx}{\salpha}{\sx}}}}

      \inferrule*[right=\defrule{M-T-Code-Prop}]
      {\mjudg{\tlenv}{\tApr}{\tUpr}{\sApr} \\
       \mjudg{\tlenv,\txpr:\tApr}{\tA}{\tU}{\sA} \\
      \mjudg{\tlenv,\txpr:\tApr,\tx:\tA}{\tB}{\tpropty}{\sB}}
      {\mjudg{\tlenv}{\tcodety{\txpr:\tApr,\tx:\tA}{\tB}}{\tpropty}{\spity{\sxpr}{\sApr}{\spity{\sx}{\sA}{\sB}}}}

      \inferrule*[right=\defrule{M-T-Code-Type}]
      {\mjudg{\tlenv}{\tApr}{\tUpr}{\sApr} \\
       \mjudg{\tlenv,\txpr:\tApr}{\tA}{\tU}{\sA} \\
      \mjudg{\tlenv,\txpr:\tApr,\tx:\tA}{\tB}{\ttypety{i}}{\sB}}
      {\mjudg{\tlenv}{\tcodety{\txpr:\tApr,\tx:\tA}{\tB}}{\ttypety{i}}{\spity{\sxpr}{\sApr}{\spity{\sx}{\sA}{\sB}}}}

      \inferrule*[right=\defrule{M-Code}]
      {\mjudg{\tlenv}{\tApr}{\tUpr}{\sApr} \\
       \mjudg{\tlenv,\txpr:\tApr}{\tA}{\tU}{\sA} \\
       \mjudg{\tlenv,\txpr:\tApr,\tx:\tA}{\tB}{\tU}{\sB} \\
       \mjudg{\tlenv,\txpr:\tApr,\tx:\tA}{\te}{\tB}{\se}}
      {\mjudg{\tlenv}{\tnfune{(\txpr:\tApr,\tx:\tA)}{\te}}{\tcodety{\txpr:\tApr,\tx:\tA}{\tB}}
        {\sfune{\sxpr}{\sApr}{\sfune{\sx}{\sA}{\se}}}}

      \inferrule*[right=\defrule{M-Clo}]
      {\mjudg{\tlenv}{\te}{\tcodety{\txpr:\tApr,\tx:\tA}{\tB}}{\se} \\
       \mjudg{\tlenv}{\tepr}{\tApr}{\sepr}}
      {\mjudg{\tlenv}{\tcloe{\te}{\tepr}}{\tpity{\tx}{\subst{\tA}{\tepr}{\tx}}{\subst{\tB}{\tepr}{\tx}}}{\sappe{\se}{\sepr}}}

      \inferrule*[right=\defrule{M-App}]
      {\mjudg{\tlenv}{\te}{\tpity{\tx}{\tA}{\tB}}{\se} \\
       \mjudg{\tlenv}{\tepr}{\tA}{\sepr}}
      {\mjudg{\tlenv}{\tappe{\te}{\tepr}}{\subst{\tB}{\tepr}{\tx}}{\sappe{\se}{\sepr}}}
    \end{mathpar}
    \caption{Model for \tlang in \slang (Part 1/2)}
    \label{fig:abs-cc:model1}
  \end{figure}
}
\newcommand{\FigCCCCModelPartTwo}[1][t]{
  \begin{figure}[#1]
    \judgshape{\mjudg{\tlenv}{\te}{\tA}{\se}}
    \begin{mathpar}
      \inferrule*[right=\defrule{M-Prop}]
      {~}
      {\mjudg{\tlenv}{\tpropty}{\ttypety{1}}{\spropty}}

      \inferrule*[right=\defrule{M-Type}]
      {~}
      {\mjudg{\tlenv}{\ttypety{i}}{\ttypety{i+1}}{\stypety{i}}}

      \inferrule*[right=\defrule{M-Var}]
      {~}
      {\mjudg{\tlenv}{\tx}{\tA}{\sx}}

      \inferrule*[right=\defrule{M-Let}]
      {\mjudg{\tlenv}{\te}{\tA}{\se} \\
       \mjudg{\tlenv,\tx:\tA,\tx=\te}{\tepr}{\tB}{\sepr}}
      {\mjudg{\tlenv}{\tlete{\tx}{\te}{\tepr}}{\subst{\tB}{\te}{\tx}}{\slete{\sx}{\se}{\sepr}}}

      \inferrule*[right=\defrule{M-Pi-Prop}]
      {\mjudg{\tlenv}{\tA}{\tU}{\sA} \\
       \mjudg{\tlenv,\tx:\tA}{\tB}{\tpropty}{\sB}}
      {\mjudg{\tlenv}{\tpity{\tx}{\tA}{\tB}}{\tpropty}{\spity{\sx}{\sA}{\sB}}}

      \inferrule*[right=\defrule{M-Pi-Type}]
      {\mjudg{\tlenv}{\tA}{\tU}{\sA} \\
       \mjudg{\tlenv,\tx:\tA}{\tB}{\ttypety{i}}{\sB}}
      {\mjudg{\tlenv}{\tpity{\tx}{\tA}{\tB}}{\ttypety{i}}{\spity{\sx}{\sA}{\sB}}}

      \inferrule*[right=\defrule{M-Bool}]
      {~}
      {\mjudg{\tlenv}{\tboolty}{\tpropty}{\sboolty}}

      \inferrule*[right=\defrule{M-True}]
      {~}
      {\mjudg{\tlenv}{\ttruee}{\tboolty}{\struee}}

      \inferrule*[right=\defrule{M-False}]
      {~}
      {\mjudg{\tlenv}{\tfalsee}{\tboolty}{\sfalsee}}

      \inferrule*[right=\defrule{M-If}]
      {\mjudg{\tlenv}{\te}{\tboolty}{\se} \\
       \mjudg{\tlenv}{\teone}{\subst{\tB}{\ttruee}{\ty}}{\seone} \\
       \mjudg{\tlenv}{\tetwo}{\subst{\tB}{\tfalsee}{\ty}}{\setwo}}
      {\mjudg{\tlenv}{\tife{\te}{\teone}{\tetwo}}{\subst{\tB}{\te}{\ty}}{\sife{\se}{\seone}{\setwo}}}

      \inferrule*[right=\defrule{M-Sig}]
      {\mjudg{\tlenv}{\tA}{\ttypety{i}}{\sA} \\
       \mjudg{\tlenv,\tx:\tA}{\tB}{\ttypety{i}}{\sB}}
      {\mjudg{\tlenv}{\tsigmaty{\tx}{\tA}{\tB}}{\ttypety{i}}{\ssigmaty{\sx}{\sA}{\sB}}}

      \inferrule*[right=\defrule{M-Pair}]
      {\mjudg{\tlenv}{\teone}{\tA}{\seone} \\
       \mjudg{\tlenv}{\tetwo}{\subst{\tB}{\teone}{\tx}}{\setwo} \\
       \mjudg{\tlenv}{\tA}{\ttypety{i}}{\sA} \\
       \mjudg{\tlenv,\tx:\tA}{\tB}{\ttypety{i}}{\sB}}
      {\mjudg{\tlenv}{\tdpaire{\teone}{\tetwo}{\tsigmaty{\tx}{\tA}{\tB}}}{\tsigmaty{\tx}{\tA}{\tB}}{\sdpaire{\seone}{\setwo}{\ssigmaty{\sx}{\sA}{\sB}}}}

      \inferrule*[right=\defrule{M-Fst}]
      {\mjudg{\tlenv}{\te}{\tsigmaty{\tx}{\tA}{\tB}}{\se}}
      {\mjudg{\tlenv}{\tfste{\te}}{\tA}{\sfste{\se}}}

      \inferrule*[right=\defrule{M-Snd}]
      {\mjudg{\tlenv}{\te}{\tsigmaty{\tx}{\tA}{\tB}}{\se}}
      {\mjudg{\tlenv}{\tsnde{\te}}{\tA}{\ssnde{\se}}}

      \inferrule*[right=\defrule{M-Conv}]
      {\mjudg{\tlenv}{\te}{\tB}{\se}}
      {\mjudg{\tlenv}{\te}{\tA}{\se}}
    \end{mathpar}
    \caption{Model of \tlang in \slang (Part 2/2)}
    \label{fig:abs-cc:model2}
  \end{figure}
}

\newcommand{\FigCCCCModelShort}[1][t]{
  \begin{figure}
    \judgshape[\im{\where{\ttyjudg{\tlenv}{\te}{\tA}}}]{\absccmodel{\te} = \se}
    \begin{displaymath}
      \begin{array}{rcl}
        \absccmodel{\tunitty} & \defeq &
        \spity{\salpha}{\spropty}{\spity{\sx}{\salpha}{\salpha}}
        \\

        \absccmodel{\tunite} & \defeq &
        \sfune{\salpha}{\spropty}{\sfune{\sx}{\salpha}{\sx}}
        \\

        \absccmodel{\tcodety{\txpr:\tApr,\tx:\tA}{\tB}} & \defeq &
        \spity{\sxpr}{\absccmodel{\tApr}}{\spity{\sx}{\absccmodel{\tA}}{\absccmodel{\tB}}}
        \\

        \absccmodel{\tnfune{(\txpr:\tApr,\tx:\tA)}{\te}} & \defeq &
        \sfune{\sxpr}{\absccmodel{\tApr}}{\sfune{\sx}{\absccmodel{\tA}}{\absccmodel{\te}}}
        \\

        \absccmodel{\tcloe{\te}{\tepr}} & \defeq & \sappe{\absccmodel{\te}}{\absccmodel{\tepr}}
        \\

        \absccmodel{\tappe{\te}{\tepr}} & \defeq & \sappe{\absccmodel{\te}}{\absccmodel{\tepr}}
        \\
        \vdots
      \end{array}
    \end{displaymath}
    \caption{Model of \tlang in \slang (excerpts)}
    \label{fig:abs-cc:model-short}
  \end{figure}
}

\newcommand{\FigCCCCModelFull}[1][t]{
  \begin{figure}
    \judgshape[\im{\where{\ttyjudg{\tlenv}{\te}{\tA}}}]{\absccmodel{\te} = \se}
    \begin{displaymath}
      \begin{array}{rcl}
        \absccmodel{\tpropty} & \defeq & \spropty
        \\

        \absccmodel{\ttypety{i}} & \defeq & \stypety{i}
        \\

        \absccmodel{\tx} & \defeq & \sx
        \\

        \absccmodel{\tunitty} & \defeq &
        \spity{\salpha}{\spropty}{\spity{\sx}{\salpha}{\salpha}}
        \\

        \absccmodel{\tunite} & \defeq &
        \sfune{\salpha}{\spropty}{\sfune{\sx}{\salpha}{\sx}}
        \\

        \absccmodel{\tlete{\tx}{\te}{\tepr}} & \defeq & \slete{\sx}{\absccmodel{\te}}{\absccmodel{\tepr}}
        \\

        \absccmodel{\tpity{\tx}{\tA}{\tB}} & \defeq & \spity{\sx}{\absccmodel{\tA}}{\absccmodel{\tB}}
        \\

        \absccmodel{\tcloe{\te}{\tepr}} & \defeq & \sappe{\absccmodel{\te}}{\absccmodel{\tepr}}
        \\

        \absccmodel{\tcodety{\txpr:\tApr,\tx:\tA}{\tB}} & \defeq &
        \spity{\sxpr}{\absccmodel{\tApr}}{\spity{\sx}{\absccmodel{\tA}}{\absccmodel{\tB}}}
        \\

        \absccmodel{\tnfune{(\txpr:\tApr,\tx:\tA)}{\te}} & \defeq &
        \sfune{\sxpr}{\absccmodel{\tApr}}{\sfune{\sx}{\absccmodel{\tA}}{\absccmodel{\te}}}
        \\

        \absccmodel{\tappe{\te}{\tepr}} & \defeq & \sappe{\absccmodel{\te}}{\absccmodel{\tepr}}
        \\

        \absccmodel{\tboolty} & \defeq & \sboolty
        \\

        \absccmodel{\ttruee} & \defeq & \struee
        \\

        \absccmodel{\tfalsee} & \defeq & \sfalsee
        \\

        \absccmodel{\tife{\te}{\teone}{\tetwo}} & \defeq & \sife{\absccmodel{\te}}{\absccmodel{\teone}}{\absccmodel{\tetwo}}
        \\

        \absccmodel{\tsigmaty{\tx}{\tA}{\tB}} & \defeq & \ssigmaty{\sx}{\absccmodel{\tA}}{\absccmodel{\tB}}
        \\

        \absccmodel{\tdpaire{\teone}{\tetwo}{\tsigmaty{\tx}{\tA}{\tB}}} & \defeq &
        \sdpaire{\absccmodel{\teone}}{\absccmodel{\tetwo}}{\ssigmaty{\sx}{\absccmodel{\tA}}{\absccmodel{\tB}}}
        \\

        \absccmodel{\tfste{\te}} & \defeq & \sfste{\absccmodel{\te}}
        \\

        \absccmodel{\tsnde{\te}} & \defeq & \ssnde{\absccmodel{\te}}
      \end{array}
    \end{displaymath}
    \caption{Model of \tlang in \slang}
    \label{fig:abs-cc:model-full}
  \end{figure}
}

\newcommand{\FigCCCCAHetero}[1][t]{
  \begin{figure}[#1]
    \judgshape{\tK\hhw{\tM} = \tM}
    \begin{displaymath}
      \begin{array}{rcl}
        \tK\hhw{\tN} &\defeq& \tK\hw{\tN} \\
        \tK\hhw{\tlete{\tx}{\tNpr}{\tM}} &\defeq& \tlete{\tx}{\tNpr}{\tK\hhw{\tM}} \\
      \end{array}
    \end{displaymath}
    \judgshape{\tK\hhw{\tK} = \tK}
    \begin{displaymath}
      \begin{array}{rcl}
        \tK\hhw{\hole} &\defeq& \tK \\
        \tK\hhw{\tlete{\tx}{\hole}{\tM}} &\defeq& \tlete{\tx}{\hole}{\tK\hhw{\tM}} \\
      \end{array}
    \end{displaymath}
    \judgshape{\hsubst{\tM}{\tMpr}{\tx} = \tM}
    \begin{displaymath}
      \begin{array}{rcl}
        \hsubst{\tM}{\tMpr}{\tx} &\defeq& (\tlete{\tx}{\hole}{\tM})\hhw{\tMpr}
      \end{array}
    \end{displaymath}
    \caption{\tlang Composition of Configurations}
    \label{fig:abs-cc:hsubst}
  \end{figure}
}


\newcommand{\FigCCANFRed}[1][t]{
  \begin{figure}[#1]
    \judgshape{\tM \mapsto \tMpr}
    \begin{reductionrules}
      \tK\hw{\tappe{\tcloe{(\tnfune{(\tn:\tMpr_{\tcolor{A}},\tx:\tMin{A})}{\tM})}{\tVpr}}{\tV}} & \mapsto_{\beta} & \tK\hhw{\subst{\subst{\tM}{\tVpr}{\tn}}{\tV}{\tx}}
      \stepnewline

      \tK\hw{\tfste{\tpaire{\tVone}{\tVtwo}}} & \mapsto_{\pi_{1}} & \tK\hw{\tVone}
      \stepnewline

      \tK\hw{\tsnde{\tpaire{\tVone}{\tVtwo}}} & \mapsto_{\pi_{2}} & \tK\hw{\tVtwo}
      \stepnewline

      \tlete{\tx}{\tV}{\tM} & \mapsto_{\zeta} & \subst{\tM}{\tV}{\tx}
    \end{reductionrules}
    \judgshape{\tstepjudg[\mapsto^*]{}{\tM}{\tMpr}}
    \begin{mathpar}
      \inferrule*[right=\defrule{RedA-Refl}]
      {~}
      {\tstepjudg[\mapsto^*]{}{\tM}{\tM}}

      \inferrule*[right=\defrule{RedA-Trans}]
      {\tM \mapsto \tMone \\
       \tstepjudg[\mapsto^*]{}{\tMone}{\tMpr}}
      {\tstepjudg[\mapsto^*]{}{\tM}{\tMpr}}
    \end{mathpar}
    \judgshape{\teval{\tM} = \tV}
    \begin{displaymath}
      \begin{array}{rcll}
        \teval{\tM} & = & \tV & \text{if \im{\wf{}{\tM}} and \im{\tM \mapsto^* \tV} and \im{\tV \not\mapsto \tVpr}}
      \end{array}
    \end{displaymath}
    \caption{\tlang ANF Machine Evaluation}
    \label{fig:abs-cc:cc-cc:anf-eval}
  \end{figure}
}

\section{Closure-Converted Intermediate Language}
\label{sec:abs-cc:target}

\FigCCCCSyntax
The target language \tlang is based on \slang, but first-class functions are
replaced by closed \tech{code} and \tech{closures}.
I add a primitive unit type \im{\tunitty} to support encoding
\tech{environments}.
In \fullref[]{fig:abs-cc:cc-cc:syntax}, I extend the syntax of expressions with a
unit \tech{expression} \im{\tnpaire{}} and its \tech{type} \im{\tunitty}, closed
\tech{code} expressions \im{\tnfune{\tn:\tApr,\tx:\tA}{\te}} and dependent
\tech{code} \tech{types} \im{\tcodety{\tn:\tApr,\tx:\tA}{\tB}}, and \tech{closure}
expressions \im{\tcloe{\te}{\tepr}} and dependent \tech{closure} \tech{types}
\im{\tpity{\tx}{\tA}{\tB}}.
The closed code expressions will eventually be separated from closures, lifted
to the top-level and heap allocated, as described in \fullref[]{chp:type-pres}.
The syntax of application \im{\tappe{\te}{\tepr}} is unchanged, but it now
applies \tech{closures} instead of functions.

I define additional syntactic sugar for sequences of \tech{expressions}, to
support writing \tech{environments} whose length is arbitrary.
A sequence of \tech{expressions} \im{\tei\dots} represents a sequence of length
\im{\len{i}} of \tech{expressions} \im{\tein{i_0},\dots,\tein{i_n}}.
I extend the notation to patterns such as \im{\txi:\tAi\dots}, which implies two
sequences \im{\txin{i_0},\dots,\txin{i_n}} and \im{\tAin{0},\dots,\tAin{i_n}}
each of length \im{\len{i}}.
I define \tech{environments} as dependent n-tuples, written
\im{\tdnpaire{\tei\dots}{\tnsigmaty{(\txi:\tAi\dots)}}}.
I encode dependent n-tuples using \tech{dependent
  pairs}---\im{\tpaire{\tein{0}}{\tpaire{\dots}{\tpaire{\tein{i}}{\tunite}}}}---
\ie, as nested \tech{dependent pairs} followed by a unit \tech{expression} to
represent the empty n-tuple.
Similarly, the type of a dependent n-tuple is a nested \tech{dependent pair}
type; \im{\tnsigmaty{(\txi:\tAi\dots)}} is syntactic sugar for
\im{\tsigmaty{\txin{0}}{\tAin{0}}{\dots\tsigmaty{\txin{i}}{\tAin{i}}{\tunitty}}}.

As with \tech{dependent pairs}, I omit the annotation on n-tuples
\im{\tnpaire{\tei\dots}} when it is obvious from context.
I also define pattern matching on n-tuples, written
\im{\tlete{\tnpaire{\txi\dots}}{\tepr}{\te}}, to perform the necessary nested
projections, \ie,
\im{\tlete{\txin{0}}{\tfste{\tepr}}{\dots\tlete{\txi}{\tfste{\tsnde{\dots \tsnde{\tepr}}}}{\te}}}, as described in \fullref[]{sec:abs-cc:ideas}.

\FigCCCCRed
In \fullref[]{fig:abs-cc:cc-cc:red}, I present the reduction rules for
\tech{closures}.
The conversion relation \im{\tstepjudg[\stepstar]{\tlenv}{\te}{\tepr}}, and
evaluation function \im{\teval{\te}} are essentially unchanged from \slang, and
are given in full in \fullref[]{sec:abs-cc:appendix}.
Note that \(\beta\)-reduction only applies to \tech{closures}.
\tech{Code} cannot be applied directly, but must be part of a \tech{closure}.
\tech{Closures} applied to an argument \(\beta\)-reduce, applying the underlying
\tech{code} to the \tech{environment} and the argument.
All the other reduction rules remain unchanged.

\FigCCCCEqv
In \fullref[]{fig:abs-cc:cc-cc:eqv}, I present \tech{equivalence} for \tlang.
The key difference is that I replace the \(\eta\)-equivalence rules for
functions by \(\eta\)-equivalence for \tech{closures}.
Even when \(\eta\)-equivalence for functions is excluded from the source
language, \(\eta\)-equivalence for \tech{closures} is necessary for proving
\fullref{lem:anf:subst}.

\FigCCCCSub
In \fullref[]{fig:abs-cc:cc-cc:sub}, I define subtyping.
As usual, subtyping extends \tech{equivalence} to include \tech{cumulativity}.
The subtyping rules for \tech{code} types are essentially the same as for
\tech{dependent function} types.

\FigCCCCTypingNew[h!]
I give the new typing rules for \tlang in \fullref[]{fig:abs-cc:cc-cc:type}; the
full rules are given in \fullref[]{sec:abs-cc:appendix},
\fullref[]{fig:abs-cc:cc-cc:type1} and \fullref[]{fig:abs-cc:cc-cc:type2}.
Most rules are unchanged from the source language.
The most interesting rule is \refrule{Code}, which guarantees that \tech{code}
only type checks when it is closed.
This rule captures the goal of typed \tech{closure conversion} and gives us
static machine-checked guarantees that our translation produces closed
\tech{code}.
\refrule{Clo} for \tech{closures} \im{\tcloe{\te}{\tepr}} substitutes the
\tech{environment} \im{\tepr} into the type of the \tech{closure}, as discussed
in \fullref[]{sec:abs-cc:ideas}.
This is similar to \refrule[srcapp]{App} from \slang, which
substitutes a function argument into the result type of a function.
As discussed in \fullref[]{sec:abs-cc:ideas}, this is also critical to \tech{type
preservation}, since the translation must generate \tech{closure} \tech{types}
with free variables and then synchronize the \tech{closure} \tech{type}
containing free variables with a closed \tech{code} type.
As with \im{\sfontsym{\Pi}} types in \slang, we have two rules for well-typed
\im{\tfont{Code}} types.
\refrule{T-Code-Prop} allows \tech{impredicativity} in \im{\tpropty}, while
\refrule{T-Code-Type} is \tech{predicative.}

Note that the \tech{impredicative} rule, \refrule{T-Code-Prop}, is not necessary
for \tech{type preservation}; we only need to include it if the source language
has \tech{impredicative} functions.

\subsection{Meta-theory}
\label{sec:abs-cc:consistency}
I prove \tech{type safety} and \tech{consistency} of \tlang following the
standard architecture presented in \fullref[]{chp:type-pres}.

\FigCCCCModelShort
The essence of the \tech{model} translation from \tlang to \slang is given in
\fullref[]{fig:abs-cc:model-short}.
The translation is defined inductively on the syntax of \tech{expressions}.
This includes only the key translation rules.
The translation \im{\absccmodel{\te} = \se} models the \tlang expression
\im{\te} as the \slang expression \im{\se}.
The remaining rules are given in \fullref[]{sec:abs-cc:appendix}, in
\fullref[]{fig:abs-cc:model-full}.\footnote{In the previous version of
  this work~\cite{bowman2018:cccc}, this translation was defined by induction on
  typing derivations.
  That presentation is verbose and not necessary for constructing a model
  of \tlang.}
As described in \fullref[]{chp:type-pres}, I always assume I only translate
well-typed \tech{expressions}, so the typing derivation for \im{\te} is always
an implicit parameter whenever we have \im{\absccmodel{\te}}.

I \tech{model} \tech{code} \im{\tnfune{\tn:\tApr,\tx:\tA}{\te}}
as a curried function
\im{\sfune{\sn}{\absccmodel{\tApr}}{\sfune{\sx}{\absccmodel{\tA}}{\absccmodel{\te}}}},
and a \tech{code} \tech{type} \im{\tcodety{\tn:\tApr,\tx:\tA}{\tB}} via the
curried \tech{dependent function} type
\im{\spity{\sn}{\absccmodel{\tApr}}{\spity{\tx}{\absccmodel{\tA}}{\absccmodel{\tB}}}}.
Observe that the inner function produced in \slang is not closed, but that is
not a problem since the \tech{model} only exists to prove \tech{type safety} and
\tech{consistency}.
It is only in \tlang programs that \tech{code} must be closed.
I \tech{model} \tech{closures} \im{\tcloe{\te}{\tepr}}
as the application \im{\sappe{\absccmodel{\te}}{\absccmodel{\tepr}}}---\ie, the application
of the function \im{\absccmodel{\te}} to its \tech{environment} \im{\absccmodel{\tepr}}.
I \tech{model} \im{\tunite} with the standard Church-encoding as the
polymorphic identity function, since \slang does not include a unit expression.
(We could just as well add a unit expression to \slang.)
All other rules simply recursively translate subterms.

As discussed in \fullref[]{chp:type-pres}, we first must prove preservation of
falseness~\fullref[]{sec:m:false-pres}.
I encode the empty type, or invalid specification, \im{\tFalse} in \tlang as
\im{\tpity{\tA}{\tpropty}{\tA}}.
This type describes a function that takes any arbitrary \tech{specification}
\im{\tA} and returns a \tech{proof} of \im{\tA}.
There is only a \tech{proof} of \im{\tFalse} if \tlang is not \tech{consistent}.
Similar, we encode \im{\sFalse} in \slang as
\im{\spity{\sA}{\spropty}{\sA}}.
It is clear from the translation of \tech{dependent function types} that the
translation preserves falseness.
I use \im{=} as the terms are not just \tech{definitionally equivalent}, but
syntactically identical.

\begin{lemma}[Preservation of Falseness]
  \label{sec:m:false-pres}
  \im{\absccmodel{\tFalse} = \sFalse}
\end{lemma}

To prove \tech{type preservation}, I use the standard architecture from
\fullref[]{chp:type-pres}.
The proofs are straightforward, since the typing rules in \tlang essentially
correspond to partial application already.

\tech{Compositionality} is an important lemma since the type system and
\tech{conversion} are defined by substitution.
\begin{lemma}[Compositionality]
  \label{lem:abs-cc:m:subst}
  \im{\absccmodel{(\subst{\te}{\tepr}{\tx})} = \subst{\absccmodel{\te}}{\absccmodel{\te}}{\sx}}
\end{lemma}
\begin{proof}
  The proof is by induction on the structure of \im{\te}.
  I give the key proof cases.
  \begin{proofcases}
  \item \im{\te = \tpropty}

    Trivial, since \im{\te = \tpropty} cannot have free variables.

  \item \im{\te = \txpr}.

    There are two sub-cases:
    \begin{enumerate}[label={\bfseries Sub-case:},itemindent=2.5em]
      \item \im{\txpr = \tx}
      Then the proof follows since \im{\absccmodel{(\subst{\tx}{\tepr}{\tx})} =
        \absccmodel{\tepr} = \subst{\sx}{\absccmodel{\te}}{\sx}}

      \item \im{\txpr \neq \tx}
      Then the proof follows since \im{\absccmodel{(\subst{\txpr}{\tepr}{\tx})} = \sxpr
        = \subst{\sxpr}{\absccmodel{\tepr}}{\sx}}
    \end{enumerate}

  \item \im{\te = \tlete{\tx}{\teone}{\tetwo}}

    Follows easily by the inductive hypotheses, since both the translation of
    \im{\tfont{let}} and the definition of substitution are structural, except
    for the capture avoidance reasoning.

  \item \im{\te = \tpity{\tx}{\tA}{\tB}}

    Follows easily by the inductive hypotheses, since both the translation of
    \im{\tfontsym{\Pi}} and the definition of substitution are structural,
    except for the capture avoidance reasoning.

  \item \im{\te = \tcloe{\teone}{\tetwo}}

    Must show that \im{\absccmodel{\subst{\tcloe{\teone}{\tetwo}}{\tepr}{\tx}} =
      \subst{\absccmodel{\tcloe{\teone}{\tetwo}}}{\absccmodel{\tepr}}{\tx}}.
    \begin{align}
      & \absccmodel{\subst{\tcloe{\teone}{\tetwo}}{\tepr}{\tx}} \nonumber \\
      =~&  \absccmodel{\tcloe{\subst{\teone}{\tepr}{\tx}}{\subst{\tetwo}{\tepr}{\tx}}} \\
      & \text{by definition of substitution} \nonumber \\
      =~&  \sappe{\absccmodel{\subst{\teone}{\tepr}{\tx}}}{\absccmodel{\subst{\tetwo}{\tepr}{\tx}}} \\
      & \text{by definition of translation} \nonumber \\
      =~&
      \sappe{\subst{\absccmodel{\teone}}{\absccmodel{\tepr}}{\tx}}{\subst{\absccmodel{\tetwo}}{\absccmodel{\tepr}}{\tx}} \\
      & \text{by the induction hypothesis applied to \im{\teone} and \im{\tetwo}} \nonumber \\
      =~&
      \subst{\absccmodel{\tcloe{\teone}{\tetwo}}}{\absccmodel{\tepr}}{\tx} \\
      & \text{by substitution} \nonumber
    \end{align}
  \end{proofcases}
\end{proof}

Next I show that the model preserves \tech{reduction}, or that our
\tech{model} in \slang weakly simulates \tech{reduction} in \tlang.
This is used both to show that \tech{equivalence} is preserved, since
\tech{equivalence} is defined by \tech{conversion}, and to show \tech{type
  safety}.
\begin{lemma}[Preservation of Reduction]
  \label{lem:abs-cc:m:red}
  If \im{\te \step \tepr} then \im{\absccmodel{\te} \stepstar \absccmodel{\tepr}}
\end{lemma}
\begin{proof}
  By cases on \im{\te \step \tepr}.
  The only interesting case is for the reduction of \tech{closures}.
  \begin{proofcases}
  \item \im{\tappe{\tcloe{(\tnfune{\txpr:\tApr,\tx:\tA}{\tein{b}})}{\tepr}}{\te} \step_\beta \subst{\subst{\tein{b}}{\tepr}{\txpr}}{\te}{\tx}}

    \noindent We must show that

    \im{\absccmodel{(\tappe{\tcloe{(\tnfune{\txpr:\tApr,\tx:\tA}{\tein{b}})}{\tepr}}{\te})} \stepstar
      \absccmodel{(\subst{\subst{\tein{b}}{\tepr}{\txpr}}{\te}{\tx})}}
    \begin{align}
      & \absccmodel{(\tappe{\tcloe{(\tnfune{\txpr:\tApr,\tx:\tA}{\tein{b}})}{\tepr}}{\te})} \nonumber \\
      =~&
      \sappe{(\sappe{(\sfune{\sxpr}{\absccmodel{\tApr}}{\sfune{\sx}{\absccmodel{\tA}}}{\absccmodel{\tein{b}}})}{\absccmodel{\tepr}})}{\absccmodel{\te}}
      & \text{by definition} \\
      \step_\beta^2~&
      \subst{\subst{\absccmodel{\tein{b}}}{\absccmodel{\tepr}}{\sxpr}}{\absccmodel{\te}}{\sx}
      \\
      =~& \absccmodel{(\subst{\subst{\tein{b}}{\tepr}{\txpr}}{\te}{\tx})} & \text{by \fullref[]{lem:abs-cc:m:subst}}
    \end{align}
  \end{proofcases}
\end{proof}

Now I show that \tech{conversion} is preserved.
This essentially follows from preservation of reduction, \fullref[]{lem:abs-cc:m:red}.
\begin{lemma}[Preservation of Conversion]
  \label{lem:abs-cc:m:red*}
  If \im{\tlenv \vdash \te \stepstar \tepr} then \im{\absccmodel{\tlenv} \vdash
    \absccmodel{\te} \stepstar \absccmodel{\tepr}}
\end{lemma}
\begin{proof}
  The proof is by induction on derivation
  \im{\tstepjudg[\stepstar]{\tlenv}{\te}{\tepr}}.\footnote{In the prior version
    of this work~\cite{bowman2018:cccc}, this proof was incorrectly stated as by
    induction on the length of the reduction sequence.
    This version is corrected.}
  Each case is essentially uninteresting, but we give a few representative cases.
  \begin{proofcases}
    \item \refrule[refabscc]{Red-Refl}

      Trivial.

    \item \refrule[refabscc]{Red-Trans}

      We have that \im{\te \step \teone} and \im{\teone \stepstar \tepr}.
      We must show that \im{\absccmodel{\te} \stepstar \absccmodel{\tepr}}.

      By \fullref[]{lem:abs-cc:m:red} applied to \im{\te \stepstar \teone}, we know
      that \im{\absccmodel{\te} \stepstar \absccmodel{\teone}}, and by the
      induction hypothesis applied to \im{\teone \stepstar \tepr} we know
      that \im{\absccmodel{\teone} \stepstar \absccmodel{\tepr}}.

      By \slang \refrule[src]{Red-Trans}, we conclude that \im{\absccmodel{\te}
        \stepstar \absccmodel{\tepr}}.

    \item \refrule[refabscc]{Red-Cong-Let}

      We have
      \im{
      \inferrule*[right=\rulename{Red-Cong-Let}]
      {\tstepjudg[\stepstar]{\tlenv}{\teone}{\teonepr} \\
       \tstepjudg[\stepstar]{\tlenv,\tx = \tepr}{\tetwo}{\tetwopr}}
      {\tstepjudg[\stepstar]{\tlenv}{\tlete{\tx}{\teone}{\tetwo}}{\tlete{\tx}{\teonepr}{\tetwopr}}}
      }

      We must show that \im{\absccmodel{\tlete{\tx}{\teone}{\tetwo}} \stepstar
        \absccmodel{\tlete{\tx}{\teonepr}{\tetwopr}}}, which follows easily by
      the induction hypothesis and by \slang \refrule[src]{Red-Cong-Let}.

    \item \refrule[refabscc]{Red-Cong-Code}

      We have
      \im{
        \inferrule*[right=\rulename{Red-Cong-Code}]
        {\tstepjudg[\stepstar]{\tlenv}{\tAone}{\tAonepr} \\
         \tstepjudg[\stepstar]{\tlenv,\tn:\tAonepr}{\tAtwo}{\tAtwopr} \\
         \tstepjudg[\stepstar]{\tlenv,\tn:\tAonepr,\tx:\tAtwopr}{\te}{\tepr}}
        {\tstepjudg[\stepstar]{\tlenv}{\tnfune{(\tn:\tAone,\tx:\tAtwo)}{\te}}{\tnfune{(\tn:\tAonepr,\tx:\tAtwopr)}{\tepr}}}
      }

      We must show that \im{\absccmodel{\tnfune{(\tn:\tAone,\tx:\tAtwo)}{\te}}
        \stepstar \absccmodel{{\tnfune{(\tn:\tAonepr,\tx:\tAtwopr)}{\tepr}}}}.

      By definition of the translation, we must show that
      \im{\sfune{\sn}{\absccmodel{\tAone}}{\sfune{\tx}{\absccmodel{\tAtwo}}{\absccmodel{\te}}}
        \stepstar
        \sfune{\sn}{\absccmodel{\tAonepr}}{\sfune{\sx}{\absccmodel{\tAtwopr}{\absccmodel{\tepr}}}}}.

      By the induction hypothesis, we know that
      \begin{enumerate}
        \item \im{\absccmodel{\tAone} \stepstar \absccmodel{\tAonepr}}
        \item \im{\absccmodel{\tAtwo} \stepstar \absccmodel{\tAtwopr}}
        \item \im{\absccmodel{\te} \stepstar \absccmodel{\tepr}}
      \end{enumerate}

      The goal follows by two applications of \slang \refrule[srcapp]{Red-Cong-Lam}.

    \item \refrule[refabscc]{Red-Cong-Clo}

      We must show that \im{\absccmodel{{\tcloe{\teone}{\tetwo}}} \stepstar
        \absccmodel{\tcloe{\teonepr}{\tetwopr}}}, given that \im{\teone
        \stepstar \teonepr} and \im{\tetwo \stepstar \tetwopr}, which follows
      easily by the induction hypothesis and by the \slang
      \refrule[srcapp]{Red-Cong-App}.
  \end{proofcases}
\end{proof}

Next, I show that the translation preserves \tech{equivalence}.
The proof essentially follows from \fullref[]{lem:abs-cc:m:red*}, but we must
show that the \(\eta\)-equivalence for \tech{closures} is preserved.
\begin{lemma}[Preservation of Equivalence]
  \label{lem:abs-cc:m:coherence}
  If \im{\teone \equiv \tetwo} then \im{\absccmodel{\teone} \equiv \absccmodel{\tetwo}}
\end{lemma}
\begin{proof}
  The proof is by induction on the derivation \im{\te \equiv \tepr}.
  The only interesting case is for \(\eta\) equivalence of closures.
  \begin{proofcases}
    \item \refrule*{eqv}{\im{\equiv}}
      Follows by \fullref{lem:abs-cc:m:red*}.

    \item \refrule*{eqv-eta1}{\im{\equiv}-Clo\im{_1}}

    By assumption, we have the following.
    \begin{enumerate}
    \item \im{\teone \stepstar {\tcloe{\tnfune{(\txpr:\tApr,\tx:\tA)}{\teonepr}}{\tepr}}}
    \item \im{\tetwo \stepstar \tetwopr}
    \item \im{\subst{\teone}{\tepr}{\txpr} \equiv \tappe{\tetwopr}{\tx}}
    \end{enumerate}
    %
    We must show that \im{\absccmodel{\teone} \equiv \absccmodel{\tetwo}}.
    By \refrule*[src]{eqv-eta1}{\im{\equiv}-\im{\eta_1}}, it suffices to show:
    %
    \begin{enumerate}
    \item \im{\absccmodel{\teone} \stepstar
      \sfune{\sx}{\subst{\absccmodel{\tA}}{\absccmodel{\tepr}}{\sxpr}}{\subst{\absccmodel{\teonepr}}{\absccmodel{\tepr}}{\sxpr}}},
      which follows since:
      %
      \begin{align}
        \qquad\absccmodel{\teone} &~\stepstar \absccmodel{(\tcloe{\tnfune{(\txpr:\tApr,\tx:\tA)}{\teonepr}}{\tepr})} & \text{by \fullref[]{lem:abs-cc:m:red*}}\\
        &~= \sappe{(\sfune{\sxpr}{\absccmodel{\tApr}}{\sfune{\sx}{\absccmodel{\tA}}}{\absccmodel{\teonepr}})}{\absccmodel{\tepr}} \\
        &~\step {\sfune{\sx}{\subst{\absccmodel{\tApr}}{\absccmodel{\tepr}}{\sxpr}}
          {\subst{\absccmodel{\teonepr}}{\absccmodel{\tepr}}{\sxpr}}}
      \end{align}

    \item \im{\absccmodel{\tetwo} \stepstar \absccmodel{\tetwopr}} which follows by \fullref[]{lem:abs-cc:m:red*}.

    \item \im{{\subst{\absccmodel{\teonepr}}{\absccmodel{\tepr}}{\sxpr}} \equiv
      \sappe{\absccmodel{\tetwopr}}{\sx}}, which follows by the inductive hypothesis
      applied to \im{\subst{\teone}{\tepr}{\txpr} \equiv \tappe{\tetwopr}{\tx}} and \fullref[]{lem:abs-cc:m:subst}.
    \end{enumerate}

    \item \refrule*{eqv-eta2}{\im{\equiv}-Clo\im{_2}} is symmetric.
  \end{proofcases}
\end{proof}

\begin{lemma}[Preservation of Subtyping]
  \label{lem:abs-cc:m:subtype-pres}
  If \im{\tsubtyjudg{\tlenv}{\tA}{\tB}} then \im{\ssubtyjudg{\absccmodel{\tlenv}}{\absccmodel{\tA}}{\absccmodel{\tB}}}
\end{lemma}
\begin{proof}
  The proof is by induction on the derivation of \im{\tsubtyjudg{\tlenv}{\tA}{\tB}}.
  All cases are completely uninteresting, but I give a few representative
  cases anyway.
  \begin{proofcases}
    \item \refrule*{sub-equiv}{\im{\subtypesym}-\im{\equiv}}

      Follows by \fullref[]{lem:abs-cc:m:coherence}.

    \item \refrule*{sub-cum}{\im{\subtypesym}-Cum}

      Must show \im{\absccmodel{\ttypety{i}} \subtypesym
        \absccmodel{\ttypety{i+1}}}.
      By translation, we must show that \im{\stypety{i} \subtypesym
        \stypety{i+1}} which follows by the \slang
      \refrule*[src]{sub-cum}{\im{\subtypesym}-Cum}.

    \item \refrule*{sub-code}{\im{\subtypesym}-Code}

      Must show \im{\absccmodel{\tcodety{\tnone:\tAone,\txone:\tAonepr}{\tBone}} \subtypesym
        \absccmodel{\tcodety{\tntwo:\tAtwo,\txtwo:\tAtwopr}{\tBtwo}}}.
      By translation, we must show that
      \im{\spity{\snone}{\absccmodel{\tAone}}{\spity{\sxone}{\absccmodel{\tAonepr}}{\absccmodel{\tBone}}}
          \subtypesym
          \spity{\sntwo}{\absccmodel{\tAtwo}}{\spity{\sxtwo}{\absccmodel{\tAtwopr}}{\absccmodel{\tBtwo}}}}
      By \fullref{lem:abs-cc:m:coherence}, we know that
      \im{\absccmodel{\tAone} \equiv \absccmodel{\tAonepr}} and
      \im{\absccmodel{\tAtwo \equiv \tAtwopr}}.
      By the induction hypothesis, we know that \im{\absccmodel{\tBone} \subtypesym \absccmodel{\tBtwo}}.
      The goal follows by two applications of the \slang \refrule*[src]{sub-pi}{\im{\subtypesym}-Pi}.
  \end{proofcases}
\end{proof}

We can now show the final lemma for type safety and consistency.
\begin{lemma}[Type and Well-formedness Preservation]
  ~
  \label{lem:abs-cc:m:type-pres}
  \begin{enumerate}
    \item If \im{\twf{\tlenv}} then \im{\swf{\absccmodel{\tlenv}}}
    \item If \im{\ttyjudg{\tlenv}{\te}{\tA}} then \im{\styjudg{\absccmodel{\tlenv}}{\absccmodel{\te}}{\absccmodel{\tA}}}
  \end{enumerate}
\end{lemma}
\begin{proof}
  I prove parts 1 and 2 by simultaneous induction on the mutually defined judgments
  \im{\twf{\tlenv}} and \im{\ttyjudg{\tlenv}{\te}{\tA}}.
  Most cases follow easily by the induction hypothesis.
  \begin{proofcases}
  \item \refrule[refabscc]{W-Empty}

    Trivial.

  \item \refrule[refabscc]{W-Def}

    We must show that \im{\swf{\absccmodel{(\tlenv,\tx = \te)}}}.
    By \refrule[src]{W-Def} in \slang and part 1 of the inductive hypothesis, it
    suffices to show that \im{\styjudg{\absccmodel{\tlenv}}{\absccmodel{\te}}{\absccmodel{\tA}}},
    which follows by part 2 of the inductive hypothesis applied to
    \im{\ttyjudg{\tlenv}{\te}{\tA}}.

  \item \refrule[refabscc]{W-Assum}

    We must show that \im{\swf{\absccmodel{(\tlenv,\tx : \tA)}}}.
    By \refrule[src]{W-Assum} in \slang and part 1 of the inductive hypothesis,
    it suffices to show that \im{\styjudg{\absccmodel{\tlenv}}{\absccmodel{\tA}}{\absccmodel{\tU}}},
    which follows by part 2 of the inductive hypothesis applied to
    \im{\ttyjudg{\tlenv}{\tA}{\tU}}.

  \item \refrule[refabscc]{Prop}

    It suffices to show that \im{\swf{\absccmodel{\tlenv}}}, since \im{\absccmodel{\tpropty} =
      \spropty}, which follows by part 1 of the inductive hypothesis.

  \item[\vdots]
  \item \refrule{T-Code-Prop}

    We have that
    \begin{displaymath}
      \inferrule
      {\ttyjudg{\tlenv}{\tApr}{\tUpr} \\
        \ttyjudg{\tlenv,\txpr:\tApr}{\tA}{\tU} \\
        \ttyjudg{\tlenv,\txpr:\tApr,\tx:\tA}{\tB}{\tpropty}}
      {\ttyjudg{\tlenv}{\tcodety{\txpr:\tApr,\tx:\tA}{\tB}}{\tpropty}}
    \end{displaymath}
    We must show that
    \im{\styjudg{\absccmodel{\tlenv}}{\spity{\sxpr}{\absccmodel{\tApr}}{\spity{\sx}{\absccmodel{\tA}}{\absccmodel{\tB}}}}{\spropty}}

    \noindent By two applications of \refrule[srcapp]{Pi-Prop}, it suffices to
    show
    \begin{itemize}
      \item \im{\styjudg{\absccmodel{\tlenv}}{\absccmodel{\tApr}}{\absccmodel{\tUpr}}}, which follows by part
        2 of the inductive hypothesis.
      \item \im{\styjudg{\absccmodel{\tlenv},\sxpr:\absccmodel{\tApr}}{\absccmodel{\tA}}{\absccmodel{\tU}}},
        which follows by part 2 of the inductive hypothesis.
      \item \im{\styjudg{\absccmodel{\tlenv},\sxpr:\absccmodel{\tApr}}{\absccmodel{\tB}}{\spropty}},
        which follows by part 2 of the inductive hypothesis and by definition
        that \im{\absccmodel{\tpropty} = \spropty}
    \end{itemize}

    \item \refrule{Code}

    We have:
    \begin{displaymath}
      \inferrule
      {\ttyjudg{\tlenv,\txpr:\tApr,\tx:\tA}{\te}{\tB}}
      {\ttyjudg{\tlenv}{\tnfune{\txpr:\tApr,\tx:\tA}{\te}}{\tcodety{\txpr:\tApr,\tx:\tA}{\tB}}}
    \end{displaymath}
    By definition of the translation, we must show
    \im{\styjudg{\absccmodel{\tlenv}}{\sfune{\sxpr}{\absccmodel{\tApr}}{\sfune{\sx}{\absccmodel{\tA}}{\absccmodel{\te}}}}
      {\spity{\sxpr}{\absccmodel{\tApr}}{\spity{\sx}{\absccmodel{\tA}}{\absccmodel{\tB}}}}},
    which follows by two uses of \refrule[srcapp]{Lam} in \slang and part 2 of the
    inductive hypothesis.

    \item \refrule{Clo}

    We have:
    \begin{displaymath}
      \inferrule
      {\ttyjudg{\tlenv}{\te}{\tcodety{\txpr:\tApr,\tx:\tA}{\tB}} \\
        \ttyjudg{\tlenv}{\tepr}{\tApr}}
      {\ttyjudg{\tlenv}{\tcloe{\te}{\tepr}}{\tpity{\tx}{\subst{\tA}{\tepr}{\txpr}}{\subst{\tB}{\tepr}{\txpr}}}}
    \end{displaymath}
    By definition of the translation, we must show that

    \im{\styjudg{\absccmodel{\tlenv}}{\sappe{\absccmodel{\te}}{\absccmodel{\tepr}}}{\absccmodel{({\tpity{\tx}{\subst{\tA}{\tepr}{\txpr}}{\subst{\tB}{\tepr}{\txpr}}})}}}.

    By \fullref{lem:abs-cc:m:subst}, it suffices to show that

    \im{\styjudg{\absccmodel{\tlenv}}{\sappe{\absccmodel{\te}}{\absccmodel{\tepr}}}{\spity{\sx}{\subst{\absccmodel{\tA}}{\absccmodel{\tepr}}{\sxpr}}{\subst{\absccmodel{\tB}}{\absccmodel{\tepr}}{\sxpr}}}}.

    By \refrule[srcapp]{App} in \slang, it suffices to show that
    \begin{itemize}
    \item
      \im{\styjudg{\absccmodel{\tlenv}}{\absccmodel{\te}}{\spity{\sxpr}{\sApr}{\spity{\sx}{\absccmodel{\tA}}{\absccmodel{\tB}}}}},
      which follows with \im{\sApr = \absccmodel{\tApr}} by part 2 of the inductive hypothesis.
    \item \im{\styjudg{\absccmodel{\tlenv}}{\absccmodel{\tepr}}{\sApr}}, which follows by part 2 of the
      inductive hypothesis.
    \end{itemize}

    \item \refrule[refabscc]{App}

      Similar to the case for \refrule{Clo}.

    \item \refrule[refabscc]{Conv}

      Follows by part 2 of the inductive hypothesis and \fullref{lem:abs-cc:m:subtype-pres}.
  \end{proofcases}
\end{proof}

The \tech{model} of \tlang in \slang implies the desired \tech{consistency} and
\tech{type safety} theorems, as discussed earlier.

\tech{Consistency} tells us that we can only write \tech{proofs} of \tech{valid}
\tech{specifications}.
\begin{theorem}[Logical Consistency of \tlang]
  \label{thm:abs-cc:m:sound}
  There does not exist a closed expression \im{\te} such that \im{\ttyjudg{\cdot}{\te}{\tFalse}}.
\end{theorem}

\tech{Type safety} tells us that there is no undefined behavior that causes a
\tech{program} to get stuck before it produces an \tech{observation}.
\begin{theorem}[Type Safety of \tlang]
  \label{thm:abs-cc:m:safe}
  If \im{\wf{}{\te}}, then \im{\teval{\te}} is well-defined.
\end{theorem}
