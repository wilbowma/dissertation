In this chapter, I develop the second of the two front-end
\tech{type-preserving} translations, a so-called \tech{abstract closure
conversion} translation.
\deftech*{abstract closure conversion,Abstract closure conversion}{Abstract closure conversion}
produces \tech{closure converted} \tech{code} in which \tech{closures} are
primitives, rather than encoded using a well-known datatype~\cite{minamide1996}.
The goal of \deftech*{closure conversion,closure converted,closure
converting,closure-converted}{closure conversion} is to close
all \tech{computation}
abstractions with respect to free variables so the definition of
a \tech{computation} can be separate from its use.
In \slang, the only \tech{computation} abstraction is \(\lambda\), so we will be
translating all \(\lambda\) expressions into an explicit closure object.
Intuitively, the abstract closure object essentially represents closed code
partially applied to the local environment in which the code was defined.
This is abstract compared to representations of closures as a pair of the code
and environment, since a pair supports more operations (projection) than a
closure.
The abstract closure conversion ensures the only operation defined on closures
is application.

I begin by describing the key problems with type preserving closure
conversion---particularly why the standard \tech{parametric closure conversion}
fails---and the solutions, then develop the \tech{closure conversion} \tech{IL},
and finally prove \tech{type preservation} and \tech{compiler correctness}.
As this pass is meant to follow \tech{ANF}, I also prove that \tech{ANF} is
preserved.

\begin{typographical}
 In this chapter, I typeset the source language, \slang, in
 a \emph{\sfonttext{blue, non-bold, sans-serif font}}, and the target
 language, \tlang, in a \emph{\tfonttext{bold, red, serif font}}.
\end{typographical}

\begin{digression}
  A variant of the translation presented in this chapter was independently discovered by \citet{kovacs2018:cconv}. That translation differs primarily in its use of universes ala Tarski. The author also spends some time discussing type-passing polymorphism, which I do not discuss.
\end{digression}
