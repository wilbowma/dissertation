\section{Main Ideas}
\label{sec:abs-cc:ideas}
\tech{Closure conversion} makes the implicit \tech{closures} from a functional
language explicit to facilitate statically allocating \tech{code} in memory.
The idea is to translate each first-class function into an explicit
\tech{closure}, \ie, a pair of closed \tech{code} and an \deftech{environment}
data structure containing the values of the free variables.
\deftech*{codes,code,Code}{Code} refers to functions with no free variables, as in a
closure-converted language.
The \tech{environment} is created dynamically, but the closed \tech{code} can be
lifted to the top-level and statically allocated.
Consider the following example translation.
%
\begin{displaymath}
  \begin{array}{@{\hspace{0pt}}r@{\hspace{1ex}}c@{\hspace{1ex}}l}
  \cctrans{(\lambda x.y)} &=& \left<(\lambda
                      n\,x.\,\kwopen{\text{let}}y\mathrel{=}(\kwopen{\pi_1}n)\mathrel{\text{in}}
                      y), \left<y\right>\right> \\
  \cctrans{((\lambda x.y)~\text{true})} &=& \begin{stackTL}
                           \kwopen{\text{let}}\left<f, n\right>\mathrel{=}\left<(\lambda
                               n\,x.\,\kwopen{\text{let}}y\mathrel{=}(\kwopen{\pi_1}n)\mathrel{\text{in}}y),
                            \left<y\right>\right>\mathrel{\text{in}}
                           \\\quad f~n~\text{true}
                          \end{stackTL}
    \end{array}
\end{displaymath}
%
I write \im{\cctrans{e}} to indicate the translation of an expression \im{e}.
We translate each function into a pair of \tech{code} and its
\tech{environment}.
The \tech{code} accepts its free variables in an \tech{environment} argument,
\im{n} (since \im{n} sounds similar to \emph{env}).
In the body of the \tech{code}, we bind the names of all free variables by
projecting from this \tech{environment} \im{n}.
To call a \tech{closure}, we apply the code to its \tech{environment} and its
argument.

This translation is not \tech{type preserving} since the structure of the
\tech{environment} shows up in the \tech{type}.
For example, the following two functions have the same \tech{type} in the
source language.
They are both functions on booleans, and \im{y} is a free variable of type \im{\gboolty}.
\begin{displaymath}
  \begin{array}{rcl}
  \lambda x.y &:&  (\gboolty \to \gboolty)
  \\
  \lambda x.x &:&  (\gboolty \to \gboolty)
  \end{array}
\end{displaymath}
But after closure conversion when we encode closures as pairs, the two
functions have different \techs{type} in the target language.
%
\begin{displaymath}
  \begin{array}{rcl}
  \cctrans{(\lambda x.y)} &:&  ((\gboolty \times \gunitty) \to \gboolty \to \gboolty)  \times (\gboolty \times \gunitty)
  \\
  \cctrans{(\lambda x.x)} &:&  (\gunitty \to \gboolty \to \gboolty) \times \gunitty
  \end{array}
\end{displaymath}

This is a well-known problem with typed \tech{closure conversion}, so we could
try the well-known solution, called \emph{parametric closure
  conversion}~[\citealp{minamide1996}, \citealp{morrisett1998:reccc},
  \citealp{morrisett1998:ftotal}, \citealp{ahmed2008}, \citealp{perconti2014},
  \citealp{new2016}], which represents \techs{closure} as an existential package
of a pair of the \tech{code} and its \tech{environment}, whose \tech{type} is
hidden.
The existential type hides the structure of the \tech{environment} in the
\tech{type}.
(Spoiler alert: it doesn't work for \slang.)
%
\begin{displaymath}
  \begin{array}{rcl}
    \cctrans{(\lambda x.y)} &:& \exists \alpha. (\alpha \to \gboolty \to \gboolty) \times \alpha
    \\
    \cctrans{(\lambda x.x)} &:& \exists \alpha. (\alpha \to \gboolty \to \gboolty) \times \alpha
  \end{array}
\end{displaymath}

This translation works well for simply typed and polymorphic languages, but when
we move to a \tech{dependently typed} language, we have new challenges.
First, the \tech{environment} must now be ordered since the \tech{type} of each
new variable can depend on all prior variables.
Second, \techs{type} can now refer to variables in the \tech{closure}'s
\tech{environment}.
Consider the polymorphic identity function below.
%
\begin{displaymath}
  \begin{array}{rcl}
  \sfune{\sA}{\spropty}{\sfune{\sx}{\sA}{\sx}} & : & \spity{\sA}{\spropty}{\spity{\sx}{\sA}{\sA}}
  \end{array}
\end{displaymath}
%
This function takes a type variable, \im{\sA}, whose type is \im{\spropty}.
It returns a function that accepts an argument \im{\sx} of type \im{\sA} and returns it.
There are two \tech{closures} in this example: the outer \tech{closure} has no
free variables, and thus will have an empty \tech{environment}, while the inner
\tech{closure} \im{\sfune{\sx}{\sA}{\sx}} has \im{\sA} free, and thus \im{\sA}
will appear in its \tech{environment}.

Below, I present the translation of this example using the existing parametric
\tech{closure conversion} translation.
The translation produces two \tech{closures}, one nested in the other.
Note that we translate source variables \im{\sx} to \im{\tx}.
In the outer \tech{closure}, the \tech{environment} is empty \im{\tnpaire{}},
and the code simply returns the inner \tech{closure}.
The inner \tech{closure} has the argument \im{\tA} from the outer \tech{code} in
its \tech{environment}.
Since the inner \tech{code} takes an argument of type \im{\tA}, we project
\im{\tA} from the \tech{environment} \emph{in the type annotation} for \im{\tx}.
That is, the inner \tech{code} takes an \tech{environment} \im{\tntwo} that
contains \im{\tA}, and the type annotation for \im{\tx} is \im{\tx :
  \tfste{\tntwo}}.
The type \im{\tfste{\tntwo}} is unusual, but is no problem since \tech{dependent
  types} allow \techs{term} in \techs{type}.
%
\begin{displaymath}
  \begin{stackTL}
    \tcloe{\tnfune{(\tnone:\tunitty,\tA:\tpropty)}{
        \tcloe{\tnfune{(\tntwo:
            \tpairty{\tpropty}{\tunitty},\tx:\tfste{\tntwo})}{\tx}
        }{\tnpaire{\tA,\tnpaire{}}}}}{\tnpaire{}} ~~ : ~~\\
    ~~\begin{array}{@{\hspace{0pt}}l@{\hspace{0pt}}l@{\hspace{0pt}}l}
      \texistty{\talphaone}{\tpropty}{&\,\tpairty{(\tnpity{&(\tnone:\talphaone,\tA:\tpropty)}{
            \\ &&
            \texistty{\talphatwo}{\ttypety{i}}{\tpairty{(\tnpity{(\tntwo:\talphatwo,\tx:\tfste{\tntwo})}{\tfste{\tntwo}})}{\talphatwo}}})\,}{\talphaone}}
    \end{array}
    \end{stackTL}
\end{displaymath}
%
We see that the inner \tech{code} on its own is well typed with the closed type
\im{\tnpity{(\tntwo:\tpairty{\tpropty}{\tunitty},\tx:\tfste{\tntwo})}{\tfste{\tntwo}}}.
That is, the \tech{code} takes two arguments: the first argument \im{\tntwo} is
the \tech{environment}, and the second argument \im{\tx} is a value of type
\im{\tfste{\tntwo}}.
The result type of the \tech{code} is also \im{\tfste{\tntwo}}.
As discussed above, we must hide the type of the \tech{environment} to ensure
\tech{type preservation}.
That is, when we build the \tech{closure}
\im{\tcloe{\tnfune{(\tntwo:
      \tpairty{\tpropty}{\tunitty},\tx:\tfste{\tntwo})}{\tx}
    }{\tnpaire{\tA,\tnpaire{}}}}, we must hide the type of the
\tech{environment} \im{\tnpaire{\tA,\tnpaire{}}}.
We use an existential type to quantify over the type \im{\talphatwo} of the
environment, and we produce the type
\im{\tnpity{(\tntwo:\talphatwo,\tx:\tfste{\tntwo})}{\tfste{\tntwo}}} for the
\tech{code} in the inner \tech{closure}.
But this type is trying to take the \emph{first projection} of something of type
\im{\talphatwo}.
We can only project from pairs, and something of type \im{\talphatwo} isn't a
pair!
In hiding the type of the \tech{environment} to recover \tech{type
  preservation}, we've broken \tech{type preservation} for \tech{dependent
  types}.

A similar problem also arises when \tech{closure converting} System F, since
System F also features type variables~\cite{minamide1996,morrisett1998:ftotal}.
To understand my solution, it is important to understand why the solutions that
have historically worked for System F do not scale to \tech{dependent types}.
I briefly present these past results and why they do not scale before moving on
to the key idea behind my translation.
Essentially, past work using existential types relies on assumptions about
\tech{computational relevance}, \tech{parametricity}, and \tech{impredicativity}
that do not necessarily hold in \tech{full-spectrum} \tech{dependent type} systems.

\subsection{Why the well-known solution doesn't work}
\citet{minamide1996} give a translation that encodes \tech{closure} types using
\tech{existential types}, a standard type-theoretic feature that they use to
make environment hiding explicit in the types.
In essence, they encode \techs{closure} as objects; the \tech{environment} can
be thought of as the private field of an object.
Since then, existential types have been the standard way to encode
\tech{closure} types has been all work on typed \tech{closure conversion}.

However, the use of existential types to encode \techs{closure} in a dependently
typed setting is problematic.
First, let us just consider \tech{closure conversion} for System F.
As \citet{minamide1996} observed, there is a problem when code must be closed
with respect to both \tech{term} and \emph{\tech{type}} variables.
This problem is similar to the one discussed above: when \tech{closure}
\techs{environment} contain \tech{type} variables, since those \tech{type}
variables can also appear in the \tech{closure}'s type, the \tech{closure}'s
type needs to project from the \tech{closure}'s (hidden) \tech{environment}
which has type \im{\alpha}.
To fix the problem, \citet{minamide1996} extend their target language with
\deftech{translucency} (essentially, a kind of \tech{type}-level equivalence
that we now call singleton types), \tech{type}-level pairs, and kinds.
All of these features can be encoded in \slang and most \tech{dependent type}
systems, so we could extend their translation essentially as follows.
(In fact, I present the extended translation in \fullref[]{chp:param-cc}.)
%
\begin{displaymath}
  \begin{array}{rcl}
    \!\!\!\cctrans{(\spity{\sx}{\sA}{\sB})} \defeq
    \texistty{\talpha}{\tU}{\texistty{\tn}{\talpha}{\tcodety{\tnpr{\,:\,}\talpha, \ty{\,:\,}\tnpr = \tn,\tx{\,:\,}\cctrans{\sA}}{\cctrans{\sB}}}}
  \end{array}
\end{displaymath}

In this translation, we existentially quantify over the \tech{type} of the
\tech{environment} \im{\talpha}, the \emph{\tech{term}} representing the
\tech{environment} \im{\tn}, and generate \tech{code} that requires an
\tech{environment} \im{\tnpr} plus a proof that the code is only ever given the
environment \im{\tn} as the argument \im{\tnpr}.
The typing rule for an existential package copies the existentially quantified
\techs{term} into the \tech{type}.
That is, for a \tech{closure} \im{\tnpackoe{\tApr,\tv,\te}} of type
\im{\texistty{\talpha}{\tU}{\texistty{\tn}{\talpha}{\tcodety{\tnpr{\,:\,}\talpha,
        \ty{\,:\,}\tnpr = \tn,\tx{\,:\,}\cctrans{\sA}}{\cctrans{\sB}}}}}, the typing rule for
\im{\tfont{pack}} requires that we show \im{\te : \tcodety{\tnpr{\,:\,}\tApr,
        \ty{\,:\,}\tnpr = \tv,\tx{\,:\,}\cctrans{\sA}}{\cctrans{\sB}}}; notice that the variable \im{\tn}
has been replaced by the \tech{term} representing the environment \im{\tv}.
The equality \im{\tnpr = \tv} essentially unifies projections from \im{\tnpr}
with projections from \im{\tv}, the list of free variables representing the
\tech{environment}.

The problem with this translation is that it relies on
\emph{\tech{impredicativity}}.
That is, if \im{(\spity{\sx}{\sA}{\sB}) : \spropty}, then we require that
\im{\cctrans{(\spity{\sx}{\sA}{\sB})} : \tpropty}.
Since the existential type quantifies over a type in an arbitrary
\tech{universe} \im{\tU} but must be in the base \tech{universe} \im{\tpropty},
the existential type must be \tech{impredicative}.
\tech{Impredicative} existential types (weak \techs{dependent pair}) are
consistent on their own, but \tech{impredicativity} causes inconsistency when
combined with other features, including \tech{computational relevance} and
\tech{higher universes}.
For example, in Coq by default, the base \tech{computationally relevant} \tech{universe}
\texttt{Set} is \tech{predicative}, so this translation would not work.
There is a flag to enable \tech{impredicative} \texttt{Set}, but this can
introduce inconsistency with some axioms, such as a combination of the law of
excluded middle plus the axiom of choice, or ad-hoc
polymorphism~\cite{boulier2017}.
Even with \tech{impredicative} \texttt{Set}, there are \tech{computationally relevant}
\tech{higher universes} in Coq's universe hierarchy, and it would be unsound to
allow \tech{impredicativity} at more than one \tech{universe}.
Furthermore, some \tech{dependently typed} languages, such as Agda, do not allow
\tech{impredicativity} at all.

A second problem arises in developing an \(\eta\) principle, because the
existential type encoding relies on \emph{\tech{parametricity}} to hide the
\tech{environment}.
So, any \(\eta\) principle would need to be justified by a parametric relation
on \tech{environments}.
Internalizing \tech{parametricity} for dependent type theory is an active area
of research~\cite{krishnaswami2013,bernardy2012,keller2012,nuyts2017} and not
all \tech{dependent type} theories admit \tech{parametricity}~\cite{boulier2017}.

Later, \citet{morrisett1998:ftotal} improved the existential-type translation
for System F, avoiding \tech{translucency} and kinds by relying on \emph{type
  erasure} before runtime, which meant that their \tech{code} didn't have to
close over type variables.
This translation does not apply in a \tech{dependently typed} setting, since
\tech{dependent types} can contain \tech{term} variables, not just ``type
erasable'' \tech{type} variables.

\subsection{Abstract Closure Conversion}
To solve \tech{type-preserving} \tech{closure conversion} for \slang, I avoid
existential types altogether and instead take inspiration from the so-called
\tech{abstract closure conversion} of \citet{minamide1996}.
They add new forms to the target language to represent \tech{code} and
\techs{closure} for a simply typed source language.
In this chapter, I scale their design to \techs{dependent type}.

I extend \slang with primitive types for \tech{code} and \techs{closure}.
I represent \tech{code} as \im{\tnfune{(\tn:\tApr,\tx:\tA)}{\teone}} of
the \deftech{code type} \im{\tcodety{\tn:\tApr,\tx:\tA}{\tB}}.
These are still \techs{dependent type}, so \im{\tn} may appear in both \im{\tA}
and \im{\tB}, and \im{\tx} may appear in \im{\tB}.
\tech{Code} must be well-typed in an empty \tech{environment}, \ie, when it is
closed.
For simplicity, \tech{code} only takes two arguments.
\begin{mathpar}
  \inferrule*[right=\rulename{Code}]
  {\ttyjudg{\cdot,\tn:\tApr,\tx:\tA}{\te}{\tB}}
  {\ttyjudg{\tlenv}{\tnfune{\tn:\tApr,\tx:\tA}{\te}}{\tcodety{\tn:\tApr,\tx:\tA}{\tB}}}
\end{mathpar}

I represent \tech{closures} as \im{\tcloe{\te}{\tepr}} of type
\im{\tpity{\tx}{\subst{\tA}{\tepr}{\tn}}{\subst{\tB}{\tepr}{\tn}}}, where
\im{\te} is code and \im{\tepr} is its \tech{environment}.
We \emph{continue} to use \im{\tfontsym{\Pi}} types to describe \tech{closures};
note that ``functions'' in \slang are implicit \tech{closures}.
The typing rule for \tech{closures} is the following.
%
\begin{mathpar}
  \inferrule*[right=\rulename{Clo}]
  {\ttyjudg{\tlenv}{\te}{\tcodety{\tn:\tApr,\tx:\tA}{\tB}} \\
  \ttyjudg{\tlenv}{\tepr}{\tApr}}
  {\ttyjudg{\tlenv}{\tcloe{\te}{\tepr}}{\tpity{\tx}{\subst{\tA}{\tepr}{\tn}}{\subst{\tB}{\tepr}{\tn}}}}
\end{mathpar}
%
We should think of a \tech{closure} \im{\tcloe{\te}{\tepr}} not as a pair, but
as a delayed partial application of the \tech{code} \im{\te} to its
\tech{environment} \im{\tepr}.
This intuition is formalized in the typing rule since the \tech{environment} is
substituted into the \tech{type}, just as in \tech{dependent-function}
application in \slang.

To understand the translation, let us start with the translation of functions;
this is the key translation rule.
%
\begin{displaymath}
  \begin{array}{@{\hspace{0pt}}r@{\hspace{1ex}}c@{\hspace{1ex}}l}
    \cctrans{(\sfune{\sx}{\sA}{\se})} &\defeq&
                                     \tcloe{(\tnfune{\begin{stackTL}(\tn:\tsigmaty{(\txi}{\cctrans{\sAi}\dots)}{\tunitty},\tx:\tlete{\tnpaire{\txi\dots}}{\tn}{\cctrans{\sA}})}{
                                           \\
                                           \tlete{\tnpaire{\txi\dots}}{\tn}{\cctrans{\se}}})}{\tnpaire{\txi\dots}}
                                     \end{stackTL} \\
                              && \text{ where \im{\sxi:\sAi\dots} are the free variables of \im{\se} and \im{\sA}}
  \end{array}
\end{displaymath}
%
The translation of functions is simple to construct.
We know we want to produce a \tech{closure} containing \tech{code} and its
\tech{environment}.
We know the \tech{environment} should be constructed from the free variables of
the body of the function, namely \im{\se}, and, due to \tech{dependent types},
the type annotation \im{\sA}.
The type of the environment,
\im{\tsigmaty{(\txi}{\cctrans{\sAi}\dots)}{\tunitty}}, is the type of a dependent
list describing the free variables, with a final element of the unit type.
This encodes the fact that type of each variable in the environment can
depend on the value of previous variables.
(The syntax \im{\tlete{\tnpaire{\txi\dots}}{\tn}{\te}} is syntactic sugar for
nested projections from a list implemented with pairs, \ie, for
\im{\tlete{\txone}{\tfste{\tn}}{(\tlete{\txtwo}{\tfste{\tsnde{\tn}}}{(...\tlete{\txin{n}}{(\tfste{\tsnde{...\tsnde{\tn}}}}{\te})})}}.)

The question is what the translation of \im{\sfontsym{\Pi}} \tech{types}
should look like.
Let's return to the earlier example of the polymorphic identity function.
If we apply the above translation, we produce the following for the inner
\tech{closure}.
We know its \tech{type} by following the typing rules \rulename{Clo} and
\rulename{Code} above.
%
\begin{displaymath}
  \begin{array}{l}
    \tcloe{\tnfune{(\tntwo:\tpairty{\tpropty}{\tunitty},\tx:\tfste{\tntwo})}{\tx}}
    {\tnpaire{\tA,\tnpaire{}}} :
    \\ \tpity{(\tx}{\subst{(\tfste{\tntwo})}{\tnpaire{\tA,\tunite}}{\tntwo})}{\subst{(\tfste{\tntwo})}{\tnpaire{\tA,\tunite}}{\tntwo}}
  \end{array}
\end{displaymath}
%
We know that the \tech{code}
\im{\tnfune{(\tntwo:\tpairty{\tpropty}{\tunitty},\tx:\tfste{\tntwo})}{\tx}} has type
\im{\tcodety{\tpairty{\tpropty}{\tunitty},\tx:\tfste{\tntwo}}{\tfste{\tntwo}}}.
Following \rulename{Clo}, we substitute the \tech{environment} into this type, so we get the following.
%
\begin{displaymath}
\tpity{(\tx}{\subst{(\tfste{\tntwo})}{\tnpaire{\tA,\tunite}}{\tntwo})}{\subst{(\tfste{\tntwo})}{\tnpaire{\tA,\tunite}}{\tntwo}}
\end{displaymath}
%
So how do we translate the function type \im{\spity{\sx}{\sA}{\sA}} into the
\tech{closure} type
\im{\tpity{(\tx}{\subst{(\tfste{\tntwo})}{\tnpaire{\tA,\tunite}}{\tntwo})}{\subst{(\tfste{\tntwo})}{\tnpaire{\tA,\tunite}}{\tntwo}}}?
Note that this type reduces to \im{\tpity{\tx}{\tA}{\tA}}.
So by the rule \rulename{Conv}, we simply need to translate
\im{\spity{\sx}{\sA}{\sA}} to \im{\tpity{\tx}{\tA}{\tA}}!

The key translation rules are given below.
%
\begin{displaymath}
  \begin{array}{@{\hspace{0pt}}r@{\hspace{1ex}}c@{\hspace{1ex}}l}
    \cctrans{(\spity{\sx}{\sA}{\sB})} &\defeq &\tpity{\tx}{\cctrans{\sA}}{\cctrans{\sB}}\\
    \cctrans{(\sfune{\sx}{\sA}{\se})} &\defeq&
                                     \tcloe{(\tnfune{\begin{stackTL}(\tn:\tnsigmaty{(\txi:\cctrans{\sAi}\dots)},\tx:\tlete{\tnpaire{\txi\dots}}{\tn}{\cctrans{\sA}})}{
                                           \\
                                           \tlete{\tnpaire{\txi\dots}}{\tn}{\cctrans{\se}}})}{\tnpaire{\txi\dots}}
                                     \end{stackTL} \\
                              && \text{ where \im{\sxi:\sAi\dots} are the free variables of \im{\se} and \im{\sA}}
  \end{array}
\end{displaymath}

Observe that, when the source is in \tech{ANF}, this translation maintains \tech{ANF}.
A \tech{closure} ought to be a \tech*{anf:value}{value}, and therefore its
sub-expressions should be \tech*{anf:value}{values}, and the translation
guarantees this.
The body of the \tech{code} should be a \tech*{anf:config}{configuration}, which
is also true, since for any \tech*{anf:config}{configuration} \im{\tM},
\im{\tlete{\tx}{\tN}{\tM}} is a \tech*{anf:config}{configuration}.
I show this in detail in \fullref[]{sec:abs-cc:cc:anf}.

A final challenge remains in the design of our target language: we need to know
when two \tech{closures} are equivalent.
As we just saw, \slang partially evaluates terms while type checking.
If two \tech{closures} get evaluated while resolving type equivalence, we may
inline a \tech{term} into the \tech{environment} for one \tech{closure} but not
the other.
When this happens, two \tech{closures} that were syntactically identical and
thus equivalent become inequivalent.
I discuss this problem in detail in \fullref[]{sec:abs-cc:cc}, but essentially
we need to know when two syntactically distinct \tech{closures} are equivalent.
The solution is simple: get rid of the \tech{closures} and keep inlining things!
%Our solution, given below, is simple: closures be damned, keep inlining things!
%
\begin{mathpar}
  \inferrule
  {\tequivjudg{\tlenv,\tx:\tA}{\subst{\teone}{\teonepr}{\tn}}{\subst{\tetwo}{\tetwopr}{\tn}}}
  {\tequivjudg{\tlenv}{\tcloe{(\tnfune{(\tn:\tApr,\tx:\tA)}{\teone})}{\teonepr}}{\tcloe{(\tnfune{(\tn:\tApr,\tx:\tA)}{\tetwo})}{\tetwopr}}}
\end{mathpar}

Two \techs{closure} are equivalent when we inline the \tech{environment}, free
variables or not, and run the body of the \tech{code}.
We leave the argument free, too.
We run the bodies of the \tech{code} to normal forms, then compare the normal forms.
Recall that equivalence runs \techs{term} while \emph{type checking} and does
not change the program, so the free variables do no harm.

This equivalence essentially corresponds to an \(\eta\)-principle for \techs{closure}.
From it, we can derive a normal form for \techs{closure} \im{\tcloe{\te}{\tepr}}
that says the \tech{environment} \im{\tepr} contains only free variables, \ie,
\im{\tepr = \tnpaire{\txi\dots}}.

The above is an intuitive, declarative presentation, but is incomplete without
additional rules.
I use an algorithmic presentation that is similar to the \(\eta\)-equivalence
rules for functions in \slang, which I give in \fullref[]{sec:abs-cc:target}.
