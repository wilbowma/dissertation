\label{sec:cps:intro}
In this chapter, I develop a \tech{type-preserving} \tech{CPS} translation for a
subset of \slang.
As mentioned in \fullref[]{chp:anf}, \tech{CPS} presents many challenges and
does not scale well to all features of \tech{dependency}.
However, by understanding the key problems that \tech{CPS} introduces into
a \tech{dependent types} system, we can develop \tech{type
preserving} \tech{CPS} for some \tech{dependently typed} languages, which is
surprising given past impossibility results.

I start with a brief discussion of the different uses of \tech{CPS} and discuss
the past impossibility result for dependent-type-preserving \tech{CPS}.
I then develop \cpsslang, a restriction of \slang suitable for translations that
rely on \emph{parametricity}, before developing both \tech{CBN}
and \tech{CBV} \tech{type-preserving} \tech{CPS} translations for \cpsslang.

\begin{typographical}
  In this chapter, I typeset the source language, \cpsslang, in a
  \emph{\sfonttext{blue, non-bold, sans-serif font}}, and the target language,
  \cpstlang, in a \emph{\tfonttext{bold, red, serif font}}.
\end{typographical}

\section{On {{CPS}}}
The \tech{CPS} translation presented in this chapter allows
implementing \tech{type-preserving} compilation for some \tech{dependently
  typed} languages, but it will not allow implementing control effects, a common
  use for \tech{CPS}.
Many of the design decisions I make do not apply when implementing control
effects, or specifically disallow implementing control effects.
For example, my goal is to avoid restricting dependencies in the source
language so that I can compile existing languages such as Coq.
However, to allow mixing control effects and dependency, one must necessarily
restrict certain \tech{dependencies}, at least for the effectful parts of the
language.
This requires different designs for the CPS language and translation; I discuss
some related work in this vein in \fullref[]{sec:cps:related}.
Before I present my CPS translation, I discuss the context of CPS translation as it
applies to type preservation and the core features of dependency.

Typically, type-preserving \tech{compilers} use one of two common type translations: (1)
the \deftech{double-negation translation} that translates expressions of
type \im{A} into a \tech{computation} of type \im{A^\div = (A^+ \to \False) \to \False} or
\im{\lnot\lnot A^+} where \im{A^+} represents the \tech{value-type translation},
or (2) the \deftech*{locally polymorphic answer type,answer-type
polymorphism,polymorphic answer type translation,polymorphic answer
type}{locally polymorphic answer type translation} that translates
\tech{expressions} of type \im{A} into \tech{computations} of type \im{A^\div =
  \forall \alpha. (A^+ \to \alpha) \to \alpha}.
The value translation differs depending on whether the translation is
encoding \tech{CBN} or \tech{CBV} evaluation order, but essentially recursively
translates types that must be \tech{values} using the value translation \im{^+},
and types that could be \tech{computations} using the computation
translation \im{^\div}.
In each translation, the \tech{computation} expects a \tech{continuation},
either of type \im{A^+ \to \False} or \im{A^+ \to \alpha}.
Intuitively, the computation is forced to call that \tech{continuation} with
a \tech{value} of type \im{A^+}.
The \tech{double-negation translation} represents the final result of a
\tech{program} with the empty type \False to indicate that \tech{programs}
\deftech{never return}, \ie, \tech{programs} no longer behave like mathematical
functions that take inputs and produce output values but instead run to
completion and end up in some final answer state.
The \tech{locally polymorphic answer type} uses \tech{parametricity} to encode a
similar idea: if the \tech{computation} behaves \tech{parametrically} in its
answer type, then no \tech{computation} can do anything but call its
\tech{continuation} with the \deftech{underlying value} that the
\tech{computation} represents.
At the level of a whole program, this is equivalent to a \tech{program}
\tech{never returning}---each intermediate \tech{computation} cannot return
because it must, by \tech{parametricity}, invoke its \tech{continuation}.

These two translations admit different reasoning principles, however.

The \tech{double-negation translation} supports encoding control
effects but complicates compositional reasoning.
A \tech{computation} is free to duplicate its \tech{continuation} and save it
for later, or call a saved \tech{continuation}, since the type of
\tech{computations} only requires that the \tech{computation} returns \im{\False} and
every \tech{continuation} has the answer type \im{\False}.
However, it becomes difficult to give a compositional interpretation
of \tech{programs} compared to a standard \(\lambda\)-calculus.
In \slang, for example, we define the meaning of \tech{expressions} simply by
evaluation to a \tech{value}.
Using the \tech{double-negation translation}, \tech{CPS} \tech{programs}
\tech{never return}, so we cannot easily define the \tech{value} of an
\tech{expression} by evaluation; we have to complete it first to get a whole
program, then evaluate it, then look at the final state.

The \tech{locally polymorphic answer type} translation makes it difficult to
implement control effects, but easily supports compositional reasoning.
Since each \tech{computation} is \tech{parametric} in its answer type, we are
essentially forced to call the \tech{continuation} exactly once and as the final
step in the \tech{computation}.
If we want to encode control effects, this is a problem.
However, it supports compositional reasoning.
We can locally evaluate \tech{CPS}'d \tech{expressions} and get the meaning of
that \tech{expression} as a \tech{value} by picking a meaningful answer type.
For example, given the \tech{computation}
\im{e : \forall \alpha. (Bool \to \alpha) \to \alpha}, we can either treat it
as a \tech{program} and instantiate the answer type with \im{\False}
(\im{e~\False : (Bool \to \False) \to \False}), or get the meaning of
the \tech{underlying value} by instantiating the answer type with \im{Bool}, the
type of the underlying value, and applying the identity function as
the \tech{continuation} \im{e~Bool~\lambda x.x : Bool}.
The \tech{expression} \im{e~Bool~\lambda x.x} will return the
\tech{underlying value} of \im{e}, supporting compositional reasoning.

The standard translation for \tech{type-preserving} compilation is
the \tech{double-negation translation}~\cite{morrisett1998:ftotal}, but
extending this translation to \tech{dependent types} has proven challenging.
\citet{barthe1999} showed how to scale typed call-by-name
(CBN) CPS translation to a large class of Pure Type Systems (PTSs), including
the Calculus of Constructions (CC) without \(\Sigma\) types.
To avoid certain technical difficulties (which I discuss
in \fullref[]{sec:cps:cbn}), they consider only \emph{domain-free} PTSs, a
variant of PTSs where \im{\lambda} abstractions do not carry the domain of their
bound variable---\ie, they are of the form \im{\lambda x.\, e} instead
of \im{\lambda x:A.\, e} as in the \emph{domain-ful} variant.
\citet{barthe2002} tried to extend these results to the Calculus of
Inductive Constructions (\tech{CIC}), but ended up reporting a negative
result, namely that the \tech{CBN} \tech{CPS} \tech{double-negation translation}
is \emph{not type preserving} for \tech{dependent pairs}.
They go on to prove a general impossibility result: for \tech{dependent
conditionals} (in particular, sum types with dependent case analysis)
\tech{type preservation} is \emph{not possible} when the \tech{CPS}
translation admits unrestricted control effects.

In this chapter, I use the locally polymorphic CPS translation and prove type
preservation for the core features of dependency that were proven ``impossible''
by \citeauthor{barthe2002}.
The key is that the polymorphic CPS uses parametricity to guarantee absence of
control effects.
This first step is an over-approximation; we only need to prevent control
effects in the presence of certain dependencies.
However, it allows us to ignore effects and focus only on dependent-type
preservation.
