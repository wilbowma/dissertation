\newcommand{\FigCBNUnv}[1][t]{
  \begin{figure}[#1]
    \judgshape{\cbnU{\slenv}{\sU}{\cpsU}}
    \begin{mathpar}
      \inferrule*[right=\defcbnUrule{Star}]
      {~}
      {\cbnU{\slenv}{\sstarty}{\cpsstarty}}

      \inferrule*[right=\defcbnUrule{Box}]
      {~}
      {\cbnU{\slenv}{\sboxty}{\cpsboxty}}
    \end{mathpar}
    \caption{\cbnname{} of Universes}
    \label{fig:cps:cbn:unv}
  \end{figure}
}

\newcommand{\FigCBNKind}[1][t]{
  \begin{figure}[#1]
    \judgshape[\text{\fullref[]{lem:cps:cbn:type-pres} will show \im{\cpstyjudg{\slenv^+}{\sK^+}{\sU^+}}}]
    {\cbnK{\slenv}{\sK}{\sU}{\cpsK}}
    \begin{mathpar}
      \inferrule*[right=\defcbnKrule{Ax}]
      {~}
      {\cbnK{\slenv}{\sstarty}{\sboxty}{\cpsstarty}}

      \inferrule*[right=\defcbnKrule{PiK}]
      {\cbnK{\slenv}{\sK}{\sU}{\cpsK} \\
       \cbnK{\slenv,\salpha:\sK}{\sKpr}{\sUpr}{\cpsKpr}}
      {\cbnK{\slenv}{\spity{\salpha}{\sK}{\sKpr}}{\sUpr}{\cpspity{\cpsalpha}{\cpsK}{\cpsKpr}}}

      \inferrule*[right=\defcbnKrule{PiA}]
      {\cbnAdiv{\slenv}{\sA}{\sKpr}{\cpsA} \\
       \cbnK{\slenv,\sx:\sA}{\sK}{\sU}{\cpsK}}
      {\cbnK{\slenv}{\spity{\sx}{\sA}{\sK}}{\sU}{\cpspity{\cpsx}{\cpsA}{\cpsK}}}
    \end{mathpar}
    \caption{\cbnname{} of Kinds}
    \label{fig:cps:cbn:kind}
  \end{figure}
}

\newcommand{\FigCBNTypesFullOne}[1][t]{
  \begin{figure}[#1]
    % Types
    \judgshape[\text{\fullref[]{lem:cps:cbn:type-pres} will show \im{\cpstyjudg{\slenv^+}{\sA^+}{\sK^+}}}]
    {\cbnA{\slenv}{\sA}{\sK}{\cpsA}}
    \begin{mathpar}
      \inferrule*[right=\defcbnArule{Var}]
      {~}
      {\cbnA{\slenv}{\salpha}{\sK}{\cpsalpha}}

      \inferrule*[right=\defcbnArule{Pi}]
      {\cbnAdiv{\slenv}{\sA}{\sK}{\cpsA} \\
       \cbnAdiv{\slenv,\sx:\sA}{\sB}{\sKpr}{\cpsB}}
      {\cbnA{\slenv}{\spity{\sx}{\sA}{\sB}}{\sKpr}{\cpspity{\cpsx}{\cpsA}{\cpsB}}}

      \inferrule*[right=\defcbnArule{PiK}]
      {\cbnK{\slenv}{\sK}{\sU}{\cpsK} \\
       \cbnAdiv{\slenv,\sx:\sA}{\sB}{\sKpr}{\cpsB}}
      {\cbnA{\slenv}{\spity{\salpha}{\sK}{\sB}}{\sKpr}{\cpspity{\cpsalpha}{\cpsK}{\cpsB}}}

      \inferrule*[right=\defcbnArule{Constr}]
      {\cbnAdiv{\slenv}{\sA}{\sKpr}{\cpsA} \\
       \cbnA{\slenv,\sx:\sA}{\sB}{\sK}{\cpsB}}
      {\cbnA{\slenv}{\sfune{\sx}{\sA}{\sB}}{\spity{\sx}{\sA}{\sK}}{\cpsfune{\cpsx}{\cpsA}{\cpsB}}}

      \inferrule*[right=\defcbnArule{Abs}]
      {\cbnK{\slenv}{\sK}{\sU}{\cpsK} \\
       \cbnA{\slenv,\salpha:\sK}{\sB}{\sKpr}{\cpsB}}
      {\cbnA{\slenv}{\sfune{\salpha}{\sK}{\sB}}{\spity{\salpha}{\sK}{\sKpr}}{\cpsfune{\cpsalpha}{\cpsK}{\cpsB}}}

      \inferrule*[right=\defcbnArule{AppConstr}]
      {\cbnA{\slenv}{\sA}{\spity{\sx}{\sB}{\sK}}{\cpsA} \\
       \cbne{\slenv}{\se}{\sB}{\cpse}}
      {\cbnA{\slenv}{\sappe{\sA}{\se}}{\subst{\sK}{\se}{\sx}}{\cpsappe{\cpsA}{\cpse}}}

      \inferrule*[right=\defcbnArule{Inst}]
      {\cbnA{\slenv}{\sA}{\spity{\salpha}{\sKpr}{\sK}}{\cpsA} \\
       \cbnA{\slenv}{\sB}{\sKpr}{\cpsB}}
      {\cbnA{\slenv}{\sappe{\sA}{\sB}}{\subst{\sK}{\sB}{\salpha}}{\cpsappe{\cpsA}{\cpsB}}}

      \inferrule*[right=\defcbnArule{Sig}]
      {\cbnAdiv{\slenv}{\sA}{\sstarty}{\cpsA} \\
       \cbnAdiv{\slenv,\sx:\sA}{\sB}{\sstarty}{\cpsB}}
      {\cbnA{\slenv}{\ssigmaty{\sx}{\sA}{\sB}}{\sstarty}{\cpssigmaty{\cpsx}{\cpsA}{\cpsB}}}

      \inferrule*[right=\defcbnArule{Bool}]
      {~}
      {\cbnA{\slenv}{\sboolty}{\sstarty}{\cpsboolty}}

      \inferrule*[right=\defcbnArule{Let}]
      {\cbne{\slenv}{\se}{\sA}{\cpse} \\
       \cbnAdiv{\slenv}{\sA}{\sK}{\cpsA} \\
       \cbnA{\slenv,\sx=\se:\sA}{\sB}{\sKpr}{\cpsB}}
      {\cbnA{\slenv}{\salete{\sx}{\se}{\sA}{\sB}}{\sKpr}{\cpsalete{\cpsx}{\cpse}{\cpsA}{\cpsB}}}

      \cdots
    \end{mathpar}
    \caption{\cbnname{} of Types (1/2)}
    \label{fig:cps:cbn:types-full1}
  \end{figure}
}

\newcommand{\FigCBNTypesFullTwo}[1][t]{
  \begin{figure}[#1]
    % Types
    \judgshape[\text{\fullref[]{lem:cps:cbn:type-pres} will show \im{\cpstyjudg{\slenv^+}{\sA^+}{\sK^+}}}]
    {\cbnA{\slenv}{\sA}{\sK}{\cpsA}}
    \begin{mathpar}
      \cdots

      \inferrule*[right=\defcbnArule{LetK}]
      {\cbnA{\slenv}{\sA}{\sK}{\cpsA} \\
       \cbnlenv{\slenv}{\cpslenv} \\
       \cbnK{\slenv}{\sA}{\sK}{\cpsK} \\
       \cbnA{\slenv,\salpha=\sA:\sK}{\sB}{\_}{\cpsB}}
      {\cbnA{\slenv}{\salete{\salpha}{\sA}{\sK}{\sB}}{\_}{\cpsalete{\cpsalpha}{\cpsA}{\cpsK}{\cpsB}}}

      \inferrule*[right=\defcbnArule{Conv}]
      {\styjudg{\slenv}{\sA}{\sKpr} \\
       \sequivjudg{\slenv}{\sK}{\sKpr} \\
       \cbnA{\slenv}{\sA}{\sKpr}{\cpsA}}
      {\cbnA{\slenv}{\sA}{\sK}{\cpsA}}
    \end{mathpar}
    \judgshape[\text{\fullref[]{lem:cps:cbn:type-pres} will show \im{\cpstyjudg{\slenv^+}{\sA^\div}{\sstarty^+}}}]
    {\cbnAdiv{\slenv}{\sA}{\sstarty}{\cpsA}}
    \begin{mathpar}
      \inferrule*[right=\defcbnAdivrule{Comp}]
      {\cbnA{\slenv}{\sA}{\sstarty}{\cpsA}}
      {\cbnAdiv{\slenv}{\sA}{\sstarty}{\cpspity{\cpsalpha}{\cpsstarty}{\cpsfunty{(\cpsfunty{\cpsA}{\cpsalpha})}{\cpsalpha}}}}
    \end{mathpar}
    \caption{\cbnname{} of Types (1/2)}
    \label{fig:cps:cbn:types-full}
  \end{figure}
}

\newcommand{\FigCBNTypesShort}[1][t]{
  \begin{figure}[#1]
    % Types
  \judgshape[\text{\fullref[]{lem:cps:cbn:type-pres} will show \im{\cpstyjudg{\slenv^+}{\sA^+}{\sK^+}}}]
    {\cbnA{\slenv}{\sA}{\sK}{\cpsA}}
    \begin{mathpar}
      \inferrule*[right=\defcbnArule{Var}]
      {~}
      {\cbnA{\slenv}{\salpha}{\sK}{\cpsalpha}}

      \inferrule*[right=\defcbnArule{Pi}]
      {\cbnAdiv{\slenv}{\sA}{\sK}{\cpsA} \\
       \cbnAdiv{\slenv,\sx:\sA}{\sB}{\sKpr}{\cpsB}}
      {\cbnA{\slenv}{\spity{\sx}{\sA}{\sB}}{\sKpr}{\cpspity{\cpsx}{\cpsA}{\cpsB}}}

      \inferrule*[right=\defcbnArule{PiK}]
      {\cbnK{\slenv}{\sK}{\sU}{\cpsK} \\
       \cbnAdiv{\slenv,\sx:\sA}{\sB}{\sKpr}{\cpsB}}
      {\cbnA{\slenv}{\spity{\salpha}{\sK}{\sB}}{\sKpr}{\cpspity{\cpsalpha}{\cpsK}{\cpsB}}}

      \inferrule*[right=\defcbnArule{Constr}]
      {\cbnAdiv{\slenv}{\sA}{\sKpr}{\cpsA} \\
       \cbnA{\slenv,\sx:\sA}{\sB}{\sK}{\cpsB}}
      {\cbnA{\slenv}{\sfune{\sx}{\sA}{\sB}}{\spity{\sx}{\sA}{\sK}}{\cpsfune{\cpsx}{\cpsA}{\cpsB}}}

      \inferrule*[right=\defcbnArule{AppConstr}]
      {\cbnA{\slenv}{\sA}{\spity{\sx}{\sB}{\sK}}{\cpsA} \\
       \cbne{\slenv}{\se}{\sB}{\cpse}}
      {\cbnA{\slenv}{\sappe{\sA}{\se}}{\subst{\sK}{\se}{\sx}}{\cpsappe{\cpsA}{\cpse}}}

      \inferrule*[right=\defcbnArule{Sig}]
      {\cbnAdiv{\slenv}{\sA}{\sstarty}{\cpsA} \\
       \cbnAdiv{\slenv,\sx:\sA}{\sB}{\sstarty}{\cpsB}}
      {\cbnA{\slenv}{\ssigmaty{\sx}{\sA}{\sB}}{\sstarty}{\cpssigmaty{\cpsx}{\cpsA}{\cpsB}}}

      \inferrule*[right=\defcbnArule{Let}]
      {\cbne{\slenv}{\se}{\sA}{\cpse} \\
       \cbnAdiv{\slenv}{\sA}{\sK}{\cpsA} \\
       \cbnA{\slenv,\sx=\se:\sA}{\sB}{\sKpr}{\cpsB}}
      {\cbnA{\slenv}{\salete{\sx}{\se}{\sA}{\sB}}{\sKpr}{\cpsalete{\cpsx}{\cpse}{\cpsA}{\cpsB}}}

      \inferrule*[right=\defcbnArule{Conv}]
      {\styjudg{\slenv}{\sA}{\sKpr} \\
       \sequivjudg{\slenv}{\sK}{\sKpr} \\
       \cbnA{\slenv}{\sA}{\sKpr}{\cpsA}}
      {\cbnA{\slenv}{\sA}{\sK}{\cpsA}}

      \cdots
    \end{mathpar}
    \judgshape[\text{\fullref[]{lem:cps:cbn:type-pres} will show \im{\cpstyjudg{\slenv^+}{\sA^\div}{\sstarty^+}}}]
    {\cbnAdiv{\slenv}{\sA}{\sstarty}{\cpsA}}
    \begin{mathpar}
      \inferrule*[right=\defcbnAdivrule{Comp}]
      {\cbnA{\slenv}{\sA}{\sstarty}{\cpsA}}
      {\cbnAdiv{\slenv}{\sA}{\sstarty}{\cpspity{\cpsalpha}{\cpsstarty}{\cpsfunty{(\cpsfunty{\cpsA}{\cpsalpha})}{\cpsalpha}}}}
    \end{mathpar}
    \caption{\cbnname{} of Types (excerpts)}
    \label{fig:cps:cbn:types-short}
  \end{figure}
}

\newcommand{\PairRules}{
  \inferrule*[right=\defcbnerule{Pair}]
  {\cbne{\slenv}{\seone}{\sA}{\cpseone} \\
    \cbne{\slenv}{\setwo}{\subst{\sB}{\seone}{\sx}}{\cpsetwo} \\
    \cbnAdiv{\slenv}{\sA}{\sstarty}{\cpsA} \\
    \cbnAdiv{\slenv,\sx:\sA}{\sB}{\sstarty}{\cpsB}}
  {\cbne{\slenv}{\spaire{\seone}{\setwo}}{\ssigmaty{\sx}{\sA}{\sB}}{
      \begin{stackTL}\cpsfune{\cpsalpha}{\cpsstarty}{
        \cpsfune{\cpsk}{\cpsfunty{\cpssigmaty{\cpsx}{\cpsA}{\cpsB}}{\cpsalpha}}{
          \\\quad\cpsappe{\cpsk}{\cpsdpaire{\cpseone}{\cpsetwo}{\cpssigmaty{\cpsx}{\cpsA}{\cpsB}}}}}}
  \end{stackTL}}

  \inferrule*[right=\defcbnerule{Fst}]
  {\cbnAdiv{\slenv}{\sA}{\sstarty}{\cpsAto{\div}} \\
    \cbnAdiv{\slenv,\sx:\sA}{\sB}{\sstarty}{\cpsBto{\div}} \\
    \cbnA{\slenv}{\sA}{\sstarty}{\cpsAto{+}} \\
    \cbne{\slenv}{\se}{\ssigmaty{\sx}{\sA}{\sB}}{\cpse}}
  {\cbne{\slenv}{\sfste{\se}}{\sA}{
      \begin{stackTL}
        \cpsfune{\cpsalpha}{\cpsstarty}{
          \cpsfune{\cpsk}{\cpsfunty{\cpsAto{+}}{\cpsalpha}}{\\
            \quad\begin{stackTL}
              \cpscappe{\cpse}{\cpsalpha}{(\cpsfune{\cpsy}{\cpssigmaty{\cpsx}{\cpsAto{\div}}{\cpsBto{\div}}}{
                    \cpslete{\cpsz}{\cpsfste{\cpsy}}{\cpsappe{\cpsz}{\cpsalpha~\cpsk}}}
               )}}}
        \end{stackTL}
      \end{stackTL}
    }}

  \inferrule*[right=\defcbnerule{Snd}]
  {\cbnAdiv{\slenv}{\sA}{\sstarty}{\cpsAto{\div}} \\
    \cbnAdiv{\slenv,\sx:\sA}{\sB}{\sstarty}{\cpsBto{\div}} \\
    \cbnA{\slenv,\sx:\sA}{\sB}{\sstarty}{\cpsBto{+}} \\
    \cbne{\slenv}{(\sfste{\se})}{\sA}{(\sfste{\se})^{\div}} \\
    \cbne{\slenv}{\se}{\ssigmaty{\sx}{\sA}{\sB}}{\cpse}}
  {\cbne{\slenv}{\ssnde{\se}}{\subst{\sB}{(\sfste{\se})}{\sx}}{
      \begin{stackTL}
        \cpsfune{\cpsalpha}{\cpsstarty}{
          \cpsfune{\cpsk}{\cpsfunty{\subst{\cpsBto{+}}{(\sfste{\se})^{\div}}{\cpsx}}{\cpsalpha}}{\\
            \quad\begin{stackTL}
              \cpscappe{\cpse}{\cpsalpha}{\begin{stackTL}(\cpsfune{\cpsy}{\cpssigmaty{\cpsx}{\cpsAto{\div}}{\cpsBto{\div}}}{
                    \\\quad\cpslete{\cpsz}{\cpssnde{\cpsy}}{\cpsncappe{\cpsz}{\cpsalpha}{\cpsk}}}
                )\end{stackTL}}}}
        \end{stackTL}
      \end{stackTL}
    }}
}

\newcommand{\CBNEnvRules} {
   \judgshape[\text{\fullref[]{lem:cps:cbn:type-pres} will show \im{\cpswf{\slenv^+}}}]
    {\cbnlenv{\slenv}{\cpslenv}}
    \begin{mathpar}
      \inferrule*[right=\defcbnlenvrule{Empty}]
      {~}
      {\cbnlenv{\cdot}{\cdot}}

      \inferrule*[right=\defcbnlenvrule{AssumT}]
      {\cbnlenv{\slenv}{\cpslenv} \\
       \cbnAdiv{\slenv}{\sA}{\sK}{\cpsA}}
      {\cbnlenv{\slenv,\sx:\sA}{\cpslenv,\cpsx:\cpsA}}

      \inferrule*[right=\defcbnlenvrule{AssumK}]
      {\cbnlenv{\slenv}{\cpslenv} \\
       \cbnK{\slenv}{\sK}{\sU}{\cpsK}}
      {\cbnlenv{\slenv,\salpha:\sK}{\cpslenv,\cpsalpha:\cpsK}}{}

      \inferrule*[right=\defcbnlenvrule{Def}]
      {\cbnlenv{\slenv}{\cpslenv} \\
       \cbnAdiv{\slenv}{\sA}{\sK}{\cpsA} \\
       \cbne{\slenv}{\se}{\sA}{\cpse}}
      {\cbnlenv{\slenv,\sx = \se:\sA}{\cpslenv,\cpsx = \cpse:\cpsA}}

      \inferrule*[right=\defcbnlenvrule{DefT}]
      {\cbnlenv{\slenv}{\cpslenv} \\
       \cbnA{\slenv}{\sA}{\sK}{\cpsA} \\
       \cbnK{\slenv}{\sK}{\sU}{\cpsK}}
      {\cbnlenv{\slenv,\salpha = \sA:\sK}{\cpslenv,\cpsalpha = \cpsA:\cpsK}}
    \end{mathpar}
}

\newcommand{\FigCBNTermsFull}[1][t]{
  \begin{figure}[#1]
  \judgshape[\text{\fullref[]{lem:cps:cbn:type-pres} will show \im{\cpstyjudg{\slenv^+}{\se^\div}{\sA^\div}}}]
    {\cbne{\slenv}{\se}{\sA}{\cpsA}}
    \begin{mathpar}
      \inferrule*[right=\defcbnerule{Var}]
      {\cbnA{\slenv}{\sA}{\sK}{\cpsA}}
      {\cbne{\slenv}{\sx}{\sA}{\cpsfune{\cpsalpha}{\cpsstarty}{
            \cpsfune{\cpsk}{\cpsfunty{\cpsA}{\cpsalpha}}{
              \cpsncappe{\cpsx}{\cpsalpha}{\cpsk}}}}}

      \inferrule*[right=\defcbnerule{Fun}]
      {\cbnAdiv{\slenv}{\sA}{\sK}{\cpsA} \\
       \cbnAdiv{\slenv,\sx:\sA}{\sB}{\sKpr}{\cpsB} \\
       \cbne{\slenv,\sx:\sA}{\se}{\sB}{\cpse}}
      {\cbne{\slenv}{\sfune{\sx}{\sA}{\se}}{\spity{\sx}{\sA}{\sB}}
        {\cpsfune{\cpsalpha}{\cpsstarty}{
            \cpsfune{\cpsk}{\cpsfunty{(\cpspity{\cpsx}{\cpsA}{\cpsB})}{\cpsalpha}}{
                \cpsappe{\cpsk}{(\cpsfune{\cpsx}{\cpsA}{\cpse})}}}}}

      \inferrule*[right=\defcbnerule{Abs}]
      {\cbnK{\slenv}{\sK}{\_}{\cpsK} \\
        \cbnAdiv{\slenv,\salpha:\sK}{\sB}{\_}{\cpsB} \\
        \cbne{\slenv,\salpha:\sK}{\se}{\sB}{\cpse}}
      {\cbne{\slenv}{\sfune{\salpha}{\sK}{\se}}{\spity{\salpha}{\sK}{\sB}}{
          \begin{stackTL}\cpsfune{\cpsalphain{ans}}{\cpsstarty}{
            \cpsfune{\cpsk}{\cpsfunty{(\cpspity{\cpsalpha}{\cpsK}{\cpsB})}{\cpsalphain{ans}}}{
              \\\quad\cpsappe{\cpsk}{(\cpsfune{\cpsalpha}{\cpsK}{\cpse})}}}}
      \end{stackTL}}

      \inferrule*[right=\defcbnerule{App}]
      {\cbne{\slenv}{\se}{\spity{\sx}{\sA}{\sB}}{\cpse} \\
       \cbnAdiv{\slenv}{\sA}{\sK}{\cpsAto{\div}} \\
       \cbnAdiv{\slenv,\sx:\sA}{\sB}{\sKpr}{\cpsBto{\div}} \\
       \cbnA{\slenv,\sx:\sA}{\sB}{\sKpr}{\cpsBto{+}} \\
       \cbne{\slenv}{\sepr}{\sA}{\cpsepr}}
      {\cbne{\slenv}{\sappe{\se}{\sepr}}{\subst{\sB}{\sepr}{\sx}}
        {\begin{stackTL}\cpsfune{\cpsalpha}{\cpsstarty}{
            \cpsfune{\cpsk}{\cpsfunty{(\subst{\cpsBto{+}}{\cpsepr}{\cpsx})}{\cpsalpha}}{\\\quad
              \cpsncappe{\cpse}{\cpsalpha}{(\cpsfune{\cpsf}{\cpspity{\cpsx}{\cpsAto{\div}}{\cpsBto{\div}}}{
                  \cpsncappe{(\cpsappe{\cpsf}{\cpsepr})}{\cpsalpha}{\cpsk})}}}
            \end{stackTL}
          }}}

      \inferrule*[right=\defcbnerule{Inst}]
      {\cbne{\slenv}{\se}{\spity{\salpha}{\sK}{\sB}}{\cpse} \\
        \cbnAdiv{\slenv,\salpha:\sK}{\sB}{\_}{\cpsBto{\div}} \\
        \cbnA{\slenv,\salpha:\sK}{\sB}{\_}{\cpsBto{+}} \\
        \cbnA{\slenv}{\sA}{\sK}{\cpsA}}
      {\cbne{\slenv}{\sappe{\se}{\sA}}{\subst{\sB}{\sA}{\salpha}}
        {\cpsfune{\cpsalphain{ans}}{\cpsstarty}{
            \begin{stackTL}
              \cpsfune{\cpsk}{\cpsfunty{(\subst{\cpsBto{+}}{\cpsA}{\cpsalpha})}{\cpsalphain{ans}}}{\\\quad
                \cpsncappe{\cpse}{\cpsalpha}{(\cpsfune{\cpsf}{\cpspity{\cpsalpha}{\cpsK}{\cpsBto{\div}}}{
                      \cpsncappe{(\cpsappe{\cpsf}{\cpsA})}{\cpsalphain{ans}}{\cpsk})}}}
            \end{stackTL}
          }}}

      \inferrule*[right=\defcbnerule{True}]
      {~}
      {\cbne{\slenv}{\struee}{\sboolty}{\cpsfune{\cpsalpha}{\cpsstarty}{
            \cpsfune{\cpsk}{\cpsfunty{\cpsboolty}{\cpsalpha}}{
              \cpsappe{\cpsk}{\cpstruee}}}}}

      \inferrule*[right=\defcbnerule{False}]
      {~}
      {\cbne{\slenv}{\sfalsee}{\sboolty}{\cpsfune{\cpsalpha}{\cpsstarty}{
            \cpsfune{\cpsk}{\cpsfunty{\cpsboolty}{\cpsalpha}}{
              \cpsappe{\cpsk}{\cpsfalsee}}}}}

      \inferrule*[right=\defcbnerule{If}]
      {\cbne{\slenv}{\se}{\sboolty}{\cpse}\\
       \cbnA{\slenv}{\sB}{\sstarty}{\cpsB}\\
       \cbne{\slenv}{\seone}{\sB}{\cpseone}\\
       \cbne{\slenv}{\setwo}{\sB}{\cpsetwo}}
      {\cbne{\slenv}{\sife{\se}{\seone}{\setwo}}{\sB}{\begin{stackTL}\cpsfune{\cpsalpha}{\cpsstarty}{
            \cpsfune{\cpsk}{\cpsfunty{\cpsB}{\cpsalpha}}{
              \\\quad\cpscappe{\cpse}{\talpha}{(\begin{stackTL}\cpsfune{\cpsx}{\cpsboolty}{
                    \begin{stackTL}
                    \cpsife{\cpsx}{(\cpscappe{\cpseone}{\talpha}{\cpsk})\\\quad\,\,\,\,}{(\cpscappe{\cpsetwo}{\talpha}{\cpsk}))}
                    \end{stackTL}}}}}}
                  \end{stackTL}
                  \end{stackTL}}

      \inferrule*[right=\defcbnerule{Let}]
      {\cbne{\slenv}{\se}{\sA}{\cpse} \\
       \cbnAdiv{\slenv}{\sA}{\sK}{\cpsA} \\
       \cbnA{\slenv,\sx=\se:\sA}{\sB}{\sKpr}{\cpsB} \\
       \cbne{\slenv,\sx=\se:\sA}{\sepr}{\sB}{\cpsepr}}
      {\cbne{\slenv}{\salete{\sx}{\se}{\sA}{\sepr}}{\subst{\sB}{\se}{\sx}}{
          \begin{stackTL}\cpsfune{\cpsalpha}{\cpsstarty}{
            \cpsfune{\cpsk}{\cpsfunty{\subst{\cpsB}{\cpse}{\cpsx}}{\cpsalpha}}{
              \\\quad\cpsalete{\cpsx}{\cpse}{\cpsA}{\cpsncappe{\cpsepr}{\cpsalpha}{\cpsk}}}}}
      \end{stackTL}}

    \end{mathpar}
    \caption{\cbnname{} of Terms (1/2)}
    \label{fig:cps:cbn:terms-full}
  \end{figure}
}

\newcommand{\FigCBNTermsShort}[1][t]{
  \begin{figure}[#1]
  \judgshape[\text{\fullref[]{lem:cps:cbn:type-pres} will show \im{\cpstyjudg{\slenv^+}{\se^\div}{\sA^\div}}}]
    {\cbne{\slenv}{\se}{\sA}{\cpsA}}
    \begin{mathpar}
      \inferrule*[right=\defcbnerule{Var}]
      {\cbnA{\slenv}{\sA}{\sK}{\cpsA}}
      {\cbne{\slenv}{\sx}{\sA}{\cpsfune{\cpsalpha}{\cpsstarty}{
            \cpsfune{\cpsk}{\cpsfunty{\cpsA}{\cpsalpha}}{
              \cpsncappe{\cpsx}{\cpsalpha}{\cpsk}}}}}

      \inferrule*[right=\defcbnerule{Fun}]
      {\cbnAdiv{\slenv}{\sA}{\sK}{\cpsA} \\
       \cbnAdiv{\slenv,\sx:\sA}{\sB}{\sKpr}{\cpsB} \\
       \cbne{\slenv,\sx:\sA}{\se}{\sB}{\cpse}}
      {\cbne{\slenv}{\sfune{\sx}{\sA}{\se}}{\spity{\sx}{\sA}{\sB}}
        {\cpsfune{\cpsalpha}{\cpsstarty}{
            \cpsfune{\cpsk}{\cpsfunty{(\cpspity{\cpsx}{\cpsA}{\cpsB})}{\cpsalpha}}{
                \cpsappe{\cpsk}{(\cpsfune{\cpsx}{\cpsA}{\cpse})}}}}}

      \inferrule*[right=\defcbnerule{App}]
      {\cbne{\slenv}{\se}{\spity{\sx}{\sA}{\sB}}{\cpse} \\
       \cbnAdiv{\slenv}{\sA}{\sK}{\cpsAto{\div}} \\
       \cbnAdiv{\slenv,\sx:\sA}{\sB}{\sKpr}{\cpsBto{\div}} \\
       \cbnA{\slenv,\sx:\sA}{\sB}{\sKpr}{\cpsBto{+}} \\
       \cbne{\slenv}{\sepr}{\sA}{\cpsepr}}
      {\cbne{\slenv}{\sappe{\se}{\sepr}}{\subst{\sB}{\sepr}{\sx}}
        {\begin{stackTL}
          \cpsfune{\cpsalpha}{\cpsstarty}{
            \cpsfune{\cpsk}{\cpsfunty{(\subst{\cpsBto{+}}{\cpsepr}{\cpsx})}{\cpsalpha}}{\\\quad
              \cpsncappe{\cpse}{\cpsalpha}{(\cpsfune{\cpsf}{\cpspity{\cpsx}{\cpsAto{\div}}{\cpsBto{\div}}}{
                  \cpsncappe{(\cpsappe{\cpsf}{\cpsepr})}{\cpsalpha}{\cpsk})}}}
            \end{stackTL}
          }}}

      \PairRules

      \inferrule*[right=\defcbnerule{Let}]
      {\cbne{\slenv}{\se}{\sA}{\cpse} \\
       \cbnAdiv{\slenv}{\sA}{\sK}{\cpsA} \\
       \cbnA{\slenv,\sx=\se:\sA}{\sB}{\sKpr}{\cpsB} \\
       \cbne{\slenv,\sx=\se:\sA}{\sepr}{\sB}{\cpsepr}}
      {\cbne{\slenv}{\salete{\sx}{\se}{\sA}{\sepr}}{\subst{\sB}{\se}{\sx}}{
          \begin{stackTL}\cpsfune{\cpsalpha}{\cpsstarty}{
            \cpsfune{\cpsk}{\cpsfunty{\subst{\cpsB}{\cpse}{\cpsx}}{\cpsalpha}}{
              \\\quad\cpsalete{\cpsx}{\cpse}{\cpsA}{\cpsncappe{\cpsepr}{\cpsalpha}{\cpsk}}}}}
      \end{stackTL}}

      \inferrule*[right=\defcbnerule{Conv}]
      {\cbne{\slenv}{\se}{\sB}{\cpse}}
      {\cbne{\slenv}{\se}{\sA}{\cpse}}

      \cdots
    \end{mathpar}
    \caption{\cbnname{} of Terms (excerpts)}
    \label{fig:cps:cbn:terms-short}
  \end{figure}
}

\newcommand{\FigCBNTermsPairs}[1][t]{
  \begin{figure}[#1]
    \begin{mathpar}
      \inferrule*[right=\defcbnerule{LetK}]
      {\cbnA{\slenv}{\sA}{\sK}{\cpsA} \\
        \cbnK{\slenv}{\sK}{\sU}{\cpsK} \\
        \cbnA{\slenv,\salpha=\sA:\sK}{\sB}{\sKpr}{\cpsB} \\
        \cbne{\slenv,\salpha=\sA:\sK}{\se}{\sB}{\cpse}}
      {\cbne{\slenv}{\salete{\salpha}{\sA}{\sK}{\se}}{\subst{\sB}{\sA}{\salpha}}{
          \begin{stackTL}\cpsfune{\cpsalphain{ans}}{\cpsstarty}{
            \cpsfune{\cpsk}{\cpsfunty{\subst{\cpsB}{\cpsA}{\cpsx}}{\cpsalphain{ans}}}{
              \\\quad\cpsalete{\cpsalpha}{\cpsA}{\cpsK}{\cpsncappe{\cpse}{\cpsalphain{ans}}{\cpsk}}}}}
      \end{stackTL}}

      \inferrule*[right=\defcbnerule{Conv}]
      {\cbne{\slenv}{\se}{\sB}{\cpse}}
      {\cbne{\slenv}{\se}{\sA}{\cpse}}

      \PairRules
    \end{mathpar}
    \caption{\cbnname{} of Terms (2/2)}
    \label{fig:cps:cbn:terms-pairs}
  \end{figure}
}

\newcommand{\FigCBNEnv}[1][t]{
  \begin{figure}[#1]
    \CBNEnvRules
  \caption{\cbnname{} of Environments}
  \label{fig:cps:cbn:env}
  \end{figure}
}

\section{Call-by-Name {{CPS}} Translation}
\label{sec:cps:cbn}
I now present the call-by-name \tech{CPS} translation (\cbnname) of \cpsslang.
There are two main differences compared to the \tech{CBN} \tech{CPS} translation
of \citet{barthe1999}: (1) I use a \tech{locally polymorphic answer type} instead
of the \tech{double-negation translation}, which enables proving
\tech{type-preservation} with \tech{dependent pairs}, and (2) I use a domain-full
target language, which is a necessary step to get decidable type-checking for
\tech{dependently typed} languages.

For CPS, it is helpful to present the translation as a relation over typing
derivations instead of as a function over syntax.
The \tech{CPS} translation requires the type of each \tech{expression} in order
to produce type annotation on continuations.
Furthermore, the translation is not uniformly defined on \tech{expressions}:
\tech{terms} are translated differently than \tech{types}, \tech{types} are
translated either using the \tech{computation translation} or the \tech{value
  translation} depending on the derivation, and \tech{kinds} and
\tech{universes} have separate translations.
The presentation as a relation on typing derivation is verbose, but makes all of
these details explicit and thus easier to follow and understand why the
translation is \tech{type preserving}.
However, as in \fullref[]{sec:cps:ideas}, it is also useful to abbreviate these
as translations over \tech{expressions} for the purposes of examples and proofs.
I continue to use \im{\st^\div} for the \tech{computation translation} and
\im{\st^+} for the \tech{value translation}.
Below I give the abbreviations for all of the translation judgments.
Note that anywhere I use this notation, I require the typing derivation as an
implicit parameter, since formally the translation is defined over typing
derivations.
\begin{mathpar}
  \begin{array}[t]{rcl}
    \sA^{\div} & \defeq & \cpsA~\where{\cbnAdiv{\slenv}{\sA}{\sstarty}{\cpsA}} \\
    \se^{\div} & \defeq & \cpse~\where{\cbne{\slenv}{\se}{\sA}{\cpse}}
  \end{array}

  \begin{array}[t]{rcl}
    \sU^{+} & \defeq & \cpsU~\where{\cbnU{\slenv}{\sU}{\cpsU}} \\
    \sK^{+} & \defeq & \cpsK~\where{\cbnK{\slenv}{\sK}{\sU}{\cpsK}} \\
    \sA^{+} & \defeq & \cpsA~\where{\cbnA{\slenv}{\sA}{\sK}{\cpsA}} \\
  \end{array}
\end{mathpar}

\FigCBNUnv
\FigCBNKind
I start with the simple translations.
The translation on \tech{universes} and \tech{kinds} are essentially homomorphic
on the structure of the typing derivation.
I define the translation for \tech{universes} \cbnUname{} in
\fullref[]{fig:cps:cbn:unv} and \tech{kinds} \cbnKname{} in
\fullref[]{fig:cps:cbn:kind}, both of which I abbreviate with \im{^+}.
There is no separate \tech{computation translation} for \tech{kinds} or
\tech{universes}, which cannot appear as \tech{computations} in \cpsslang.

\FigCBNTypesShort[h!]
I give the key \cbnname{} translations of \tech{types} in
\fullref[]{fig:cps:cbn:types-short}; the full definition is in
\fullref[]{sec:cps:cbn:appendix} \fullref[]{fig:cps:cbn:types-full}.
For \tech{types}, I define a \tech{value translation} \cbnAname{} and a
\tech{computation translation} \cbnAdivname{}.
We need separate \tech*{computation translation}{computation} and \tech{value
  translations} for \tech{types} since we internalize the concept of evaluation
at the \tech{term}-level, and \tech{types} describe \tech{term}-level
\tech{computations} and \tech{term}-level \tech{values}.
Recall that this is the call-by-name translation, so function arguments, even
\tech{type}-level functions, are \tech{computations}.
Note, therefore, that the rule \cbnKrule{PiA} uses the \tech{computation
  translation} on the domain annotation \im{\sA} of
\im{\spity{\sx}{\sA}{\sK}}---\ie, the \tech{kind} describing a \tech{type}-level
function that abstracts over a \tech{term} of \tech{type} \im{\sA}.
Most rules are straightforward.
We translate \tech{type}-level variables \im{\salpha} in-place in \cbnArule{Var}.
Again, since this is the \tech{CBN} translation, we use the \tech{computation
  translation} on domain annotations.
\cbnArule{Constr} for the \tech{value translation} of \tech{type}-level
functions that abstract over a \tech{term}, \im{\sfune{\sx}{\sA}{\sB}},
translates the domain annotation \im{\sA} using the \tech{computation
  translation}.
The rule for the \tech{value translation} of a \tech{dependent function} type,
\cbnArule{Pi}, translates the domain annotation \im{\sA} using the
\tech{computation translation}.
This means that a function is a \tech{value} when it accepts a
\tech{computation} as an argument.
\cbnArule{Sig} produces the \tech{value translation} of a \tech{dependent pair}
type by translating both components of a pair using the \tech{computation
  translation}.
This means we consider a pair a \tech{value} when it contains
\tech{computations} as components.
Note that since the translation is defined on typing derivations, we have an
explicit translation of the conversion rule \cbnArule{Conv}.

There is only one rule for the \tech{computation translation} of a \tech{type},
\cbnAdivrule{Comp}, which is the \tech{polymorphic answer type translation}
described in \fullref[]{sec:cps:ideas}.
Notice that \cbnAdivrule{Comp} is defined only for \tech{types} of \tech{kind}
\im{\sstarty}, since only \tech{types} of kind \im{\sstarty} have inhabitants in
\cpsslang.
For example, we cannot apply \cbnAdivrule{Comp} to \tech{type}-level function since no
\tech{term} inhabits a \tech{type}-level function.

\FigCBNTermsShort[h!]
The main \cbnname{} translation rules for \tech{terms} are given in
\fullref[]{fig:cps:cbn:terms-short}; the full definition is in
\fullref[]{sec:cps:cbn:appendix} \fullref[]{fig:cps:cbn:terms-full} and
\fullref[]{fig:cps:cbn:terms-pairs}.
Intuitively, we translate each \tech{term} \im{\se} of type \im{\sA} to a term
\im{\cpse} of type
\im{\cpspity{\cpsalpha}{\cpsstarty}{\cpsfunty{(\cpsfunty{\cpsA}{\cpsalpha})}{\cpsalpha}}},
where \im{\cpsA} is the \tech{value translation} of \im{\sA}.
This \tech{type} represents a \tech{computation} that, when given a
\tech{continuation} \im{\cpsk} that expects a \tech{value} of type \im{\cpsA},
promises to call \im{\cpsk} with a \tech{value} of type \im{\cpsA}.
Since we have only two \tech{value} forms in the call-by-name translation, we do
not explicitly define a separate \tech{value translation} for \tech{terms}, but
inline that translation.
Note that the \tech{value} cases, \cbnerule{Fun} and \cbnerule{Pair}, feature the same
pattern---\ie, produce a \tech{computation}
\im{\cpsnfune{\cpsalpha}{\cpsnfune{\cpsk}{\cpsappe{\cpsk}{\cpsv}}}} that expects
a \tech{continuation} and then immediately calls that \tech{continuation} on the
\tech{value} \im{\cpsv}.
In the case of \cbnerule{Fun}, the \tech{value} \im{\cpsv} is the function
\im{\cpsfune{\cpsx}{\cpsA}{\cpse}} produced by translating the source function
\im{\sfune{\sx}{\sA}{\se}} using the \tech{computation} \tech{type} translation from
\im{\sA \cbnsymAdiv \cpsA} and the \tech{computation} \tech{term} translation
\im{\se \cbnsyme \cpse}.
In the case of \cbnerule{Pair}, the \tech{value} we produce
\im{\cpspaire{\cpseone}{\cpsetwo}} contains \tech{computations}, not
\tech{values}.

The rest of the \tech{term} translation rules are for \tech{computations}.
Notice that while all \tech{terms} produced by the \tech{term} translation have
a \tech{computation} type, all \tech{continuations} take a \tech{value} type.
Since this is a \tech{CBN} translation, we consider variables as
\tech{computations} in \cbnerule{Var}.
We translate \tech{term} variables as an \(\eta\)-expansion of a \tech{CPS'd}
\tech{computation}.
We must \(\eta\)-expand the variable case to guarantee \tech{CBN} evaluation
order, as I discuss shortly.
In \cbnerule{App}, we encode the \tech{CBN} evaluation order for function
application \im{\sappe{\se}{\sepr}} in the usual way.
We translate the \tech{computations} \im{\se \cbnsyme \cpse} and \im{\sepr
  \cbnsyme \cpsepr}.
First, we evaluate \im{\cpse} to a \tech{value} \im{\cpsf}, then apply
\im{\cpsf} to the \emph{\tech{computation}} \im{\cpsepr}.
The application \im{\cpsappe{\cpsf}{\cpsepr}} is itself a \tech{computation},
which we call with the \tech{continuation} \im{\cpsk}.

Notice that only the translation rules \cbnerule{Fst} and \cbnerule{Snd} use the
new \im{\tfontsym{@}} form.
As discussed in, \fullref[]{sec:cps:ideas}, to type check the translation of
\im{\ssnde{\se}} produced by \cbnerule{Snd}, we require the rule
\refrule{T-Cont} when type checking the \tech{continuation} that performs the
second projection.
While type checking the \tech{continuation,} we know that the \tech{value} \im{\cpsy}
that the \tech{continuation} receives is equivalent to
\im{\cpsncappe{\se^\div}{\cpsalpha}{\cpsidk}}.
The reason we must use the \im{\tfontsym{@}} syntax in the in the translation
\cbnerule{Fst}, whose type is not dependent, is so that we can apply the
\refrule*{eqv-cont}{\im{\equiv}-Cont} rule to resolve the \tech{equivalence} of
the two \emph{first projections} in the type of the second projection.
That is, as we saw in \fullref[]{sec:cps:ideas}, \tech{type preservation} fails because
we must show \tech{equivalence} between \im{(\sfste{\se})^\div} and
\im{\cpsfste{\cpsy}}.
Since these are the only two cases that require our new rules, these are the
only cases where we need to use the \im{\tfontsym{@}} form in the translation;
all other translation rules use standard application.
In \fullref[]{sec:cps:cbv}, I show that that the \tech{CBV} translation must use
the \im{\tfontsym{@}} form much more frequently since, intuitively, the new
\tech{equivalence} rule recovers a notion of ``\tech{value}'' in the
\tech{CPS'd} language, and in call-by-\emph{value} types can only depend on
\tech{values}.

The \tech{CPS} translation encodes the \tech{CBN} evaluation order explicitly so
that the evaluation order of compiled \tech{terms} is independent of the target
language's evaluation order.
This property is not immediately obvious since \cbnerule{Let} binds a
variable \im{\cpsx} to an expression \im{\cpse}, making it seem like there are
two possible evaluation orders: either evaluate \im{\cpse} first, or substitute
\im{\cpse} for \im{\cpsx} first.
Note, however, that the \tech{CBN} translation always produces a
\im{\tfontsym{\lambda}} \tech{term}---even in the variable case since
\cbnerule{Var} employs \(\eta\)-expansion as noted above.
Therefore, in \cbnerule{Let}, \im{\cpse} will always be a \tech{value}, which
means it doesn't evaluate in either \tech{CBN} or \tech{CBV}.
Therefore, there is no ambiguity in how to evaluate the translation of
\im{\sfont{let}}.

The translation rule \cbnerule{Conv} is deceptively simple.
We could equivalently write this translation as follows, which makes its
subtlety apparent.
\begin{displaymath}
\inferrule*[right=\rulename{\cbnename-Conv}]
{\styjudg{\slenv}{\se}{\sB} \\
 \sequivjudg{\slenv}{\sA}{\sB} \\
 \cbne{\slenv}{\se}{\sB}{\cpse} \\
 \cbnA{\slenv}{\sA}{\sK}{\cpsA} \\
 \cbnA{\slenv}{\sB}{\sK}{\cpsB}}
{\cbne{\slenv}{\se}{\sA}{\cpsfune{\cpsalpha}{\cpsstarty}{
      \cpsfune{\cpsk}{\cpsfunty{\cpsA}{\cpsalpha}}{
        \cpsncappe{\cpse}{\cpsalpha}{(\cpsfune{\cpsx}{\cpsB}{\cpsappe{\cpsk}{\cpsx}})}}}}}
\end{displaymath}
Notice now that while the \tech{continuation} \im{\cpsk} expects a \tech{term}
of type \im{\cpsA}, we call \im{\cpsk} with a \tech{term} of type \im{\cpsB}.
Intuitively, \im{\cpsappe{\cpsk}{\cpsx}} should be well-typed since \im{\cpsA}
and \im{\cpsB} should be equivalent.
However, recall from \fullref[]{chp:type-pres} that we cannot prove that
\im{\cpsA} and \im{\cpsB} are equivalence until we prove compositionality, which
requires that we appeal to this translation rule.
As a result, we require equivalence to be non-type-directed.

\FigCBNEnv
I lift the translations to typing environments in the usual way in
\fullref[]{fig:cps:cbn:env}.
Since this is the \tech{CBN} translation, we recur over the environment applying
the \tech{computation translation}.

\subsection{Type Preservation}
\label{sec:cps:cbn:proof}
{
\allowdisplaybreaks
The proof of \tech{type preservation} follows the standard architecture from
\fullref[]{chp:type-pres}.

First I show \fullref{lem:cps:cbn:subst}, which states that the \cbnname{}
translation commutes with substitution.
The formal statement of the lemma is somewhat complicated since we have the
cross product of four syntactic categories and two translations.
However, the intuition is simple: first substituting and then translating is
equivalent to translating and then substituting.

\begin{lemma}[\cbnname{} Compositionality]
  \label{lem:cps:cbn:subst}
  ~
  \begin{multicols}{2}
  \begin{enumerate}
    \item \im{(\subst{\sK}{\sA}{\salpha})^{+} \equiv \subst{\sK^{+}}{\sA^{+}}{\cpsalpha}}
    \item \im{(\subst{\sK}{\se}{\sx})^{+} \equiv \subst{\sK^{+}}{\se^{\div}}{\cpsx}}
    \item \im{(\subst{\sA}{\sB}{\salpha})^{+} \equiv \subst{\sA^{+}}{\sB^{+}}{\cpsalpha}}
    \item \im{(\subst{\sA}{\se}{\sx})^{+} \equiv \subst{\sA^{+}}{\se^{\div}}{\cpsx}}
    \item \im{(\subst{\sA}{\sB}{\salpha})^{\div} \equiv \subst{\sA^{\div}}{\sB^{+}}{\cpsalpha}}
    \item \im{(\subst{\sA}{\se}{\sx})^{\div} \equiv \subst{\sA^{\div}}{\se^{\div}}{\cpsx}}
    \item \im{(\subst{\se}{\sA}{\salpha})^{\div} \equiv \subst{\se^{\div}}{\sA^{+}}{\cpsalpha}}
    \item \im{(\subst{\se}{\sepr}{\sx})^{\div} \equiv \subst{\se^{\div}}{\se^{\sprime\div}}{\cpsx}}
  \end{enumerate}
  \end{multicols}
\end{lemma}
\begin{proof}
  In the PTS syntax, we represent source expressions as \im{\subst{\st}{\stpr}{\sx}}.
  The proof is by induction on the typing derivations for \im{\st}.
  Note that our \im{^\div} and \im{^+} notation implicitly require the typing derivations as
  premises.
  The proof is completely straightforward.
    Part 6 follows immediately by part 3.
    Part 7 follows immediately by part 4.
    I give representative cases for the other parts.
    \begin{proofcases}
    \item \refrule[cpssrc]{*}, parts 1 and 2. Trivial, since no free variables appear in \im{\sstarty}.
    \item \refrule[cpssrc]{Pi-*} and \refrule*[cpssrc]{Pi-Square}{Pi-\im{\square}}: \im{\st = \spity{\sx}{\sB}{\sKpr}}
    \item[{\bfseries Sub-case:}] Part 1. We must show that
      \im{(\subst{(\spity{\sx}{\sB}{\sKpr})}{\sA}{\salpha})^+ =
        \subst{(\spity{\sx}{\sB}{\sKpr})^+}{\sA^+}{\cpsalpha}}.
      \begin{align}
        & (\subst{(\spity{\sx}{\sB}{\sKpr})}{\sA}{\salpha})^+ \nonumber \\
        =~&(\spity{\sx}{\subst{\sB}{\sA}{\salpha}}{\subst{\sKpr}{\sA}{\salpha}})^+ \\
        & \text{by definition of substitution} \nonumber \\
        =~&\cpspity{\cpsxpr}{(\subst{\sB}{\sA}{\salpha})^\div}{(\subst{\sKpr}{\sA}{\salpha})^+} \\
        & \text{by definition of the translation} \nonumber \\
        =~&\cpspity{\cpsxpr}{\subst{\sB^\div}{\sA^+}{\salpha}}{\subst{\sK^{\sprime+}}{\sA^+}{\salpha}}\\
        & \text{by parts 1 and 5 of the induction hypothesis} \nonumber \\
        =~& \subst{(\cpspity{\cpsxpr}{\sB^\div}{\sK^{\sprime+}})}{\sA^+}{\cpsalpha} \\
        & \text{by definition of substitution} \nonumber \\
        =~&\subst{(\spity{\sxpr}{\sB}{\sKpr})^+}{\sA^+}{\cpsalpha} \\
        & \text{by definition of the translation} \nonumber
      \end{align}
    \item[{\bfseries Sub-case:}] Part 2. Similar to the previous sub-case.

    \item \refrule[cpssrc]{Var}
      \item[{\bfseries Sub-case:}] Part 3 \im{\st = \salphapr}. Part 4 is trivial since \im{\sx} is not free in
        \im{\salpha}.

        We must show that \im{(\subst{\salphapr}{\sA}{\salpha})^+ = \subst{\cpsalphapr}{\sA^+}{\cpsalpha}}.

      \item[{\bfseries Sub-sub-case:}] \im{\salpha = \salphapr}. It suffices to show that \im{\sA^+ = \sA^+}, which is trivial.
      \item[{\bfseries Sub-sub-case:}] \im{\salpha \neq \salphapr}. \im{\salpha^+ = \cpsalpha} by definition.

      \item[{\bfseries Sub-case:}] \im{\sx}, parts 7 and 8. Similar to previous case.

      \item \refrule[cpssrc]{App}

      \item[{\bfseries Sub-case:}] \im{\st = \sappe{\seone}{\setwo}}, Part 7

      We must show \im{(\subst{(\sappe{\seone}{\setwo})}{\sApr}{\salphapr})^\div =
        \subst{(\sappe{\seone}{\setwo})^\div}{\sA^{\sprime+}}{\cpsalphapr}}.
      \begin{align}
        & (\subst{(\sappe{\seone}{\setwo})}{\sApr}{\salphapr})^\div \nonumber \\
        =~& (\sappe{\subst{\seone}{\sA}{\salphapr}}{\subst{\setwo}{\sApr}{\salphapr}})^\div \\
        & \text {by definition of substitution} \nonumber \\
        =~& {\begin{stackTL}
            \cpsfune{\cpsalpha}{\cpsstarty}{
              \cpsfune{\cpsk}{\cpsfunty{(\subst{(\subst{\sB}{\sApr}{\salphapr})^+}{(\subst{\setwo}{\sApr}{\salphapr})^\div}{\cpsx})}{\cpsalpha}}{
                \\\quad\cpsncappe{(\subst{\seone}{\sApr}{\salphapr})^\div}{\cpsalpha}{(\begin{stackTL}
                    \cpsfune{\cpsf}{\cpspity{\cpsx}{(\subst{\sA}{\sApr}{\salphapr})^{\div}}{
                       (\subst{\sB}{\sApr}{\salphapr})^{\div}}}{
                     \\\quad\cpsncappe{(\cpsappe{\cpsf}{(\subst{\setwo}{\sApr}{\salphapr})^{\div}})}{\cpsalpha}{\cpsk}})
                 \end{stackTL}}
             }}
         \end{stackTL}} \\
        & \text{by def. of translation} \nonumber \\
        =~& {\begin{stackTL}
            \cpsfune{\cpsalpha}{\cpsstarty}{
              \cpsfune{\cpsk}{\cpsfunty{(\subst{\subst{\sB^+}{\sA^{\sprime+}}{\cpsalphapr}}{\subst{\setwo^\div}{\sA^{\sprime+}}{\cpsalphapr}}{\cpsx})}{\cpsalpha}}{
                \\\quad\cpsncappe{\subst{\seone^{\div}}{\sA^{\sprime+}}{\cpsalphapr}}{\cpsalpha}{(\begin{stackTL}
                    \cpsfune{\cpsf}{\cpspity{\cpsx}{\subst{\sA^\div}{\sA^{\sprime+}}{\cpsalphapr}}{
                        \subst{\sB^{\div}}{\sA^{\sprime+}}{\cpsalphapr}}}{
                      \\\quad\cpsncappe{(\cpsappe{\cpsf}{\subst{\setwo^{\div}}{\sA^{\sprime+}}{\cpsalphapr}})}{\cpsalpha}{\cpsk}})}}}
            \end{stackTL}\end{stackTL}} \\
        & \text{by part 3, 5, and 7 of IH} \nonumber \\
        =~& {(\begin{stackTL}
            \cpsfune{\cpsalpha}{\cpsstarty}{
                \cpsfune{\cpsk}{\cpsfunty{(\subst{\sB^+}{\setwo^\div}{\cpsx})}{\cpsalpha}}{
                \\\quad\cpsncappe{\seone^{\div}}{\cpsalpha}{(\begin{stackTL}
                    \cpsfune{\cpsf}{\cpspity{\cpsx}{\sA^\div}{\sB^{\div}}}{
                      \\\quad\cpsncappe{(\cpsappe{\cpsf}{\setwo^{\div}})}{\cpsalpha}{\cpsk}})}}})
              [{\sA^{\sprime+}}/{\cpsalphapr}]
            \end{stackTL}
          \end{stackTL}
        } \\
        & \text{by definition of substitution} \nonumber \\
        =~& \subst{(\sappe{\seone}{\setwo})^{\div}}{\sA^{\sprime+}}{\cpsalphapr} \\
            & \text{by definition of translation} \nonumber
      \end{align}

      \item[{\bfseries Sub-case:}] Part 8

      We must show \im{(\subst{(\sappe{\seone}{\setwo})}{\se}{\sx})^\div =
        \subst{(\sappe{\seone}{\setwo})^\div}{\se^\div}{\cpsx}}.

      Similar to previous case.

    \item \refrule[cpssrc]{Conv}. The proof is trivial, now that we have staged the proof appropriately. I
give part 8 as an example.

    \im{\inferrule
      {\styjudg{\slenv}{\se}{\sB} \\
        \styjudg{\slenv}{\sA}{\sU} \\
       \sequivjudg{\slenv}{\sA}{\sB}}
      {\styjudg{\slenv}{\se}{\sA}}}

    We must show that \im{(\subst{\se}{\sepr}{\sx})^\div \equiv
      \subst{\se^\div}{\se^{\sprime\div}}{\cpsx}} (at type \im{\sA}).
    Note that by part 8 of the induction hypothesis, we know that
    \im{(\subst{\se}{\sepr}{\sx})^\div \equiv
      \subst{\se^\div}{\se^{\sprime\div}}{\cpsx}} (at the smaller derivation for type \im{\sB}).
    But recall that equivalence is not type directed, so the proof is complete. \qedhere
    \end{proofcases}
\end{proof}

I next prove that the translation preserves \tech{reduction} and
\tech{conversion}, \fullref[]{lem:cps:cbn:pres-red} and
\fullref[]{lem:cps:cbn:pres-red*}.
Note that \tech{kinds} cannot take steps in the \tech{reduction} relation, but
can in the \tech{conversion} relation since it reduces under all contexts.
Note that we can only preserve \tech{reduction} and \tech{conversion} up to
\tech{equivalence}, in particular \(\eta\)-equivalence.
The intuition for this is simple.
The \tech{computation translation} of a term \im{\se^{\sprime\div}} always
produce a \(\lambda\)-expression
\im{\cpsnfune{\cpsalpha}{\cpsnfune{\cpsk}{\cpsepr[2]}}}.
However, when \im{\se^\div \stepstar \cpsepr}, we do not know that the term
\im{\cpsepr} is equal to a \(\lambda\)-expression, although it is
\(\eta\)-equivalent to one.
\begin{lemma}[\cbnname{} Preservation of Reduction]
  \label{lem:cps:cbn:pres-red}
  ~
  \begin{itemize}
    \item If \im{\styjudg{\slenv}{\se}{\sA}} and \im{\se \step \sepr} then
      \im{\se^{\div} \stepstar \cpsepr} and \im{\cpsepr \equiv \se^{\sprime\div}}
    \item If \im{\styjudg{\slenv}{\sA}{\sK}} and \im{\sA \step \sApr} then
      \im{\sA^{+} \stepstar \cpsApr} and \im{\cpsApr \equiv \sA^{\sprime+}}
    \item If \im{\styjudg{\slenv}{\sA}{\sstarty}} and \im{\sA \step \sApr} then
      \im{\sA^{\div} \stepstar \cpsApr} and \im{\cpsApr \equiv \sA^{\sprime\div}}
  \end{itemize}
\end{lemma}
\begin{proof}
  The proof is straightforward by cases on the \tech{reduction} relation.
  I give some representative cases.
  \begin{proofcases}
    \item \im{\sx \step_{\delta} \sepr~\where{\sx = \sepr : \sApr \in \slenv}}

    It suffices to show that \im{\sx^\div \step_{\delta} \se^{\sprime\div}~\where{\sx^\div =
        \se^{\sprime\div} : \sA^{\sprime\div} \in \slenv^+}}, which follows by \cbnlenvrule[refcpsN]{Def}.

    \item \im{\sappe{(\sfune{\sx}{\_}{\seone})}{\setwo} \step_{\beta} \subst{\seone}{\setwo}{\sx}}

    We must show that \im{(\sappe{(\sfune{\sx}{\_}{\seone})}{\setwo})^{\div}
      \stepstar \cpsepr} and \im{\cpsepr \equiv (\subst{\seone}{\setwo}{\sx})^\div}.
    \begin{align}
      & (\sappe{(\sfune{\sx}{\_}{\seone})}{\setwo}) \nonumber \\
      =~& \begin{stackTL}\cpsfune{\cpsalpha}{\cpsstarty}{
          \cpsfune{\cpsk}{\_}{
            \\\quad\cpsncappe{(\cpsfune{\cpsalpha}{\cpsstarty}{
              \cpsfune{\cpsk}{\_}{
                \cpsappe{\cpsk}{(\cpsfune{\cpsx}{\_}{\seone^{\div}})}}})}
          {\cpsalpha}{({\cpsfune{\cpsf}{\_}{\cpsncappe{(\cpsappe{\cpsf}{\setwo^{\div}})}{\cpsalpha}{\cpsk}}})}}}
        \end{stackTL} \\
        & \text{by definition of translation} \nonumber \\
      \stepstar~& \cpsfune{\cpsalpha}{\cpsstarty}{
                  \cpsfune{\cpsk}{\_}{
                  (\cpsncappe{(\cpsappe{(\cpsfune{\cpsx}{\_}{\seone^{\div}})}{\setwo^{\div}})}{\cpsalpha}{\cpsk})}}
                  \\
        & \text{by \im{\step_\beta}} \nonumber \\
      \stepstar ~& \cpsfune{\cpsalpha}{\cpsstarty}{
                   \cpsfune{\cpsk}{\_}{
                   \cpsncappe{(\subst{\seone^{\div}}{\setwo^\div}{\cpsx})}{\cpsalpha}{\cpsk}}} \\
      \equiv ~& \subst{\seone^{\div}}{\setwo^\div}{\cpsx} \\
                & \text{by \refrule*[refcps]{eqv-eta1}{\im{\equiv}-\im{\eta}}} \nonumber \\
      = ~& (\subst{\seone}{\setwo}{\sx})^{\div} \\
           & \text{by \fullref[]{lem:cps:cbn:subst}} \nonumber
    \end{align}
    \item \im{\ssnde{\spaire{\seone}{\setwo}} \step_{\pi_2} \setwo}

    We must show that \im{(\ssnde{\spaire{\seone}{\setwo}})^{\div}
      \stepstar \cpsepr} and \im{\cpsepr \equiv \setwo^\div}.
    \begin{align}
      & (\ssnde{\spaire{\seone}{\setwo}})^\div \nonumber \\
      =~& \begin{stackTL} \cpsfune{\cpsalpha}{\cpsstarty}{
            \cpsfune{\cpsk}{\_}{
              \\\quad\cpscappe{(\cpsfune{\cpsalpha}{\cpsstarty}{\cpsfune{\cpsk}{\_}{
                \cpsappe{\cpsk}{\cpspaire{\seone^\div}{\setwo^\div}}}})}
          {\cpsalpha}{(\cpsfune{\cpsy}{\_}{\cpslete{\cpsz}{\cpssnde{\cpsy}}{\cpsncappe{\cpsz}{\cpsalpha}{\cpsk}}})}}}
      \end{stackTL}
      \\
      \stepstar~& \cpsfune{\cpsalpha}{\cpsstarty}{
                    \cpsfune{\cpsk}{\_}{
                  {\cpslete{\cpsz}{\cpssnde{{\cpspaire{\seone^\div}{\setwo^\div}}}}{\cpsncappe{\cpsz}{\cpsalpha}{\cpsk}}}}}
      \\
      \stepstar~& \cpsfune{\cpsalpha}{\cpsstarty}{
                    \cpsfune{\cpsk}{\_}{
                  {\cpsncappe{\setwo^\div}{\cpsalpha}{\cpsk}}}} \\
      \equiv~& \setwo^\div \quad \text{by \refrule*[refcps]{eqv-eta1}{\im{\equiv}-\im{\eta}}}
    \end{align}
  \end{proofcases}
\end{proof}

\begin{lemma}[\cbnname{} Preservation of Conversion]
  \label{lem:cps:cbn:pres-red*}
  ~
  \begin{itemize}
    \item If \im{\styjudg{\slenv}{\se}{\sA}} and \im{\se \stepstar \sepr} then
      \im{\se^{\div} \stepstar \cpsepr} and \im{\cpsepr \equiv \se^{\sprime\div}}
    \item If \im{\styjudg{\slenv}{\sA}{\sK}} and \im{\sA \stepstar \sApr} then
      \im{\sA^{+} \stepstar \cpsApr} and \im{\cpsApr \equiv \sA^{\sprime+}}
    \item If \im{\styjudg{\slenv}{\sA}{\sstarty}} and \im{\sA \stepstar \sApr} then
      \im{\sA^{\div} \stepstar \cpsApr} and \im{\cpsApr \equiv \sA^{\sprime\div}}
    \item If \im{\styjudg{\slenv}{\sK}{\sU}} and \im{\sK \stepstar \sKpr} then
      \im{\sK^{+} \stepstar \cpsKpr} and \im{\cpsKpr \equiv \sK^{\sprime+}}
  \end{itemize}
\end{lemma}
\begin{proof}
  The proof is straightforward by induction on the \tech{conversion} derivation,
  \ie, on \im{\sstepjudg[\stepstar]{\slenv}{\st}{\stpr}}.\footnote{In the
    previous version of this work~\cite{bowman2018:cps-sigma}, this proof was
    incorrectly stated as by induction on the length of reduction sequences.}
\end{proof}

\begin{lemma}[\cbnname{} Preservation of Equivalence]
  \label{lem:cps:cbn:pres-equiv}
  ~
  \begin{multicols}{2}
  \begin{itemize}
    \item If \im{\se \equiv \sepr} then \im{\se^{\div} \equiv \se^{\sprime\div}}
    \item If \im{\sA \equiv \sApr} then \im{\sA^{+} \equiv \sA^{\sprime+}}
    \item If \im{\sA \equiv \sApr} then \im{\sA^{\div} \equiv \sA^{\sprime\div}}
    \item If \im{\sK \equiv \sKpr} then \im{\sK^{+} \equiv \sK^{\sprime+}}
  \end{itemize}
  \end{multicols}
\end{lemma}
\begin{proof}
  The proof is by induction on the derivation of \im{\se \equiv \sepr}.
  The base case follows by \fullref[]{lem:cps:cbn:pres-red*}, and the cases of \(\eta\)-equivalence follow
  from \fullref[]{lem:cps:cbn:pres-red*}, the induction hypothesis, and the fact that we have the same
  \(\eta\)-equivalence rules in the \cps{}.
\end{proof}

I first prove type and well-formedness preservation,
\fullref[]{lem:cps:cbn:type-pres}, using the explicit syntax on which I defined
\cbnname{}.
In this lemma, proving that the translation of \im{\ssnde{\se}} preserves typing
requires both the new typing rule \refrule{T-Cont} and the equivalence rule
\refrule*{eqv-cont}{\im{\equiv}-Cont}.
The rest of the proof is straightforward.
\begin{lemma}[\cbnname{} Type and Well-formedness Preservation]
  \label{lem:cps:cbn:type-pres}
  ~
  \begin{multicols}{2}
  \begin{enumerate}
  \item If \im{\swf{\slenv}} then \im{\cpswf{\slenv^{+}}}
  \item If \im{\styjudg{\slenv}{\se}{\sA}} then
    \im{\styjudg{\slenv^{+}}{\se^{\div}}{\sA^{\div}}}
  \item If \im{\styjudg{\slenv}{\sA}{\sK}} then
    \im{\cpstyjudg{\slenv^{+}}{\sA^{+}}{\sK^+}}
  \item If \im{\styjudg{\slenv}{\sA}{\sstarty}} then
    \im{\styjudg{\slenv^{+}}{\sA^{\div}}{\sstarty^+}}
  \item If \im{\styjudg{\slenv}{\sK}{\sU}} then
    \im{\cpstyjudg{\slenv^{+}}{\sK^{+}}{\sU^+}}
  \end{enumerate}
  \end{multicols}
\end{lemma}
\begin{proof}
  All cases are proven simultaneously by simultaneous induction on the type derivation and well-formedness
  derivation.
  Part 4 follows easily by part 3 in every case, so I elide its proof.
  Most cases follow easily from the induction hypotheses.
  \begin{proofcases}
  \item Part 5, \refrule[cpssrc]{*}: \im{\styjudg{\slenv}{\sstarty}{\sboxty}}

    We must show that
    \im{\cpstyjudg{\slenv^+}{\sstarty^+}{\sboxty^+}}, which follows by part 1 of the
    induction hypothesis and by definition of the translation, since \im{\sstarty^+ = \cpsstarty}
    and \im{\sboxty^+ = \cpsboxty}.

  \item \refrule[cpssrc]{Pi-*}: \im{\styjudg{\slenv}{\spity{\sx}{\seone}{\setwo}}{\sK}}

      There are two sub-cases: either \im{\setwo} is a \tech{type}, or a \tech{kind}.
    \item[{\bfseries Sub-case:}] Part 3, \im{\setwo = \sB}, \ie, is a \tech{type}.

      There are two sub-cases: either \im{\seone} is a \tech{type} or a
      \tech{kind}.
    \item[{\bfseries Sub-sub-case:}] \im{\seone = \sA}, \ie, is a \tech{type}.

    We have \im{\styjudg{\slenv}{\spity{\sx}{\sA}{\sB}}{\sstarty}}.

    We must show that \im{\cpstyjudg{\slenv^+}{(\spity{\sx}{\sA}{\sB})^+}{\sstarty^+}}

    By definition, must show
    \im{\cpstyjudg{\slenv^+}{\cpspity{\cpsx}{\sA^\div}{\sB^\div}}{\cpsstarty}}, which follows by the
    part 4 of the induction hypothesis applied to \im{\sA} and \im{\sB}.

    \item[{\bfseries Sub-sub-case:}] \im{\seone = \sK}, \ie, is a \tech{kind}.

    We have \im{\styjudg{\slenv}{\spity{\salpha}{\sK}{\sB}}{\sstarty}}.

    We must show that \im{\cpstyjudg{\slenv^+}{(\spity{\salpha}{\sK}{\sB})^+}{\sstarty^+}}

    By definition, must show
    \im{\cpstyjudg{\slenv^+}{\cpspity{\cpsalpha}{\sK^+}{\sB^\div}}{\cpsstarty}}, which follows by the
    part 4 of the induction hypothesis applied to \im{\sB}, and part 5 of the induction hypothesis
    applied to \im{\sK}.

    \item[{\bfseries Sub-case:}] Part 5,  \im{\setwo = \sKpr}, \ie, is a \tech{kind}.

    There are two sub-cases: either \im{\seone} is a \tech{type} or a \tech{kind}.
    \item[{\bfseries Sub-sub-case:}] \im{\seone = \sA}, \ie, is a \tech{type}.

    We have \im{\styjudg{\slenv}{\spity{\sx}{\sA}{\sKpr}}{\sU}}.

    We must show that \im{\cpstyjudg{\slenv^+}{(\spity{\sx}{\sA}{\sKpr})^+}{\sstarty^+}}

    By definition, must show
    \im{\cpstyjudg{\slenv^+}{\cpspity{\cpsx}{\sA^\div}{\sK^{\sprime+}}}{\cpsstarty}}, which follows
    by the part 4 of the induction hypothesis applied to \im{\sA} and part 5 of the induction
    hypothesis applied to \im{\sK}.

    \item[{\bfseries Sub-sub-case:}] \im{\seone = \sK}, \ie, is a \tech{kind}.

    We have \im{\styjudg{\slenv}{\spity{\salpha}{\sK}{\sKpr}}{\sstarty}}.

    We must show that \im{\cpstyjudg{\slenv^+}{(\spity{\salpha}{\sK}{\sKpr})^+}{\sstarty^+}}

    By definition, must show
    \im{\cpstyjudg{\slenv^+}{\cpspity{\cpsalpha}{\sK^+}{\sK^{\sprime+}}}{\cpsstarty}}, which
    follows by the
    part 5 of the induction hypothesis applied to \im{\sK} and \im{\sK^{\sprime+}}.

    \item \refrule*[cpssrc]{Pi-Square}{Pi-\im{\square}} Similar to the previous
      case, except with \im{\sstarty} replaced by \im{\sboxty}; there are two
      fewer cases since this must be a \tech{kind}.

    \item \refrule[cpssrc]{Sig} \im{\styjudg{\slenv}{\ssigmaty{\sx}{\sA}{\sB}}{\sstarty}}

    We must show that
    \im{\cpstyjudg{\slenv^+}{\cpssigmaty{\cpsx}{\sA^\div}{\sB^\div}}{\cpsstarty}}, which
    follows easily by the part 4 of the induction hypothesis applied to \im{\sA} and \im{\sB}.

    \item \refrule[cpssrc]{Pair} \im{\styjudg{\slenv}{\spaire{\seone}{\setwo}}{\ssigmaty{\sx}{\sA}{\sB}}}

    By definition of the translation, we must show that

    \im{
      \cpstyjudg{\slenv^+}{\begin{stackTL}\cpsfune{\cpsalpha}{\cpsstarty}{
            \cpsfune{\cpsk}{(\cpsfunty{\cpssigmaty{\cpsx}{\sA^\div}{\sB^\div}}{\cpsalpha})}{
              \\\quad\cpsappe{\cpsk}{\cpsdpaire{\seone^\div}{\setwo^\div}{\cpssigmaty{\cpsx}{\sA^\div}{\sB^\div}}}}}}
        {
          {\cpspity{\cpsalpha}{\cpsstarty}{\cpsfunty{(\cpsfunty{\cpssigmaty{\cpsx}{\sA^\div}{\sB^\div}}{\cpsalpha})}{\cpsalpha}}}
        }
      \end{stackTL}}

    It suffices to show that
    \im{\styjudg{\slenv^+}{\cpsdpaire{\seone^\div}{\setwo^\div}{\cpssigmaty{\cpsx}{\sA^\div}{\sB^\div}}}{\cpssigmaty{\cpsx}{\sA^\div}{\sB^\div}}},
    which follows easily by part 2 of the induction hypothesis applied to \im{\styjudg{\slenv}{\seone}{\sA}} and
    \im{\styjudg{\slenv}{\setwo}{\subst{\sB}{\seone}{\sx}}}.

  \item \refrule[cpssrc]{Snd} \im{\styjudg{\slenv}{\ssnde{\se}}{\subst{\sB}{\sfste{\se}}{\sx}}}

    We must show that
    \begin{displaymath}
      \begin{stackTL}
          \cpsfune{\cpsalpha}{\cpsstarty}{
            \cpsfune{\cpsk}{\cpsfunty{\subst{\sB^+}{(\sfste{\se})^{\div}}{\cpsx}}{\cpsalpha}}{\\
              \quad\begin{stackTL}
              \cpscappe{\se^\div}{\cpsalpha}{(\cpsfune{\cpsy}{\cpssigmaty{\cpsx}{\sA^\div}{\sB^\div}}{
              \cpslete{\cpsz}{\cpssnde{\cpsy}}{\cpsncappe{\cpsz}{\cpsalpha}{\cpsk}}})}}}
          \end{stackTL}
          \end{stackTL}
    \end{displaymath}
    has type \im{(\subst{\sB}{\sfste{\se}}{\sx})^\div}.

      By part 6 of \fullref[]{lem:cps:cbn:subst}, and definition of the
      translation, this type is equivalent to
      \begin{displaymath}
        {\cpspity{\cpsalpha}{\cpsstarty}{\cpsfunty{(\cpsfunty{\subst{\sB^+}{(\sfste{\se})^{\div}}{\cpsx}}{\cpsalpha})}{\cpsalpha}}}
      \end{displaymath}

      By \refrule[refcps]{Lam}, it suffices to show that
      \begin{displaymath}
      \cpstyjudg{\slenv^+,\cpsalpha:\cpsstarty,\cpsk:\cpsfunty{\subst{\sB^+}{(\sfste{\se})^\div}{\sx}}{\cpsalpha}}
        {\cpscappe{\se^\div}{\cpsalpha}{\begin{stackTL}(\cpsfune{\cpsy}{\cpssigmaty{\cpsx}{\sA^\div}{\sB^\div}}{
                \\\quad\,\cpslete{\cpsz}{\cpssnde{\cpsy}}{\cpsncappe{\cpsz}{\cpsalpha}{\cpsk}}})}
          }{\cpsalpha}
        \end{stackTL}
      \end{displaymath}
    This is the key difficulty in the proof.
    Recall from \fullref[]{sec:cps:ideas} that the term
    \im{\cpsappe{\cpsz}{\cpsalpha}} has type
    \im{\cpsfunty{(\cpsfunty{\subst{\sB^+}{\cpsfste{\cpsy}}{\cpsx}}{\cpsalpha})}{\cpsalpha}}
    while the term \im{\cpsk} has type
    \im{\cpsfunty{\subst{\sB^+}{(\sfste{\se})^\div}{\cpsx}}{\cpsalpha}}.
    To show that \im{\cpsncappe{\cpsz}{\cpsalpha}{\cpsk}} is well-typed, we
    must show that \im{(\sfste{\se})^\div \equiv \cpsfste{\cpsy}}.
    I proceed by the new typing rule \refrule{T-Cont}, which will help us prove this.

    First, note that \im{\cpsncappe{\se^\div}{(\cpssigmaty{\cpsx}{\sA^\div}{\sB^\div})}{\cpsidk}}
    is well-typed.
    By part 4 of the induction hypothesis we know that
    \im{\cpstyjudg{\slenv^+}{\sA^\div}{\cpsstarty}} and
    \im{\cpstyjudg{\slenv^+,\cpsx:\sA^\div}{\sB^\div}{\cpsstarty}}.
    By part 2 of the induction hypothesis applied to
    \im{\styjudg{\slenv}{\se}{\ssigmaty{\sx}{\sA}{\sB}}}, we know
    \im{\cpstyjudg{\slenv^+}{\se^\div}{\cpspity{\cpsalpha}{\cpsstarty}{\cpsfunty{(\cpsfunty{\cpssigmaty{\cpsx}{\sA^\div}{\sB^\div}}{\cpsalpha})}{\cpsalpha}}}}.

    Now, by \refrule{T-Cont}, it suffices to show that
    \begin{displaymath}
    \cpstyjudg{\slenv^+,\cpsalpha:\cpsstarty,\cpsk:\cpsfunty{\subst{\sB^+}{(\sfste{\se})^\div}{\cpsx}}{\cpsalpha}
        ,\cpsy = \cpsappe{\se^\div}{\cpssigmaty{\cpsx}{\sA^\div}{\sB^\div}}{\cpsidk}}
      {\cpslete{\cpsz}{\cpssnde{\cpsy}}{\cpsncappe{\cpsz}{\cpsalpha}{\cpsk}}}{\cpsalpha}
    \end{displaymath}
    Note that we now have the \tech{definitional equivalence} \im{\cpsy =
      {\cpsncappe{\se^\div}{(\cpssigmaty{\cpsx}{\sA^\div}{\sB^\div})}{\cpsidk}}}.
    By \refrule[refcps]{Let} it suffices to show
    \begin{displaymath}
    \cpstyjudg{\slenv^+,\begin{stackTL}\cpsalpha:\cpsstarty,\cpsk:\cpsfunty{\subst{\sB^+}{(\sfste{\se})^\div}{\cpsx}}{\cpsalpha}
        ,\\\cpsy = \cpsncappe{\se^\div}{\cpssigmaty{\cpsx}{\sA^\div}{\sB^\div}}{\cpsidk},\\\cpsz =
        \cpssnde{\cpsy} : \subst{\sB^\div}{\cpsfste{\cpsy}}{\cpsx}
      \end{stackTL}}
      {\cpsncappe{\cpsz}{\cpsalpha}{\cpsk}}{\cpsalpha}
    \end{displaymath}

    Note that
    \begin{align}
      \cpsz :~&\subst{\sB^\div}{\cpsfste{\cpsy}}{\cpsx}\\
       =~&\cpspity{\cpsalpha}{\cpsstarty}{\cpsfunty{(\cpsfunty{\subst{\sB^+}{\cpsfste{\cpsy}}{\cpsx}}{\cpsalpha})}{\cpsalpha}}
      & \text{by definition of \im{\sB^\div}} \\
      \label{line:delta}\equiv~&\cpspity{\cpsalpha}{\cpsstarty}{\cpsfunty{(\cpsfunty{\subst{\sB^+}{\cpsfste{(\cpsncappe{\se^\div}{\_}{\cpsidk})}}{\cpsx}}{\cpsalpha})}{\cpsalpha}}
         & \text{by \im{\delta} reduction on \im{\cpsy}}
    \end{align}

    \fullref[]{line:delta} above, in which we \(\delta\)-reduce \im{\cpsy}, is impossible
    without \refrule{T-Cont}.

    By \refrule[refcps]{Conv}, and since \im{\cpsk :
      \cpsfunty{\subst{\sB^+}{(\sfste{\se})^\div}{\cpsx}}{\cpsalpha}},
     to show \im{\cpsncappe{\cpsz}{\cpsalpha}{\cpsk} : \cpsalpha} it suffices to show that
    \im{(\sfste{\se})^\div \equiv \cpsfste{(\cpsncappe{\se^\div}{\_}{\cpsidk})}}.

    Note that \im{
      (\sfste{\se})^\div = \begin{stackTL}
        \cpsfune{\cpsalpha}{\cpsstarty}{\cpsfune{\cpskpr}{(\cpsfunty{\sA^+}{\cpsalpha})}{
            \\\quad\cpscappe{\se^\div}{\cpsalpha}{(\cpsfune{\cpsy}{\cpssigmaty{\cpsx}{\sA^\div}{\sB^\div}}{
                \cpsalete{\cpszpr}{\cpsfste{\cpsy}}{\sA^\div}{\cpsncappe{\cpszpr}{\cpsalpha}{\cpskpr}}})}}}
        \end{stackTL}}
    by definition of the translation.

    By \refrule*[refcps]{eqv-eta1}{\im{\equiv}-\im{\eta}}, it suffices to show that
    \begin{align}
      & \cpscappe{\se^\div}{\cpsalpha}{(\cpsfune{\cpsy}{\cpssigmaty{\cpsx}{\sA^\div}{\sB^\div}}
        \cpsalete{\cpszpr}{\cpsfste{\cpsy}}{\sA^\div}{\cpsncappe{\cpszpr}{\cpsalpha}{\cpskpr}})} \\
      \label{line:cont} & \equiv \cpsappe{(\cpsfune{\cpsy}{\cpssigmaty{\cpsx}{\sA^\div}{\sB^\div}}{
        \cpslete{\cpszpr}{\cpsfste{\cpsy}}{\cpsncappe{\cpszpr}{\cpsalpha}{\cpskpr}})}}{(\cpsncappe{\se^\div}{\_}{\cpsidk})}
      & \text{\refrule*{eqv-cont}{\im{\equiv}-Cont}} \\
      & \equiv \cpsncappe{(\cpsfste{(\cpsncappe{\se^\div}{\_}{\cpsidk})})}{\cpsalpha}{\cpskpr}
        & \text{by reduction}
    \end{align}
    Notice that \fullref[]{line:cont} requires
    \refrule*{eqv-cont}{\im{\equiv}-Cont} applied to the translation of the
    \im{\sfont{fst}}.
  \item \refrule[cpssrc]{Lam}

    I give proofs for only the \tech{term}-level functions; the \tech{type}-level functions follow
    exactly the same structure as \tech{type}-level function types.
    There are two subcases.

    \begin{proofcases}
    \item[{\bfseries Sub-case:}] The function abstracts over a term, \im{\styjudg{\slenv}{\sfune{\sx}{\sA}{\se}}{\spity{\sx}{\sA}{\sB}}}

    We must show

    \im{\cpstyjudg{\slenv^+}{(\sfune{\sx}{\sA}{\se})^\div}{(\spity{\sx}{\sA}{\sB})^\div}}.

    By definition of the translation, we must show

    \im{\cpstyjudg{\slenv^+}{
        \begin{stackTL}
          \cpsfune{\cpsalpha}{\cpsstarty}{\cpsfune{\cpsk}{\cpsfunty{(\cpspity{\cpsx}{\sA^\div}{\sB^{\div}})}{\cpsalpha}}
            {\\\quad(\cpsappe{\cpsk}{(\cpsfune{\cpsx}{\sA^\div}{\se^\div})})}}}
        {\cpspity{\cpsalpha}{\cpsstarty}{
            \cpsfunty{(\cpsfunty{\cpspity{\cpsx}{\sA^\div}{\sB^\div}}{\cpsalpha})}{\cpsalpha}}}
        \end{stackTL}
      }

    It suffices to show that

    \im{\cpstyjudg{\slenv^+,\cpsalpha:\cpsstarty,
        \cpsk:\cpsfunty{\cpspity{\cpsx}{\sA^\div}{\sB^{\div}}}{\cpsalpha}}
      {\cpsappe{\cpsk}{(\cpsfune{\cpsx}{\sA^\div}{\se^\div})}}{\cpsalpha}}.

    By \refrule[refcps]{App}, it suffices to show that

    \im{\cpstyjudg{\slenv^+,\cpsalpha:\cpsstarty,
        \cpsk:\cpsfunty{\cpspity{\cpsx}{\sA^\div}{\sB^{\div}}}{\cpsalpha}}
      {(\cpsfune{\cpsx}{\sA^\div}{\se^\div})}{\cpspity{\cpsx}{\sA^\div}{\sB^{\div}}}}

    By part 2 of the induction hypothesis applied to
    \im{\styjudg{\slenv,\sx:\sA}{\se}{\sB}}, we know that

    \im{\cpstyjudg{\slenv^+,\cpsx:\sA^\div}{\se^\div}{\sB^\div}}

    It suffices to show that

    \im{\cpswf{\slenv^+,\cpsalpha:\cpsstarty,
        \cpsk:\cpsfunty{\cpspity{\cpsx}{\sA^\div}{\sB^{\div}}}{\cpsalpha}}}
    which follows easily by part 4 of the induction hypothesis applied to the typing derivations for
    \im{\sA} and \im{\sB}.

    \item[{\bfseries Sub-case:}] The function abstracts over a type,
    \im{\styjudg{\slenv}{\sfune{\salpha}{\sK}{\se}}{\spity{\salpha}{\sK}{\sB}}}

    We must show
    \im{\cpstyjudg{\slenv^+}{(\sfune{\salpha}{\sK}{\se})^\div}
      {(\spity{\salpha}{\sK}{\sB})^\div}}.

    By definition of the translation, we must show that

    \im{\cpstyjudg{\slenv^+}{\begin{stackTL}
          \cpsfune{\cpsalphain{ans}}{\cpsstarty}{\cpsfune{\cpsk}{\cpsfunty{(\cpspity{\cpsalpha}{\sK^+}{\sB^{\div}})}{\cpsalpha}}
            {\\\quad(\cpsappe{\cpsk}{(\cpsfune{\cpsalpha}{\sK^+}{\se^\div})})}}}
        {\cpspity{\cpsalphain{ans}}{\cpsstarty}{
            \cpsfunty{(\cpsfunty{\cpspity{\cpsalpha}{\sK^+}{\sB^\div}}{\cpsalphain{ans}})}{\cpsalphain{ans}}}}
        \end{stackTL}
      }

    It suffices to show that

    \im{\cpstyjudg{\slenv^+,\cpsalphain{ans}:\cpsstarty,
        \cpsk:\cpsfunty{\cpspity{\cpsalpha}{\sK^+}{\sB^{\div}}}{\cpsalphain{ans}}}
      {\cpsappe{\cpsk}{(\cpsfune{\cpsalpha}{\sK^+}{\se^\div})}}{\cpsalphain{ans}}}.

    By \refrule[refcps]{App}, it suffices to show that

    \im{\cpstyjudg{\slenv^+,\cpsalphain{ans}:\cpsstarty,
        \cpsk:\cpsfunty{\cpspity{\cpsalpha}{\sK^+}{\sB^{\div}}}{\cpsalphain{ans}}}
      {(\cpsfune{\cpsx}{\sA^\div}{\se^\div})}{\cpspity{\cpsalpha}{\sK^+}{\sB^{\div}}}}

    By part 2 of the induction hypothesis applied to
    \im{\styjudg{\slenv,\salpha:\sK}{\se}{\sB}}, we know that

    \im{\cpstyjudg{\slenv^+,\cpsalpha:\sK^+}{\se^\div}{\sB^\div}}

    It suffices to show that
    \im{\cpswf{\slenv^+,\cpsalphain{ans}:\cpsstarty,
        \cpsk:\cpsfunty{\cpspity{\cpsalpha}{\sK^+}{\sB^{\div}}}{\cpsalphain{ans}}}}
    which follows easily by parts 5 and 4 of the induction hypothesis applied to the typing derivations for
    \im{\sK} and \im{\sB}.
    \end{proofcases}

    \item \refrule[cpssrc]{App}
      \begin{proofcases}
    \item[{\bfseries Sub-case:}] A \tech{term}-level function applied to a term \im{\styjudg{\slenv}{\sappe{\seone}{\setwo}}{\subst{\sB}{\setwo}{\sx}}}

    We must show that

    \im{\styjudg{\slenv^+}{(\sappe{\seone}{\setwo})^\div}{(\subst{\sB}{\setwo}{\sx})^\div}}

    By definition of the translation, we must show:

    \im{\styjudg{\slenv^+}{\begin{stackTL}\cpsfune{\cpsalpha}{\cpsstarty}{\cpsfune{\cpsk}{\cpsfunty{(\subst{\sB}{\setwo}{\sx})^+}{\cpsalpha}}{
            \\\quad\,\cpsncappe{\seone^{\div}}{\cpsalpha}{(\cpsfune{\cpsf}{\cpspity{\cpsx}{\sA^{\div}}{\sB^{\div}}}{\cpsncappe{(\cpsappe{\cpsf}{\setwo^{\div}})}{\cpsalpha}{\cpsk}})}}}}{(\subst{\sB}{\setwo}{\sx})^\div}\end{stackTL}}

    By part 6 of \fullref[]{lem:cps:cbn:subst} and definition of \im{\sB^{\div}}, we must show:

    %% NB: Crazy manual formatting and spacing
    \im{\styjudg{\slenv^+}{\cpsfune{\cpsalpha}{\cpsstarty}{\begin{stackTL}
            \cpsfune{\cpsk}{\cpsfunty{(\subst{\sB^+}{\setwo^{\div}}{\cpsx})}{\cpsalpha}}{
              \\\quad \cpsncappe{\seone^{\div}}{\cpsalpha}{
                (\begin{stackTL}\cpsfune{\cpsf}{\cpspity{\cpsx}{\sA^{\div}}{\sB^{\div}}}{
                    \\\quad\,\cpsncappe{(\cpsappe{\cpsf}{\setwo^{\div}})}{\cpsalpha}{\cpsk}}
                  )~~~~}}}}
        {\cpspity{\cpsalpha}{\cpsstarty}{
            \cpsfunty{(\cpsfunty{\subst{\sB^+}{\setwo^{\div}}{\cpsx}}{\cpsalpha})}{\cpsalpha}}}
                \end{stackTL}
          \end{stackTL}
    }

    It suffices to show that
    \begin{itemize}
      \item \im{\cpstyjudg{\slenv^+}{\subst{\sB^+}{\setwo^{\div}}{\cpsx}}{\cpsK}}
        By part 3 of the induction hypothesis we know that
        \im{\cpstyjudg{\slenv^+,\cpsx:\sA^\div}{\sB^+}{\cpsK}}, and by part 2 of the induction
        hypothesis we know that \im{\cpstyjudg{\slenv^+}{\setwo^{\div}}{\sA^{\div}}}, hence the
        goal follows by substitution.
      \item
        \im{\cpstyjudg{\slenv^+}{\seone^\div}{\cpspity{\cpsalpha}{\cpsstarty}{\cpsfunty{(\cpsfunty{\cpspity{\cpsx}{\sA^{\div}}{\sB^\div}}{\cpsalpha})}{\cpsalpha}}}},
        which follows by part 2 of the induction hypothesis and by definition of \im{(\spity{\sx}{\sA}{\sB})^{\div}}.
      \item \im{\cpstyjudg{\slenv^+,\cpsalpha:\cpsstarty,\cpsk:\cpsfunty{(\subst{\sB^+}{\setwo^{\div}}{\cpsx})}{\cpsalpha}}{(\cpsfune{\cpsf}{\cpspity{\cpsx}{\sA^{\div}}{\sB^{\div}}}{
                  \cpsncappe{(\cpsappe{\cpsf}{\setwo^{\div}})}{\cpsalpha}{\cpsk}})}{\cpsfunty{\cpspity{\cpsx}{\sA^{\div}}{\sB^\div}}{\cpsalpha}}},
            which follows since by part 2 of the induction hypothesis \im{\setwo^\div:\sA^\div} we
            know \im{(\cpsappe{\cpsf}{\setwo^{\div}}) : \subst{\sB^{\div}}{\setwo^\div}{\cpsx}}
            and by definition \im{\subst{\sB^{\div}}{\setwo^\div}{\cpsx} = \cpspity{\cpsalpha}{\cpsstarty}{\cpsfunty{(\cpsfunty{\subst{\sB^+}{\setwo^{\div}}{\cpsx}}{\cpsalpha})}{\cpsalpha}}}.
    \end{itemize}

    \item[{\bfseries Sub-case:}] A \tech{term}-level function applied to a type \im{\styjudg{\slenv}{\sappe{\seone}{\sA}}{\subst{\sB}{\sA}{\salpha}}}

    The proof is similar to the previous case, but relies on showing that
    \im{\cpstyjudg{\slenv^+}{\sA^+}{\sK^+}}, which follows by part 3 of the induction hypothesis.

    \item[{\bfseries Sub-case:}] A type-level function applied to a term \im{\styjudg{\slenv}{\sappe{\sA}{\se}}{\subst{\sK}{\se}{\sx}}}

    This case is straightforward by the part 3 and part 2 of the induction hypothesis.

    \item[{\bfseries Sub-case:}] A type-level function applied to a type \im{\styjudg{\slenv}{\sappe{\sA}{\sB}}{\subst{\sK}{\sB}{\salpha}}}

    This case is straightforward by the part 3 of the induction hypothesis.
    \end{proofcases}

  \item \refrule[cpssrc]{Conv} \im{\styjudg{\slenv}{\se}{\sA}} such that
  \im{\styjudg{\slenv}{\se}{\sB}} and \im{\sA \equiv \sB}.

  We must show that
  \im{\se^\div} has type \im{\sA^\div = \cpspity{\cpsalpha}{\cpsstarty}{\cpsfunty{(\cpsfunty{\sA^+}{\cpsalpha})}{\cpsalpha}}}.

  By the induction hypothesis, we know that \im{\se^\div : {\sB^\div =
      \cpspity{\cpsalpha}{\cpsstarty}{\cpsfunty{(\cpsfunty{\sB^+}{\cpsalpha})}{\cpsalpha}}}}.
  By \refrule[refcps]{Conv} it suffices to show that \im{\sA^+ \equiv \sB^+}, which follows
  by \fullref[]{lem:cps:cbn:pres-equiv}. \qedhere
  \end{proofcases}
\end{proof}

To recover a simple statement of the \tech{type-preservation} theorem over the
\tech{PTS} syntax, I define two meta-functions for translating
\tech{expressions} depending on their use.
I define \im{\cpsterm{\st}} to translate a \tech{PTS} \tech{expression} in
``\tech{term}'' position, \ie, when used on the left side of a type annotation as in
\im{\st : \stpr}, and define \im{\cpstype{\stpr}} to translate an \tech{expression}
in ``\tech{type}'' position, \ie, when used on the right side of a type annotation.
I define these in terms of the translation shown above, noting that for every
\im{\st : \stpr} in the \tech{PTS} syntax, one of the following is true:
\im{\st} is a \tech{term} \im{\se} and \im{\stpr} is a \tech{type} \im{\sA} in
the explicit syntax; \im{\st} is a \tech{type} \im{\sA} and \im{\stpr} is a
\tech{kind} \im{\sK} in the explicit syntax; or \im{\st} is a \tech{kind}
\im{\sK} and \im{\stpr} is a \tech{universe} \im{\sU} in the explicit syntax.
\begin{displaymath}
  \begin{array}{rcl}
    \cpsterm{\st} & \defeq & \se^{\div}\text{ when \im{\st} is a term} \\
    \cpsterm{\st} & \defeq & \sA^{+}\text{ when \im{\st} is a type} \\
    \cpsterm{\st} & \defeq & \sK^{+}\text{ when \im{\st} is a kind} \\
  \end{array}
  \qquad\qquad\!\!\!\!\!
  \begin{array}{rcl}
    \cpstype{\stpr} & \defeq & \sA^{\div}\text{ when \im{\stpr} is a type} \\
    \cpstype{\stpr} & \defeq & \sK^{+}\text{ when \im{\stpr} is a kind} \\
    \cpstype{\stpr} & \defeq & \sU^+\text{ when \im{\stpr} is a universe} \\
  \end{array}
\end{displaymath}
This notation is based on \citet{barthe2002}.

\begin{theorem}[\cbnname{} Type Preservation]
  \label{thm:cps:cbn:type-pres}
  \im{\styjudg{\slenv}{\st}{\stpr}} then
  \im{\styjudg{\slenv^{+}}{\cpsterm{\st}}{\cpstype{\stpr}}}
\end{theorem}

\subsection{Compiler Correctness}
\label{sec:cps:cbn:correct}
Recall from \fullref[]{chp:type-pres} that since we preserve
\tech{conversion}, proving compiler correctness is simple.
I use the standard definition of linking by substitution and the standard
cross-language relation.

I extend the \cbnname{} translation in a straightforward way to translate closing
substitutions, written \im{\ssubst^\div}, and allow translated \tech{terms} to
be linked with the translation of any valid closing substitution \im{\ssubst}.
This definition supports a separate compilation theorem that allows linking with
the output of this translation, but not with the output of other compilers.

\FigCPSProg
\FigCPSEval
Now I can show that the \cbnname{} translation is correct with respect to
separate compilation---if we first link and run to a \tech{value}, we get a related
\tech{value} when we compile and then link with the compiled closing
substitution.
I first define well-formed programs and the evaluation function for \cpstlang.
These are different than described in \fullref[]{chp:source}.
I present well-formed programs and components in \fullref[]{fig:cps:wfprog}.
CPS programs are well-formed when they are computations that produce a ground
values, \ie, when they expect a continuation that expects a value of type
\im{\cpsboolty}.
The evaluation function is given in \fullref[]{fig:cps:eval}
Since the target language is in \tech{CPS}, we must apply the halt continuation
\im{\cpsidk} at the top-level to evaluate a program to an observation.
\begin{theorem}[Separate Compilation Correctness]
  \label{thm:cps:cbn:sep-comp}
  If \im{\wf{\slenv}{\se}} and
  \im{\ssubstok{\slenv}{\ssubst}},
  then\\% manual break
  \im{\seval{\ssubst(\se)} \approx \teval{\ssubst^+(\se^\div)}}.
\end{theorem}
\begin{proof}
  Since \tech{conversion} implies \tech{equivalence}, we reason in terms of
  \tech{equivalence}.
  By \fullref[]{lem:cps:cbn:pres-red*}, \im{(\ssubst(\se))^\div \stepstar \cpse} and \im{\sv^\div
    \equiv \cpse}.
  By \fullref[]{lem:cps:cbn:subst}, \im{(\ssubst(\se))^\div \equiv
    \ssubst^{\div}(\se^\div)}, hence \im{\ssubst^{\div}(\se^\div) \stepstar \cpse} and
  \im{\sv^\div \equiv \cpse}.
  Since the translation on all \tech{observations} is \im{\sv^\div =
    \cpsnfune{\cpsalpha}{\cpsnfune{\cpsk}{\cpsappe{\cpsk}{\cpsv}}}}, where \im{\sv \approx \cpsv},
  we know \im{\cpsncappe{\sv^\div}{\sA^+}{\cpsidk} \stepstar \cpsv} such that \im{\sv \approx \cpsv}.
  Since \im{\sv^\div \equiv \cpse \equiv \ssubst^{\div}(\se^\div)}, we also know that
  \im{\cpsappe{\ssubst^{\div}(\se^\div)}{\sA^+\cpsidk} \stepstar \cpsvpr} and \im{\cpsvpr \equiv
    \cpsv}.
  Since \im{\cpsv} is an \tech{observation}, \im{\cpsvpr = \cpsv} and \im{\sv \approx \cpsvpr}.
\end{proof}

\begin{corollary}[Whole-Program Correctness]
  \label{thm:cps:cbn:correctness}
  If \im{\wf{}{\se}} then
  \im{\seval{\se} \approx \teval{\se^\div}}.
\end{corollary}
}
