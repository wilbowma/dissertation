\section{Related and Future Work}
\label{sec:cps:related}

\paragraph{Dependent Conditionals}
In addition to showing that the traditional double-negation CPS for \(\Sigma\)
types is not type preserving, \citet{barthe2002} demonstrate an impossibility
result for CPS and dependent conditionals.
They prove that no type-correct CPS translation can exist for sums with
dependent case.
However, careful inspection of their proof reveals that this impossibility
result only applies if the CPS translation allows unrestricted control effects.
As my translation does not allow control effects, I conjecture it is possible to
prove type preservation for dependent conditionals.

The impossibility result by \citeauthor{barthe2002} relies on the ability to
implement \texttt{call/cc} via the CPS translation.
Assuming there is a type-preserving CPS translation, they construct a model of CoC extended with
\texttt{call/cc} and sum types (CC\(\Delta+\)) in CoC with sum types (CoC+).
Since CoC+ is consistent, this model proves that CoC\(\Delta+\) is consistent.
However, it is known that CoC\(\Delta+\) is inconsistent~\cite{coquand1989}.
Therefore, the type-preserving CPS translation of CoC\(\Delta\)+ cannot exist.

Their proof is valid; however, it is well known that the polymorphic answer type
CPS translation that I use cannot implement
\texttt{call/cc}~\cite{ahmed2011}.
Therefore, my translation does not give a model of CoC\(\Delta+\) in CoC+.

I further conjecture that my CPS translation is type preserving for dependent
conditionals.
Since the original work by \citeauthor{barthe2002} uses sums with dependent
case, I use these instead of the dependent if presented in
\fullref[]{chp:source}, but the same core idea applies to either implementation
of dependent conditionals.
The typing rule for dependent case is the following.
\begin{displaymath}
  \inferrule
  {\styjudg{\slenv,\sy:\ssumty{\sA}{\sB}}{\sApr}{\sstarty} \\
   \styjudg{\slenv}{\se}{\ssumty{\sA}{\sB}} \\
   \styjudg{\slenv,\sx:\sA}{\seone}{\subst{\sApr}{\ssume{1}{\sx}}{\sy}} \\
   \styjudg{\slenv,\sx:\sB}{\setwo}{\subst{\sApr}{\ssume{2}{\sx}}{\sy}}
 }
  {\styjudg{\slenv}{\scasee{\se}{\sx}{\seone}{\sx}{\setwo}}{\subst{\sApr}{\se}{\sy}}}
\end{displaymath}

I would extend the \cbnname{} with the following rule.
\begin{displaymath}
  (\scasee{\se}{\sx}{\seone}{\sx}{\setwo})^\div =
  \begin{stackTL}
    \cpsfune{\cpsalpha}{\cpsstarty}{
      \cpsfune{\cpsk}{\cpsfunty{\subst{\sA^{\sprime+}}{\se^\div}{\cpsy}}{\cpsalpha}}{
        \\\quad
        \cpscappe{\se^\div}{\cpsalpha}{(\cpsfune{\cpsy}{\cpssumty{\sA^+}{\sB^+}}{
          \cpscasee{\cpsy}{\cpsx}{(\cpsappe{\seone^\div}{\cpsalpha~\cpsk})}{\cpsx}{(\cpsappe{\setwo^\div}{\cpsalpha~\cpsk})}})}}}
  \end{stackTL}
\end{displaymath}

Focusing on the first branch of the \im{\tfont{case}} above, note that
\im{\cpsappe{\seone^\div}{\cpsalpha} : 
  \cpsfunty{(\cpsfunty{\subst{\sA^{\sprime+}}{(\ssume{1}{\sx})^\div}{\cpsy}}{\cpsalpha})}{\cpsalpha}},
however \im{\cpsk : {\cpsfunty{\subst{\sA^{\sprime+}}{\se^\div}{\cpsy}}{\cpsalpha}}}.
We need to show that \im{\se^\div \equiv (\ssume{1}{\sx})^\div}, similar to the problem with
the second projection of dependent pairs.  This time, however, to show
\im{\se^\div \equiv (\ssume{1}{\sx})^\div} we need to reason
about the flow of the underlying value of $\se^\div$ into $\cpsy$
and also about the relationship between $\cpsy$ and $\cpsx$.
Specifically, we need to first use my new \rulename{T-Cont} rule, which
allows us to assume \im{\cpsy =
  \cpsappe{\se^\div}{\cpsalpha~\cpsidk}}.  Next, 
we need to know that since the case analysis is on the value $\cpsy$,
in the first branch \im{\cpsy \equiv \cpssume{1}{\cpsx}} (and similarly
for the other branch), but the problem is that the existing typing
rule for dependent case does not let us assume that.

Nonetheless, I conjecture the same extension I propose in \fullref[]{chp:anf}
works to allow type preservation for CPS.
We simply change the typing rule for case to record the equality \im{\se \equiv
  \ssume{1}{\sx}} while typing the first branch, and similarly for the second
branch, like in the following rule.
\begin{displaymath}
  \inferrule
  {\styjudg{\slenv,\sy:\ssumty{\sA}{\sB}}{\sApr}{\sstarty} \\
   \styjudg{\slenv}{\se}{\ssumty{\sA}{\sB}} \\
   \styjudg{\slenv,\sx:\sA,\sp:\se = \ssume{1}}{\seone}{\subst{\sApr}{\ssume{1}{\sx}}{\sy}} \\
   \styjudg{\slenv,\sx:\sB,\sp:\se = \ssume{2}}{\setwo}{\subst{\sApr}{\ssume{2}{\sx}}{\sy}}
 }
  {\styjudg{\slenv}{\scasee{\se}{\sx}{\seone}{\sx}{\setwo}}{\subst{\sApr}{\se}{\sy}}}
\end{displaymath}
With this modification in the target language \cps{}, we can type check the
above translation of dependent case.
I have not yet shown the consistency of the target language \cps{} once it is
extended with this modified typing rule for dependent case, primarily due to
problems scaling the CPS translation to higher universes, which I discuss next,
and further in \fullref[]{chp:conclusions}.

\paragraph{Higher Universes}
It is unclear how to scale this CPS translation to higher universes.
In this work, I have a single impredicative universe \im{\cpsstarty}, and the
locally polymorphic answer type \im{\cpsalpha} lives in that universe.
With an infinite hierarchy, it is not clear what the universe of \im{\cpsalpha} should be.
Furthermore, the translation relies on impredicativity in \im{\cpsstarty}.
We can only use impredicativity at one universe level and the locally
polymorphic answer-type translation does not seem possible in the context of
predicative polymorphism, so it's unclear how to adapt our translation to the
infinite predicative hierarchy.

I initially conjectured that the right solution for handling universes was
universe polymorphism~\cite{sozeau2014}.
My thought was that since the type is provided by the calling context, it seems
sensible that the calling context should also provide the universe.
Then, we could modify the type translation to be
\im{\cpspity{\tfont{\ell}}{\tfont{Level}}{\cpspity{\cpsalpha}{\cpstypety{\tfont{\ell}}}{\cpsfunty{(\cpsfunty{\sA^+}{\cpsalpha})}{\cpsalpha}}}}. 
However, one of the authors of \citeauthor{sozeau2014} (Sozeau) suggested to me
that this would not work.\footnote{Personal communication. Jan. 2018.}

\paragraph{Control Operators and Dependent Types}
This chapter explicitly avoids control effects and dependent types to focus on
type preservation.
However, in general, we may want to combine the two.
\citet{herbelin2005} shows that unrestricted use of \texttt{call/cc} and
\texttt{throw} in a language with \(\Sigma\) types and equality leads to an inconsistent system.
The inconsistency is caused by type dependency on terms involving control effects.
\citet{herbelin2012} solves the inconsistency by constraining types to depend only
on \emph{negative-elimination-free (NEF)} terms, which are free of effects.
This restriction makes dependent types compatible with classical reasoning
enabled by the control operators.

After the initial publication of the work in this
chapter~\cite{bowman2018:cps-sigma}, \citet{cong2018:lam-pi-s-r} extended the
CPS translation to control effects in the NEF fragment, and use this locally
polymorphic translation to disallow control effects for terms that use unsafe dependencies.

This is similar to recent work by \citet{miquey2017} uses the NEF restriction to
soundly extend the \im{\bar{\lambda}\mu\tilde{\mu}}-calculus of
\citet{curien2000}, a computational classic calculus, with
dependent types.
He then extends the language with delimited continuations, and defines a
type-preserving CPS to a standard intuitionistic dependent type theory.
By comparison, our source language \cpsslang is effect-free,
therefore we do not need the NEF condition to restrict dependency.
My use of the identity function serves a similar role to their delimited
continuations---allowing local evaluation of a CPS'd computation.

\paragraph{Effects and Dependent Types}
\citet{pedrot2017:weaning} define the \emph{weaning translation}, a monadic translation for
adding effects to dependent type theory.
The weaning translation allows us to represent self-algebraic monads, \ie,
monads whose algebras' universe itself forms an algebra, such as exception and
non-determinism monads.
However, it does not apply to the standard continuation monad, which is not self-algebraic.
The paper extends the translation to inductive types, with a restricted form of
dependent elimination.
Full dependent elimination can be implemented only for terms whose type is
first-order, and this comes at the cost of inconsistency, although one can
recover consistency by requiring that every inductive term be parametric.
My translation does not lead to inconsistency, and requires no restrictions on
the type to be translated.
However, my translation appears to impose the self-algebraic structure on the
computation types, and my use of parametricity to cast computations to values is
similar to their
parametricity restriction.

\paragraph{Internalizing Parametricity}
My work internalizes a specific free theorem, but ongoing work focuses on how to internalize
parametricity more generally in a dependent type theory.
\citet{krishnaswami2013} develop a technique for adding new rules to the extensional CoC.
They present a logical relation for terms that are not syntactically well typed, but are
semantically well behaved and equivalent at a particular type.
Using this logical relation, they prove the consistency of several extensions to extensional CoC,
including sum types, dependent records, and natural numbers.
\citet{bernardy2012,keller2012} give translations from one dependent type theory
into another that yield a parametric model of the original theory.
These essentially encode the logical relation in the target type theory, similar to the approach we
took in \fullref{sec:cps:consistent}.
Recent work by \citet{nuyts2017} develops a theory that internalizes parametricity, including the
important concept of \emph{identity extension}, and gives a thorough comparison of the literature.
By building a target language based on one of these systems, it's possible that
we could eliminate the rule \rulename{\im{\equiv}-Cont} as an axiom and instead
derive it internally.
This could allow the translation to scale to theories that are orthogonal to
parametricity.

\paragraph{Pervasive Translation}
In this chapter, I only CPS translate \emph{terms}, \ie, run-time expressions.
However, other work~\cite{barthe2001} studies \emph{pervasive} translation of PTSs.
A pervasive CPS translation internalizes evaluation order for all
expressions: terms, types, and kinds. 
This is necessary in general, since in a language such as Coq, we cannot easily
distinguish between terms, types, and kinds.
A pervasive translation may be necessary to scale type-preserving translation to
higher universes and more type-level computation.
The translations in \fullref[]{chp:anf} and \fullref[]{chp:abs-cc} both use
pervasive translation for this reason.

A pervasive CPS translation is also used in a partial solution to the
Barendregt-Geuvers-Klop conjecture~\cite{geuvers1993} which essentially states
that every weakly normalizing PTS is also strongly normalizing.
The conjecture has been solved for non-dependent PTSs, but the solution for
dependent PTSs remains open.
