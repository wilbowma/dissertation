\newcommand{\FigCBVUnv}[1][t]{
  \begin{figure}[#1]
    \judgshape{\cbvU{\slenv}{\sU}{\cpsU}}
    \begin{mathpar}
      \inferrule*[right=\defcbvUrule{Star}]
      {~}
      {\cbvU{\slenv}{\sstarty}{\cpsstarty}}

      \inferrule*[right=\defcbvUrule{Box}]
      {~}
      {\cbvU{\slenv}{\sboxty}{\cpsboxty}}
    \end{mathpar}
    \caption{\cbvname{} of Universes}
    \label{fig:cps:cbv:unv}
  \end{figure}
}

\newcommand{\FigCBVKinds}[1][t]{
  \begin{figure}[#1]
    \judgshape[\text{\fullref[]{lem:cps:cbv:type-pres} will show \im{\cpstyjudg{\slenv^+}{\sK^+}{\sU^+}}}]
     {\cbvK{\slenv}{\sK}{\sU}{\cpsK}}
    \begin{mathpar}
      \inferrule*[right=\defcbvKrule{Ax}]
      {~}
      {\cbvK{\slenv}{\sstarty}{\sboxty}{\cpsstarty}}

      \inferrule*[right=\defcbvKrule{PiK}]
      {\cbvK{\slenv}{\sK}{\sU}{\cpsK} \\
       \cbvK{\slenv,\salpha:\sK}{\sKpr}{\sU}{\cpsKpr}}
      {\cbvK{\slenv}{\spity{\salpha}{\sK}{\sKpr}}{\sU}{\cpspity{\cpsalpha}{\cpsK}{\cpsKpr}}}

      \inferrule*[right=\defcbvKrule{PiA}]
      {\cbvA{\slenv}{\sA}{\sKpr}{\cpsA} \\
       \cbvK{\slenv,\sx:\sA}{\sK}{\sU}{\cpsK}}
      {\cbvK{\slenv}{\spity{\sx}{\sA}{\sK}}{\sU}{\cpspity{\cpsx}{\cpsA}{\cpsK}}}
    \end{mathpar}
    \caption{\cbvname{} of Kinds}
    \label{fig:cps:cbv:kind}
  \end{figure}
}

\newcommand{\FigCBVTypesFull}[1][t]{
  \begin{figure}[#1]
  \judgshape[\text{\fullref[]{lem:cps:cbv:type-pres} will show \im{\cpstyjudg{\slenv^+}{\sA^+}{\sK^+}}}]
     {\cbvA{\slenv}{\sA}{\sK}{\cpsA}}
    \begin{mathpar}
      \inferrule*[right=\cbvArule{Var}]
      {~}
      {\cbvA{\slenv}{\salpha}{\sK}{\cpsalpha}}

      \inferrule*[right=\cbvArule{Pi}]
      {\cbvA{\slenv}{\sA}{\sKpr}{\cpsA} \\
       \cbvAdiv{\slenv,\sx:\sA}{\sB}{\sK}{\cpsB}}
      {\cbvA{\slenv}{\spity{\sx}{\sA}{\sB}}{\sK}{\cpspity{\cpsx}{\cpsA}{\cpsB}}}

      \inferrule*[right=\cbvArule{PiK}]
      {\cbvK{\slenv}{\sK}{\sUpr}{\cpsK} \\
       \cbvAdiv{\slenv,\sx:\sA}{\sB}{\sU}{\cpsB}}
      {\cbvA{\slenv}{\spity{\salpha}{\sK}{\sB}}{\sU}{\cpspity{\cpsalpha}{\cpsK}{\cpsB}}}

      \inferrule*[right=\cbvArule{Constr}]
      {\cbvA{\slenv}{\sA}{\sKpr}{\cpsA} \\
       \cbvA{\slenv,\sx:\sA}{\sB}{\sK}{\cpsB}}
      {\cbvA{\slenv}{\sfune{\sx}{\sA}{\sB}}{\spity{\sx}{\sA}{\sK}}{\cpsfune{\cpsx}{\cpsA}{\cpsB}}}

      \inferrule*[right=\cbvArule{Abs}]
      {\cbvK{\slenv}{\sK}{\sU}{\cpsK} \\
       \cbvA{\slenv,\salpha:\sK}{\sB}{\sKpr}{\cpsB}}
      {\cbvA{\slenv}{\sfune{\salpha}{\sK}{\sB}}{\spity{\salpha}{\sK}{\sKpr}}{\cpsfune{\cpsalpha}{\cpsK}{\cpsB}}}

      \inferrule*[right=\cbvArule{AppConstr}]
      {\cbvA{\slenv}{\sA}{\spity{\sx}{\sB}{\sK}}{\cpsA} \\
       \cbvA{\slenv,\sx:\sA}{\sB}{\sKpr}{\cpsB} \\
       \cbve{\slenv}{\se}{\sB}{\cpse}}
      {\cbvA{\slenv}{\sappe{\sA}{\se}}{\subst{\sK}{\se}{\sx}}{\cpsappe{\cpsA}{(\cpsncappe{\cpse}{\cpsB}{\cpsidk})}}}

      \inferrule*[right=\cbvArule{Inst}]
      {\cbvA{\slenv}{\sA}{\spity{\salpha}{\sKpr}{\sK}}{\cpsA} \\
       \cbvA{\slenv}{\sB}{\sKpr}{\cpsB}}
      {\cbvA{\slenv}{\sappe{\sA}{\sB}}{\subst{\sK}{\sB}{\salpha}}{\cpsappe{\cpsA}{\cpsB}}}

      \inferrule*[right=\cbvArule{Sigma}]
      {\cbvA{\slenv}{\sA}{\sstarty}{\cpsA} \\
       \cbvA{\slenv,\sx:\sA}{\sB}{\sstarty}{\cpsB}}
      {\cbvA{\slenv}{\ssigmaty{\sx}{\sA}{\sB}}{\sstarty}{\cpssigmaty{\cpsx}{\cpsA}{\cpsB}}}

      \inferrule*[right=\cbvArule{Bool}]
      {~}
      {\cbvA{\slenv}{\sboolty}{\sstarty}{\cpsboolty}}

      \inferrule*[right=\cbvArule{Let}]
      {\cbve{\slenv}{\se}{\sA}{\cpse} \\
       \cbvA{\slenv}{\sA}{\sKpr}{\cpsA} \\
       \cbvA{\slenv,\sx=\se:\sA}{\sB}{\sK}{\cpsB}}
      {\cbvA{\slenv}{\salete{\sx}{\se}{\sA}{\sB}}{\sK}{\cpsalete{\cpsx}{\cpsncappe{\cpse}{\cpsAto}{\cpsidk}}{\cpsA}{\cpsB}}}

      \inferrule*[right=\cbvArule{LetK}]
      {\cbvA{\slenv}{\sA}{\sK}{\cpsA} \\
       \cbvlenv{\slenv}{\cpslenv} \\
       \cpstyjudg{\cpslenv}{\cpsA}{\cpsKpr} \\
       \cbve{\slenv,\salpha=\sA:\sK}{\sB}{\_}{\cpsB}}
      {\cbvA{\slenv}{\salete{\salpha}{\sA}{\sK}{\sB}}{\_}{\cpsalete{\cpsalpha}{\cpsA}{\cpsKpr}{\cpsB}}}

      \inferrule*[right=\cbvArule{Conv}]
      {\styjudg{\slenv}{\sA}{\sKpr} \\
       \sequivjudg{\slenv}{\sK}{\sKpr} \\
       \cbnA{\slenv}{\sA}{\sKpr}{\cpsA}}
      {\cbnA{\slenv}{\sA}{\sK}{\cpsA}}
    \end{mathpar}
    \judgshape[\text{\fullref[]{lem:cps:cbv:type-pres} will show \im{\cpstyjudg{\slenv^+}{\sA^\div}{\sstarty^+}}}]
     {\cbvAdiv{\slenv}{\sA}{\sstarty}{\cpsA}}
    \begin{mathpar}
      \inferrule*[right=\cbvArule{Comp}]
      {\cbvA{\slenv}{\sA}{\sstarty}{\cpsA}}
      {\cbvAdiv{\slenv}{\sA}{\sstarty}{\cpspity{\cpsalpha}{\cpsstarty}{\cpsfunty{(\cpsfunty{\cpsA}{\cpsalpha})}{\cpsalpha}}}}
    \end{mathpar}
    \caption{\cbvname{} of Types}
    \label{fig:cps:cbv:types-full}
  \end{figure}
}

\newcommand{\FigCBVTypesShort}[1][t]{
  \begin{figure}[#1]
  \judgshape[\text{\fullref[]{lem:cps:cbv:type-pres} will show \im{\cpstyjudg{\slenv^+}{\sA^+}{\sK^+}}}]
     {\cbvA{\slenv}{\sA}{\sK}{\cpsA}}
    \begin{mathpar}
      \inferrule*[right=\cbvArule{Pi}]
      {\cbvA{\slenv}{\sA}{\sKpr}{\cpsA} \\
       \cbvAdiv{\slenv,\sx:\sA}{\sB}{\sK}{\cpsB}}
      {\cbvA{\slenv}{\spity{\sx}{\sA}{\sB}}{\sK}{\cpspity{\cpsx}{\cpsA}{\cpsB}}}

      \inferrule*[right=\cbvArule{PiK}]
      {\cbvK{\slenv}{\sK}{\sUpr}{\cpsK} \\
       \cbvAdiv{\slenv,\sx:\sA}{\sB}{\sU}{\cpsB}}
      {\cbvA{\slenv}{\spity{\salpha}{\sK}{\sB}}{\sU}{\cpspity{\cpsalpha}{\cpsK}{\cpsB}}}

      \inferrule*[right=\cbvArule{Constr}]
      {\cbvA{\slenv}{\sA}{\sKpr}{\cpsA} \\
       \cbvA{\slenv,\sx:\sA}{\sB}{\sK}{\cpsB}}
      {\cbvA{\slenv}{\sfune{\sx}{\sA}{\sB}}{\spity{\sx}{\sA}{\sK}}{\cpsfune{\cpsx}{\cpsA}{\cpsB}}}

      \inferrule*[right=\cbvArule{AppConstr}]
      {\cbvA{\slenv}{\sA}{\spity{\sx}{\sB}{\sK}}{\cpsA} \\
       \cbvA{\slenv,\sx:\sA}{\sB}{\sKpr}{\cpsB} \\
       \cbve{\slenv}{\se}{\sB}{\cpse}}
      {\cbvA{\slenv}{\sappe{\sA}{\se}}{\subst{\sK}{\se}{\sx}}{\cpsappe{\cpsA}{(\cpsncappe{\cpse}{\cpsB}{\cpsidk})}}}

      \inferrule*[right=\cbvArule{Sigma}]
      {\cbvA{\slenv}{\sA}{\sstarty}{\cpsA} \\
       \cbvA{\slenv,\sx:\sA}{\sB}{\sstarty}{\cpsB}}
      {\cbvA{\slenv}{\ssigmaty{\sx}{\sA}{\sB}}{\sstarty}{\cpssigmaty{\cpsx}{\cpsA}{\cpsB}}}

      \inferrule*[right=\cbvArule{Let}]
      {\cbve{\slenv}{\se}{\sA}{\cpse} \\
       \cbvA{\slenv}{\sA}{\sKpr}{\cpsA} \\
       \cbvA{\slenv,\sx=\se:\sA}{\sB}{\sK}{\cpsB}}
      {\cbvA{\slenv}{\salete{\sx}{\se}{\sA}{\sB}}{\sK}{\cpsalete{\cpsx}{\cpsncappe{\cpse}{\cpsAto}{\cpsidk}}{\cpsA}{\cpsB}}}

      \cdots
    \end{mathpar}
    \caption{\cbvname{} of Types (excerpts)}
    \label{fig:cps:cbv:types-short}
  \end{figure}
}

\newcommand{\CBVEnvRules}[1][t]{
   \judgshape[\text{\fullref[]{lem:cps:cbv:type-pres} will show \im{\cpswf{\slenv^+}}}]
     {\cbvlenv{\slenv}{\cpslenv}}
    \begin{mathpar}
      \inferrule*[right=\cbvlenvrule{Empty}]
      {~}
      {\cbvlenv{\cdot}{\cdot}}

      \inferrule*[right=\cbvlenvrule{AssumT}]
      {\cbvlenv{\slenv}{\cpslenv} \\
       \cbvA{\slenv}{\sA}{\sK}{\cpsA}}
      {\cbvlenv{\slenv,\sx:\sA}{\cpslenv,\cpsx:\cpsA}}

      \inferrule*[right=\cbvlenvrule{AssumK}]
      {\cbvlenv{\slenv}{\cpslenv} \\
       \cbvK{\slenv}{\sK}{\sU}{\cpsK}}
      {\cbvlenv{\slenv,\salpha:\sK}{\cpslenv,\cpsalpha:\cpsK}}{}

      \inferrule*[right=\cbvlenvrule{Def}]
      {\cbvlenv{\slenv}{\cpslenv} \\
       \cbvA{\slenv}{\sA}{\sK}{\cpsA} \\
       \cbve{\slenv}{\se}{\sA}{\cpse}}
      {\cbvlenv{\slenv,\sx = \se:\sA}{\cpslenv,\cpsx = \cpsncappe{\cpse}{\cpsA}{\cpsidk}:\cpsA}}

      \inferrule*[right=\cbvlenvrule{DefT}]
      {\cbvlenv{\slenv}{\cpslenv} \\
       \cbvA{\slenv}{\sA}{\sK}{\cpsA} \\
       \cbvK{\slenv}{\sK}{\sU}{\cpsK}}
      {\cbvlenv{\slenv,\salpha = \sA:\sK}{\cpslenv,\cpsalpha = \cpsA:\cpsK}}
    \end{mathpar}
}


\newcommand{\FigCBVTerms}[1][t]{
  \begin{figure}[#1]
  \judgshape[\text{\fullref[]{lem:cps:cbv:type-pres} will show \im{\cpstyjudg{\slenv^+}{\se^\div}{\sA^\div}}}]
     {\cbve{\slenv}{\se}{\sA}{\cpse}}
    \begin{mathpar}
      \inferrule*[right=\cbverule{Var}]
      {\cbvA{\slenv}{\sA}{\sK}{\cpsA}}
      {\cbve{\slenv}{\sx}{\sA}{
          \cpsfune{\cpsalpha}{\cpsstarty}{
            \cpsfune{\cpsk}{\cpsfunty{\cpsA}{\cpsalpha}}{
              \cpsappe{\cpsk}{\cpsx}}}}}

      \inferrule*[right=\cbverule{Fun}]
      {\cbvA{\slenv}{\sA}{\sKpr}{\cpsA} \\
       \cbvAdiv{\slenv,\sx:\sA}{\sB}{\sK}{\cpsB} \\
       \cbve{\slenv,\sx:\sA}{\se}{\sB}{\cpse}}
      {\cbve{\slenv}{\sfune{\sx}{\sA}{\se}}{\spity{\sx}{\sA}{\sB}}
        {\cpsfune{\cpsalpha}{\cpsstarty}{
            \cpsfune{\cpsk}{\cpsfunty{(\cpspity{\cpsx}{\cpsA}{\cpsB})}{\cpsalpha}}{
                \cpsappe{\cpsk}{(\cpsfune{\cpsx}{\cpsA}{\cpse})}}
            }}}

        \inferrule*[right=\cbverule{Abs}]
        {\cbvK{\slenv}{\sK}{\_}{\cpsK} \\
          \cbvAdiv{\slenv,\salpha:\sK}{\sB}{\_}{\cpsB} \\
          \cbve{\slenv,\salpha:\sK}{\se}{\sB}{\cpse}}
        {\cbve{\slenv}{\sfune{\salpha}{\sK}{\se}}{\spity{\salpha}{\sK}{\sB}}{
            \begin{stackTL}\cpsfune{\cpsalphain{ans}}{\cpsstarty}{
              \cpsfune{\cpsk}{\cpsfunty{(\cpspity{\cpsalpha}{\cpsK}{\cpsB})}{\cpsalphain{ans}}}{\\\quad
                  \cpsappe{\cpsk}{(\cpsfune{\cpsalpha}{\cpsK}{\cpse})}}
              \end{stackTL}}}}

      \inferrule*[right=\cbverule{App}]
      {\cbve{\slenv}{\se}{\spity{\sx}{\sA}{\sB}}{\cpse} \\
       \cbvAdiv{\slenv,\sx:\sA}{\sB}{\sK}{\cpsBto{\div}} \\
       \cbvA{\slenv,\sx:\sA}{\sB}{\sK}{\cpsBto{+}} \\
       \cbve{\slenv}{\sepr}{\sA}{\cpsepr} \\
       \cbvA{\slenv}{\sA}{\sKpr}{\cpsA}}
      {\cbve{\slenv}{\sappe{\se}{\sepr}}{\subst{\sB}{\sepr}{\sx}}
        {\cpsfune{\cpsalpha}{\cpsstarty}{
            \begin{stackTL}
            \cpsfune{\cpsk}{\cpsfunty{(\subst{\cpsBto{+}}{(\cpsncappe{\cpsepr}{\cpsA}{\cpsidk})}{\cpsx})}{\cpsalpha}}{\\\quad
              \cpsncappe{\cpse}{\cpsalpha}{\begin{stackTL}(\cpsfune{\cpsf}{\cpspity{\cpsx}{\cpsA}{\cpsBto{\div}}}{\\\quad
                    {\cpscappe{\cpsepr}{\cpsalpha}{(\cpsfune{\cpsx}{\cpsA}{
                        \cpsncappe{(\cpsappe{\cpsf}{\cpsx})}{\cpsalpha}{\cpsk}})}})
                  \end{stackTL}
                }}}
            \end{stackTL}
          }}}

      \inferrule*[right=\cbverule{Inst}]
        {\cbve{\slenv}{\se}{\spity{\salpha}{\sK}{\sB}}{\cpse} \\
          \cbvAdiv{\slenv,\salpha:\sK}{\sB}{\_}{\cpsB} \\
          \cbve{\slenv}{\sA}{\sK}{\cpsA}}
        {\cbve{\slenv}{\sappe{\se}{\sA}}{\slete{\sx}{\sA}{\sB}}
          {\begin{stackTL}
              \cpsfune{\cpsalphain{ans}}{\cpsstarty}{
                \cpsfune{\cpsk}{\cpsfunty{(\subst{\cpsB}{\cpsA}{\cpsalpha})}{\cpsalphain{ans}}}{\\\quad
                  \cpsncappe{\cpse}{\cpsalpha}{\begin{stackTL}(\cpsfune{\cpsf}{\cpspity{\cpsalpha}{\cpsK}{\cpsB}}{\\\quad
                        \cpsncappe{(\cpsappe{\cpsf}{\cpsA})}{\cpsalphain{ans}}{\cpsk})
                      \end{stackTL}
                    }}}
              \end{stackTL}
            }}}

      \inferrule*[right=\defcbnerule{True}]
      {~}
      {\cbne{\slenv}{\struee}{\sboolty}{\cpsfune{\cpsalpha}{\cpsstarty}{
            \cpsfune{\cpsk}{\cpsfunty{\cpsboolty}{\cpsalpha}}{
              \cpsappe{\cpsk}{\cpstruee}}}}}

      \inferrule*[right=\defcbnerule{False}]
      {~}
      {\cbne{\slenv}{\sfalsee}{\sboolty}{\cpsfune{\cpsalpha}{\cpsstarty}{
            \cpsfune{\cpsk}{\cpsfunty{\cpsboolty}{\cpsalpha}}{
              \cpsappe{\cpsk}{\cpsfalsee}}}}}

      \inferrule*[right=\defcbnerule{If}]
      {\cbve{\slenv}{\se}{\sboolty}{\cpse}\\
       \cbvA{\slenv}{\sB}{\sstarty}{\cpsB}\\
       \cbve{\slenv}{\seone}{\sB}{\cpseone}\\
       \cbve{\slenv}{\setwo}{\sB}{\cpsetwo}}
      {\cbve{\slenv}{\sife{\se}{\seone}{\setwo}}{\sB}{\begin{stackTL}\cpsfune{\cpsalpha}{\cpsstarty}{
            \cpsfune{\cpsk}{\cpsfunty{\cpsB}{\cpsalpha}}{
              \\\quad\cpscappe{\cpse}{\talpha}{(\begin{stackTL}\cpsfune{\cpsx}{\cpsboolty}{
                    \begin{stackTL}
                    \cpsife{\cpsx}{(\cpscappe{\cpseone}{\talpha}{\cpsk})\\\quad\,\,\,\,}{(\cpscappe{\cpsetwo}{\talpha}{\cpsk}))}
                    \end{stackTL}}}}}}
                  \end{stackTL}
                  \end{stackTL}}

      \inferrule*[right=\cbverule{Let}]
      {\cbve{\slenv}{\se}{\sA}{\cpse} \\
       \cbvA{\slenv}{\sA}{\sKpr}{\cpsA} \\
       \cbvA{\slenv}{\sB}{\sK}{\cpsB} \\
       \cbve{\slenv,\sx=\se:\sA}{\sepr}{\sB}{\cpsepr}}
      {\cbve{\slenv}{\salete{\sx}{\se}{\sA}{\sepr}}{\subst{\sB}{\se}{\sx}}{
          \begin{stackTL}\cpsfune{\cpsalpha}{\cpsstarty}{
            \cpsfune{\cpsk}{\cpsfunty{\subst{\cpsB}{(\cpsncappe{\cpse}{\cpsA}{\cpsidk})}{\cpsx}}{\cpsalpha}}{
              \\\quad\cpscappe{\cpse}{\cpsalpha}{(\cpsfune{\cpsx}{\cpsA}{
                  \cpsncappe{\cpsepr}{\cpsalpha}{\cpsk}})}}}}
        \end{stackTL}}

    \end{mathpar}
    \caption{\cbvname{} of Terms (1/2)}
    \label{fig:cps:cbv:terms}
  \end{figure}
}

\newcommand{\CBVPairRules}{
  \inferrule*[right=\cbverule{Pair}]
      {\cbve{\slenv}{\seone}{\sA}{\cpseone} \\
       \cbve{\slenv}{\setwo}{\subst{\sB}{\seone}{\sx}}{\cpsetwo} \\
       \cbvA{\slenv}{\sA}{\sstarty}{\cpsA} \\
       \cbvA{\slenv,\sx:\sA}{\sB}{\sstarty}{\cpsB}}
      {\cbve{\slenv}{\spaire{\seone}{\setwo}}{\ssigmaty{\sx}{\sA}{\sB}}{
          \begin{stackTL}
          \cpsfune{\cpsalpha}{\cpsstarty}{
              \cpsfune{\cpsk}{\cpsfunty{\cpssigmaty{\cpsx}{\cpsA}{\cpsB}}{\cpsalpha}}{
              \\\!\!\quad\cpscappe{\cpseone}{\cpsalpha}{\begin{stackTL}(\cpsfune{\cpsxone}{\cpsA}{
                  \\\!\quad\cpscappe{\cpsetwo}{\cpsalpha}{\begin{stackTL}(\cpsfune{\cpsxtwo}{\subst{\cpsB}{(\cpsncappe{\cpseone}{\cpsA}{\cpsidk})}{\cpsx}}{
                      \\\quad\cpsappe{\cpsk}{\cpsdpaire{\cpsxone}{\cpsxtwo}{\cpssigmaty{\cpsx}{\cpsA}{\cpsB}}}})}})}}}}
      \end{stackTL}
      \end{stackTL}
      \end{stackTL}
      }

      \inferrule*[right=\cbverule{Fst}]
      {\cbvA{\slenv}{\sA}{\sstarty}{\cpsA} \\
       \cbve{\slenv}{\se}{\ssigmaty{\sx}{\sA}{\sB}}{\cpse}}
      {\cbve{\slenv}{\sfste{\se}}{\sA}{
          \begin{stackTL}
          \cpsfune{\cpsalpha}{\cpsstarty}{
            \cpsfune{\cpsk}{\cpsfunty{\cpsAto{+}}{\cpsalpha}}{\\
              \quad\begin{stackTL}
              \cpscappe{\cpse}{\cpsalpha}{\begin{stackTL}(\cpsfune{\cpsy}{\cpssigmaty{\cpsx}{\cpsA}{\cpsB}}{
              \cpslete{\cpsz}{\cpsfste{\cpsy}}{\cpsappe{\cpsk}{\cpsz}}}
            )}}}
          \end{stackTL}
          \end{stackTL}
          \end{stackTL}
        }}

      \inferrule*[right=\cbverule{Snd}]
      {\cbvA{\slenv}{\sA}{\sstarty}{\cpsA} \\
       \cbvA{\slenv,\sx:\sA}{\sB}{\sstarty}{\cpsB} \\
       \cbve{\slenv}{(\sfste{\se})}{\sA}{(\sfste{\se})^{\div}} \\
       \cbve{\slenv}{\se}{\ssigmaty{\sx}{\sA}{\sB}}{\cpse}}
      {\cbve{\slenv}{\ssnde{\se}}{\subst{\sB}{\sfste{\se}}{\sx}}{
          \begin{stackTL}
          \cpsfune{\cpsalpha}{\cpsstarty}{
            \cpsfune{\cpsk}{\cpsfunty{\subst{\cpsB}{(\cpsncappe{(\sfste{\se})^\div}{\cpsA}{\cpsidk})}{\cpsx}}{\cpsalpha}}{\\
              \quad\begin{stackTL}
              \cpscappe{\cpse}{\cpsalpha}{\begin{stackTL}(\cpsfune{\cpsy}{\cpssigmaty{\cpsx}{\cpsA}{\cpsB}}{
              \cpslete{\cpsz}{\cpssnde{\cpsy}}{\cpsappe{\cpsk}{\cpsz}}}
            )}}}
          \end{stackTL}
          \end{stackTL}
          \end{stackTL}
        }}
}

\newcommand{\FigCBVTermsPairs}[1][t]{
  \begin{figure}[#1]
    \begin{mathpar}
      \CBVPairRules

        \inferrule*[right=\cbverule{Pair-Alt}]
        {\cbve{\slenv}{\seone}{\sA}{\cpseone} \\
          \cbve{\slenv}{\setwo}{\subst{\sB}{\seone}{\sx}}{\cpsetwo} \\
          \cbvA{\slenv}{\sA}{\sstarty}{\cpsA} \\
          \cbvA{\slenv,\sx:\sA}{\sB}{\sstarty}{\cpsB}}
        {\cbve{\slenv}{\spaire{\seone}{\setwo}}{\ssigmaty{\sx}{\sA}{\sB}}{
            \begin{stackTL}
            \cpsfune{\cpsalpha}{\cpsstarty}{
                \cpsfune{\cpsk}{\cpsfunty{\cpssigmaty{\cpsx}{\cpsA}{\cpsB}}{\cpsalpha}}{
                  \\\quad\cpscappe{\cpseone}{\cpsalpha}{\begin{stackTL}(\cpsfune{\cpsx}{\cpsA}{
                        \\\quad\cpscappe{\cpsetwo}{\cpsalpha}{(\cpsfune{\cpsxtwo}{\cpsB}{
                            \cpsappe{\cpsk}{\cpspaire{\cpsx}{\cpsxtwo}}})}})}}}}
            \end{stackTL}
          \end{stackTL}
      }

        \inferrule*[right=\cbverule{LetK}]
        {\cbvA{\slenv}{\sA}{\sK}{\cpsA} \\
          \cbvK{\slenv}{\sK}{\sU}{\cpsK} \\
          \cbvA{\slenv,\salpha=\sA:\sK}{\sB}{\sKpr}{\cpsB} \\
          \cbve{\slenv,\salpha=\sA:\sK}{\se}{\sB}{\cpse}}
        {\cbve{\slenv}{\salete{\salpha}{\sA}{\sK}{\se}}{\subst{\sB}{\sA}{\salpha}}{
            \begin{stackTL}\cpsfune{\cpsalphain{ans}}{\cpsstarty}{
              \cpsfune{\cpsk}{\cpsfunty{\subst{\cpsB}{\cpsA}{\cpsalpha}}{\cpsalphain{ans}}}{
                  \\\quad\cpsalete{\cpsalpha}{\cpsA}{\cpsK}{\cpsncappe{\cpse}{\cpsalphain{ans}}{\cpsk}}
        \end{stackTL}}}}}

      \inferrule*[right=\cbverule{Conv}]
      {\cbve{\slenv}{\se}{\sB}{\cpse}}
      {\cbve{\slenv}{\se}{\sA}{\cpse}}
    \end{mathpar}
    \caption{\cbvname{} of Terms (2/2)}
    \label{fig:cps:cbv:terms-pairs}
  \end{figure}
}

\newcommand{\FigCBVEnv}[1][t] {
  \begin{figure}[#1]
    \CBVEnvRules
    \caption{\cbvname{} of Environments}
    \label{fig:cps:cbv:env}
  \end{figure}
}

\section{Call-by-Value {{CPS}} Translation}
\label{sec:cps:cbv}
In this section, I present the call-by-value (\tech{CBV}) \tech{CPS} translation
(\cbvname{}) of \cpsslang.
First, I redefine the short-hand from \fullref[]{sec:cps:cbn} to refer to
call-by-value translation.
\begin{displaymath}
  \begin{array}{rcl}
    \sA^{\div} & \defeq & \cpsA~\where{\cbvAdiv{\slenv}{\sA}{\sstarty}{\cpsA}} \\
    \se^{\div} & \defeq & \cpse~\where{\cbve{\slenv}{\se}{\sA}{\cpse}}
  \end{array}
  \qquad\qquad
  \begin{array}{rcl}
    \sU^{+} & \defeq & \cpsU~\where{\cbvU{\slenv}{\sU}{\cpsU}} \\
    \sK^{+} & \defeq & \cpse~\where{\cbvK{\slenv}{\sK}{\sU}{\cpsK}} \\
    \sA^{+} & \defeq & \cpsA~\where{\cbvA{\slenv}{\sA}{\sK}{\cpsA}} \\
  \end{array}
\end{displaymath}

In general, \cbvname{} differs from \cbnname{} in two ways.
First, all \tech{term} variables must have \tech{value} types, so the
translation rules for all binding constructs now use the \tech{value
  translation} for type annotations.
Second, we change the definition of \tech{value} types so that functions must receive
\tech{values} are arguments and pairs must contain \tech{values} as components.
Since the translation must force every \tech{computation} to a \tech{value}, the
translation of every feature of \tech{dependency} requires the new typing rule
\refrule{T-Cont}.
Furthermore, all substitutions of a \tech{term} into a \tech{type} must
substitute \emph{\tech{values}} instead of \tech{computations}, so all
\tech{dependent type} annotations must explicitly convert \tech{computations} to
\tech{values} by supplying the identity \tech{continuation}.

The translation of \tech{universes} is unchanged compared to the \tech{CBN}
translation, so I leave its definition in \fullref[]{sec:cps:cbv:appendix}
\fullref[]{fig:cps:cbv:unv}.

\FigCBVKinds
I define the translation of \tech{kinds} in \fullref[]{fig:cps:cbv:kind}.
The only difference is in \cbvKrule{PiA}.
Now \cbvKrule{PiA} translates the \tech{kind} of \tech{type}-level functions
\im{\spity{\sx}{\sA}{\sK}} to accept a \emph{\tech{value}} as argument \im{\cpsx :
  \sA^+}.
In the \tech{CBN} translation, the domain of a \tech{dependent function} is a
\tech{computation}, so the domain annotation is translated with the
\tech{computation translation} on \tech{types}.
Now, in the \tech{CBV} translation, all arguments to \tech{dependent functions}
are \tech{values}, so \cbvKrule{PiA} uses the \tech{value translation} on
types to translate the domain annotation.

\FigCBVTypesShort
I present the translation on \tech{types} in \fullref[]{fig:cps:cbv:types-short}.
The \tech{type} translation has multiple rules with type annotations that have
changed from \tech{CBN}.
The \tech{computation translation} of types is unchanged.
In the \tech{value translation} of \tech{types}, similar to the \tech{kind}
translation, \tech{dependent function} types that abstract over \tech{terms} now
translate the domain annotation \im{\sx:\sA} using the \tech{value translation}.
After \tech{CBV} translation, \tech{dependent pairs} must contain \tech{values},
so the translation of \im{\ssigmaty{\sx}{\sA}{\sB}} uses the \tech{value
  translation} on the component types, \ie, the \tech{CBV} translation is
\im{\cpssigmaty{\cpsx}{\sA^+}{\sB^+}}.
When \tech{terms} appear in the \tech{type} language, such as in
\cbvArule{AppConstr} and \cbvArule{Let}, we must explicitly convert the
\tech{computation} to a \tech{value} to maintain the invariant that all
\tech{term} variables refer to \tech{values}.
For example, in \cbvArule{AppConstr} we translate a \tech{type}-level application
with a \tech{term} argument \im{\sappe{\sA}{\se}} to
\im{\cpsappe{\sA^+}{(\cpsncappe{\se^\div}{\sB^+}{\cpsidk})}}.
We similarly translate let-bound \tech{terms} \cbvArule{Let} by casting the
\tech{computation} to a \tech{value}.
Recall from \fullref[]{sec:cps:ideas} that, while expressions of the form
\im{\cpsappe{\sA^+}{(\cpsncappe{\se^\div}{\sB^+}{\cpsidk})}} are not in CPS,
this \tech{expression} is a \tech{type} and will be evaluated during type
checking.
\tech{Terms} that evaluate at run time are always in CPS and \tech{never
  return}.

\FigCBVTerms[h!]
\FigCBVTermsPairs[h!]
\FigCBVEnv[h!]
The \tech{term} translation (\fullref[]{fig:cps:cbv:terms} and
  \fullref[]{fig:cps:cbv:terms-pairs}) changes in three major ways.
As in \fullref[]{sec:cps:cbn}, we implicitly have a \tech*{computation
  translation}{computation} and a \tech{value translation} on \tech{term}
\tech{values}, with the latter inlined into the former.
First, unlike in \cbnname{}, variables are \tech{values}, whereas the translation must
produce a \tech{computation}.
Therefore, we translate \im{\sx} by ``\tech{value} \(\eta\)-expansion'' into
\im{\cpsnfune{\cpsalpha}{\cpsnfune{\cpsk}{\cpsappe{\cpsk}{\cpsx}}}}, a
\tech{computation} that immediately applies the \tech{continuation} to the
\tech{value}.
Second, as discussed above, we change the translation of application
\cbverule{App} to force the evaluation of the function argument.
Third, in the translation of pairs \cbverule{Pair}, we force the
evaluation of the components of the pair and produce a pair of \tech{values} for
the \tech{continuation}.
Note that in cases of the translation where we have types with
\tech{dependency}---\cbverule{App}, \cbverule{Let}, \cbverule{Pair}, and
\cbverule{Snd}---we cast \tech{computations} to \tech{values} in the
\tech{types} by applying the identity continuation, and require the
\im{\tfont{@}} form to use the new typing rule \refrule{T-Cont}.

\paragraph{Digression}
  Interestingly, because typing the application of a \tech{continuation} is
  essentially a \tech{dependent let}, we can simplify the translation of pairs.
  I present this in the rule \cbverule{Pair-Alt}.
  Instead of explicitly substituting the \tech{value} of \im{\cpseone} into the
  \tech{type} \im{\cpsB}, we simply use the same variable name \im{\cpsx} to
  bind the \tech{value} of \im{\cpseone} in both the \tech{term} and the \tech{type}.
  Since that variable is free in the \tech{type} annotation \im{\cpsB} on the
  variable \im{\cpsxtwo}, we implicitly substitute its \tech{value} into \im{\cpsB}
  rather than being so explicitly.
  This is rather subtle so I prefer the more direct and explicit translation,
  \cbverule{Pair}.

Given the translation of binding constructs in the language, the translation of
the typing environment (\fullref[]{fig:cps:cbv:env}) should be unsurprising.
Since all variables are \tech{values}, we translate \tech{term} variables
\im{\sx : \sA} using the \tech{value translation} on \tech{types} to produce
\im{\cpsx : \sA^+} instead of \im{\cpsx : \sA^\div}.
We must also translate \tech{term} \tech{definitions} \im{\sx = \se : \sA} by
casting the \tech{computation} to a \tech{value}, producing \im{\cpsx =
  \cpsncappe{\se^\div}{\sA^+}{\cpsidk} : \sA^+}.

\subsection{Type Preservation}
\label{sec:cps:cbv:proof}
The \tech{type-preservation} proof follows the standard architecture presented
in \fullref[]{chp:type-pres}.
First we prove \tech{compositionality}, then preservation of \tech{reduction},
preservation of \tech{conversion}, then preservation of \tech{equivalence}, then
preservation of \tech{subtyping}, then \tech{type preservation}.
However, we must reason about \tech{CPS'd} ``values'' of the form
\im{\cpscappe{\cpse}{\cpsA}{\cpsk}} in all uses of \tech{dependency}.
This requires a few extra steps of reasoning, particularly in the proof of
\fullref{lem:cps:cbv:subst}.
Otherwise, the proofs of the supporting lemmas are essentially the same as in
\fullref[]{sec:cps:cbn}.

{
\allowdisplaybreaks % Let latex break proofs on a page boundary
I begin with a technical lemma that is essentially an \(\eta\)-principle for
\tech{CPS'd} \tech{computations}, which simplifies the aforementioned reasoning
about \tech{CPS'd} ``values''.\footnote{The proofs for the \tech{CBN} setting only
  require a specialized instance of this property although the general form
  holds.}
The lemma states that any \tech{CPS'd} \tech{computation} \im{\se^\div} is
\tech{equivalent} to a new \tech{CPS'd} \tech{computation} that accepts a
\tech{continuation} \im{\cpsk} simply applies \im{\se^\div} to \im{\cpsk}.
The proof is straightforward.
Note the type annotations are apparently mismatched, as in our explanation of
the translation of \refrule[cpssrc]{Conv} in \fullref[]{sec:cps:cbn} and the discussion
of untyped vs type \tech{equivalence} in \fullref[]{chp:type-pres},
but the behaviors of the terms are the same and \tech{equivalence} is untyped.
\begin{lemma}[\cbvname{} Computation \im{\eta}]
  \label{lem:cps:cbv:break}
    \im{\se^\div \equiv
    \cpsfune{\cpsalpha}{\cpsstarty}{\cpsfune{\cpsk}{\cpsfunty{\cpsA}{\cpsalpha}}{
        \cpscappe{\se^\div}{\cpsalpha}{(\cpsfune{\cpsx}{\cpsB}{\cpsappe{\cpsk}{\cpsx}})}}}}
\end{lemma}
\begin{proof}
  Note that
  \im{\se^\div \equiv
    \cpsfune{\cpsalpha}{\cpsstarty}{\cpsfune{\cpsk}{\cpsfunty{\cpsA}{\cpsalpha}}{
        \cpsncappe{\se^\div}{\cpsalpha}{(\cpsfune{\cpsx}{\cpsB}{\cpsappe{\cpsk}{\cpsx}})}}}}, by
  \(\eta\)-equivalence.
  By transitivity, it suffices to show that

  \im{\cpsfune{\cpsalpha}{\cpsstarty}{\cpsfune{\cpsk}{\cpsfunty{\cpsA}{\cpsalpha}}{
        \cpsncappe{\se^\div}{\cpsalpha}{(\cpsfune{\cpsx}{\cpsB}{\cpsappe{\cpsk}{\cpsx}})}}}
    \equiv
    \cpsfune{\cpsalpha}{\cpsstarty}{\cpsfune{\cpsk}{\cpsfunty{\cpsA}{\cpsalpha}}{
        \cpscappe{\se^\div}{\cpsalpha}{(\cpsfune{\cpsx}{\cpsB}{\cpsappe{\cpsk}{\cpsx}})}}}}

  Intuitively, this is true since \im{\tfontsym{@}} dynamically behaves exactly like
  application, only changing which typing rule is used.
  Since \tech{equivalence} is untyped, the semantics of \im{\tfontsym{@}} as far
  as \tech{equivalence} is concerned is no different than normal application.

  Note that by definition of the translation, \im{\se^\div} must be of
  the form \im{\cpsnfune{\cpsalpha}{\cpsnfune{\cpsk}{\cpsepr}}}.

  The goal follows since

  \im{\cpsfune{\cpsalpha}{\cpsstarty}{\cpsfune{\cpsk}{\cpsfunty{\cpsA}{\cpsalpha}}{
        \cpsncappe{(\cpsnfune{\cpsalpha}{\cpsnfune{\cpsk}{\cpsepr}})}{\cpsalpha}{(\cpsfune{\cpsx}{\cpsB}{\cpsappe{\cpsk}{\cpsx}})}}}
    \stepstar
    \cpsfune{\cpsalpha}{\cpsstarty}{\cpsfune{\cpsk}{\cpsfunty{\cpsA}{\cpsalpha}}{\cpsepr}}}

  and

  \im{\cpsfune{\cpsalpha}{\cpsstarty}{\cpsfune{\cpsk}{\cpsfunty{\cpsA}{\cpsalpha}}{
        \cpscappe{(\cpsnfune{\cpsalpha}{\cpsnfune{\cpsk}{\cpsepr}})}{\cpsalpha}{(\cpsfune{\cpsx}{\cpsB}{\cpsappe{\cpsk}{\cpsx}})}}}
    \stepstar
    \cpsfune{\cpsalpha}{\cpsstarty}{\cpsfune{\cpsk}{\cpsfunty{\cpsA}{\cpsalpha}}{\cpsepr}}}
\end{proof}

Since variables are \tech{values} in call-by-value, we adjust the statement
of \fullref[]{lem:cps:cbv:subst} to cast \tech{computations} to \tech{values}.
Proving this lemma now requires the new \tech{equivalence} rule
\refrule*{eqv-cont}{\im{\equiv}-Cont} for cases involving substitution of
\tech{terms}.
Recall that all \tech{terms} being translated have an implicit typing
derivation, so the omitted types are easy to reconstruct.

\begin{lemma}[\cbvname{} Compositionality]
  \label{lem:cps:cbv:subst}
  ~
  \begin{multicols}{2}
  \begin{enumerate}
    \item \im{(\subst{\sK}{\sA}{\salpha})^{+} \equiv \subst{\sK^{+}}{\sA^{+}}{\cpsalpha}}
    \item \im{(\subst{\sK}{\se}{\sx})^{+} \equiv \subst{\sK^{+}}{\cpsncappe{\se^{\div}}{\_}{\cpsidk}}{\cpsx}}
    \item \im{(\subst{\sA}{\sB}{\salpha})^{+} \equiv \subst{\sA^{+}}{\sB^{+}}{\cpsalpha}}
    \item \im{(\subst{\sA}{\se}{\sx})^{+} \equiv \subst{\sA^{+}}{\cpsncappe{\se^{\div}}{\_}{\cpsidk}}{\cpsx}}
    \item \im{(\subst{\sA}{\sB}{\salpha})^{\div} \equiv \subst{\sA^{\div}}{\sB^{+}}{\cpsalpha}}
    \item \im{(\subst{\sA}{\se}{\sx})^{\div} \equiv \subst{\sA^{\div}}{\cpsncappe{\se^{\div}}{\_}{\cpsidk}}{\cpsx}}
    \item \im{(\subst{\se}{\sA}{\salpha})^{\div} \equiv \subst{\se^{\div}}{\sA^{+}}{\cpsalpha}}
    \item \im{(\subst{\se}{\sepr}{\sx})^{\div} \equiv
        \subst{\se^{\div}}{\cpsncappe{\se^{\sprime\div}}{\_}{\cpsidk}}{\cpsx}}
  \end{enumerate}
  \end{multicols}
\end{lemma}
\begin{proof}
  The proof is straightforward by induction on the typing derivation of the
  expression \im{\st} being substituted into.

    Part 6 follows immediately by part 3 of the induction hypothesis.
    Part 7 follows immediately by part 4 of the induction hypothesis.
    I give representative cases for the other parts.
    In most cases, it suffices to show that the two \tech{terms} are
    syntactically identical.
    \begin{proofcases}
    \item \refrule[cpssrc]{*} \im{\st = \sU}, parts 1 and 2. Trivial, since no free variables appear in \im{\sU}.
    \item \refrule[cpssrc]{Pi-*} \im{\st = \spity{\sx}{\sB}{\sKpr}}
    \item[{\bfseries Sub-case:}] Part 1. We must show that
      \im{(\subst{(\spity{\sx}{\sB}{\sKpr})}{\sA}{\salpha})^+ =
        \subst{(\spity{\sx}{\sB}{\sKpr})^+}{\sA^+}{\cpsalpha}}.
      \begin{align}
        & (\subst{(\spity{\sx}{\sB}{\sKpr})}{\sA}{\salpha})^+ \nonumber \\
        =~&(\spity{\sx}{\subst{\sB}{\sA}{\salpha}}{\subst{\sKpr}{\sA}{\salpha}})^+ \\
        & \text{by definition of substitution} \nonumber \\
        =~& \cpspity{\cpsxpr}{(\subst{\sB}{\sA}{\salpha})^+}{(\subst{\sKpr}{\sA}{\salpha})^+} \\
        & \text{by definition of the translation} \nonumber \\
        =~& \cpspity{\cpsxpr}{\subst{\sB^+}{\sA^+}{\salpha}}{\subst{\sK^{\sprime+}}{\sA^+}{\salpha}} \\
        & \text{by parts 1 and 3 of the induction hypothesis} \nonumber \\
        =~& \subst{(\cpspity{\cpsxpr}{\sB^+}{\sK^{\sprime+}})}{\sA^+}{\cpsalpha} \\
        & \text{by definition of substitution} \nonumber \\
        =~& \subst{(\spity{\sxpr}{\sB}{\sKpr})^+}{\sA^+}{\cpsalpha} \\
        & \text{by definition of the translation} \nonumber
      \end{align}
      \item[{\bfseries Sub-case:}] Part 2. Similar to the previous subcase.

      \item \refrule[cpssrc]{Var}
      \item[{\bfseries Sub-case:}] \im{\st = \salphapr}, part 3. Part 4 is trivial since \im{\sx} is not free in
      \im{\salpha}.

      \noindent We must show that \im{(\subst{\salphapr}{\sA}{\salpha})^+ = \subst{\cpsalphapr}{\sA^+}{\cpsalpha}}.

      \item[{\bfseries Sub-case:}] \im{\salpha = \salphapr}. It suffices to show that \im{\sA^+ = \sA^+}, which is trivial.
      \item[{\bfseries Sub-case:}] \im{\salpha \neq \salphapr}. Trivial.

      \item[{\bfseries Sub-case:}] \im{\st = \sx}, part 7 is trivial.
      \item[{\bfseries Sub-case:}] Part 8

    \item \refrule[cpssrc]{Var} Part 8, \im{(\subst{\sx}{\se}{\sxpr})^\div}

      We must show \im{(\subst{\sx}{\se}{\sxpr})^\div =
        \subst{\sx^\div}{\cpsncappe{\se^{\div}}{\_}{\cpsidk}}{\cpsxpr}}.

      W.l.o.g., assume \im{\sx = \sxpr}.
      \begin{align}
        & (\subst{\sx}{\se}{\sx})^\div \nonumber \\
        =~& \se^\div & \text{by definition of substitution} \\
        \equiv~& \cpsnfune{\cpsalpha}{\cpsnfune{\cpsk}{(\cpscappe{\se^\div}{\cpsalpha}{\cpsnfune{\cpsx}{\cpsappe{\cpsk}{\cpsx}}})}}
          & \text{by \fullref[]{lem:cps:cbv:break}} \\
        \equiv~& \cpsnfune{\cpsalpha}{\cpsnfune{\cpsk}{\cpsappe{(\cpsnfune{\cpsx}{\cpsappe{\cpsk}{\cpsx}})}{(\cpsncappe{\se^\div}{\_}{\cpsidk})}}}
          & \text{by \refrule*{eqv-cont}{\im{\equiv}-Cont}} \\
        =~& \subst{(\cpsnfune{\cpsalpha}{\cpsnfune{\cpsk}{\cpsappe{(\cpsnfune{\cpsx}{\cpsappe{\cpsk}{\cpsx}})}{\cpsx}}})}{(\cpsncappe{\se^\div}{\_}{\cpsidk})}{\cpsx}
          & \text{by substitution} \\
        \equiv~&
        \subst{(\cpsnfune{\cpsalpha}{\cpsnfune{\cpsk}{\cpsappe{\cpsk}{\cpsx}}})}{(\cpsncappe{\se^\div}{\_}{\cpsidk})}{\cpsx}
            & \text{by \im{\step_{\beta}}} \\
        =~& \subst{\sx^\div}{(\cpsncappe{\se^\div}{\_}{\cpsidk})}{\cpsx}
          & \text{by definition of translation}
      \end{align}

    \item \refrule[cpssrc]{App} \im{\sappe{\seone}{\setwo}}

        \item[{\bfseries Sub-case:}] Part 7

        We must show \im{(\subst{(\sappe{\seone}{\setwo})}{\sApr}{\salphapr})^\div =
          \subst{(\sappe{\seone}{\setwo})^\div}{\sA^{\sprime+}}{\cpsalphapr}}.
        \begin{align}
          & (\subst{(\sappe{\seone}{\setwo})}{\sApr}{\salphapr})^\div \nonumber \\
          =~& (\sappe{\subst{\seone}{\sA}{\salphapr}}{\subst{\setwo}{\sApr}{\salphapr}})^\div \qquad
            \text {by substitution}\\
          =~& {\begin{stackTL}
              \cpsfune{\cpsalpha}{\cpsstarty}{
                \cpsfune{\cpsk}{\cpsfunty{(\subst{(\subst{\sB}{\sApr}{\salphapr})^+}{(\subst{\setwo}{\sApr}{\salphapr})^\div}{\cpsx})}{\cpsalpha}}{
                  \\\quad\cpsncappe{(\subst{\seone}{\sApr}{\salphapr})^\div}{\cpsalpha}{(\begin{stackTL}
                      \cpsfune{\cpsf}{\cpspity{\cpsx}{(\subst{\sA}{\sApr}{\salphapr})^{+}}{
                          (\subst{\sB}{\sApr}{\salphapr})^{\div}}}{
                        \\\quad\cpscappe{(\subst{\setwo}{\sApr}{\salphapr})^\div}{\cpsalpha}{(\cpsfune{\cpsx}{{(\subst{\sA}{\sApr}{\salphapr})^{+}}}{
                            \cpsncappe{(\cpsappe{\cpsf}{\cpsx})}{\cpsalpha}{\cpsk}})}})
                    \end{stackTL}}
                }}
            \end{stackTL}} \nonumber \\
          & \text{by translation} \\
          =~& {\begin{stackTL}
              \cpsfune{\cpsalpha}{\cpsstarty}{
                \cpsfune{\cpsk}{\cpsfunty{(\subst{\subst{\sB^+}{\sA^{\sprime+}}{\cpsalphapr}}{\subst{\setwo^\div}{\sA^{\sprime+}}{\cpsalphapr}}{\cpsx})}{\cpsalpha}}{
                  \\\quad\cpsncappe{\subst{\seone^\div}{\sA^{\sprime+}}{\cpsalphapr}}{\cpsalpha}{(\begin{stackTL}
                      \cpsfune{\cpsf}{\cpspity{\cpsx}{\subst{\sA^+}{\sA^{\sprime+}}{\salphapr}}{
                          \subst{\sB^{\div}}{\sA^{\sprime+}}{\cpsalphapr}}}{
                        \\\quad\cpscappe{\subst{\setwo^{\div}}{\sA^{\sprime+}}{\cpsalphapr}}{\cpsalpha}{(\cpsfune{\cpsx}{{\subst{\sA^+}{\sA^{\sprime+}}{\cpsalphapr}}}{
                            \cpsncappe{(\cpsappe{\cpsf}{\cpsx})}{\cpsalpha}{\cpsk}})}})
                    \end{stackTL}}
                }}
            \end{stackTL}} \nonumber \\
          & \text{by IH (3,5,7)} \\
          =~& {(\begin{stackTL}
              \cpsfune{\cpsalpha}{\cpsstarty}{
                \cpsfune{\cpsk}{\cpsfunty{(\subst{\sB^+}{\setwo^\div}{\cpsx})}{\cpsalpha}}{
                  \\\quad\cpsncappe{\seone^{\div}}{\cpsalpha}{(\begin{stackTL}
                      \cpsfune{\cpsf}{\cpspity{\cpsx}{\sA^\div}{\sB^{\div}}}{
                        \\\quad\cpscappe{\setwo^{\div}}{\cpsalpha}{(\cpsfune{\cpsx}{\sA^+}{
                            \cpsncappe{(\cpsappe{\cpsf}{\cpsx})}{\cpsalpha}{\cpsk}})}})}}})
                [{\sA^{\sprime+}}/{\cpsalphapr}]
              \end{stackTL}
            \end{stackTL}
          } \\
          & \text{by substitution} \nonumber \\
          =~& \subst{(\sappe{\seone}{\setwo})^{\div}}{\sA^{\sprime+}}{\cpsalphapr} \qquad
            \text{by translation}
        \end{align}
        \qedhere
    \end{proofcases}
\end{proof}

\begin{lemma}[\cbvname{} Preservation of Reduction]
  \label{lem:cps:cbv:pres-red}
  ~
  \begin{itemize}
    \item If \im{\styjudg{\slenv}{\se}{\sA}} and \im{\se \step \sepr} then
      \im{\se^{\div} \stepstar \cpsepr} and \im{\cpsepr \equiv \se^{\sprime\div}}
    \item If \im{\styjudg{\slenv}{\sA}{\sK}} and \im{\sA \step \sApr} then
      \im{\sA^{+} \stepstar \cpsApr} and \im{\cpsApr \equiv \sA^{\sprime+}}
    \item If \im{\styjudg{\slenv}{\sA}{\sstarty}} and \im{\sA \step \sApr} then
      \im{\sA^{\div} \stepstar \cpsApr} and \im{\cpsApr \equiv \sA^{\sprime\div}}
  \end{itemize}
\end{lemma}
\begin{proof}
  The proof is straightforward by cases on the \tech{reduction} (\im{\step}) relation.
  I give some representative cases.
  \begin{proofcases}
    \item \im{\sx \step_{\delta} \sepr~\where{\sx = \sepr : \sApr \in \slenv}}

    We must show that \im{\sx^\div \step_{\delta} \cpse} such that
    \im{\cpse \equiv  \cpsncappe{\se^{\sprime\div}}{\_}{\cpsidk}~\where{\sx^\div =
        \cpsncappe{\se^{\sprime\div}}{\_}{\cpsidk} : \sA^{\sprime+} \in \slenv^+}}, which
    follows by the same argument as Sub-Case Part 8 of the \im{\sx} case of
    \fullref[]{lem:cps:cbv:pres-red}.

    \item \im{\sappe{(\sfune{\sx}{\_}{\seone})}{\setwo} \step_{\beta} \subst{\seone}{\setwo}{\sx}}

    We must show that \im{(\sappe{(\sfune{\sx}{\_}{\seone})}{\setwo})^{\div}
      \stepstar \cpsepr} and \im{\cpsepr \equiv (\subst{\seone}{\setwo}{\sx})^\div}.
    \begin{align}
      & (\sappe{(\sfune{\sx}{\_}{\seone})}{\setwo}) \nonumber \\
      =~&
          {(\begin{stackTL}
              \cpsnfune{\cpsalpha}{
                \cpsnfune{\cpsk}{
                  \\\quad\cpsncappe{(\cpsnfune{\cpsalpha}{
                      \cpsnfune{\cpsk}{
                        \cpsappe{\cpsk}{(\cpsnfune{\cpsx}{\seone^{\div}})}}})}{\cpsalpha}{(\begin{stackTL}
                      \cpsnfune{\cpsf}{
                        \cpscappe{\setwo^{\div}}{\cpsalpha}{(\cpsnfune{\cpsx}{
                            \cpsncappe{(\cpsappe{\cpsf}{\cpsx})}{\cpsalpha}{\cpsk}})}})}}})
              \end{stackTL}
            \end{stackTL}
      } \\ & \text{by translation} \nonumber \\
      \stepstar~&
                  (\cpsnfune{\cpsalpha}{
                        \cpsnfune{\cpsk}{
                          \cpscappe{\setwo^{\div}}{\cpsalpha}{(\cpsnfune{\cpsx}{
                              \cpsncappe{(\cpsappe{(\cpsnfune{\cpsx}{\seone^{\div}})}{\cpsx})}{\cpsalpha}{\cpsk}})}}})
      \\ & \text{by \im{\step_\beta}} \nonumber \\
      \stepstar~& (\cpsnfune{\cpsalpha}{
                        \cpsnfune{\cpsk}{
                          \cpscappe{\setwo^{\div}}{\cpsalpha}{(\cpsnfune{\cpsx}{
                  \cpsncappe{\seone^\div}{\cpsalpha}{\cpsk}})}}})
        \\ & \text{by \im{\step_\beta}} \nonumber \\
      \equiv~& (\cpsnfune{\cpsalpha}{
               \cpsnfune{\cpsk}{
               \cpsappe{(\cpsnfune{\cpsx}{
               \cpsncappe{\seone^\div}{\cpsalpha}{\cpsk}})}{(\cpsncappe{\setwo^{\div}}{\_}{\cpsidk})}}})
               \\ & \text{by \refrule*{eqv-cont}{\im{\equiv}-Cont}} \nonumber \\
      \stepstar ~& (\cpsnfune{\cpsalpha}{
                   \cpsnfune{\cpsk}{
                   \subst{(\cpsncappe{\seone^\div}{\cpsalpha}{\cpsk})}{(\cpsncappe{\setwo^{\div}}{\_}{\cpsidk})}{\cpsx}}})
        \\ & \text{by \im{\step_\beta}} \nonumber \\
      =~& \subst{(\cpsnfune{\cpsalpha}{
                   \cpsnfune{\cpsk}{
          (\cpsncappe{\seone^\div}{\cpsalpha}{\cpsk})}})}
          {(\cpsncappe{\setwo^{\div}}{\_}{\cpsidk})}{\cpsx}
          \\ & \text{by substitution} \nonumber \\
      \equiv ~& \subst{\seone^{\div}}{\cpsncappe{\setwo^\div}{\_}{\cpsidk}}{\cpsx}
                \\ & \text{by \refrule*[refcps]{eqv-eta1}{\im{\equiv}-\im{\eta}}} \nonumber \\
      = ~& (\subst{\seone}{\setwo}{\sx})^{\div}
           \\ & \text{by \fullref[]{lem:cps:cbv:subst}} \nonumber
    \end{align}\qedhere
  \end{proofcases}
\end{proof}

Note that \tech{kinds} do not take steps in the \tech{reduction} relation, but
can in the \tech{conversion} relation.
\begin{lemma}[\cbvname{} Preservation of Conversion]
  \label{lem:cps:cbv:pres-red*}
  ~
  \begin{itemize}
    \item If \im{\styjudg{\slenv}{\se}{\sA}} and \im{\se \stepstar \sepr} then
      \im{\se^{\div} \stepstar \cpsepr} and \im{\cpsepr \equiv \se^{\sprime\div}}
    \item If \im{\styjudg{\slenv}{\sA}{\sK}} and \im{\sA \stepstar \sApr} then
      \im{\sA^{+} \stepstar \cpsApr} and \im{\cpsApr \equiv \sA^{\sprime+}}
    \item If \im{\styjudg{\slenv}{\sA}{\sstarty}} and \im{\sA \stepstar \sApr} then
      \im{\sA^{\div} \stepstar \cpsApr} and \im{\cpsApr \equiv \sA^{\sprime\div}}
    \item If \im{\styjudg{\slenv}{\sK}{\sU}} and \im{\sK \stepstar \sKpr} then
      \im{\sK^{+} \stepstar \cpsKpr} and \im{\cpsKpr \equiv \sK^{\sprime+}}
  \end{itemize}
\end{lemma}
\begin{proof}
  The proof is straightforward by induction on the derivation of \im{\st
    \stepstar \stpr}.\footnote{In the previous version of this
    work~\cite{bowman2018:cps-sigma}, this proof was incorrectly stated as by
    induction on the length of reduction sequences.}
\end{proof}

\begin{lemma}[\cbvname{} Preservation of Equivalence]
  \label{lem:cps:cbv:pres-equiv}
  ~
  \begin{multicols}{2}
  \begin{itemize}
    \item If \im{\se \equiv \sepr} then \im{\se^{\div} \equiv \se^{\sprime\div}}
    \item If \im{\sA \equiv \sApr} then \im{\sA^{+} \equiv \sA^{\sprime+}}
    \item If \im{\sA \equiv \sApr} then \im{\sA^{\div} \equiv \sA^{\sprime\div}}
    \item If \im{\sK \equiv \sKpr} then \im{\sK^{+} \equiv \sK^{\sprime+}}
  \end{itemize}
  \end{multicols}
\end{lemma}
%
\begin{lemma}[\cbvname{} Type and Well-formedness Preservation]
  \label{lem:cps:cbv:type-pres}
  ~
  \begin{enumerate}
  \item If \im{\swf{\slenv}} then \im{\cpswf{\slenv^{+}}}
  \item If \im{\styjudg{\slenv}{\se}{\sA}} then
    \im{\styjudg{\slenv^{+}}{\se^{\div}}{\sA^{\div}}}
  \item If \im{\styjudg{\slenv}{\sA}{\sK}} then
    \im{\cpstyjudg{\slenv^{+}}{\sA^{+}}{\sK^+}}
  \item If \im{\styjudg{\slenv}{\sA}{\sstarty}} then
    \im{\styjudg{\slenv^{+}}{\sA^{\div}}{\sstarty^+}}
  \item If \im{\styjudg{\slenv}{\sK}{\sU}} then
    \im{\cpstyjudg{\slenv^{+}}{\sK^{+}}{\sU^+}}
  \end{enumerate}
\end{lemma}
\begin{proof}
  All cases are proven simultaneously by simultaneous induction on the type
  derivation and well-formedness derivation.
  Part 4 follows easily by part 3 in every case, so we elide its proof.
  Most cases follow easily from the induction hypotheses.
  \begin{proofcases}
  \item \refrule[refcpssrc]{W-Assum} \im{\swf{\slenv,\sx:\sA}}

    There are two sub-cases: either \im{\sA} is a type or a kind.

    \item[{\bfseries Sub-case:}] \im{\sA} is a type

    We must show \im{\cpswf{\slenv^+,\cpsx:\sA^+}}.

    It suffices to show that \im{\cpstyjudg{\slenv^+}{\sA^+}{\cpsK}}, which follows by
    part 3 of the induction hypothesis.

    \item[{\bfseries Sub-case:}] \im{\sA} is a kind; similar to the previous case, except the goal follows by part 5 of
    the induction hypothesis.

    \item \refrule[refcpssrc]{W-Def} \im{\swf{\slenv,\sx = \se:\sA}}

    We give the case for when \im{\sA} is a type; the case when \im{\sA} is a kind is
    similar.

    We must show \im{\cpswf{\slenv^+,\cpsx = \cpsncappe{\se^\div}{\sA^+}{\cpsidk}:\sA^+}}.

    It suffices to show that
    \im{\cpstyjudg{\slenv^+}{\cpsncappe{\se^\div}{\sA^+}{\cpsidk}}{\sA^+}}.

    By part 2 of the induction hypothesis and definition of the translation, we know that
    \im{\cpstyjudg{\slenv^+}{\se^{\div}}{\cpspity{\cpsalpha}{\cpsstarty}{\cpsfunty{(\cpsfunty{\sA^+}{\cpsalpha})}{\cpsalpha}}}},
    easily which implies the goal.

    \item \refrule[cpssrc]{Var} \im{\styjudg{\slenv}{\sx}{\sA}}

    We give the case for when \im{\sA} is a type; the case when \im{\sA} is a
    kind is simple since the translation on type variables is the identity.

    We must show that
    \im{\cpstyjudg{\slenv^+}{\cpsfune{\cpsalpha}{\cpsstarty}{\cpsfune{\cpsk}{\cpsfunty{\sA^+}{\cpsalpha}}{\cpsappe{\cpsk}{\cpsx}}}}{\sA^\div}}

    By the part 1 of the induction hypothesis, we know
    \im{\cpstyjudg{\slenv^+}{\cpsx}{\sA^+}}, which implies the goal.

    \item \refrule[cpssrc]{App}
    \im{\styjudg{\slenv}{\sappe{\seone}{\setwo}}{\subst{\sB}{\setwo}{\sx}}}.

    There are four sub-cases: \im{\seone} can be either a \tech{term} or a \tech{type}, and \im{\setwo}
    can be either a \tech{term} or a \tech{type}.
    The interesting case is when both are \tech{terms}, since this is the case most
    affected by the \tech{CPS} translation.

    \item[{\bfseries Sub-case:}] \cbverule{App}, both \im{\seone} and \im{\setwo} are terms.

    We must show that

    \begin{displaymath}
      {\cpsfune{\cpsalpha}{\cpsstarty}{
          \begin{stackTL}
            \cpsfune{\cpsk}{\cpsfunty{(\subst{\sB^{+}}{(\cpsncappe{\setwo^\div}{\sA^+}{\cpsidk})}{\cpsx})}{\cpsalpha}}{\\\quad
              \cpsncappe{\seone^\div}{\cpsalpha}{\begin{stackTL}(\cpsfune{\cpsf}{\cpspity{\cpsx}{\sA^+}{\sB^{\div}}}{
                    {\cpscappe{\setwo^\div}{\cpsalpha}{(\cpsfune{\cpsx}{\sA^{+}}{
                          \cpsncappe{(\cpsappe{\cpsf}{\cpsx})}{\cpsalpha}{\cpsk}})}})
                \end{stackTL}
            }}}
          \end{stackTL}
      }}
    \end{displaymath}
    has type \im{(\subst{\sB}{\setwo}{\sx})^\div}.

    Note that,
    \begin{align}
      & (\subst{\sB}{\setwo}{\sx}) \nonumber \\
      \equiv~& \subst{\sB^\div}{\cpsncappe{\setwo^\div}{\sA^+}{\cpsidk}}{\cpsx}
             & \text{by \fullref[]{lem:cps:cbv:subst}} \\
      \equiv~&
               \cpspity{\cpsalpha}{\cpsstarty}{\cpsfunty{
               (\cpsfunty{(\subst{\sB^+}{\cpsncappe{\setwo^\div}{\sA^+}{\cpsidk}}{\cpsx})}{\cpsalpha})}
               {\cpsalpha}}
               & \text{by translation}
    \end{align}
    Hence it suffices to show that

    \im{\cpstyjudg{\slenv^+,\cpsalpha:\cpsstarty,
        \cpsk:{\cpsfunty{(\subst{\sB^{+}}{(\cpsncappe{\setwo^\div}{\sA^+}{\cpsidk})}{\cpsx})}{
            \cpsalpha}}}{
        \cpsncappe{\seone^\div}{\cpsalpha}{%
            \begin{stackTL}(\cpsfune{\cpsf}{\cpspity{\cpsx}{\sA^+}{\sB^{\div}}}{
              \\\quad\,{\cpscappe{\setwo^\div}{\cpsalpha}{\begin{stackTL}(\cpsfune{\cpsx}{\sA^{+}}{
                      \\\cpsncappe{(\cpsappe{\cpsf}{\cpsx})}{\cpsalpha}{\cpsk}})
              \end{stackTL}}}})
            \end{stackTL}}}{\cpsalpha}}

    By part 2 of the induction hypothesis, we know that

    \im{\cpstyjudg{\slenv^+}{\seone^\div}{\cpspity{\cpsalpha}{\cpsstarty}{\cpsfunty{(\cpsfunty{(\cpspity{\cpsx}{\sA^+}{\sB^\div})}{\cpsalpha})}{\cpsalpha}}}},

    hence it suffices to show that

    \im{\cpstyjudg{\slenv^+,\begin{stackTL}\cpsalpha:\cpsstarty,
        \\\cpsk:\cpsfunty{(\subst{\sB^{+}}{(\cpsncappe{\setwo^\div}{\sA^+}{\cpsidk})}{\cpsx})}
        {\cpsalpha},
        \\
        \cpsf:\cpspity{\cpsx}{\sA^+}{\sB^{\div}}
        \end{stackTL}
      }{
        \cpscappe{\setwo^\div}{\cpsalpha}{(\cpsfune{\cpsx}{\sA^{+}}{
            \cpsncappe{(\cpsappe{\cpsf}{\cpsx})}{\cpsalpha}{\cpsk}})}}{\cpsalpha}}

    By \refrule{T-Cont}, we must show

    \im{\cpstyjudg{\slenv^+,\begin{stackTL}\cpsalpha:\cpsstarty,
        \\\cpsk:\cpsfunty{(\subst{\sB^{+}}{(\cpsncappe{\setwo^\div}{\sA^+}{\cpsidk})}{\cpsx})}
        {\cpsalpha},
        \\\cpsf:\cpspity{\cpsx}{\sA^+}{\sB^{\div}},
        \\\cpsx = \cpsncappe{\setwo^\div}{\sA^+}{\cpsidk},
        \end{stackTL}
      }{\cpsncappe{(\cpsappe{\cpsf}{\cpsx})}{\cpsalpha}{\cpsk}}{\cpsalpha}}

    Note that \im{\cpsappe{\cpsf}{\cpsx} : \subst{\sB^\div}{\cpsx}{\cpsx}} and
    \im{\subst{\sB^\div}{\cpsx}{\cpsx} =
      \cpspity{\cpsalpha}{\cpsstarty}{\cpsfunty{\cpsfunty{(\subst{\sB^+}{\cpsx}{\cpsx})}{\cpsalpha}}{\cpsalpha}}}.

    But \im{\cpsk : \cpsfunty{(\subst{\sB^{+}}{(\cpsncappe{\setwo^\div}{\sA^+}{\cpsidk})}{\cpsx})}{\cpsalpha}}.

    Hence it suffices to show that \im{(\subst{\sB^+}{\cpsx}{\cpsx}) \equiv
      (\subst{\sB^{+}}{(\cpsncappe{\setwo^\div}{\sA^+}{\cpsidk})}{\cpsx})}, which follows by
    \im{\delta} reduction on \im{\cpsx} since we have \im{\cpsx =
      \cpsncappe{\setwo^\div}{\sA^+}{\cpsidk}} by \refrule{T-Cont}.

    Note that without the new typing rule, we would be here stuck.
    However, thanks to \refrule{T-Cont}, we have the equality that \im{\cpsx =
      \cpsncappe{\setwo^\div}{\sA^+}{\cpsidk}}, and we are able to complete the
    proof.

  \item[{\bfseries Sub-case:}] \im{\seone} is a \tech{term} but \im{\setwo} is a \tech{type} \im{\sApr}.
    This case is similar to the application case of the \tech{CBN} translation.
    It does not require the new typing rule \refrule{T-Cont}, as the argument
    is a \tech{type}, the argument is not \tech{CPS} translated.

  \item[{\bfseries Sub-case:}] \im{\seone} is a \tech{type} and \im{\setwo} is a
    \tech{term}.
    This case is simple; note that the translate \cbvArule{AppConstr} translates
    the argument \im{\setwo} into \im{\cpsncappe{\setwo^\div}{\sA^+}{\cpsidk}}
    since the \tech{term} variable must have a \tech{value} type.

  \item[{\bfseries Sub-case:}] Both \im{\seone} and \im{\setwo} are \tech{types}.
    This case is trivial by the induction hypothesis.

    \item \refrule[cpssrc]{Let} \im{\styjudg{\slenv}{\salete{\sx}{\seone}{\sA}{\setwo}}{\subst{\sB}{\seone}{\sx}}}

    There are four sub-cases, as in the case of application, and the proofs are
    nearly identical.
    This should be unsurprising, since the new typing rule \refrule{T-Cont}
    essentially gives the typing of application of a \tech{continuation} the same
    expressive power as \tech{dependent let}.
    I give the case for both \im{\seone} and \im{\setwo} are \tech{terms}, since this
    is the most interesting case.

    \item[{\bfseries Sub-case:}] \cbverule{Let}

    \noindent We must show that \im{\cpstyjudg{\slenv^+}{
        \cpsfune{\cpsalpha}{\cpsstarty}{
            \begin{stackTL}\cpsfune{\cpsk}{\cpsfunty{\subst{\sB^+}{\cpsncappe{\setwo^\div}{\sA^+}{\cpsidk}}{\cpsx}}{\cpsalpha}}{
              \\\quad\cpscappe{\seone^\div}{\cpsalpha}{(\cpsfune{\cpsx}{\sA^+}{
                  \cpsncappe{\setwo^\div}{\cpsalpha}{\cpsk}})}}}
             \end{stackTL}}{(\subst{\sB}{\setwo}{\sx})^\div}}

    By \fullref[]{lem:cps:cbv:subst} and the definition of the translation, it
    suffices to show

    \im{\cpstyjudg{\slenv^+,\cpsalpha:\cpsstarty,
        \cpsk:{\cpsfunty{\subst{\sB^+}{\cpsncappe{\setwo^\div}{\sA^+}{\cpsidk}}{\cpsx}}{
            \cpsalpha}}}{
        \cpscappe{\seone^\div}{\cpsalpha}{(\cpsfune{\cpsx}{\sA^+}{
            \cpsncappe{\setwo^\div}{\cpsalpha}{\cpsk}})}}{\cpsalpha}}

    By \refrule{T-Cont}, we must show

    \im{\cpstyjudg{\slenv^+,\cpsalpha:\cpsstarty,
        \cpsk:{\cpsfunty{\subst{\sB^+}{\cpsncappe{\setwo^\div}{\sA^+}{\cpsidk}}{\cpsx}}
          {\cpsalpha}},
        \cpsx = \cpsncappe{\seone^\div}{\sA^+}{\cpsidk}
      }{\cpsncappe{\setwo^\div}{\cpsalpha}{\cpsk}}{\cpsalpha}}

    Note that by the induction hypothesis,

    \im{\cpstyjudg{\slenv^+,\cpsx = \cpsncappe{\seone^\div}{\sA^+}{\cpsidk}}{\setwo^\div}{\cpspity{\cpsalpha}{\cpsstarty}{\cpsfunty{(\cpsfunty{\sB^+}{\cpsalpha})}{\cpsalpha}}}}

    Hence by \(\delta\)-reduction and \refrule[refcps]{Conv},

    \im{\cpstyjudg{\slenv^+,\cpsx = \cpsncappe{\seone^\div}{\sA^+}{\cpsidk}}{\setwo^\div}{\cpspity{\cpsalpha}{\cpsstarty}{\cpsfunty{(\cpsfunty{\subst{\sB^+}{(\cpsncappe{\se^\div}{\sA^+}{\cpsidk})}{\cpsx}}{\cpsalpha})}{\cpsalpha}}}}
    which implies the goal.

    \item \refrule[cpssrc]{Pair}

    Note that the translation of \tech{dependent pairs}, \cbverule{Pair}, also requires a
    use of the rule \refrule{T-Cont}.
    Since the source language allows pairs of \tech{expressions}, but our target language
    for the \tech{CBV} translation should not, we must evaluate both components
    of the pair before calling the \tech{continuation}.
    However, since the type of second component depends on the \tech{value} of
    the first component, we must apply \refrule{T-Cont} when typing the
    application of the \tech{continuation} to the first component so that we
    have \im{\cpsxone = \cpsncappe{\seone^\div}{\sA^+}{\cpsidk}} when typing the
    continuation for the second component.

    The proof is similar to the case for \refrule[cpssrc]{App}.

    \item \refrule[cpssrc]{Snd} The proof is exactly like the case for the \tech{CBN} translation. \qedhere
  \end{proofcases}
\end{proof}

\begin{theorem}[\cbvname{} Type Preservation]
  \label{thm:cps:cbv:type-pres}
  ~
  If \im{\styjudg{\slenv}{\se}{\sA}} then
  \im{\styjudg{\slenv^{+}}{\cpsterm{\se}}{\cpstype{\sA}}}.
\end{theorem}

\subsection{Compiler Correctness}
To prove correctness of separate compilation for \cbvname, I follow the standard
architecture from \fullref[]{chp:type-pres} and used in
\fullref[]{sec:cps:cbn:correct}.
I use the same cross-language relation \im{\approx} on \tech{observation}.
However, note that in \tech{CBV}, we should only link with \tech{values}, so I
restrict closing substitutions \im{\ssubst} to \tech{values} and use the
\tech{value translation} on substitutions \im{\ssubst^+}.
The proofs follow exactly the same structure as in
\fullref[]{sec:cps:cbn:correct}.

\begin{theorem}[Separate Compilation Correctness]
  \label{thm:cps:cbv:sep-comp}
  If \im{\wf{\slenv}{\se}} and \im{\wf{\slenv}{\ssubst}}, then \\
  \im{\seval{\ssubst(\se)} \approx \teval{\ssubst^{+}(\se^\div)}}.
\end{theorem}

\begin{corollary}[Whole-Program Correctness]
  \label{thm:cps:cbv:correctness}
  If \im{\wf{}{\se}} then
  \im{\seval{\se} \approx \teval{\se^\div}}.
\end{corollary}
}
