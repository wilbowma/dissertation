\newcommand{\FigCoCCCTermShort}[1][t]{
  \begin{figure}[#1]
    \judgshape[~where~{\im{\styjudg{\slenv}{\se}{\sA}}}]{\cctrans{\se} = \pcce}
    \begin{displaymath}
      \begin{array}{rcl}
        \cctrans{\sx} & \defeq & \tx
        \\

        \cctrans{\sstarty} & \defeq & \pccstarty
        \\

        \cctrans{\spity{\sx}{\sA}{\sB}} & \defeq & \pccnexistty{(\pccalpha:\pccstarty,\pccn:\pccalpha)}{\pcctrlufunty{\pccn}{\pcccodety{\pccx:\pccA}{\pccB}}}
        \\

        \cctrans{\sfune{\sx}{\sA}{\se}} & \defeq &
        \pccnpackoe{
          \begin{stackTL}
            \pccnsigmaty{(\pccxi:\cctrans{\sAi}\dots)},
          \pccnpaire{\pccxi\dots},
            \\\begin{stackTL}
              (\pccnfune{\begin{stackTL}(\pccn:\pccnsigmaty{(\pccxi:\cctrans{\sAi}\dots)},\pccx:\pcclete{\pccnpaire{\pccxi\dots}}{\pccn}{\cctrans{\sA}})}{
                  \\
                  \pcclete{\pccnpaire{\pccxi\dots}}{\pccn}{\cctrans{\se}}})}
                \end{stackTL}
            \end{stackTL}
          \end{stackTL}
        \\&& \sxi:\sAi\dots{}=\DFV{\slenv}{\sfune{\sx}{\sA}{\se},\spity{\sx}{\sA}{\sB}} \\

        \cctrans{\sappe{\seone}{\setwo}} & \defeq & \pccnunpacke{\pccalpha,\pccn,\pccf}{\cctrans{\seone}}{\pccappe{\pccf}{\pccn~\cctrans{\setwo}}}
        \\

        \cctrans{\salete{\sx}{\se}{\sA}{\sepr}} & \defeq & \pccalete{\tx}{\cctrans{\se}}{\cctrans{\sA}}{\cctrans{\sepr}}
        \\

        & \vdots &
      \end{array}
    \end{displaymath}
    \caption{Parametric Closure Conversion from \pccslang to \pcctlang (excerpts)}
    \label{fig:param-cc:cc:short}
  \end{figure}
}

\newcommand{\FigCCTermOne}[1][t]{
  \begin{figure}[#1]
    \judgshape{\pccjudg{\slenv}{\se}{\sA}{\pcce}}
    \begin{mathpar}
      \inferrule*[right=\defrule{CC-Pi}]
      {\pccjudg{\slenv}{\sA}{\sU}{\pccA} \\
       \pccjudg{\slenv,\sx:\sA}{\sB}{\sstarty}{\pccB}}
      {\pccjudg{\slenv}{\spity{\sx}{\sA}{\sB}}{\sUpr}{
          \pccnexistty{(\pccalpha:\pccstarty,\pccn:\pccalpha)}{\pcctrlufunty{\pccn}{\pcccodety{\pccx:\pccA}{\pccB}}}}}

      \inferrule*[right=\defrule{CC-Lam}]
       {\pccjudg{\slenv,\sx:\st}{\se}{\stpr}{\pcce} \\
        \pccjudg{\slenv}{\st}{\sU}{\pcct} \\
        \pccjudg{\slenv,\sx:\st}{\stpr}{\sU}{\pcctpr} \\
        \sxi:\sAi\dots = \DFV{\slenv}{\sfune{\sx}{\sA}{\se},\spity{\sx}{\sA}{\sB}} \\
        \pccjudg{\slenv}{\sAi}{\sU}{\pccAi} \dots}
       {\pccjudg{\slenv}{\sfune{\sx}{\st}{\se}}{\spity{\sx}{\st}{\stpr}}{
            \pccnpackoe{\begin{stackTL}\pccnsigmaty{(\pccxi:\pccAi\dots)}, \pccnpaire{\pccxi\dots},
                   \\\pccnfune{\begin{stackTL}
                           (\begin{stackTL}\pccn:\pccnsigmaty{(\pccxi:\pccAi\dots)},\\\pccx:\pcclete{\pccnpaire{\pccxi\dots}}{\pccn}{\pcct}\end{stackTL})}
                          {\\\pcclete{\pccnpaire{\pccxi\dots}}{\pccn}{\pcce
          }}}}
                              \end{stackTL}\end{stackTL}}

      \inferrule*[right=\defrule{CC-App}]
      {\pccjudg{\slenv}{\seone}{\_}{\pcceone} \\
       \pccjudg{\slenv}{\setwo}{\_}{\pccetwo}}
      {\pccjudg{\slenv}{\sappe{\seone}{\setwo}}{\st}{
          \pccnunpacke{\pccalpha,\pccn,\pccf}{\pcceone}{\pccappe{\pccf}{\pccn~\pccetwo}}}}
    \end{mathpar}
    \caption{Parametric Closure Conversion from \pccslang to \pcctlang (1/2)}
    \label{fig:param-cc:cc:full1}
  \end{figure}
}

\newcommand{\FigCCTermTwo}[1][t]{
  \begin{figure}[#1]
    \judgshape{\pccjudg{\slenv}{\se}{\sA}{\pcce}}
    \begin{mathpar}
      \inferrule*[right=\defrule{CC-Var}]
      {~}
      {\pccjudg{\slenv}{\sx}{\sA}{\pccx}}

      \inferrule*[right=\defrule{CC-*}]
      {~}
      {\pccjudg{\slenv}{\sstarty}{\sboxty}{\pccstarty}}

      \inferrule*[right=\defrule{CC-Sig}]
      {\ccjudg{\slenv}{\sA}{\stypety{i}}{\tA} \\
       \ccjudg{\slenv,\sx:\sA}{\sB}{\stypety{i}}{\tB}}
      {\ccjudg{\slenv}{\ssigmaty{\sx}{\sA}{\sB}}{\stypety{i}}{\tsigmaty{\tx}{\tA}{\tB}}}

      \inferrule*[right=\defrule{CC-Pair}]
      {\ccjudg{\slenv}{\seone}{\sA}{\teone} \\
       \ccjudg{\slenv}{\setwo}{\subst{\sB}{\seone}{\sx}}{\tetwo} \\
       \ccjudg{\slenv}{\sA}{\stypety{i}}{\tA} \\
       \ccjudg{\slenv,\sx:\sA}{\sB}{\stypety{i}}{\tB}}
      {\ccjudg{\slenv}{\sdpaire{\seone}{\setwo}{\ssigmaty{\sx}{\sA}{\sB}}}{\ssigmaty{\sx}{\sA}{\sB}}{\tdpaire{\teone}{\tetwo}{\tsigmaty{\tx}{\tA}{\tB}}}}

      \inferrule*[right=\defrule{CC-Fst}]
      {\ccjudg{\slenv}{\se}{\ssigmaty{\sx}{\sA}{\sB}}{\te}}
      {\ccjudg{\slenv}{\sfste{\se}}{\sA}{\tfste{\te}}}

      \inferrule*[right=\defrule{CC-Snd}]
      {\ccjudg{\slenv}{\se}{\ssigmaty{\sx}{\sA}{\sB}}{\te}}
      {\ccjudg{\slenv}{\ssnde{\se}}{\subst{\sB}{\sfste{\se}}{\sx}}{\tsnde{\te}}}

      \inferrule*[right=\defrule{CC-Bool}]
      {~}
      {\pccjudg{\slenv}{\sboolty}{\sstarty}{\pccboolty}}

      \inferrule*[right=\defrule{CC-True}]
      {~}
      {\pccjudg{\slenv}{\struee}{\sboolty}{\pcctruee}}

      \inferrule*[right=\defrule{CC-False}]
      {~}
      {\pccjudg{\slenv}{\sfalsee}{\sboolty}{\pccfalsee}}

      \inferrule*[right=\defrule{CC-If}]
      {\pccjudg{\slenv}{\se}{\sboolty}{\pcce} \\
       \pccjudg{\slenv}{\seone}{\sB}{\pcceone} \\
       \pccjudg{\slenv}{\setwo}{\sB}{\pccetwo}}
      {\pccjudg{\slenv}{\sife{\se}{\seone}{\setwo}}{\sB}{\pccife{\pcce}{\pcceone}{\pccetwo}}}

      \inferrule*[right=\defrule{CC-Let}]
      {\ccjudg{\slenv}{\se}{\sA}{\te} \\
       \ccjudg{\slenv,\sx:\sA}{\sepr}{\sB}{\tepr}}
      {\ccjudg{\slenv}{\slete{\sx}{\se}{\sepr}}{\subst{\sB}{\se}{\sx}}{\tlete{\tx}{\te}{\tepr}}}

      \inferrule*[right=\defrule{CC-Conv}]
      {\ccjudg{\slenv}{\se}{\sA}{\te}}
      {\ccjudg{\slenv}{\se}{\sB}{\te}}
    \end{mathpar}
    \judgshape[~where~{\im{\swf{\slenv}}}]{\ccenvjudg{\slenv}{\tlenv}}
    \begin{mathpar}
      \inferrule*[right=\defrule{CC-Empty}]
      {~}
      {\ccenvjudg{\cdot}{\cdot}}

      \inferrule*[right=\defrule{CC-Assum}]
      {\ccenvjudg{\slenv}{\tlenv} \\
       \ccjudg{\slenv}{\sA}{\sU}{\tA}}
      {\ccenvjudg{\slenv,\sx:\sA}{\tlenv,\tx:\tA}}

      \inferrule*[right=\defrule{CC-Def}]
      {\ccenvjudg{\slenv}{\tlenv} \\
       \ccjudg{\slenv}{\se}{\sA}{\te} \\
       \pccjudg{\slenv}{\sA}{\sU}{\pccA}}
      {\ccenvjudg{\slenv,\sx = \se : \sA}{\tlenv,\tx = \te : \pccA}}
    \end{mathpar}
    \caption{Parametric Closure Conversion from \pccslang to \pcctlang (2/2)}
    \label{fig:param-cc:cc:full2}
  \end{figure}
}

\section{Parametric Closure Conversion Translation}
\label{sec:param-cc:cc}

\FigCoCCCTermShort
In \fullref[]{fig:param-cc:cc:short}, I present the key parametric
closure-conversion rules.
As in \fullref[]{chp:abs-cc}, formally, this translation must be defined on
typing derivations, but for brevity I present the key translation rules on syntax.
I define the complete translation in \fullref[]{sec:param-cc:cc:appendix}
\fullref[]{fig:param-cc:cc:full1} and \fullref[]{fig:param-cc:cc:full2}.
The translation of \tech{dependent function} types is the key to understanding
our translation.
As described in \fullref[]{sec:param-cc:ideas}, the idea in this translation is
to translate \tech{dependent functions} of type \im{\spity{\sx}{\sA}{\sB}} to
\tech{existential packages} of type
\im{\pccnexistty{\pccalpha:\pccstarty,\pccn:\pccalpha}{(\pcctrlufunty{\pccn}{\pcccodety{\pccx:\pccA}{\pccB}})}}.
In the translation of \tech{types}, we leave type variables free in the \tech{type} of the
\tech{closure}--\ie, in \im{\pccA} and \im{\pccB}--but we leave them free under
a \tech{translucent type}
\im{(\pcctrlufunty{\pccn}{\pcccodety{\pccx:\pccA}{\pccB}})} that describes the
new closed \tech{Code}.
In the translation of functions, \im{\cctrans{\sfune{\sx}{\sA}{\se}}}, we produce
closed type annotations that are only valid with respect to the
\tech{environment} we produce in the \tech{closure}.
When we check that the \tech{closure} produced by the translation has the
\tech{type} produced by the translation, \ie, when showing \tech{type
  preservation} for functions, the \tech{translucent type} introduces an
equality \im{\pccn = env} into the type, unifying the closed \tech{type} of the
\tech{closure} and the open \tech{type} of the \tech{code}.

The rest of the translation is straightforward.
We translate each \im{\sstarty} to \im{\pccstarty}.
Similarly, we translate names \im{\sx} into \im{\pccx}.
All omitted rules are structural.

\subsection{Type Preservation}
In this section, I show that \tech{parametric closure-conversion} is \tech{type
  preserving}.
I follow the standard architecture presented in \fullref[]{chp:type-pres}.

Recall from \fullref[]{sec:param-cc:ideas} that the \tech{type preservation}
argument relies on an equivalence between a \tech{type} \im{\pccA} with free
variables \im{\pccxi\dots} and the same \tech{type} with those free variables
projected out of the \tech{closure}'s \tech{environment}
\im{\pccnpaire{\pccxi\dots}}, \ie,
\im{\pccA \equiv \subst{\pccA}{\pccnpaire{\pccxi\dots}}{\pccnpaire{\pccxi\dots}}}.
Since the \tech{environment} is always generated from the same free variables,
this substitution is essentially a no-op.
This argument is used in several proofs.
I formalize this \tech{equivalence} in \fullref[]{lem:param-cc:cc:env-id}.
The proof is a straightforward computation.
\begin{lemma}
  \label{lem:param-cc:cc:env-id}
  \im{\pccequivjudg{\pcclenv,\pccn=\pccnpaire{\pccxi\dots}:\pccApr}{(\pcclete{\pccnpaire{\pccxi\dots}}{\pccn}{\pcce})}{\pcce}}.
\end{lemma}
\begin{proof}
  \im{\pcclete{{\pccnpaire{\pccxi\dots}}}{\pccn}{\pcce} \step_{\delta} \pcclete{{\pccnpaire{\pccxi\dots}}}{{\pccnpaire{\pccxi\dots}}}{\pcce} \step_{\zeta}^{n}
    \subst{\pcce}{\pccxi}{\pccxi} = \pcce}
\end{proof}

Next I prove \tech{compositionality}.
As in \fullref[]{chp:abs-cc}, this is where we use most of the complexity of the
target language, such as \tech{translucent types} and the \(\eta\)-principle for
\tech{closures}.
I go into detail for the key proof cases, \ie, those for the encoding of \tech{closures}.
\begin{lemma}[Compositionality]
  \label{lem:cc:pres-subst}
  \im{\cctrans{(\subst{\se}{\sepr}{\sx})} \equiv \subst{\cctrans{\se}}{\cctrans{\sepr}}{\sx}}

\end{lemma}
\begin{proof}
  By cases on structure of \im{\se}.
  Omitted cases are straightforward.
  \begin{proofcases}
  \item \im{\se = \spity{\sy}{\sA}{\sB}}

    Show \im{\cctrans{(\subst{(\spity{\sy}{\sA}{\sB})}{\sepr}{\sx})} \equiv \subst{\cctrans{(\spity{\sy}{\sA}{\sB})}}{\cctrans{\sepr}}{\sx}}.
    \begin{align}
      & \cctrans{(\subst{(\spity{\sy}{\sA}{\sB})}{\sepr}{\sx})} \nonumber \\
      =~& \cctrans{(\spity{\sy}{\subst{\sA}{\sepr}{\sx}}{\subst{\sB}{\sepr}{\sx}})} \\
      & \text{by substitution} \nonumber \\
      =~& \pccnexistty{(\pccalpha:\pccstarty,\pccn:\pccalpha)}{\pcctrlufunty{\pccn}{\pcccodety{\pccy:\subst{\cctrans{\sA}}{\cctrans{\sepr}}{\pccx}}{\subst{\cctrans{\sB}}{\cctrans{\sepr}}{\pccx}}}} \\
      & \text{by definition of translation} \nonumber \\
      =~& \subst{(\pccnexistty{(\pccalpha:\pccstarty,\pccn:\pccalpha)}{\pcctrlufunty{\pccn}{\pcccodety{\pccy:\cctrans{\sA}}{\cctrans{\sB}}}})}{\cctrans{\sepr}}{\sx} \\
      & \text{by substitution} \nonumber \\
      =~& \subst{\cctrans{(\spity{\sy}{\sA}{\sB})}}{\cctrans{\sepr}}{\sx}
    \end{align}
  \item \im{\se = \sfune{\sx}{\sA}{\seone}}.\\
    This case essentially asks ``when are two \tech{closures} equivalent?'',
    since substitution on a function changes its \tech{closure} by changing the
    \tech{environment}.
    The \(\eta\) rules for \tech{closures} allows us to answer that question.
    When substituting into a \tech{closure} before translation, we produce a
    \tech{closure} whose \tech{environment} contains only free variables.
    When substituting into a \tech{closure} after translation, we can put a
    (potentially open) term into the \tech{environment} of a \tech{closure}.
    \begin{displaymath}
      \begin{stackTL}
        p_L = \cctrans{(\subst{\se}{\sepr}{\sx})} = \pccnpackoe{\Sigma_{env}',
          env', \begin{stackTL}\pccnfune{\pccn:\Sigma_{env}',\pccy:\pccAin{L}}{\\\quad\pcclete{env'}{\pccn}{\subst{\cctrans{\seone}}{\cctrans{\sepr}}{\pccx}}}}\end{stackTL}
        \\\qquad\where{
          \begin{stackTL}
            env' =
            \pccnpaire{\pccxin{0},\dots,\pccxin{i-1},\pccxin{i+1},\dots,\pccxin{m},\pccxin{m+1},\dots,\pccxin{n}}\\
            \pccxin{0},\dots,\pccxin{i-1},\pccx,\pccxin{i+1},\dots,\pccxin{m} = \FV{(\cctrans{\se})}\\
            \pccxin{m+1},\dots,\pccxin{n} = \FV{(\cctrans{\sepr})}\\
            \pccAin{L} = \pcclete{env'}{\pccn}{\cctrans{\sA}}\\
          \end{stackTL}
        } \\
        p_R =\subst{\cctrans{\se}}{\cctrans{\sepr}}{\sx} = \pccnpackoe{\Sigma_{env}, env, \pccnfune{\pccn:\Sigma_{env},\pccy:\pccAin{R}}{\pcclete{env}{\pccn}{\cctrans{\se}}}}
        \\
        \qquad\where{
          \begin{stackTL}
            env =
            \pccnpaire{\pccxin{0},\dots,\pccxin{i-1},\cctrans{\sepr},\pccxin{i+1},\dots,\pccxin{m}}\\
            \pccxin{0},\dots,\pccxin{i-1},\pccx,\pccxin{i+1},\dots,\pccxin{m} = \FV{(\cctrans{\st})}\\
            \pcctin{R} = \pcclete{env}{\pccn}{\cctrans{\sA}}\\
          \end{stackTL}
        }
    \end{stackTL}
  \end{displaymath}
  Note that in \im{p_L}, the \tech{environment} \im{env'} is missing \im{\pccx}
  between \im{\pccxin{i-1}} and \im{\pccxin{i+1}} and contains the free
  variables from \im{\cctrans{\sepr}}.
  In \im{p_R}, the \tech{environment} \im{env} contains \im{\cctrans{\sepr}} in
  place of \im{\pccx}.
  To show that \im{p_L \equiv p_R}, it suffices by
  \refrule*{eqv-eta1}{CC-\im{\equiv}-\im{\eta_1}} to show
    \im{\subst{\cctrans{\se}}{\cctrans{\sepr}}{\pccx}
    \equiv
    \pccnunpacke{\pccalpha, \pccn, \pccf}{p_R}{\pccappe{\pccf}{\pccn~\pccy}}}. Observe that
    \begin{displaymath}
      \begin{stackTL}
        \pccnunpacke{\pccalpha, \pccn, \pccf}{p_R}{\pccappe{\pccf}{\pccn~\pccy}} \\
        \stepstar \pcclete{env}{env}{\cctrans{\se}} \\
        \step_\zeta \pcclete{\pccnpaire{\pccxin{0},\dots,\pccxin{i-1},\pccxin{i+1},\dots,\pccxin{m}}}{\pccnpaire{\pccxin{0},\dots,\pccxin{i-1},\pccxin{i+1},\dots,\pccxin{m}}}{\subst{\cctrans{\se}}{\cctrans{\sepr}}{\pccx}}\\
        \equiv \subst{\cctrans{\se}}{\cctrans{\sepr}}{\pccx}\text{ by \fullref[]{lem:param-cc:cc:env-id}.}
        \end{stackTL}
      \end{displaymath}
  \end{proofcases}
\end{proof}

Now we can show that the translation preserves \tech{reduction} up to
\tech{equivalence}, \ie, that the translation \im{\pcce} of a term \im{\se}
weakly simulates one step of reduction of \im{\se}.
Since \fullref[]{lem:cc:pres-subst} is in terms of \tech{definitional
  equivalence}, I show weak simulation up to \tech{definitional equivalence}.
I give the key cases.
\begin{lemma}[Preservation of Reduction]
  \label{lem:cc:pres-red}
  If \im{\sstepjudg[\step_x]{\slenv}{\se}{\sepr}} then
  \im{\pccstepjudg[\stepstar]{\cctrans{\slenv}}{\cctrans{\se}}{\pcce}} and \im{\pcce \equiv \cctrans{\sepr}}
\end{lemma}
\begin{proof}
  By cases on the judgment \im{\se \step_x \sepr}.
  All cases are straightforward by computation
  and \fullref[]{lem:cc:pres-subst}.
  \begin{proofcases}
  \item \im{\sx \step_\delta \sepr~\where{\sx = \sepr \in \slenv}}

    Suffices to show that \im{\pccx \step_\delta \cctrans{\sepr}~\where{\pccx = \cctrans{\sepr} \in \slenv}}, which follows by definition of \im{\pccenvjudg{\slenv}{\pcclenv}}.
  \item \im{\sappe{(\sfune{\sx}{\sA}{\seone})}{\setwo} \step_\beta \subst{\seone}{\setwo}{\sx}}

    We must show that \im{\cctrans{(\sappe{(\sfune{\sx}{\sA}{\seone})}{\setwo})}
      \stepstar \pccepr \equiv  \cctrans{(\subst{\seone}{\setwo}{\sx})}}.

    \begin{align}
      & \cctrans{(\sappe{(\sfune{\sx}{\sA}{\seone})}{\setwo})} \nonumber \\
      =~&
      \begin{stackTL}
      \pccnunpacke{\pccalpha,\pccn,\pccf}{(\pccnpackoe{\pccApr,\pccnpaire{\pccxi\dots},\begin{stackTL}
            \pccnfune{(\pccn:\pccApr,\pccx:\pccA)}{
              \\\quad
              \pcclete{\pccnpaire{\pccxi\dots}}{\pccn}{\cctrans{\seone}}}}
      )\end{stackTL}}{\\\quad\pccappe{\pccf}{\pccn~\cctrans{\setwo}}}
      \end{stackTL} \\
      \step_\exists~& \pccappe{(\begin{stackTL}
          \pccnfune{(\pccn:\pccApr,\pccx:\pccA)}{
              \pcclete{\pccnpaire{\pccxi\dots}}{\pccn}{\cctrans{\seone}}}
      \end{stackTL})}{\pccnpaire{\pccxi\dots}~\cctrans{\setwo}} \\
      \step_\beta~& \subst{\pcclete{\pccnpaire{\pccxi\dots}}{\pccnpaire{\pccxi\dots}}{\cctrans{\seone}}}{\cctrans{\setwo}}{\pccx} \\
      \equiv~& \subst{\cctrans{\seone}}{\cctrans{\setwo}}{\pccx} \\
      & \text{by \fullref{lem:param-cc:cc:env-id}} \nonumber \\
      \equiv~& \cctrans{(\subst{\seone}{\setwo}{\sx})} \\
      & \text{by \fullref{lem:cc:pres-subst}} \nonumber
    \end{align}
  \end{proofcases}
\end{proof}

\begin{lemma}[Preservation of Conversion]
  \label{lem:cc:pres-norm}
  If \im{\sstepjudg[\stepstar]{\slenv}{\se}{\sepr}} then
  \im{\pccstepjudg[\stepstar]{\cctrans{\slenv}}{\cctrans{\se}}{\pcce}} and \im{\pccequivjudg{\cctrans{\slenv}}{\pcce}{\cctrans{\sepr}}}.
\end{lemma}
\begin{proof}
  The proof is by induction on the derivation of \im{\sstepjudg[\stepstar]{\slenv}{\se}{\sepr}}.
  The proof is entirely uninteresting, following essentially from \fullref{lem:cc:pres-red}.
\end{proof}

To show that \tech{equivalence} is preserved, we require \(\eta\)-equivalence
for \tech{closures} in the case of \(\eta\)-equivalent source terms.
\begin{lemma}[Preservation of Equivalence]
  \label{lem:cc:pres-equiv}
  If \im{\sequivjudg{\slenv}{\se}{\sepr}}, then
  \im{\pccequivjudg{\cctrans{\slenv}}{\cctrans{\se}}{\seto{\sprime+}}}.
\end{lemma}
\begin{proof}
  By induction on the \im{\se \equiv \sepr} judgment.
  \begin{proofcases}
      \item \refrule*[refcpssrc]{eqv}{\im{\equiv}}
      By assumption, \im{\se \stepstar \seone} and \im{\sepr \stepstar \seone}.

      By \fullref[]{lem:cc:pres-norm}, \im{\cctrans{\se} \stepstar \pcce} and
      \im{\pcce \equiv \cctrans{\seone}}, and similarly.
      \im{\cctrans{\sepr} \stepstar \pccepr} and \im{\pccepr \equiv \cctrans{\seone}}.
      The result follows by symmetry and transitivity of \im{\equiv}.
    \item \refrule*[refcpssrc]{eqv-eta1}{\im{\equiv}-\im{\eta_1}}
    By assumption, \im{\se \stepstar \sfune{\sx}{\st}{\seone}}, \im{\sepr \stepstar \setwo} and
    \im{\seone \equiv \sappe{\setwo}{\sx}}.

    Must show \im{\cctrans{\se} \equiv \cctrans{\sepr}}.

    By \fullref{lem:cc:pres-norm}, \im{\cctrans{\se} \stepstar \pcce} and \im{\pcce \equiv
      \cctrans{\sfune{\sx}{\st}{\seone}}}, and similarly \im{\cctrans{\sepr} \stepstar \pccepr} and \im{\pccepr
      \equiv \cctrans{\setwo}}.

    By transitivity, it suffices to show \im{\cctrans{\sfune{\sx}{\st}{\seone}}
      \equiv \cctrans{\setwo}}.

    By \refrule*{eqv-eta1}{\im{\equiv}-\im{\eta_1}} and
    \fullref{lem:param-cc:cc:env-id}, it suffices to show
    \begin{displaymath}
      \cctrans{\seone} \equiv
      {\pccnunpacke{\pccalpha,\pccn,\pccf}{\cctrans{\setwo}}{\pccappe{\pccf}{\pccn~\pccx}}}.
    \end{displaymath}

    Note that
    \im{\pccnunpacke{\pccalpha,\pccn,\pccf}{\cctrans{\setwo}}{\pccappe{\pccf}{\pccn~\pccx}}}
    is exactly \im{\cctrans{(\sappe{\setwo}{\sx})}}, so the result follows by the
    induction hypothesis.

    \item \refrule*[refcpssrc]{eqv-eta2}{\im{\equiv}-\im{\eta_2}}
    Symmetric to the previous case; requires
    \refrule*{eqv-eta2}{\im{\equiv}-\im{\eta_2}} instead of
    \refrule*{eqv-eta1}{\im{\equiv}-\im{\eta_1}}.
  \end{proofcases}
\end{proof}

Now we can prove the central lemma necessary for showing \tech{type
  preservation}.
\begin{lemma}[Type and Well-formedness Preservation]
  \label{lem:cc:type-pres}
  ~
  \begin{enumerate}
    \item If \im{\wf{\slenv}} then \im{\pccwf{\cctrans{\slenv}}}
    \item If \im{\styjudg{\slenv}{\se}{\sA}} then \im{\pcctyjudg{\cctrans{\slenv}}{\cctrans{\se}}{\cctrans{\sA}}}
  \end{enumerate}
\end{lemma}
\begin{proof}
  Parts 1 and 2 proven by simultaneous induction on the mutually defined judgments
    \im{\wf{\slenv}} and \im{\styjudg{\slenv}{\se}{\sA}}.

  Part 1 follows easily by induction and part 2.
  I give the key cases for part 2.

  \begin{proofcases}
    \item \refrule[refcpssrc]{*}, follows by part 1.
    \item \refrule[refcpssrc]{Var}, follows by part 1.
    \item \refrule[refcpssrc]{Pi-*}

    We have that \im{\styjudg{\slenv}{\spity{\sx}{\sA}{\sB}}{\sstarty}}.

    We must show that \im{\pcctyjudg{\pcclenv}{\pccnexistty{(\pccalpha:\pccstarty,\pccn:\pccalpha)}{\pcctrlufunty{\pccn}{\pcccodety{\pccx:\cctrans{\sA}}}{\cctrans{\sB}}}}{\pccstarty}}

    By \refrule{Exist}, it suffices to show
    \im{\pcctyjudg{\pcclenv,\pccalpha:\pccstarty,\pccn:\pccalpha}{(\pcctrlufunty{\pccn}{\pcccodety{\pccx:\cctrans{\sA}}}{\cctrans{\sB}})}{\pccstarty}}.

    By \refrule*{trfunty}{\im{\trlufunarrow}}, it suffices to show \im{\pcctyjudg{\pcclenv,\pccalpha:\pccstarty,\pccn:\pccalpha}{\pcccodety{\pccnpr:\pccalpha,\pccx:\cctrans{\sA}}{\cctrans{\sB}}}{\pccstarty}}

    By \refrule[refpcc]{Code-*}, it suffices to show
    \im{\pcctyjudg{\pcclenv,\pccalpha:\pccstarty,\pccn:\pccalpha,\pccnpr:\pccalpha,\pccx:\cctrans{\sA}}{\cctrans{\sB}}{\pccstarty}},
    which follows by the induction hypothesis applied to \im{\styjudg{\slenv,\sx:\sA}{\sB}{\sstarty}}.
    \item \refrule[refcpssrc]{Lam}

    We must show that
    \im{\pcctyjudg{\pcclenv}{\cctrans{(\sfune{\sx}{\sA}{\se})}}{\cctrans{(\spity{\sx}{\sA}{\sB})}}}

    The key to this proof is to show that

    \im{\pcctyjudg{\pcclenv}{(\pccnfune{(\pccn:\pccApr,\pccx:\pcclete{env}{\pccn}{\cctrans{\sA}})}{\pcclete{env}{\pccn}{\cctrans{\se}}})}{\pcctrlufunty{env}{\pcccodety{\pccx:\cctrans{\sA}}{\cctrans{\sB}}}}}

    \im{\where{\begin{stackTL}
          env = \tnpaire{\pccxin{0},\dotsm\pccxin{n}} \\
          \pccApr = \pccnsigmaty{(\pccxin{0}:\pccAin{0},\dots,\pccxin{n}:\pccAin{n})}
      \end{stackTL}}}

    Note that by definition, \im{\pcctyjudg{\pcclenv}{env}{\pccApr}}.

    Note also that
    \im{\pccequivjudg{\pcclenv,\pccn=env:\pccApr}{\pcclete{env}{\pccn}{\cctrans{\sA}}}{\cctrans{\sA}}},
    by \fullref{lem:param-cc:cc:env-id}.

    So, by \refrule[refpcc]{Conv} and \refrule*{trfunty}{\im{\trlufunarrow}}, it suffices to show:

    \im{\pcctyjudg{\pcclenv}{\begin{stackTL}\pccnfune{(\pccn:\pccApr,\pccx:\pcclete{env}{\pccn}{\cctrans{\sA}})}{\pcclete{env}{\pccn}{\cctrans{\se}}}}{
        \\\pcccodety{(\pccn:\pccApr,\pccx:\pcclete{env}{\pccn}{\cctrans{\sA}})}{\pcclete{env}{\pccn}{\cctrans{\sB}}}}
    \end{stackTL}}

    By \refrule[refpcc]{Code} and \refrule[refpcc]{Let}, it suffices to show that

    \im{\pcctyjudg{\pccn:\pccApr,\pccx:\pcclete{env}{\pccn}{\cctrans{\sA}},\pccxin{0} =
          \pccprje{0}{\pccn} : \pcctin{0},\dots,\pccxin{n} = \pccprje{n}{\pccn} : \pccAin{n}}{\cctrans{\se}}{\cctrans{\sB}}}

      Note that by the induction hypothesis we know that
      \im{\pcctyjudg{\pcclenv,\pccx:\cctrans{\sA}}{\cctrans{\se}}{\cctrans{\sB}}}.

      The goal follows since, by definition of the translation, the free
      variables of \im{\cctrans{\se}} are exactly \im{\pccxin{0},\pccxin{n}},
      and all other variables of \im{\pcclenv} are not referenced in
      \im{\cctrans{\se}} or \im{\cctrans{\sB}}.

    \item \refrule[refcpssrc]{App}

    We must show that
    \im{\pcctyjudg{\pcclenv}{\cctrans{(\sappe{\se}{\sepr})}}{\cctrans{(\subst{\sB}{\sepr}{\sx})}}}

    That is, we must show

    \im{\pcctyjudg{\pcclenv}{\pccnunpacke{\pccalpha,\pccn,\pccf}{\cctrans{\se}}{\pccappe{\pccf}{\pccn~\cctrans{\sepr}}}}{\cctrans{(\subst{\sB}{\sepr}{\sx})}}}

    By \refrule{Unpack}, it suffices to show, (1):

    \im{\pcctyjudg{\pcclenv}{\cctrans{\se}}{\pccnexistty{\pccalpha:\pccstarty,\pccn:\pccalpha}{\pcctrlufunty{\pccn}{\pcccodety{\pccx:\cctrans{\sA}}{\cctrans{\sB}}}}}},

    which follows immediately from the induction hypothesis, and (2):

    \im{\pcctyjudg{\pcclenv,\pccalpha:\pccstarty,\pccn:\pccalpha,\pccf:\pcctrlufunty{\pccn}{\pcccodety{\pccx:\cctrans{\sA}}{\cctrans{\sB}}}}{\pccappe{\pccf}{\pccn~\cctrans{\sepr}}}{\cctrans{(\subst{\sB}{\sepr}{\sx})}}}

    By \refrule[refpcc]{Conv} and \fullref{lem:cc:pres-subst}, it suffices to show

    \im{\pcctyjudg{\pcclenv,\pccalpha:\pccstarty,\pccn:\pccalpha,\pccf:\pcctrlufunty{\pccn}{\pcccodety{\pccx:\cctrans{\sA}}{\cctrans{\sB}}}}{\pccappe{\pccf}{\pccn~\cctrans{\sepr}}}{\subst{\cctrans{\sB}}{\cctrans{\sepr}}{\pccx}}}

    By \refrule{TrApp}, it suffices to show that

    \im{\pcctyjudg{\pcclenv,\pccalpha:\pccstarty,\pccn:\pccalpha,\pccy:\pcctrlufunty{\pccn}{\pcccodety{\pccx:\cctrans{\sA}}{\cctrans{\sB}}}}{\cctrans{\sepr}}{\cctrans{\sA}}},
    which follows by the induction hypothesis.
  \end{proofcases}
\end{proof}

\begin{corollary}[Type Preservation]
  If \im{\styjudg{\slenv}{\se}{\sA}} then \im{\pcctyjudg{\cctrans{\slenv}}{\cctrans{\se}}{\cctrans{\sA}}}.
\end{corollary}

\subsection{Compiler Correctness}
The proof of correctness with respect to separate compilation follows exactly
the pattern in \fullref[]{chp:type-pres}.
I define linking by substitution, as usual, and define observations and whole
program as evaluation to booleans.
\begin{theorem}[Separate Compilation Correctness]
  \label{thm:cc:comp-correct}
  If \im{\wf{\slenv}{\se}},
  \im{\wf{\slenv}{\ssubst}},
  \im{\wf{\cctrans{\slenv}}{\pccsubst}},
  and \im{\cctrans{\ssubst} \equiv \pccsubst}
  then
  \im{\seval{\ssubst(\se)} \approx \teval{\pccsubst(\cctrans{\se})}}
\end{theorem}
\begin{proof}
  Since \im{\equiv} on \tech{observations} implies \im{\approx}, it suffices to show
  \im{\seval{\ssubst(\se)} \equiv \teval{\pccsubst(\cctrans{\se})}}.
  Since the translation commutes with substitution, preserves equivalence, reduction implies
  equivalence, and equivalence is transitive, the following diagram commutes.

  \vtop{%
    \begin{tikzcd}
    \cctrans{(\ssubst(\se))} \arrow[r, "\equiv"] \arrow[d, "\equiv"]
    & \pccsubst(\cctrans{\se}) \arrow[d, "\equiv"] \\
      \cctrans{\sv} \arrow[r, "\equiv"]
      & \pccvpr
  \end{tikzcd}}\qedhere
\end{proof}

\begin{corollary}[Whole-Program Correctness]
  If \im{\wf{\cdot}{\se}}
  then \im{\seval{\se} \approx \teval{\cctrans{\se}}}
\end{corollary}
