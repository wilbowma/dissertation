\section{Main Ideas}
\label{sec:param-cc:ideas}
As review, recall from \fullref[]{chp:abs-cc} the problem with
preserving \tech{dependent types} through \tech{closure conversion} using the
\tech{existential types}.
I reproduce the example from \fullref[]{chp:abs-cc} below.
%
\begin{displaymath}
  \begin{array}{rcl}
  \sfune{\sA}{\sstarty}{\sfune{\sx}{\sA}{\sx}} & : & \spity{\sA}{\sstarty}{\spity{\sx}{\sA}{\sA}}
  \end{array}
\end{displaymath}
%
Here we have the polymorphic identity function, where the first
argument \im{\sA}, a \tech{type}, is free in the second function and is used in
the type of the function.
The standard \tech{parametric closure conversion} translation of this example
is below, reproduced from \fullref[]{chp:abs-cc}.
%
\begin{displaymath}
  \begin{stackTL}
    \tcloe{\tnfune{(\tnone:\tunitty,\tA:\tstarty)}{
        \tcloe{\tnfune{(\tntwo:
            \tpairty{\tstarty}{\tunitty},\tx:\tfste{\tntwo})}{\tx}
        }{\tnpaire{\tA,\tnpaire{}}}}}{\tnpaire{}} ~~ : ~~\\
    ~~\begin{array}{@{\hspace{0pt}}l@{\hspace{0pt}}l@{\hspace{0pt}}l}
      \texistty{\talphaone}{\tstarty}{&\,\tpairty{(\tnpity{&(\tnone:\talphaone,\tA:\tstarty)}{
            \\ &&
            \texistty{\talphatwo}{\tboxty}{\tpairty{(\tnpity{(\tntwo:\talphatwo,\tx:\tfste{\tntwo})}{\tfste{\tntwo}})}{\talphatwo}}})\,}{\talphaone}}
    \end{array}
    \end{stackTL}
\end{displaymath}
%
Unfortunately, while the \tech{code} is well-typed, the \tech{closure} is not.
Recall that the code of the inner \tech{closure} has
type \im{\tnpity{(\tntwo:\talphatwo,\tx:\tfste{\tntwo})}{\tfste{\tntwo}}}.
In the type of the \tech{closure}, this projection \im{\tfste{\tntwo}} is not
well-typed since the type of the \tech{environment} is
hidden---\im{\tfste{\tntwo}} is invalid when \im{\tntwo} has
type \im{\pccalphatwo}.

A similar problem presents itself in the \tech{type-preserving} \tech{closure
conversion} of System F by~\citet{minamide1996}.
In that work, \tech{type} variables in System F are included in the
\tech{environment}, introducing the same problem as above of \tech{closures} not
being well-typed.
In subsequent work, \citet{morrisett1998:ftotal} observed that in System F
\tech{type} variables can be considered \tech{computationally irrelevant} (and will,
therefore, be erased before runtime).
This justified a simple translation where \tech{closure-converted} functions
were allowed to have free \tech{type} (though not \tech{term}) variables.
In \tech{dependently typed} languages, this solution does not apply since
\tech{term} variables can also appear in \tech{types}.

As before in \fullref[]{chp:abs-cc}, and in the work of \citet{minamide1996}, we
have a dilemma.
We must close the \tech{type} of \tech{code} since \tech{code} must be closed
after \tech{closure conversion}, but we must also leave the \tech{type} of
the \tech{closure} open since we cannot project from the
hidden \tech{environment}.
The challenge is to unify the closed type of the \tech{code} and the open type of
the \tech{closure}.

In this chapter, I adapt the \tech{parametric closure conversion} of
\citet{minamide1996} to \tech{dependent types}.
Previously, in \fullref[]{chp:abs-cc}, I essentially axiomatize the target language
with \tech{closures}, adapting the \tech{abstract closure conversion}
of \citet{minamide1996} to \tech{dependent types}.
The primitive \tech{closure} captures exactly the above needs.
I use \tech{existential types} to encode closures, and use a form
of \tech{singleton types}, called \tech{translucent types}, to encode an
equality between the hidden \tech{environment} and the free variables.
The \tech{translucent type} unifies the hidden \tech{environment} with the type
variables, thus unifying the closed type of the \tech{closure} and the open type
of the \tech{code}.

Below is the nearly complete translation to demonstrate how this unifies the two
seemingly different types.%
\marginpar{Notice that this type translation relies on \tech{impredicativity},
  since the \tech{existential type} must be in the same \tech{universe} as the
  type variable \im{\alpha} over which it quantifies.}%
\begin{displaymath}
  \begin{stackTL}
    \cctrans{(\spity{\sx}{\sA}{\sB})} =
    \pccexistty{\pccalpha}{\pccstarty}{(\pccpairty{\pccalpha}{\pcccodety{\pccn:\pccalpha,\pccx:\sA^\plus}{\cctrans{\sB}}})}\\
    \cctrans{(\sfune{\sx}{\sA}{\se})} = \begin{stackTL}\pccpackoe{\pccApr}{\pccpaire{\pccnpaire{\pccxi\dots}}{\begin{stackTL}\pccnfune{(\pccn:\pccApr, \pccx:\pcclete{\pccnpaire{\pccxi\dots}}{\pccn}{\cctrans{\sA}})}{
              \\\quad\pcclete{\pccnpaire{\pccxi\dots}}{\pccn}{\cctrans{\se}}}}}
      \end{stackTL}\\
    \where{\begin{stackTL}
        \pccApr = \pccnsigmaty{\pccxi:\cctrans{\sAi}\dots} \\
        \sxi:\sAi\dots \text{ are the free variables of \im{\se} and \im{\sA}}}
    \end{stackTL}
  \end{stackTL}
  \end{stackTL}
\end{displaymath}
%
This translation uses the \tech{code} type as in \fullref[]{chp:abs-cc} to
distinguish closed \tech{code} from \tech{closures}, but lacks a \(\Pi\)-type
since \tech{closures} are encoded using existential types.
Using this translation, we need to complete the following derivation in order to
prove \tech{type preservation}.
As we will see, we cannot complete the derivation for the
standard \tech{existential type translation}.
%
\begin{displaymath}
        \inferrule
            {
              \inferrule
              {\text{stuck}}
              {\pcctyjudg{\pccn:\pccApr,\pccx:\pccA}{\pcclete{\pccnpaire{\pccxi\dots}}{\pccn}{\pcce}}{\cctrans{\sB}}}}
            {\pcctyjudg{\pcclenv}{\pccnfune{(\pccn:\pccApr,\pccx:\pccA)}{{\pcclete{\pccnpaire{\pccxi\dots}}{\pccn}{\pcce}}}}{\pcccodety{\pccn:\pccApr,\pccx:\cctrans{\sA}}{\cctrans{\sB}}}}
\end{displaymath}
%
I focus on showing the \tech{code} is well-typed at the translated type, since
that is where the problem arises.
Notice that \im{\cctrans{\sB}} has free variables in the \tech{type}.
To complete this derivation, we must show
\im{{\pcclete{\pccnpaire{\pccxi\dots}}{\pccn}{\pcce}}:\cctrans{\sB}}.
However, by the typing rule for \tech{dependent let}, we can only conclude
\im{\pcclete{\pccnpaire{\pccxi\dots}}{\pccn}{\pcce} : \subst{\cctrans{\sB}}{\pccprje{i}{\pccn}}{\pccxi}}.
We cannot complete the derivation and this translation is not \tech{type
  preserving} because \tech{dependent types} allow \tech{types} to depend on
free \tech{term} variables from the \tech{environment}.

However, intuitively, it should be the case that \im{\cctrans{\sB} \equiv
  \subst{\cctrans{\sB}}{\pccprje{i}{\pccn}}{\pccxi}}.
By \tech{parametricity}, we know \im{\pccn} will only get replaced by
exactly the \tech{environment} generated by our compiler.
\marginpar{This is why we rely on \tech{parametricity}, and where the \tech{type
    preservation} argument fails if we cannot have \tech{parametricity} in the
  target language.}
Since the compiler generates the \tech{environment} \im{\pccn =
  \pccnpaire{\pccxi\dots}}, this substitution is essentially a no-op, replacing
\im{\pccxi} by \im{\pccxi}.
To develop a \tech{type-preserving} translation, we need to find a way to
internalize this reasoning so that the type system can decide this equivalence.

To solve this, I use a form of singleton types called \tech{translucent
types}~\cite{harper1994,lillibridge1997} to encode this \tech{equivalence} in a
\tech{type}.
The \deftech*{Translucent types,translucent types,translucent type,translucent
  function,translucent function type,translucent function types}{translucent
  function type}\footnote{This particular presentation of translucent function
  types is due to \citet{minamide1996}.},
written \im{\pcctrlufunty{\pccepr}{\pccB}}, represents a function (or in our
case, \tech{code}) that must be applied to (an expression \tech{equivalent} to)
the term \im{\pccepr} and produces something of type \im{\pccB}.
The \tech{translucent function} type rules are the following.
Essentially, any function \im{\pccf} can be statically specialized to a
particular argument \im{\pccepr}.
This has the effect of instantiating the \tech{dependent function} type before
the function is actually applied.
After that, it must be dynamically applied to that argument.
\begin{mathpar}
  \inferrule*[right=\rulename{TrFun}]
  {\pcctyjudg{\pcclenv}{\pccf}{\pccpity{\pccn}{\pccApr}{\pccB}}\\
   \pcctyjudg{\pcclenv}{\pccepr}{\pccApr}}
  {\pcctyjudg{\pcclenv}{\pccf}{\pcctrlufunty{\pccepr}{\subst{\pccB}{\pccepr}{\pccn}}}}

  \inferrule*[right=\rulename{TrApp}]
  {\pcctyjudg{\pcclenv}{\pccf}{\pcctrlufunty{\pccepr}{\pccB}}}
  {\pcctyjudg{\pcclenv}{\pccappe{\pccf}{\pccepr}}{\pccB}}
\end{mathpar}

To encode \tech{closures}, we existentially quantify over the \emph{value} of
the \tech{environment} \im{\pccn}, in addition to the \tech{type} of the
\tech{environment} \im{\pccalpha}, leveraging \tech{dependent types}.
Then we describe type of the \tech{code} \im{\pccf} using a \tech{translucent type}
\im{(\pcctrlufunty{\pccn}{\pcccodety{\pccx:\cctrans{\sA}}{\cctrans{sB}}})}, which
requires that the \tech{code} be applied to exactly the \tech{environment}
\im{\pccn} and an arbitrary argument \im{\pccx} of type \im{\cctrans{\sA}}, and
produces an output of type \im{\cctrans{\sB}}.
The translation becomes the following.
\begin{displaymath}
  \begin{stackTL}
  \cctrans{(\spity{\sx}{\sA}{\sB})} =
\pccnexistty{(\pccalpha:\pccstarty,\pccn:\pccalpha,\pccf:(\pcctrlufunty{\pccn}{\pcccodety{\pccx:\cctrans{\sA}}{\cctrans{\sB}}}))} \\
\cctrans{(\sfune{\sx}{\sA}{\se})} =
\begin{stackTL}\pccnpackoe{\pccApr,\pccnpaire{\pccxi\dots},\begin{stackTL}\pccnfune{(\pccn:\pccApr,\pccx:\pcclete{\pccnpaire{\pccxi\dots}}{\pccn}{\cctrans{\sA}})}{\\\quad\pcclete{\pccnpaire{\pccxi\dots}}{\pccn}{\cctrans{\se}}}}
\end{stackTL}\\
\where{
    \begin{stackTL}
      \pccApr = \pccnsigmaty{(\pccxi:\cctrans{\sAi}\dots)} \\
      \sxi:\sAi \dots \text{ are the free variables of \im{\se} and \im{\sA}}
    \end{stackTL}}
  \end{stackTL}
  \end{stackTL}
\end{displaymath}
Now, the \tech{environment} is still hidden from the client of the
\tech{closure}, but when checking that the \tech{closure} is well-typed, the
variable \im{\pccn} is replaced by the value of the \tech{closure}.
Then all we need is to prove \im{\cctrans{\sB} \equiv
  \subst{\cctrans{\sB}}{\pccprje{i}{\pccnpaire{\pccxi\dots}}}{\pccxi}}, which is trivial.
This was essentially the insight in \citet{minamide1996}, although in a simpler setting.

As we descend into the \pccfonttext{pack} expression, the existentially
quantified environment flows into the \tech{type} and the \tech{translucent
type} yields an additional \tech{equivalence}.
The completed derivation is the following.
{\small
\begin{mathpar}
  \inferrule
      {
        \inferrule
            {
              \inferrule
              {\vdots}
              {\pcctyjudg{\pcclenv}{\pccnfune{(\pccn:\pccApr,\pccx:\cctrans{\sA})}{\pccepr}}{\pcccodety{\pccn:\pccApr,\pccx:\cctrans{\sA}}{\subst{\cctrans{\sB}}{\pccprje{i}{\pccn}}{\pccxi}}}}
              \quad
              \inferrule
                  {
                    \text{trivial, since } env = \pccnpaire{\pccxi\dots}}
              {\subst{\subst{\cctrans{\sB}}{\pccprje{i}{\pccn}}{\pccxi}}{env}{\pccn} \equiv \cctrans{\sB}}}
            {\vdots \qquad\qquad\qquad \pcctyjudg{\pcclenv}{\pccnfune{(\pccn:\pccApr,\pccx:\cctrans{\sA})}{\pcclete{\pccnpaire{\pccxi\dots}}{\pccn}{\cctrans{\se}}}}{\pcctrlufunty{env}{\pcccodety{\pccx:\cctrans{\sA}}{\cctrans{\sB}}}}}
      }
      {\pcctyjudg{\pcclenv}{\pccnpackoe{\pccApr,env,\pccnfune{(\pccn:\pccApr,\pccx:\cctrans{\sA})}{\pccepr}}}{\pccnexistty{\pccalpha:\pccstarty,\pccn:\pccalpha,\pccf:(\pcctrlufunty{\pccn}{\pcccodety{\pccx:\cctrans{\sA}}})}{\cctrans{\sB}}}}
\end{mathpar}
}

While the translation is similar in essence to \citet{minamide1996}, \tech{dependent types}
introduce additional difficulties.
The above translation disrupts the syntactic reasoning about functions
that the type system relies on, in particular, \(\eta\)-equivalence of functions.

To preserve \(\eta\)-equivalence, as in \fullref[]{chp:abs-cc}, I develop a
principle of \tech{equivalence} for \tech{closures}.
Essentially, \tech{closures} are \tech{equivalent} when we compute \tech{equivalent}
results under \tech{equivalent} \tech{environments}.
This gives us an \(\eta\)-principle for \tech{closures} that extends
the \(\eta\)-principle for functions to include the \tech{environment}.
This principle also relies on \tech{parametricity} to justify.
