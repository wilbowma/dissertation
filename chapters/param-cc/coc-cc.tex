\newcommand{\FigCoCCCSyntax}[1][t]{
  \begin{figure}[#1]
    \begin{bnfgrammar}
      \bnflabel{Universes} &
      \pccU & \bnfdef & \pccstarty \bnfalt \pccboxty
      \bnfnewline

      \bnflabel{Expressions} &
      \pcce,\pccA,\pccB & \bnfdef & \pccx \bnfalt \pccstarty
                                    \bnfalt \pcccodety{(\pccn:\pccApr,\pccx:\pccA)}{\pccB}
                                    \\ && \bnfalt & \pcctrlufunty{\pcce}{\pcccodety{\pccn:\pccApr}{\pccB}}
                                    \bnfalt \pccnfune{(\pccn:\pccApr,\pccx:\pccA)}{\pcce}
                                    \bnfalt \pccappe{\pccein{f}}{\pcceone~\pccetwo}
                                    \\ && \bnfalt & \pccsigmaty{\pccx}{\pccA}{\pccB}
                                    \bnfalt \pccdpaire{\pcce}{\pcce}{\pccsigmaty{\pccx}{\pccA}{\pccB}}
                                    \bnfalt \pccfste{\pcce}
                                    \bnfalt \pccsnde{\pcce}
                                    \\&& \bnfalt & \pccalete{\pccx}{\pcce}{\pccA}{\pcce}
                                    \bnfalt \pccnexistty{(\pccalpha:\pccA,\pccn:\pccApr)}{\pccB}
                                    \\&& \bnfalt & \pccnpacke{\pccA,\pccepr,\pcce}{\pccnexistty{(\pccalpha:\pccA,\pccn:\pccApr)}{\pccB}}
                                    \\&& \bnfalt & \pccnunpacke{\pccx,\pccx,\pccx}{\pcce}{\pcce} \bnfalt \cdots
      \bnfnewline

      \bnflabel{Environments} &
      \pcclenv & \bnfdef & \cdot \bnfalt \pcclenv,\pccx:\pccA \bnfalt \pcclenv,\pccx = \pcce:\pccA
    \end{bnfgrammar}
    \caption{\pcctlang Syntax (excerpts)}
    \label{fig:coc-cc:syntax}
  \end{figure}
}

\newcommand{\FigCoCCCSyntaxFull}[1][t]{
  \begin{figure}[#1]
    \begin{bnfgrammar}
      \bnflabel{Universes} &
      \pccU & \bnfdef & \pccstarty \bnfalt \pccboxty
      \bnfnewline

      \bnflabel{Expressions} &
      \pcce,\pccA,\pccB & \bnfdef & \pccx \bnfalt \pccstarty
                                    \bnfalt \pcccodety{(\pccn:\pccApr,\pccx:\pccA)}{\pccB}
                                    \\ && \bnfalt & \pcctrlufunty{\pcce}{\pcccodety{\pccn:\pccApr}{\pccB}}
                                    \bnfalt \pccnfune{(\pccn:\pccApr,\pccx:\pccA)}{\pcce}
                                    \bnfalt \pccappe{\pccein{f}}{\pcceone~\pccetwo}
                                    \\&& \bnfalt& \pccsigmaty{\pccx}{\pccA}{\pccB} \bnfalt
                                    \pccdpaire{\pcce}{\pcce}{\pccsigmaty{\pccx}{\pccA}{\pccB}}
                                    \bnfalt \pccfste{\pcce}
                                    \bnfalt \pccsnde{\pcce}
                                    \\ && \bnfalt & \pccboolty
                                    \bnfalt \pcctruee
                                    \bnfalt \pccfalsee
                                    \bnfalt \pccife{\pcce}{\pcceone}{\pccetwo}
                                    \\ && \bnfalt & \pccalete{\pccx}{\pcce}{\pccA}{\pcce}
                                    \bnfalt \pccnexistty{(\pccalpha:\pccA,\pccn:\pccApr)}{\pccB}
                                    \\&& \bnfalt & \pccnpacke{\pccA,\pccepr,\pcce}{\pccnexistty{(\pccalpha:\pccA,\pccn:\pccApr)}{\pccB}}
                                    \\&& \bnfalt & \pccnunpacke{\pccx,\pccx,\pccx}{\pcce}{\pcce}
      \bnfnewline

      \bnflabel{Environments} &
      \pcclenv & \bnfdef & \cdot \bnfalt \pcclenv,\pccx:\pccA \bnfalt \pcclenv,\pccx = \pcce:\pccA
    \end{bnfgrammar}
    \caption{\pcctlang Syntax}
    \label{fig:coc-cc:syntax-full}
  \end{figure}
}

\newcommand{\FigCoCCCSugar}[1][t]{
  \begin{figure}[#1]
    \begin{displaymath}
      \begin{array}{rcl}
        \pcce:\pcct & \defeq & \pcctyjudg{\pcclenv}{\pcce}{\pccA}~\where{\pcclenv\text{
            is obvious from context}}
        \\
        \pccnpackoe{\pccei\dots} & \defeq &
        \pccnpacke{\pccei\dots}{\pccnexistty{\pccxi:\pccti\dots}}~\where{\pccei:\subst{\pccti}{\pccein{i-1}}{\pccxin{i-1}}}
        \\
        \pccnpaire{\pccei\dots} & \defeq & \pccdnpaire{\pccei\dots}{\pccnsigmaty{(\pccxi:\pccti\dots)}}~\where{\pccei:\subst{\pccti}{\pccein{i-1}}{\pccxin{i-1}}}
        \\
        \pcclete{\pccx}{\pcce}{\pccepr} & \defeq & \pccalete{\pccx}{\pcce}{\pcct}{\pccepr}~\where{\pcce:\pcct}
        \\
        \pcclete{\pccnpaire{\pccxi\dots}}{\pcce}{\pccepr} & \defeq & \pcclete{\pccxin{0}}{\pccprje{0}{\pcce}}{\dots\pcclete{\pccxin{\len{i}}}{\pccprje{\len{i}}}{\pccepr}}
        \\
        \subst{\pcce}{\pccnpaire{\pccxi\dots}}{\pcce} & \defeq & \subst{\pcce}{\pccxin{i}}{\pccprje{i}{\pcce}}
      \end{array}
    \end{displaymath}
    \caption{\pcctlang Syntactic Sugar}
    \label{fig:coc-cc:sugar}
  \end{figure}
}

\newcommand{\FigCoCCCRedFull}[1][t]{
  \begin{figure}[#1]
    \judgshape{\pccstepjudg[\step]{\pcclenv}{\pcce}{\pccepr}}
    \begin{reductionrules}
      \tlenv \vdash\tappe{(\tnfune{\txpr:\tApr,\tx:\tA}{\teone})}{\tepr~\te}
      & \step_{\beta} & \subst{\subst{\teone}{\tepr}{\txpr}}{\te}{\tx}
      \stepnewline

      \tlenv \vdash \tfste{\tpaire{\teone}{\tetwo}} & \step_{\pi_{1}} & \teone
      \stepnewline

      \tlenv \vdash \tsnde{\tpaire{\teone}{\tetwo}} & \step_{\pi_{2}} & \tetwo
      \stepnewline

      \tlenv \vdash \tife{\ttruee}{\teone}{\tetwo} & \step_{\iota_{1}} & \teone
      \stepnewline

      \tlenv \vdash \tife{\tfalsee}{\teone}{\tetwo} & \step_{\iota_{2}} & \tetwo
      \stepnewline

      \tlenv \vdash \tx & \step_{\delta} & \te \\
      && \where{\tx = \te : \pccA \in \tlenv}
      \stepnewline

      \tlenv \vdash \talete{\tx}{\te}{\tA}{\teone} & \step_{\zeta} & \subst{\teone}{\te}{\tx}
      \stepnewline

      \pcclenv \vdash \pccnunpacke{\pccx,\pccxpr,\pccxin{b}}{\pccnpackoe{\pcce,\pccepr,\pccein{b}}}{\pccetwo} & \step_{\exists} & \subst{\pccetwo}{\pccnpaire{\pcce,\pccepr,\pccein{b}}}{\pccnpaire{\pccx,\pccxpr,\pccxin{b}}}
    \end{reductionrules}
    \caption{\pcctlang Reduction}
    \label{fig:coc-cc:red-full}
  \end{figure}
}

\newcommand{\FigCoCCCRed}[1][t]{
  \begin{figure}[#1]
    \judgshape{\pccstepjudg[\step]{\pcclenv}{\pcce}{\pccepr}}
    \begin{reductionrules}
      & \vdots & \\
      \pccnunpacke{\pccx,\pccxpr,\pccxin{b}}{\pccnpackoe{\pcce,\pccepr,\pccein{b}}}{\pccetwo} & \step_{\exists} & \subst{\pccetwo}{\pccnpaire{\pcce,\pccepr,\pccein{b}}}{\pccnpaire{\pccx,\pccxpr,\pccxin{b}}}
    \end{reductionrules}
    \caption{\pcctlang Reduction (excerpts)}
    \label{fig:coc-cc:red}
  \end{figure}
}

\newcommand{\FigCoCCCConv}[1][t]{
  \begin{figure}[#1]
    \judgshape{\pccstepjudg[\stepstar]{\pcclenv}{\pcce}{\pccepr}}
    \begin{mathpar}
      \inferrule*[right=\defrule{Red-Cong-T-Code}]
      {\pccstepjudg[\stepstar]{\pcclenv}{\pccAone}{\pccAonepr} \\
       \pccstepjudg[\stepstar]{\pcclenv,\pccn:\pccAonepr}{\pccAtwo}{\pccAtwopr} \\
       \pccstepjudg[\stepstar]{\pcclenv,\pccn:\pccAonepr,\pccx:\pccAtwopr}{\pccB}{\pccBpr}}
      {\pccstepjudg[\stepstar]{\pcclenv}{\pcccodety{\pccn:\pccAone,\pccx:\pccAtwo}{\pccB}}{\pcccodety{\pccn:\pccAonepr,\pccx:\pccAtwopr}{\pccBpr}}}

      \inferrule*[right=\defrule*{Red-Cong-trlufunty}{Red-Cong-\im{\trlufunarrow}}]
      {\pccstepjudg[\stepstar]{\pcclenv}{\pcce}{\pccepr} \\
       \pccstepjudg[\stepstar]{\pcclenv}{\pccAtwo}{\pccAtwopr} \\
       \pccstepjudg[\stepstar]{\pcclenv,\pccx:\pccAtwopr}{\pccB}{\pccBpr}}
      {\pccstepjudg[\stepstar]{\pcclenv}{\pcctrlufunty{\pcce}{\pcccodety{\pccx:\pccAtwo}{\pccB}}}
        {\pcctrlufunty{\pccepr}{\pcccodety{\pccx:\pccAtwopr}{\pccBpr}}}}


      \inferrule*[right=\defrule{Red-Cong-Code}]
      {\pccstepjudg[\stepstar]{\tlenv}{\tAone}{\tAonepr} \\
       \pccstepjudg[\stepstar]{\tlenv,\tn:\tAonepr}{\tAtwo}{\tAtwopr} \\
       \pccstepjudg[\stepstar]{\tlenv,\tn:\tAonepr,\tx:\tAtwopr}{\te}{\tepr}}
      {\pccstepjudg[\stepstar]{\tlenv}{\tnfune{(\tn:\tAone,\tx:\tAtwo)}{\te}}{\tnfune{(\tn:\tAonepr,\tx:\tAtwopr)}{\tepr}}}

      \inferrule*[right=\defrule{Red-Cong-App}]
      {\pccstepjudg[\stepstar]{\pcclenv}{\pcceone}{\pcceonepr} \\
       \pccstepjudg[\stepstar]{\pcclenv}{\pccetwo}{\pccetwopr}}
      {\pccstepjudg[\stepstar]{\pcclenv}{\pccappe{\pcceone}{\pccetwo}}{\pccappe{\pcceonepr}{\pccetwopr}}}

      \inferrule*[right=\defrule{Red-Cong-Sig}]
      {\pccstepjudg[\stepstar]{\tlenv}{\tA}{\tApr} \\
       \pccstepjudg[\stepstar]{\tlenv,\tx:\tApr}{\te}{\tepr}}
      {\pccstepjudg[\stepstar]{\tlenv}{\tsigmaty{\tx}{\tA}{\te}}{\tsigmaty{\tx}{\tApr}{\tepr}}}

      \inferrule*[right=\defrule{Red-Cong-Pair}]
      {\pccstepjudg[\stepstar]{\tlenv}{\teone}{\teonepr} \\
       \pccstepjudg[\stepstar]{\tlenv}{\tetwo}{\tetwopr} \\
       \pccstepjudg[\stepstar]{\tlenv}{\tA}{\tApr}}
      {\pccstepjudg[\stepstar]{\tlenv}{\tdpaire{\teone}{\tetwo}{\tA}}{\tdpaire{\teonepr}{\tetwopr}{\tApr}}}

      \inferrule*[right=\defrule{Red-Cong-Fst}]
      {\pccstepjudg[\stepstar]{\tlenv}{\te}{\tepr}}
      {\pccstepjudg[\stepstar]{\tlenv}{\tfste{\te}}{\tfste{\tepr}}}

      \inferrule*[right=\defrule{Red-Cong-Snd}]
      {\pccstepjudg[\stepstar]{\tlenv}{\te}{\tepr}}
      {\pccstepjudg[\stepstar]{\tlenv}{\tsnde{\te}}{\tsnde{\tepr}}}

      \inferrule*[right=\defrule{Red-Cong-If}]
      {\tstepjudg[\stepstar]{\tlenv}{\te}{\tepr} \\
       \tstepjudg[\stepstar]{\tlenv}{\teone}{\teonepr} \\
       \tstepjudg[\stepstar]{\tlenv}{\tetwo}{\tetwopr}}
      {\sstepjudg[\stepstar]{\tlenv}{\tife{\te}{\teone}{\tetwo}}{\tife{\tepr}{\teonepr}{\tetwopr}}}

      \inferrule*[right=\defrule{Red-Cong-Let}]
      {\pccstepjudg[\stepstar]{\pcclenv}{\pcceone}{\pcceonepr} \\
       \pccstepjudg[\stepstar]{\pcclenv}{\pccA}{\pccApr} \\
       \pccstepjudg[\stepstar]{\pcclenv,\pccx = \pccepr:\pccApr}{\pccetwo}{\pccetwopr}}
      {\pccstepjudg[\stepstar]{\pcclenv}{\pccalete{\pccx}{\pcceone}{\pccA}{\pccetwo}}{\pccalete{\pccx}{\pcceonepr}{\pccApr}{\pccetwopr}}}

      \inferrule*[right=\defrule{Red-Cong-Exist}]
      {\pccstepjudg[\stepstar]{\pcclenv}{\pccAone}{\pccAonepr} \\
       \pccstepjudg[\stepstar]{\pcclenv,\pccx:\pccAonepr}{\pccAtwo}{\pccAtwopr} \\
       \pccstepjudg[\stepstar]{\pcclenv,\pccx:\pccAonepr,\pccxpr:\pccAtwopr}{\pccB}{\pccBpr}}
      {\pccstepjudg[\stepstar]{\pcclenv}{\pccnexistty{(\pccx:\pccA,\pccxpr:\pccApr)}{\pccB}}{\pccnexistty{(\pccx:\pccAone,\pccxpr:\pccAtwo)}{\pccBpr}}}

      \inferrule*[right=\defrule{Red-Cong-Pack}]
      {\pccstepjudg[\stepstar]{\pcclenv}{\pccA}{\pccApr} \\
       \pccstepjudg[\stepstar]{\pcclenv}{\pcceone}{\pcceonepr} \\
       \pccstepjudg[\stepstar]{\pcclenv}{\pccetwo}{\pccetwopr} \\
       \pccstepjudg[\stepstar]{\pcclenv}{\pccethree}{\pccethreepr}}
      {\pccstepjudg[\stepstar]{\pcclenv}{\pccnpacke{\pcceone,\pccetwo,\pccein{3}}{\pccA}}{\pccnpacke{\pcceonepr,\pccetwopr,\pccethreepr}{\pccApr}}}

      \inferrule*[right=\defrule{Red-Cong-Unpack}]
      {\pccstepjudg[\stepstar]{\pcclenv}{\pcceone}{\pcceonepr} \\
       \pccstepjudg[\stepstar]{\pcclenv,\pccx:\pccA,\pccxpr:\pccApr,\pccxin{b}:\pccB}{\pccetwo}{\pccetwo}}
      {\pccstepjudg[\stepstar]{\pcclenv}{\pccnunpacke{\pccx,\pccxpr,\pccxin{b}}{\pcceone}{\pccetwo}}
        {\pccnunpacke{\pccx,\pccxpr,\pccxin{b}}{\pcceonepr}{\pccetwopr}}}
    \end{mathpar}
    \caption{\pcctlang Conversion}
    \label{fig:coc-cc:conv}
  \end{figure}
}

\newcommand{\FigCoCCCEquivFull}[1][t]{
  \begin{figure}[#1]
    \judgshape{\pccequivjudg{\pcclenv}{\pcce}{\pccepr}}
    \begin{mathpar}
      \inferrule*[right=\defrule*{eqv}{\im{\equiv}}]
      {\tstepjudg[\stepstar]{\tlenv}{\teone}{\te} \\
       \tstepjudg[\stepstar]{\tlenv}{\tetwo}{\te}}
      {\tequivjudg{\tlenv}{\teone}{\tetwo}}

      \inferrule*[right=\defrule*{eqv-eta1}{\im{\equiv}-\im{\eta_1}}]
      {\pccstepjudg[\stepstar]{\pcclenv}{\pcceone}{\pccnpackoe{\pccApr,\pccepr,
            \pccnfune{(\pccn:\pccApr,\pccx:\pccA)}{\pcceonepr}}} \\
       \pccstepjudg[\stepstar]{\pcclenv}{\pccetwo}{\pccetwopr} \\
       \pccequivjudg{\pcclenv,\pccx:\subst{\pccA}{\pccepr}{\pccn}}{\subst{\pcceonepr}{\pccepr}{\pccn}}{\pccnunpacke{\pccalpha,\pccn,\pccf}{\pccetwopr}{\pccappe{\pccf}{\pccn~\pccx}}}}
      {\pccequivjudg{\pcclenv}{\pcceone}{\pccetwo}}

      \inferrule*[right=\defrule*{eqv-eta2}{\im{\equiv}-\im{\eta_2}}]
      {\pccstepjudg[\stepstar]{\pcclenv}{\pcceone}{\pcceonepr} \\
       \pccstepjudg[\stepstar]{\pcclenv}{\pccetwo}{\pccnpackoe{\pccApr,\pccepr,
           \pccnfune{(\pccn:\pccApr,\pccx:\pccA)}{\pccetwopr}}} \\
       \pccequivjudg{\pcclenv,\pccx:\subst{\pccA}{\pccepr}{\pccn}}{\pccnunpacke{\pccalpha,\pccn,\pccf}{\pcceonepr}{\pccappe{\pccf}{\pccn~\pccx}}}{\subst{\pccetwopr}{\pccepr}{\pccn}}}
      {\pccequivjudg{\pcclenv}{\pcceone}{\pccetwo}}
    \end{mathpar}
    \caption{\pcctlang Equivalence}
  \end{figure}
}

\newcommand{\FigCoCCCEquiv}[1][t]{
  \begin{figure}[#1]
    \judgshape{\pccequivjudg{\pcclenv}{\pcce}{\pccepr}}
    \begin{mathpar}
      \cdots

      \inferrule*[right=\defrule*{eqv-eta1}{\im{\equiv}-\im{\eta_1}}]
      {\pccstepjudg[\stepstar]{\pcclenv}{\pcceone}{\pccnpackoe{\pccApr,\pccepr,
            \pccnfune{(\pccn:\pccApr,\pccx:\pccA)}{\pcceonepr}}} \\
       \pccstepjudg[\stepstar]{\pcclenv}{\pccetwo}{\pccetwopr} \\
       \pccequivjudg{\pcclenv,\pccx:\subst{\pccA}{\pccepr}{\pccn}}{\subst{\pcceonepr}{\pccepr}{\pccn}}{\pccnunpacke{\pccalpha,\pccn,\pccf}{\pccetwopr}{\pccappe{\pccf}{\pccn~\pccx}}}}
      {\pccequivjudg{\pcclenv}{\pcceone}{\pccetwo}}

      \inferrule*[right=\defrule*{eqv-eta2}{\im{\equiv}-\im{\eta_2}}]
      {\pccstepjudg[\stepstar]{\pcclenv}{\pcceone}{\pcceonepr} \\
       \pccstepjudg[\stepstar]{\pcclenv}{\pccetwo}{\pccnpackoe{\pccApr,\pccepr,
           \pccnfune{(\pccn:\pccApr,\pccx:\pccA)}{\pccetwopr}}} \\
       \pccequivjudg{\pcclenv,\pccx:\subst{\pccA}{\pccepr}{\pccn}}{\pccnunpacke{\pccalpha,\pccn,\pccf}{\pcceonepr}{\pccappe{\pccf}{\pccn~\pccx}}}{\subst{\pccetwopr}{\pccepr}{\pccn}}}
      {\pccequivjudg{\pcclenv}{\pcceone}{\pccetwo}}
    \end{mathpar}
    \caption{\pcctlang Equivalence (excerpts)}
    \label{fig:coc-cc:equiv}
  \end{figure}
}

\newcommand{\FigCoCCCTypingOne}[1][t]{
  \begin{figure}[#1]
    \judgshape{\pcctyjudg{\pcclenv}{\pcce}{\pccA}}
    \begin{mathpar}
      \inferrule*[right=\defrule{Var}]
      {(\cpsx : \cpsA \in \cpslenv \text{ or }
        \cpsx = \cpse : \cpsA \in \cpslenv) \\
        \cpswf{\cpslenv}}
      {\cpstyjudg{\cpslenv}{\cpsx}{\cpsA}}

      \inferrule*[right=\defrule{*}]
      {\cpswf{\cpslenv}}
      {\cpstyjudg{\cpslenv}{\cpsstarty}{\cpsboxty}}

      \inferrule*[right=\defrule{Sig}]
      {\cpstyjudg{\cpslenv}{\cpsA}{\cpsstarty} \\
       \cpstyjudg{\cpslenv,\cpsx:\cpsA}{\cpsB}{\cpsstarty}}
      {\cpstyjudg{\cpslenv}{\cpssigmaty{\cpsx}{\cpsA}{\cpsB}}{\cpsstarty}}

      \inferrule*[right=\defrule{Pair}]
      {\cpstyjudg{\cpslenv}{\cpseone}{\cpsA} \\
       \cpstyjudg{\cpslenv}{\cpsetwo}{\subst{\cpsB}{\cpseone}{\cpsx}}}
      {\cpstyjudg{\cpslenv}{\cpsdpaire{\cpseone}{\cpsetwo}{\cpssigmaty{\cpsx}{\cpsA}{\cpsB}}}{\cpssigmaty{\cpsx}{\cpsA}{\cpsB}}}

      \inferrule*[right=\defrule{Fst}]
      {\cpstyjudg{\cpslenv}{\cpse}{\cpssigmaty{\cpsx}{\cpsA}{\cpsB}}}
      {\cpstyjudg{\cpslenv}{\cpsfste{\cpse}}{\cpsA}}

      \inferrule*[right=\defrule{Snd}]
      {\cpstyjudg{\cpslenv}{\cpse}{\cpssigmaty{\cpsx}{\cpsA}{\cpsB}}}
      {\cpstyjudg{\cpslenv}{\cpssnde{\cpse}}{\subst{\cpsB}{\cpsfste{\cpse}}{\cpsx}}}

      \inferrule*[right=\defrule{Bool}]
      {\cpswf{\cpslenv}}
      {\cpstyjudg{\cpslenv}{\cpsboolty}{\cpsstarty}}

      \inferrule*[right=\defrule{True}]
      {\cpswf{\cpslenv}}
      {\cpstyjudg{\cpslenv}{\cpstruee}{\cpsboolty}}

      \inferrule*[right=\defrule{False}]
      {\cpswf{\cpslenv}}
      {\cpstyjudg{\cpslenv}{\cpsfalsee}{\cpsboolty}}

      \inferrule*[right=\defrule{If}]
      {\cpstyjudg{\cpslenv}{\cpse}{\cpsboolty} \\
       \cpstyjudg{\cpslenv}{\cpseone}{\cpsB} \\
       \cpstyjudg{\cpslenv}{\cpsetwo}{\cpsB}}
      {\cpstyjudg{\cpslenv}{\cpsife{\cpse}{\cpseone}{\cpsetwo}}{\cpsB}}

      \inferrule*[right=\defrule{Let}]
      {\cpstyjudg{\cpslenv}{\cpse}{\cpsA} \\
       \cpstyjudg{\cpslenv,\cpsx=\cpse:\cpsA}{\cpsepr}{\cpsB}}
      {\cpstyjudg{\cpslenv}{\cpsalete{\cpsx}{\cpse}{\cpsA}{\cpsepr}}{\subst{\cpsB}{\cpse}{\cpsx}}}

      \inferrule*[right=\defrule{Conv}]
      {\cpstyjudg{\cpslenv}{\cpse}{\cpsA} \\
       \cpstyjudg{\cpslenv}{\cpsB}{\cpsU} \\
       \cpsequivjudg{\cpslenv}{\cpsA}{\cpsB}}
      {\cpstyjudg{\cpslenv}{\cpse}{\cpsB}}
    \end{mathpar}
    \caption{\pcctlang Typing (1/2)}
  \end{figure}
}

\newcommand{\FigCoCCCTypingTwo}[1][t]{
  \begin{figure}[#1]
    \judgshape{\pcctyjudg{\pcclenv}{\pcce}{\pccA}}
    \begin{mathpar}
      \inferrule*[right=\defrule{Code-*}]
      {\pcctyjudg{\pcclenv}{\pccB}{\pccstarty}}
      {\pcctyjudg{\pcclenv}{\pcccodety{(\pccxpr:\pccApr,\pccx:\pccA)}{\pccB}}{\pccstarty}}

      \inferrule*[right=\defrule*{trfunty}{\im{\trlufunarrow}}]
      {\pcctyjudg{\pcclenv}{\pcce}{\pccApr} \\
       \pcctyjudg{\pcclenv}{\pcccodety{(\pccxpr:\pccApr,\pccx:\pccA)}{\pccB}}{\pccU}}
      {\pcctyjudg{\pcclenv}{\pcctrlufunty{\pcce}{\pcccodety{\pccx:\pccA}{\pccB}}}{\pccU}}

      \inferrule*[right=\defrule{TrFun}]
      {\pcctyjudg{\pcclenv}{\pcce}{\pcccodety{(\pccxpr:\pccApr,\pccx:\pccA)}{\pccB}} \\
       \pcctyjudg{\pcclenv}{\pccepr}{\pccApr}}
      {\pcctyjudg{\pcclenv}{\pcce}{\pcctrlufunty{\pccepr}{\pcccodety{(\pccx:\subst{\pccA}{\pccepr}{\pccxpr})}{\subst{\pccB}{\pccepr}{\pccxpr}}}}}

      \inferrule*[right=\defrule{Code}]
      {\pcctyjudg{\cdot,\pccxpr:\pccApr,\pccx:\pccA}{\pcce}{\pccB}}
      {\pcctyjudg{\pcclenv}{\pccnfune{(\pccxpr:\pccApr,\pccx:\pccA)}{\pcce}}{\pcccodety{\pccxpr:\pccApr,\pccx:\pccA}{\pccB}}}

      \inferrule*[right=\defrule{TrApp}]
      {\pcctyjudg{\pcclenv}{\pccf}{\pcctrlufunty{\pccepr}{\pcccodety{\pccx:\pccA}{\pccB}}} \\
       \pcctyjudg{\pcclenv}{\pcce}{\pccA}}
      {\pcctyjudg{\pcclenv}{\pccappe{\pccf}{\pccepr\:\pcce}}{\subst{\pccB}{\pcce}{\pccx}}}

      \inferrule*[right=\defrule{Exist}]
      {\pcctyjudg{\pcclenv}{\pccA}{\pccU} \\
       \pcctyjudg{\pcclenv,\pccx:\pccA}{\pccApr}{\pccUpr} \\
       \pcctyjudg{\pcclenv,\pccx:\pccA,\pccxpr:\pccApr}{\pccB}{\pccstarty}}
      {\pcctyjudg{\pcclenv}{\pccnexistty{(\pccx:\pccA,\pccxpr:\pccApr)}{\pccB}}{\pccstarty}}

      \inferrule*[right=\defrule{Pack}]
      {\pcctyjudg{\pcclenv}{\pccnexistty{(\pccx:\pccA,\pccxpr:\pccApr)}{\pccB}}{\pccU} \\
       \pcctyjudg{\pcclenv}{\pcce}{\pccA} \\
       \pcctyjudg{\pcclenv}{\pccepr}{\subst{\pccApr}{\pcce}{\pccx}} \\
       \pcctyjudg{\pcclenv}{\pccein{b}}{\subst{\subst{\pccB}{\pcce}{\pccx}}{\pccepr}{\pccxpr}}}
      {\pcctyjudg{\pcclenv}{\pccnpacke{\pcce,\pccepr,\pccein{b}}{\pccnexistty{(\pccx:\pccA,\pccxpr:\pccApr)}{\pccB}}}{\pccnexistty{(\pccx:\pccA,\pccxpr:\pccApr)}{\pccB}}}

      \inferrule*[right=\defrule{Unpack}]
      {\pcctyjudg{\pcclenv}{\pcce}{\pccnexistty{(\pccx:\pccA,\pccxpr:\pccApr)}{\pccB}} \\
       \pcctyjudg{\pcclenv,\pccx:\pccA,\pccxpr:\pccApr,\pccxin{b}:\pccB}{\pccepr}{\pccBpr} \\
       \pcctyjudg{\pcclenv}{\pccBpr}{\pccU}}
      {\pcctyjudg{\pcclenv}{\pccnunpacke{\pccx,\pccxpr,\pccxin{b}}{\pcce}{\pccepr}}{\pccBpr}}
    \end{mathpar}
    \caption{\pcctlang Typing (2/2)}
  \end{figure}
}

\newcommand{\FigCoCCCTypingNew}[1][t]{
  \begin{figure}[#1]
    \judgshape{\pcctyjudg{\pcclenv}{\pcce}{\pccA}}
    \begin{mathpar}
      \cdots

      \inferrule*[right=\defrule{Code-*}]
      {\pcctyjudg{\pcclenv}{\pccB}{\pccstarty}}
      {\pcctyjudg{\pcclenv}{\pcccodety{(\pccxpr:\pccApr,\pccx:\pccA)}{\pccB}}{\pccstarty}}

      \inferrule*[right=\defrule*{trfunty}{\im{\trlufunarrow}}]
      {\pcctyjudg{\pcclenv}{\pcce}{\pccApr} \\
       \pcctyjudg{\pcclenv}{\pcccodety{(\pccxpr:\pccApr,\pccx:\pccA)}{\pccB}}{\pccU}}
      {\pcctyjudg{\pcclenv}{\pcctrlufunty{\pcce}{\pcccodety{\pccx:\pccA}{\pccB}}}{\pccU}}

      \inferrule*[right=\defrule{TrFun}]
      {\pcctyjudg{\pcclenv}{\pcce}{\pcccodety{(\pccxpr:\pccApr,\pccx:\pccA)}{\pccB}} \\
       \pcctyjudg{\pcclenv}{\pccepr}{\pccApr}}
      {\pcctyjudg{\pcclenv}{\pcce}{\pcctrlufunty{\pccepr}{\pcccodety{(\pccx:\subst{\pccA}{\pccepr}{\pccxpr})}{\subst{\pccB}{\pccepr}{\pccxpr}}}}}

      \inferrule*[right=\defrule{Code}]
      {\pcctyjudg{\cdot,\pccxpr:\pccApr,\pccx:\pccA}{\pcce}{\pccB}}
      {\pcctyjudg{\pcclenv}{\pccnfune{(\pccxpr:\pccApr,\pccx:\pccA)}{\pcce}}{\pcccodety{\pccxpr:\pccApr,\pccx:\pccA}{\pccB}}}

      \inferrule*[right=\defrule{TrApp}]
      {\pcctyjudg{\pcclenv}{\pccf}{\pcctrlufunty{\pccepr}{\pcccodety{\pccx:\pccA}{\pccB}}} \\
       \pcctyjudg{\pcclenv}{\pcce}{\pccA}}
      {\pcctyjudg{\pcclenv}{\pccappe{\pccf}{\pccepr\:\pcce}}{\subst{\pccB}{\pcce}{\pccx}}}

      \inferrule*[right=\defrule{Exist}]
      {\pcctyjudg{\pcclenv}{\pccA}{\pccU} \\
       \pcctyjudg{\pcclenv,\pccx:\pccA}{\pccApr}{\pccUpr} \\
       \pcctyjudg{\pcclenv,\pccx:\pccA,\pccxpr:\pccApr}{\pccB}{\pccstarty}}
      {\pcctyjudg{\pcclenv}{\pccnexistty{(\pccx:\pccA,\pccxpr:\pccApr)}{\pccB}}{\pccstarty}}

      \inferrule*[right=\defrule{Pack}]
      {\pcctyjudg{\pcclenv}{\pccnexistty{(\pccx:\pccA,\pccxpr:\pccApr)}{\pccB}}{\pccU} \\
       \pcctyjudg{\pcclenv}{\pcce}{\pccA} \\
       \pcctyjudg{\pcclenv}{\pccepr}{\subst{\pccApr}{\pcce}{\pccx}} \\
       \pcctyjudg{\pcclenv}{\pccein{b}}{\subst{\subst{\pccB}{\pcce}{\pccx}}{\pccepr}{\pccxpr}}}
      {\pcctyjudg{\pcclenv}{\pccnpacke{\pcce,\pccepr,\pccein{b}}{\pccnexistty{(\pccx:\pccA,\pccxpr:\pccApr)}{\pccB}}}{\pccnexistty{(\pccx:\pccA,\pccxpr:\pccApr)}{\pccB}}}

      \inferrule*[right=\defrule{Unpack}]
      {\pcctyjudg{\pcclenv}{\pcce}{\pccnexistty{(\pccx:\pccA,\pccxpr:\pccApr)}{\pccB}} \\
       \pcctyjudg{\pcclenv,\pccx:\pccA,\pccxpr:\pccApr,\pccxin{b}:\pccB}{\pccepr}{\pccBpr} \\
       \pcctyjudg{\pcclenv}{\pccBpr}{\pccU}}
      {\pcctyjudg{\pcclenv}{\pccnunpacke{\pccx,\pccxpr,\pccxin{b}}{\pcce}{\pccepr}}{\pccBpr}}
    \end{mathpar}
    \caption{\pcctlang Typing (excerpts)}
    \label{fig:coc-cc:type-new}
  \end{figure}
}

\section{Parametric Closure Conversion {{IL}}}
\label{sec:param-cc:coc-cc}

\FigCoCCCSyntax
In \fullref[]{fig:coc-cc:syntax}, I present the syntax of \pcctlang, a
\tech{dependently typed} \tech{closure converted} language based on
\pccslang.
As in \fullref[]{chp:abs-cc}, closed \tech{code} is written as
\im{\pccnfune{(\pccn:\pccApr,\pccx:\pccA)}{\pcce}} and has type
\im{\pcccodety{(\pccn:\pccApr,\pccx:\pccA)}{\pccB}}.
All \tech{code} takes two arguments, which are intuitively the
\tech{environment} \im{\pccn} and the original argument \im{\pccx}.
I add 2-ary \tech{existential types}
\im{\pccnexistty{(\pccalpha:\pccA,\pccn:\pccApr)}{\pccB}} to encode
\tech{closures}, with the introduction form
\im{\pccnpacke{\pccA,\pccepr,\pcce}{\pccnexistty{(\pccalpha:\pccA,\pccn:\pccApr)}{\pccB}}}
and the elimination form
\im{\pccnunpacke{\pccalpha,\pccn,\pccf}{\pccepr}{\pcce}}.
As with \tech{dependent pairs}, I omit the type annotation on
\im{\pccfont{pack}} for brevity, as in \im{\pccnpackoe{\pccA,\pccepr,\pcce}}.
Finally, I add a variant of \emph{translucent function types}, specialized to
\tech{code} types, written
\im{\pcctrlufunty{\pcce}{\pcccodety{\pccx:\pccA}{\pccB}}}.
Since all \tech{code} is 2-ary, the \tech{translucent function type} is
specialized to only curry \tech{code} types.
This type represents a \tech{code} that can only be applied to the term
\im{\pcce} as its first argument, and something of type \im{\pccA} as its second
argument, and results in a term of the type \im{\pccB}.
Note that this type is not \tech{dependent} in the usual sense on its first
argument---\ie, there is no name bound for the term \im{\pcce}---since
\im{\pcce} has already been substituted into the types \im{\pccA} and
\im{\pccB}.
%
\marginpar{We could simplify the type system by encoding 2-ary constructs such as
\pccfonttext{code} and \pccfonttext{pack} the same way we encode n-ary
\tech{dependent pairs}.
However, this complicates \(\eta\)-equivalence for \tech{closures}, which relies
on the canonical forms.}

\FigCoCCCTypingNew
In \fullref[]{fig:coc-cc:type-new}, I define the key typing rules for the target
language; the complete definition is given in \fullref[]{sec:param-cc:appendix}.
I explain \tech{translucent function types} in detail shortly.
I add the standard rules for \tech{existential types}.
Notice that \tech{existential types} are \tech{impredicative} (\refrule{Exist});
unlike \tech{dependent pairs} (strong dependent pairs with projection),
\tech{existential types} (weak dependent pairs with pattern matching) are
consistent with \tech{impredicativity}.
Furthermore, \tech{impredicativity} is necessary for expressing the type of
\tech{closures}.
\refrule{TrFun} is the introduction rule for the \tech{translucent type}.
Intuitively, any function can be statically specialized to a specific argument.
To ascribe a term a \tech{translucent type}, we must check that the specific
argument is of the same type as expected by the \tech{code}.
\refrule{TrApp} eliminates a \tech{translucent type}.
When applying \tech{code} ascribed a \tech{translucent type}, we check that the
first argument is \emph{equal} to the argument specified by the
\tech{translucent type}.

As an example of the \tech{translucent type}, consider the following example in
which we ascribe two different types to the polymorphic identity function.
\begin{displaymath}
  \begin{stackTL}
    \pccfont{id} : \pcccodety{(\pccA:\pccstarty,\pccx:\pccA)}{\pccA}\\
    \pccfont{id} : \pcctrlufunty{\pccboolty}{\pcccodety{\pccx:\pccboolty}{\pccboolty}}\\
  \end{stackTL}
\end{displaymath}
In the first case, we ascribe \im{\pccfont{id}} the type
\im{\pcccodety{(\pccA:\pccstarty,\pcca:\pccA)}{\pccA}}.
This is the standard type, compiled to a \tech{code} type.
In the second case, we ascribe \im{\pccfont{id}} a \tech{translucent type}.
Notice that we do not actually apply \im{\pccfont{id}} to any arguments; we
specialize only its \tech{type}.
By ascribing it a \tech{translucent type}, we know statically that when this
function is applied to the type \im{\pccboolty}, it accepts and return a
\im{\pccboolty}.
This gives us the ability to reason about the result of applying
\im{\pccfont{id}} before applying it.
As a result, after ascribing this type we \emph{must} only apply
\im{\pccfont{id}} to \im{\pccboolty}, or the additional static reasoning would
not be valid.

\FigCoCCCRed
In \fullref[]{fig:coc-cc:red} I give the new reduction rule for \tech{existential types}.
The rule is completely standard.
\tech{Translucent types} do not introduce new dynamic semantics.

\begin{digression}
  If we had recursion, and therefore recursive \tech{closures}, we would need a
  non-standard reduction rule for \pccfonttext{unpack} to ensure normalization.
  In the same way that recursive functions in Coq only reduce when applied to a
  constant, a recursive \tech{closure} should only reduce when
  \pccfonttext{unpack}ed and applied to a constant.
\end{digression}

\FigCoCCCEquiv
In \fullref[]{fig:coc-cc:equiv} I present new \(\eta\)-equivalence rules for
\tech{closures}.
These new rules are the result of \tech{closure converting} the
\(\eta\)-equivalence rules from \pccslang.

This \(\eta\)-equivalence is specialized to \tech{closures} and necessarily
differs from the standard \(\eta\)-equivalence for \tech{existential types}.
To see why, consider the normal \(\eta\)-equivalence for \tech{existential types}.
\begin{displaymath}
  C[\pcce] \equiv \pccunpacke{\pccalpha}{\pccx}{\pcce}{C[\pccpackoe{\pccalpha}{\pccx}]}\\
\end{displaymath}
This states that the expression \im{\pcce} in some arbitrary program
\deftech{context} (a program with a \emph{hole}, or single linear variable,
\im{\hole}) \im{C}
is \tech{equivalent} to \pccfonttext{unpack}ing \im{\pcce} and executing \im{C}
with the canonical form of the \tech{existential type}
\im{\pccpackoe{\pccalpha}{\pccx}}.
In a \tech{closure-converted} language, we can't use this rule, since
\im{C} may be \tech{code} \im{\pccnfune{\pccy:\pccA}{\hole}},
and we can't push this context under the scope of the free variables
\im{\pccalpha, \pccx}.
In general, after \tech{closure conversion} we cannot introduce references to
free variables in arbitrary contexts.
Since the normal \(\eta\)-equivalence for \tech{existential types} introduces
references to free variables in an arbitrary context, it is not valid in our
\tech{closure-converted} language.

\subsection{Meta-theory}
As the target language is relatively standard, I do not prove any meta-theoretic
results about \pcctlang; I sketch most of a model below.
I conjecture that \pcctlang is \tech{consistent} and \tech{type safe}, strongly
normalizing, and satisfies \tech{subject reduction}.
However, decidability is an open question; the presentation of \tech{translucent
  types} is not syntax directed and likely requires changes to ensure
decidability.

Both \tech{translucent types} and \tech{existential types} are easily modeled in
\tech{CIC}.
\tech{Existential types} are included in Coq's standard library, so I focus on
\tech{translucent types}.
I give an implementation in \fullref[]{sec:param-cc:appendix}, but give an
intuitive presentation here.

The \tech{translucent type} is easily modeled with the \tech{identity type} as
follows.
\begin{displaymath}
  \begin{array}{rcl}
  \sembrace{\pcctrlufunty{\pcce}{\pccB}}^\circ &\defeq& \spity{\sx}{\sembrace{\pccA}^\circ}{
    \pccfunty{{\sx = \sembrace{\pcce}^\circ}}{\sembrace{\pccB}^\circ}}
    \\&& \where{{\pcce : \pccA}}
  \end{array}
\end{displaymath}
Here, I write \im{\sembrace{\pcce}^\circ} to give a model of \im{\pcce} in
\slang extended with \tech{existential types} and the \tech{identity type}.
The \tech{translucent type} is modeled as a function that demands some
argument \im{\sx} of type \im{\pccA} and a proof that \im{\sx} is equal to
\im{\pcce}, using the \tech{identity type}.

The introduction and elimination forms are modeled as the following.
\begin{displaymath}
  \begin{array}{rcl}
  \sembrace{\pcce : \pcctrlufunty{\pccepr}{\pccB}}^\circ &\defeq& \snfune{(\sx:\sembrace{\pccA}^\circ,\sy:\sx = \sembrace{\pccepr}^\circ)}{\sfont{case}~\sy~\sfont{of}~\sfont{refl}.\sappe{\sembrace{\pcce}^\circ}{\sx}}
  \\
  \sembrace{\pccappe{\pcce}{\pccepr}}^\circ & \defeq & \sappe{\sembrace{\pcce}^\circ}{\sembrace{\pccepr}^\circ}{\sfont{refl}}
  \\&& \where{{\pcce : \pcctrlufunty{\pccepr}{\pccB}}}
  \end{array}
\end{displaymath}
The introduction form is modeled as a function of two arguments, which pattern
matches on the equality proof and applies the underlying function \im{\pcce} to
the argument \im{\pccx}.
The elimination form simply applies the \tech{translucent function} \im{\pcce}
to its argument \im{\pccepr} and the proof of equality \im{\sfont{refl}}.

The new \(\eta\)-equivalence for \tech{closures} is the only non-standard rule
and constructing a model of the rule requires a parametricity argument that I
leave for future work.
While essentially based on the \(\eta\)-principle for \tech{existential types},
it is not exactly the same.
It should be similar in some ways to the model for the \tech{CPS} target
language in \fullref[]{chp:cps}.
That is, a model should be possible using the parametricity translation into the
extensional Calculus of Constructions.
However, it is unclear how to formalize the relation between two
\tech{environments} of type \im{\pccalpha}.
Intuitively, we must allow any two \tech{environments}, with potentially
different sets of free variables, to be related as long as the \tech{codes} are
related after inlining the \tech{environments} into the \tech{codes}.
