\label{sec:param-cc:intro}
In this chapter, I develop a \deftech{parametric closure conversion}
translation, \ie, a \tech{closure conversion} translation that does not add
primitive \tech{closures} to the target language but instead encodes them using
\tech{existential types}.
As discussed in \fullref[]{chp:abs-cc}, this translation is not suitable in all
\tech{dependent type} theories, since it relies on \tech{parametricity} and
\tech{impredicativity}.
However, parametric \tech{closure conversion} has some advantages over \tech{abstract
  closure conversion} which makes studying this translation worthwhile.
In particular, known optimizations for recursive \tech{closures} that avoid
rebuilding the \tech{closure} every time through the loop use \tech{parametric
  closure conversion}~\cite{minamide1996,morrisett1998:reccc}.
We can develop a \tech{parametric closure conversion} if we admit
\tech{impredicativity} and \tech{parametricity}, and avoid \tech{higher
  universes}.

I begin with a review of the problems with \tech{parametric closure conversion}
discussed in \fullref[]{chp:abs-cc}.
To accommodate the restrictions, the \tech{parametric closure conversion}
translation is defined on \pccslang, the same source language used
in \fullref[]{chp:cps}, and a restriction of \slang presented
in \fullref[]{chp:source}.
I then design a target language, \pcctlang, before developing the full
translation, and proofs of type preservation and compiler correctness.

\begin{typographical}
 In this chapter, I typeset the source language, \pccslang, in
 a \emph{\sfonttext{blue, non-bold, sans-serif font}}, and the target
 language, \pcctlang, in a \emph{\tfonttext{bold, red, serif font}}.
\end{typographical}
