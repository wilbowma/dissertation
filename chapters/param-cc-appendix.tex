\renewcommand{\techprefix}{refpcc}
\renewcommand{\tlang}{\pcctlang}
\renewcommand{\slang}{\pccslang}

\chapter{Reference for \pcctlang}
\label{sec:param-cc:appendix}
This appendix contains the complete definitions for \pcctlang.
All \pcctlang figures from \fullref[]{chp:param-cc} are reproduced and completed
with elided parts here, and elided figures are presented here.
These figures contained extensions with definitions required to fully formalize
the compiler correctness results, including booleans with non-dependent
elimination.

\begin{typographical}
  In this appendix, I typeset \pcctlang in a \emph{\tfonttext{bold, red, serif
      font}}.
\end{typographical}

\FigCoCCCSyntaxFull[ht]
\FigCoCCCSugar[ht]
\FigCoCCCRedFull[ht]
\FigCoCCCConv[ht]
\FigCoCCCEquivFull[ht]
\FigCoCCCTypingOne[ht]
\FigCoCCCTypingTwo[ht]

\begingroup
\let\label\discard
\renewcommand{\cpstlang}{\pcctlang}
\renewcommand{\slang}{\pcctlang}
\renewcommand{\sfont}{\pccfont}
\renewcommand{\sfontsym}{\pccfontsym}
\renewcommand{\scolor}{\pcccolor}

\FigCPSWf[ht]
\FigECCObs[ht]
\FigECCProg[ht]
\FigECCEval[ht]

\endgroup

\begingroup
\let\label\discard
\renewcommand{\cpsslang}{\pcctlang}
\renewcommand{\sfont}{\pccfont}
\renewcommand{\sfontsym}{\pccfontsym}
\renewcommand{\scolor}{\pcccolor}

\FigCOCLinking
\endgroup

\FloatBarrier

\begin{listing}[h]
\begin{minted}{coq}
Definition translucent {A} e B := forall (x : A), (x = e) -> B.

Notation "e ⇒ B" := (translucent e B) (at level 42).

Definition translucent_intro {A B} {e' : A}
  (e : forall (x : A), B x) : e' ⇒ B e'
:= fun x w => match w with eq_refl => e x end.

Notation "e :T: A" := (translucent_intro e : A) (at level 99).

Definition translucent_elim {A B} {e' : A} (e : e' ⇒ B) : B
:= e e' eq_refl.

Notation "e @ m" := (@translucent_elim _ _ m e) (at level 99).

Inductive sigS {A:Type} (P:A -> Type) : Prop :=
  existS : forall x:A, P x -> sigS P.

Notation "'∃' x : A , P" := (sigS (A:=A) (fun x => P))
  (at level 0, x at level 99) : type_scope.

Definition Closure A B : Prop :=
  ∃ α : Type , ∃ env : α , (env ⇒ (A -> B)).
\end{minted}
\caption{Model of \pcctlang in Coq}
\end{listing}

\cleardoublepage
\chapter{Reference for Parametric Closure Conversion}
\label{sec:param-cc:cc:appendix}
This appendix contains the complete definitions for the parametric closure
conversion from \cpsslang to \pcctlang.
All translation figures from \fullref[]{chp:param-cc} are reproduced and completed
with elided parts here, and elided figures are presented here.

\begin{typographical}
  In this appendix, I typeset the source language, \pccslang, in a
  \emph{\sfonttext{blue, non-bold, sans-serif font}}, and the target language,
  \pcctlang, in a \emph{\tfonttext{bold, red, serif font}}.
\end{typographical}

\begingroup
\let\label\discard
\renewcommand{\cpstlang}{\pcctlang}

\FigCPSObsRel[ht]

\endgroup

\begingroup
\let\label\discard

\FigDFVs[ht]

\endgroup

\FigCCTermOne[ht]
\FigCCTermTwo[ht]
