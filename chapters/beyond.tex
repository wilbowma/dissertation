\chapter{Beyond ``Just'' Dependency}
\label{chp:beyond}
\todo{How might this scale to recursion, relevance, universe polymorphism (both
  explicit and implicit), univalence, effects, ???}
Future and related work, etc

\todo{Copy pasted stub:}
{
\paragraph{Toward a (Full-Spectrum) Dependently Typed Assembly Language}
\label{sec:future}
Ultimately we want to compile to a dependently typed machine-like language in which we can safely link
and then generate machine code.
A typical compiler for a functional language would perform CPS translation, closure-conversion, heap
allocation, and machine-code generation.
We have solved one of these issues, but the others introduce further challenges.

Past work shows that a standard CPS translation is not type-preserving in our
setting~\cite{barthe2002}.
Other CPS translations have not been studied, such as using a locally polymorphic answer type~\cite{thielecke04}.
This translation has proven useful in compiler verification work already~\cite{ahmed2011}.
Alternatively, we may try to target a stack-based machine language such as stack-based TAL~\cite{morrisett2002}.
This avoid the problems of CPS, but complicates the target language.

We have begun investigating heap allocation.
This pass raises theoretical and practical questions.
On the theoretical side, we must allow cycles in the heap to support recursive functions but still
ensure soundness and termination.
We may be able to continue adapting the guard condition to allow this.
Linear types have been used to allow cycles in the heap but still guarantee strong
normalization~\cite{morrisett2005}.
Unfortunately, linear types and dependent types are difficult to integrate~\cite{McBride2016}.
On the practical side, we must figure out how to efficiently allocate inductive data structures.
Past work shows how to avoid allocating computationally irrelevant indices from inductive
types~\cite{brady2003}, which we believe we can adapt to CIC.

The design of a dependently typed assembly language will be challenging for similar reasons that CPS
is not type-preserving.
\citet{herbelin2005} show that the ability to express \verb|call/cc| with $\Sigma$ types results in an
inconsistent logic.
That work also shows how to recover soundness by restricting the type of \verb|call/cc|.
We may need to use this design to build a sound dependently typed assembly.
}
